\documentclass{article}
\usepackage[utf8]{inputenc}
\usepackage{amsmath,latexsym}
\usepackage{amsfonts}
\usepackage{amsthm}
\usepackage{graphicx}
\usepackage[colorlinks=true, allcolors=blue]{hyperref}




\title{Preskill - QFT}
\author{ch.thierfelder }
\date{March 2021}

\begin{document}

\maketitle

\section{The free scalar field}
%1.1
To begin with, we will approach quantum field theory by starting with a (relativistic) classical theory, and then performing canonical quantization (canonical quantization is a procedure for formulating a quantum theory with a given classical limit). This may be a perverse choice of a place to begin if our real interest is in particles, but we will be able to justify it in retrospect.

So to get started we need to specify some relativistically invariant (frame independent) classical dynamics for \textcolor{red}{a field on set of fields}, and we'll consider at first the simplest case, that of a single scalar field.

Scalar refers to the transformation properties of the field under a Lorentz transformation.

\end{document}