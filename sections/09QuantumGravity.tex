\documentclass[../main.tex]{subfiles}

%\graphicspath{{\subfix{../images/}}}

\begin{document}

\section{{\sc Hartman} - Lectures on Quantum Gravity and Black Holes}

\section{{\sc Ammon, Erdmenger} - Gauge/Gravity Duality - Foundations and Applications}
The authors use $d-1$ spacial dimension and the sign convention 
\begin{align}
\eta_{\mu\nu}=diag(-1,1,...,1)
\end{align}
which implies 
\begin{align}
    \square&=\partial^\mu\partial_\mu=-\partial_t^2+\triangle\\
    kx&=-k^0x^0+\vec{k}\vec{x}
\end{align}
and results in a minus sign in the KG equation.

\subsection{Problem 1.1.1 - Fourier representation of free scalar field}
Ansatz (because KG equation looks quite similar to wave equation) $\phi(x)=a\cdot e^{ikx}$ with $x^\mu=(t,\vec{x})$, $k^\mu=(\omega,\vec{k})$ and $a\in\mathbb{C}$ meaning 
\begin{align}
    e^{ikx}\equiv e^{ik^{\mu}x_{\mu}}=e^{i\eta_{\mu\nu}k^{\mu}x^{\nu}}=e^{i(-k^0x^0+\vec{k}\vec{x})}
\end{align}
Inserting into the equation of motion
\begin{align}
    (\square - m^2)\phi(x)&=(\partial^t\partial_t + \triangle - m^2)\phi(x)\\
    &=a(-\partial_t^2 + \triangle - m^2)e^{i(-\omega t+\vec{k}\vec{x})}\\
    &=a\left(\omega^2 + i^2\vec{k}^2 - m^2\right)e^{i(-\omega t+\vec{k}\vec{x})}=0 
\end{align}
This implies $\omega^2-\vec{k}^2-m^2=0$ and therefore $\omega_k\equiv\omega=\sqrt{\vec{k}^2+m^2}$. One particular solution is therefore $\phi(x)=a\cdot e^{ikx}|_{k^0=\omega_k}$. The general solution is then given by a superposition
\begin{align}
    \phi(x)=\int d^{d-1}\vec{k}\left[a(\vec{k})e^{ikx}\right]
\end{align}
to ensure a real valued $\phi{x}$ we add the conjugate complex solution
\begin{align}
    \phi(x)=\int d^{d-1}\vec{k}\left[a(\vec{k})e^{ikx} + a^*(\vec{k})e^{-ikx}\right].
\end{align}
The factor $(2\pi)^{1-d}/2\omega_k$ can be absorbed into $a(k)$.

\subsection{Problem 1.1.2 - Lagrangian of self-interacting scalar field}
The Lagrangian is then
\begin{align}
    \mathcal{L}&=\mathcal{L}_\text{free}+\mathcal{L}_\text{int}\\
                &=-\frac{1}{2}\eta^{\mu\nu}\partial_\mu\phi(x)\partial_\nu\phi(x)-\frac{1}{2}m^2\phi(x)^2-\frac{g}{4!}\phi(x)^4.
\end{align}
with the Euler-Lagrange equations
\begin{align}
    \partial_\alpha\left(\frac{\partial\mathcal{L}}{\partial(\partial_\alpha\phi)}\right)-\frac{\partial\mathcal{L}}{\partial\phi}=0.
\end{align}
Therefore
\begin{align}
    \partial_\alpha\left(\frac{\partial\mathcal{L}}{\partial(\partial_\alpha\phi)}\right)
    &=\partial_\alpha\left(-\frac{1}{2}\eta^{\mu\nu}[\delta_{\mu\alpha}\partial_\nu\phi+\partial_\mu\phi\delta_{\nu\alpha}]\right)\\
    &=\partial_\alpha\left(-\frac{1}{2}\eta^{\alpha\nu}\partial_\nu\phi-\frac{1}{2}\eta^{\mu\alpha}\partial_\mu\phi\right)\\
    &=-\partial_\alpha\left(\eta^{\alpha\beta}\partial_\beta\phi\right)\\
    &=-\partial^\beta\partial_\beta\phi\\
    &=-\square\phi
\end{align}
and
\begin{align}
    \frac{\partial\mathcal{L}}{\partial\phi} = -m^2\phi-\frac{g}{3!}\phi^3.
\end{align}

The relevant term in the Euler-Lagrange equations is $\partial\mathcal{L}_\text{int}/\partial\phi=-g\phi^3/3!$. The modified equation of motion is therefore
\begin{align}
    (\square - m^2)\phi(x)-\frac{g}{3!}\phi(x)^3=0
\end{align}

\subsection{Problem 1.1.3 - Complex scalar field }
\begin{align}
    \mathcal{L}_\text{free}&=-\partial_\mu\phi^*\partial^\mu\phi-m^2\phi^*\phi\\
    &=-\eta^{\mu\nu}\partial_\mu\phi^*\partial_\nu\phi-m^2\phi^*\phi\\
    &=-\frac{1}{2}\eta^{\mu\nu}\partial_\mu(\phi_1-i\phi_2)\partial_\nu(\phi_1+i\phi_2)-\frac{1}{2}m^2(\phi_1^2+\phi_2^2)\\
    &=-\frac{1}{2}\eta^{\mu\nu}\left(
    \partial_\mu\phi_1\partial_\nu\phi_1
    +i\partial_\mu\phi_1\partial_\nu\phi_2
    -i\partial_\mu\phi_2\partial_\nu\phi_1
    +\partial_\mu\phi_2\partial_\nu\phi_2
    \right)-\frac{1}{2}m^2(\phi_1^2+\phi_2^2)\\
    &=-\frac{1}{2}\eta^{\mu\nu}\left(\partial_\mu\phi_1\partial_\nu\phi_1+\partial_\mu\phi_2\partial_\nu\phi_2\right)-\frac{1}{2}m^2(\phi_1^2+\phi_2^2)\\
    &=-\frac{1}{2}\eta^{\mu\nu}\partial_\mu\phi_1\partial_\nu\phi_1-\frac{1}{2}m^2\phi_1^2
    -\frac{1}{2}\eta^{\mu\nu}\partial_\mu\phi_2\partial_\nu\phi_2-\frac{1}{2}m^2\phi_2^2\\
    &=\mathcal{L}_\text{free1}+\mathcal{L}_\text{free2}
\end{align}

Equations of motion for $\phi$ and $\phi^*$ are given by
\begin{align}
    \partial_\alpha\left(\frac{\partial\mathcal{L}}{\partial(\partial_\alpha\phi^*)}\right)-\frac{\partial\mathcal{L}}{\partial\phi^*}=0\\
    %
    -\partial_\mu\partial^\mu\phi+m^2\phi=0\\
    (\square-m^2)\phi=0
\end{align}
and
\begin{align}
    \partial_\alpha\left(\frac{\partial\mathcal{L}}{\partial(\partial_\alpha\phi^*)}\right)-\frac{\partial\mathcal{L}}{\partial\phi}=0\\
    %
    -\partial_\mu\partial^\mu\phi+m^2\phi^*=0\\
    (\square-m^2)\phi^*=0
\end{align}

\subsection{Problem 1.2.1 - Time-independence of Noether charge}
The conserved current is
\begin{align}
    \partial_\mu\mathcal{J}^\mu\equiv-\partial_0\mathcal{J}^0+\partial_i\mathcal{J}^i=0.
\end{align}
Spacial integration using Gauss law on the right hand side gives
\begin{align}
    \int_{\mathbb{R}^{d-1}} d^{d-1}\vec{x}\;\partial_0\mathcal{J}^0&=\int_{\mathbb{R}^{d-1}} d^{d-1}\vec{x}\;\partial_i\mathcal{J}^i\\
    %
    \partial_0\int_{\mathbb{R}^{d-1}} d^{d-1}\vec{x}\;\mathcal{J}^0&=\int_{\partial\mathbb{R}^{d-1}} dS\;\mathcal{J}^i\\
    \partial_0\mathcal{Q}&=0
\end{align}
where we used that $\mathcal{J}^i$ is vanishing at infinity.

\subsection{Problem 1.2.2 - Hamiltonian of scalar field}
The Lagrangian of the real free scalar field is given by 
\begin{align}
    \mathcal{L}=-\frac{1}{2}\eta^{\mu\nu}\partial_\mu\phi(x)\partial_\nu\phi(x)-\frac{1}{2}m^2\phi(x)^2.
\end{align}
The canonical momentum is therefore
\begin{align}
    \Pi &= \frac{\partial\mathcal{L}}{\partial(\partial_t\phi)}\\
    &=-\frac{1}{2}2\eta^{ti}\partial_i\phi -\frac{1}{2}2\eta^{tt}\partial_t\phi\\
    &=\partial_t\phi.
\end{align}
Using $\eta_{\mu\nu}=diag(-1,1,...,1)$ the Hamiltonian $\mathcal{H}=\Theta^{tt}=\eta^{t\nu}\Theta^t_{\;\nu}=-\Theta^t_{\;t}$ is 
\begin{align}
    \Theta^t_{\;t}
    &=-\frac{\partial\mathcal{L}}{\partial(\partial_t\phi)}\partial_t\phi+\mathcal{L}\\
    &=-\Pi\cdot\partial_t\phi+\mathcal{L}
\end{align}
and therefore
\begin{align}
    \mathcal{H}&=\Pi\partial_t\phi-\mathcal{L}\\
    &=\Pi^2-\left(-\frac{1}{2}\eta^{\mu\nu}\partial_\mu\phi(x)\partial_\nu\phi(x)-\frac{1}{2}m^2\phi(x)^2\right)\\
    &=\Pi^2-\left(\frac{1}{2}(\partial_t\phi)^2-\frac{1}{2}(\nabla\phi)^2-\frac{1}{2}m^2\phi(x)^2\right)\\
    &=\frac{1}{2}\Pi^2+\frac{1}{2}(\nabla\phi)^2+\frac{1}{2}m^2\phi(x)^2
\end{align}

\subsection{Problem 1.2.3 - Symmetric energy-momentum tensor}
The Lorentz transformation
\begin{align}
    \Lambda^\mu_{\;\nu}=\delta^\mu_{\;\nu}+\omega^\mu_{\;\nu}
\end{align}
implies the field transformation
\begin{align}
    \phi(x^\mu)\rightarrow\tilde\phi(x^\mu)&=\phi(x^\mu-\omega^\mu_{\;\rho}x^\rho)\\
    &=\phi(x^\mu)-\omega^\mu_{\;\rho}x^\rho\partial_\mu\phi
\end{align}
under which the Lagrangian transforms as
\begin{align}
    \mathcal{L}\rightarrow\tilde{\mathcal{L}}&=\mathcal{L}+\frac{\partial\mathcal{L}}{\partial x^\mu}dx^\mu\\
    &=\mathcal{L}-\omega^\nu_{\;\rho}x^\rho\partial_\mu(\delta^\mu_{\;\nu}\mathcal{L})\\
    &=\mathcal{L}+\partial_\mu(\omega^\nu_{\;\rho}x^\rho)\cdot(\delta^\mu_{\;\nu}\mathcal{L})-\partial_\mu(\omega^\nu_{\;\rho}x^\rho\delta^\mu_{\;\nu}\mathcal{L})\\
    &=\mathcal{L}+\omega^\nu_{\;\rho}\delta^\rho_\mu\cdot(\delta^\mu_{\;\nu}\mathcal{L})-\partial_\mu(\omega^\nu_{\;\rho}x^\rho\delta^\mu_{\;\nu}\mathcal{L})\\
    &=\mathcal{L}+\omega^\rho_{\;\rho}\mathcal{L}-\partial_\mu(\omega^\nu_{\;\rho}x^\rho\delta^\mu_{\;\nu}\mathcal{L})\\
    &=\mathcal{L}-\partial_\mu(\omega^\nu_{\;\rho}x^\rho\delta^\mu_{\;\nu}\mathcal{L})
\end{align}
where we used $\omega_{\mu\nu}=-\omega_{\nu\mu}$ meaning
\begin{align}
    \omega^\rho_{\;\rho}&=\eta^{\alpha\rho}\omega_{\alpha\rho}\\
    &=\sum_\rho\eta^{0\rho}\omega_{0\rho}+\eta^{1\rho}\omega_{1\rho}+\eta^{2\rho}\omega_{2\rho}+\eta^{3\rho}\omega_{3\rho}\\
    &=0
\end{align}    
in the last step (as $\eta$ has only diagonal elements and the diagonal elements of $\omega$ are zero). With $\delta\phi=-\omega^\mu_{\;\rho}x^\rho\partial_\mu\phi$ and $X^\mu=-\omega^\nu_{\;\rho}x^\rho\delta^\mu_{\;\nu}\mathcal{L}$ we obtain for the conserved current  
\begin{align}
    \mathcal{J}^\mu&=-\frac{\partial\mathcal{L}}{\partial(\partial_\mu\phi)}\delta\phi+X^\mu\\
    &=-\frac{\partial\mathcal{L}}{\partial(\partial_\mu\phi)}(-\omega^\nu_{\;\rho}x^\rho\partial_\nu\phi)+(-\omega^\nu_{\;\rho}x^\rho\delta^\mu_{\;\nu}\mathcal{L})\\
    &=(-\omega^\nu_{\;\rho}x^\rho)\left(-\frac{\partial\mathcal{L}}{\partial(\partial_\mu\phi)}\partial_\nu\phi+(\delta^\mu_{\;\nu}\mathcal{L})\right)\\
    &=(-\omega^\nu_{\;\rho}x^\rho)\Theta^\mu_{\;\nu}\\
    &=(-\eta^{\nu\alpha}\omega_{\alpha\rho}x^\rho)\Theta^\mu_{\;\nu}\\
    &=-\omega_{\alpha\rho}x^\rho\Theta^{\mu\alpha}\\
    &=-\frac{1}{2}\omega_{\alpha\rho}(x^\rho\Theta^{\mu\alpha}-x^\alpha\Theta^{\mu\rho})\\
    &=-\frac{1}{2}\omega_{\alpha\rho}N^{\mu\rho\alpha}
\end{align}
With $\partial_\mu\Theta^\mu_{\;\nu}=0$ and $\partial_\mu N^{\mu\nu\rho}=0$ we see
\begin{align}
    0&=\partial_\mu N^{\mu\nu\rho}\\
    &= \partial_\mu\left( x^\nu\Theta^{\mu\rho}-x^\rho\Theta^{\mu\nu}\right)\\
    &= (\partial_\mu x^\nu) \Theta^{\mu\rho}+  x^\nu (\partial_\mu \Theta^{\mu\rho}) -(\partial_\mu x^\rho)\Theta^{\mu\nu} - x^\rho (\partial_\mu\Theta^{\mu\nu})\\
    &= \delta_\mu^\nu \Theta^{\mu\rho}+  x^\nu (\partial_\mu \Theta^{\mu\rho}) -\delta_\mu^\rho\Theta^{\mu\nu} - x^\rho (\partial_\mu\Theta^{\mu\nu})\\
    &= \Theta^{\nu\rho} - \Theta^{\rho\nu}.
\end{align}
which means that the (canonical) energy-momentum tensor for Poincare invariant field theories is symmetric $\Theta^{\nu\rho} = \Theta^{\rho\nu}$.


\subsection{Problem 1.2.4 - Callan-Coleman-Jackiw energy-momentum tensor}
For the scalar field we have with $\mathcal{L}=-\frac{1}{2}\eta^{\alpha\beta}\partial_\alpha\phi\partial_\beta\phi-\frac{1}{2}m^2\phi^2$
\begin{align}
    \Theta^\mu_{\;\nu}&=-\frac{\partial\mathcal{L}}{\partial(\partial_\mu\phi)}\partial_\nu\phi+(\delta^\mu_{\;\nu}\mathcal{L})\\
    &=-\left(-\frac{1}{2}\eta^{\alpha\beta}\delta^\mu_{\alpha}\partial_\beta\phi-\frac{1}{2}\eta^{\alpha\beta}\partial_\alpha\phi\delta^\mu_{\beta}\right)\partial_\nu\phi +\delta^\mu_{\;\nu}\left(-\frac{1}{2}\eta^{\alpha\beta}\partial_\alpha\phi\partial_\beta\phi-\frac{1}{2}m^2\phi^2\right)\\
    &=\partial^\mu\phi\partial_\nu\phi -\frac{1}{2}\delta^\mu_{\;\nu}(\partial^\beta\phi\partial_\beta\phi+m^2\phi^2)
\end{align}
which gives in the massless case
\begin{align}
    \Theta^\mu_{\;\nu\text{, massless}}&=\partial^\mu\phi\partial_\nu\phi -\frac{1}{2}\delta^\mu_{\;\nu}\partial^\beta\phi\partial_\beta\phi\\
    \Theta_{\mu\nu\text{, massless}}&=\partial_\mu\phi\partial_\nu\phi -\frac{1}{2}\eta_{\mu\nu}\partial^\beta\phi\partial_\beta\phi
\end{align}

The new improved or Callan–Coleman–Jackiw energy-momentum tensor for a single, real, massless scalar field in $d$-dimensional Minkowski space is obtained by adding a term proportional to $(\partial_\mu\partial_\nu-\eta_{\mu\nu}\square)\phi^2$ where the proportionality constant is chosen to make the tensor traceless
\begin{align}
    T_{\mu\nu}=\partial_\mu\phi\partial_\nu\phi-\frac{1}{2}\eta_{\mu\nu}\partial_\rho\phi\partial^\rho\phi-\frac{d-2}{4(d-1)}\left(\partial_\mu\partial_\nu-\eta_{\mu\nu}\square\right)\phi^2
\end{align}
Let us now check the properties
\begin{enumerate}
    \item symmetric: obvious
    \item conserved: we use the equation of motion $\partial^\mu\partial_\mu\phi=\square\phi=0$
    \begin{align}
        \partial_\mu T^{\mu\nu}&=(\partial_\mu\partial^\mu\phi)\partial^\nu\phi+\partial^\mu\phi(\partial_\mu\partial^\nu\phi)\\
        &\quad-\frac{1}{2}\eta^{\mu\nu}\left[(\partial_\mu\partial_\rho\phi)\partial^\rho\phi + \partial_\rho\phi(\partial_\mu\partial^\rho\phi)\right]\\
        &\quad-\frac{d-2}{4(d-1)}\square\partial^\nu\phi^2+\frac{d-2}{4(d-1)}\eta^{\mu\nu}\partial_\mu\square\phi^2\\
        &=\partial^\mu\phi(\partial_\mu\partial^\nu\phi)-\frac{1}{2}\left[(\partial^\nu\partial_\rho\phi)\partial^\rho\phi + \partial_\rho\phi(\partial^\nu\partial^\rho\phi)\right]\\
        &=0
    \end{align}
    \item traceless: 
    \begin{align}
    T^\mu_{\;\mu}&=\partial^\mu\phi\partial_\mu\phi-\frac{1}{2}\eta^\mu_{\;\mu}\partial_\rho\phi\partial^\rho\phi-\frac{d-2}{4(d-1)}\left(\partial^\mu\partial_\mu-\eta^\mu_{\;\mu}\square\right)\phi^2\\
    &=\partial^\mu\phi\partial_\mu\phi-\frac{d}{2}\partial_\rho\phi\partial^\rho\phi-\frac{d-2}{4(d-1)}\left(\partial^\mu\partial_\mu-d\cdot\partial^\mu\partial_\mu\right)\phi^2\\
    &=\frac{2-d}{2}\partial_\rho\phi\partial^\rho\phi-\frac{d-2}{4(d-1)}(1-d)\partial^\mu\partial_\mu\phi^2\\
    &=\frac{2-d}{2}\partial_\rho\phi\partial^\rho\phi+\frac{d-2}{4}\partial^\mu\partial_\mu\phi^2\\
    &=\frac{2-d}{2}\partial_\rho\phi\partial^\rho\phi+\frac{d-2}{4}\partial^\mu(2\phi\partial_\mu\phi)\\
    &=\frac{2-d}{2}[\partial_\rho\phi\partial^\rho\phi-\partial^\mu\phi\partial_\mu\phi]+\frac{d-2}{2}\phi\cdot\square\phi\\
    &=0.
\end{align}
\end{enumerate}

\subsection{Problem 1.2.5 - Noether currents of complex scalar field}
\begin{align}
    \mathcal{L}_\text{free}&=-\partial^\mu\phi^*\partial_\mu\phi-m^2\phi^*\phi\\
    &=-\eta^{\mu\nu}\partial_\nu\phi^*\partial_\nu\phi-m^2\phi^*\phi
\end{align}
with the field transformations
\begin{align}
    \phi\rightarrow\phi'&=e^{i\alpha}\phi=\phi+i\alpha\phi\\
    \phi^*\rightarrow\phi^{*'}&=e^{-i\alpha}\phi^*=\phi^*-i\alpha\phi^*\\
    \mathcal{L}\rightarrow\mathcal{L}'&=\mathcal{L}
\end{align}
we have $\delta\phi=i\alpha\phi$ and $\delta\phi^*=-i\alpha\phi^*$ and $X^\mu=0$. With 
\begin{align}
    \mathcal{J}^\sigma&=-\frac{\partial\mathcal{L}}{\partial(\partial_\sigma\phi)}\delta\phi+X^\sigma
\end{align}
we obtain the the two fields
\begin{align}    
    \mathcal{J}^\sigma&=-\frac{\partial\mathcal{L}}{\partial(\partial_\sigma\phi)}\delta\phi-\frac{\partial\mathcal{L}}{\partial(\partial_\sigma\phi^*)}\delta\phi^*\\
    &=-(\eta^{\sigma\nu}\partial_\nu\phi^*)i\alpha\phi+(\eta^{\sigma\nu}\partial_\nu\phi)i\alpha\phi^*\\
    &=i\alpha\left[\phi^*(\partial^\sigma\phi)-\phi(\partial^\sigma\phi^*)\right]
\end{align}

\subsection{Problem 1.2.6 - \texorpdfstring{$O(n)$}{Lg} invariance of action of \texorpdfstring{$n$}{Lg} free scalar fields}
For the $n$ real scalar fields with equal mass $m$ we have
\begin{align}
    \mathcal{L}=-\frac{1}{2}\sum_{j=1}^n\left[\eta^{\alpha\beta}(\partial_\alpha\phi_j)(\partial_\beta\phi^j)+m^2(\phi^j)^2\right]
\end{align}
the action functional is then
\begin{align}
    S&=\int d^dx\mathcal{L}\\
    &=-\frac{1}{2}\sum_{j=1}^n\int d^dx\left[\eta^{\alpha\beta}(\partial_\alpha\phi_j)(\partial_\beta\phi^j)+m^2(\phi_j\phi^j)\right]
\end{align}
With $\phi'^{j}=R^j_{\;k}\phi^k$ and the definition of an orthogonal matrix $R$ (inner product is invariant under rotation)
\begin{align}
    x^ix_i&=x^i\delta_{ij}x^j\\
    &\stackrel{!}{=}R^i_{\;a}x^a\delta_{ij}R^j_{\;b}x^b\\
    &=\delta_{ij}R^j_{\;b}R^i_{\;a}x^ax^b\\
    &=R_{ib}R^i_{\;a}x^ax^b
\end{align}
we require $R_{ib}R^i_{\;a}=\delta_{ba}$. Then we can recalculate the action
\begin{align}
    S'&=-\frac{1}{2}\sum_{j=1}^n\int d^dx\left[\eta^{\alpha\beta}(\partial_\alpha R_{ja}\phi^a)(\partial_\beta R^j_{\;b}\phi^b)+m^2(R_{ja}\phi^a\cdot R^j_{\;b}\phi^b)\right]\\
    &=-\frac{1}{2}\sum_{j=1}^n\int d^dx\left[\eta^{\alpha\beta}R_{ja}R^j_{\;b}(\partial_\alpha \phi^a)(\partial_\beta \phi^b)+m^2R_{ja}R^j_{\;b}(\phi^a\cdot \phi^b)\right]\\
    &=-\frac{1}{2}\sum_{b=1}^n\int d^dx\left[\eta^{\alpha\beta}\delta_{ab}(\partial_\alpha \phi^a)(\partial_\beta \phi^b)+m^2\delta_{ab}(\phi^a\cdot \phi^b)\right]\\
    &=-\frac{1}{2}\sum_{b=1}^n\int d^dx\left[\eta^{\alpha\beta}(\partial_\alpha \phi_b)(\partial_\beta \phi^b)+m^2(\phi_b\cdot \phi^b)\right]   
\end{align}
\textcolor{red}{Analog for the complex case.}

\subsection{Problem 1.3.1 - Field commutators of scalar field}
From the field  
\begin{align}
    \hat\phi(x)&=\frac{1}{(2\pi)^{d-1}}\int \frac{d^{d-1}\vec{k}}{2\omega_k}\left[\hat a(\vec{k})e^{ikx} + \hat a^\dagger(\vec{k})e^{-ikx}\right]_{k^0=\omega_k}
\end{align}
we can derive the conjugated momentum
\begin{align}
    \hat\Pi(x)&=\partial_t\hat\phi\\
    &=\frac{1}{(2\pi)^{d-1}}\int \frac{d^{d-1}\vec{k}}{2\omega_k}\partial_t\left[\hat a(\vec{k})e^{-i\omega_kt}e^{i\vec{k}\vec{x}} + \hat a^\dagger(\vec{k})e^{i\omega_kt}e^{-i\vec{k}\vec{x}}\right]\\
    &=\frac{1}{(2\pi)^{d-1}}\int \frac{d^{d-1}\vec{k}}{2\omega_k}\left[\hat a(\vec{k})(-i\omega_k)e^{ikx} + \hat a^\dagger(\vec{k})(i\omega_k)e^{-ikx}\right]_{k^0=\omega_k}\\
    &=\frac{i}{2(2\pi)^{d-1}}\int d^{d-1}\vec{k}\left[-\hat a(\vec{k})e^{ikx} + \hat a^\dagger(\vec{k})e^{-ikx}\right]_{k^0=\omega_k}.
\end{align}
Now calculating the three commutation relations
\begin{itemize}
\item $[\hat\phi(t,\vec{x}),\hat\phi(t,\vec{y})]$
\begin{align}
    &=\frac{1}{(2\pi)^{2(d-1)}}\int \frac{d^{d-1}\vec{k}d^{d-1}\vec{q}}{4\omega_k \omega_q}
    \left((\hat a(\vec{k})e^{ikx} + \hat a^\dagger(\vec{k})e^{-ikx})
    (\hat a(\vec{q})e^{iqy} + \hat a^\dagger(\vec{q})e^{-iqy})\right. - \\ 
    &\quad\left.(\hat a(\vec{q})e^{iqy} + \hat a^\dagger(\vec{q})e^{-iqy})
    (\hat a(\vec{k})e^{ikx} + \hat a^\dagger(\vec{k})e^{-ikx}) \right)
\end{align}
the bracket can then be simplified
\begin{align}
    (\hat a(\vec{k})&e^{ikx} + \hat a^\dagger(\vec{k})e^{-ikx})(\hat a(\vec{q})e^{iqy} + \hat a^\dagger(\vec{q})e^{-iqy})-(\hat a(\vec{q})e^{iqy} + \hat a^\dagger(\vec{q})e^{-iqy})
    (\hat a(\vec{k})e^{ikx} + \hat a^\dagger(\vec{k})e^{-ikx}) \\
    %
    &=[\hat a(\vec{k}),\hat a(\vec{q})]e^{i(kx+qy)}+[\hat a(\vec{k}),\hat a^\dagger(\vec{q})]e^{i(kx-qy)}+[\hat a^\dagger(\vec{k}),\hat a(\vec{q})]e^{i(-kx+qy)}+[\hat a^\dagger(\vec{k}),\hat a^\dagger(\vec{q})]e^{i(-kx-qy)}\\
    &=[\hat a(\vec{k}),\hat a^\dagger(\vec{q})]e^{i(kx-qy)}-[\hat a(\vec{q}),\hat a^\dagger(\vec{k})]e^{i(-kx+qy)}\\
    &=2\omega_k(2\pi)^{d-1}\left(\delta^{d-1}(\vec{k}-\vec{q})e^{i(kx-qy)}-\delta^{d-1}(\vec{q}-\vec{k})e^{i(-kx+qy)}\right)
\end{align}
where we used the given commutation relations for $\hat a(\vec{k})$.
\begin{align}
    [\hat\phi(t,\vec{x}),\hat\phi(t,\vec{y})] 
    &= \frac{1}{(2\pi)^{2(d-1)}}\int \frac{d^{d-1}\vec{k}d^{d-1}\vec{q}}{4\omega_k \omega_q}2\omega_k(2\pi)^{d-1}\left(\delta^{d-1}(\vec{k}-\vec{q})e^{i(kx-qy)}-\delta^{d-1}(\vec{q}-\vec{k})e^{i(-kx+qy)}\right)\\
    &= \frac{1}{(2\pi)^{d-1}}\int \frac{d^{d-1}\vec{k}d^{d-1}\vec{q}}{2 \omega_q}\left(\delta^{d-1}(\vec{k}-\vec{q})e^{i(kx-qy)}-\delta^{d-1}(\vec{q}-\vec{k})e^{i(-kx+qy)}\right)\\
    &= \frac{1}{(2\pi)^{d-1}}\int \frac{d^{d-1}\vec{k}d^{d-1}\vec{q}}{2 \omega_q}\left(\delta^{d-1}(\vec{k}-\vec{q})e^{i(-\omega_kt+\vec{k}\vec{x}-[-\omega_qt+\vec{q}\vec{y}]))}\right.\\
    &\qquad\qquad\qquad\qquad\left.-\delta^{d-1}(\vec{q}-\vec{k})e^{-i(-\omega_kt+\vec{k}\vec{x}-[-\omega_qt+\vec{q}\vec{y}]))}\right)\\
    &= \frac{1}{(2\pi)^{d-1}}\int \frac{d^{d-1}\vec{k}d^{d-1}\vec{q}}{2 \omega_q}\left(\delta^{d-1}(\vec{k}-\vec{q})e^{i(-[\omega_k-\omega_q]t+\vec{k}\vec{x}-\vec{q}\vec{y})}\right.\\
    &\qquad\qquad\qquad\qquad\left.-\delta^{d-1}(\vec{q}-\vec{k})e^{-i(-[\omega_k-\omega_q]t+\vec{k}\vec{x}-\vec{q}\vec{y})}\right)\\
    &= \frac{1}{(2\pi)^{d-1}}\int \frac{d^{d-1}\vec{k}}{2 \omega_k}\left(e^{i\vec{k}(\vec{x}-\vec{y})}-e^{-i\vec{k}(\vec{x}-\vec{y})}\right)\\
    &= \frac{1}{2 \omega_k}\left( \delta^{d-1}(\vec{y}-\vec{x})-\delta^{d-1}(\vec{x}-\vec{y})\right)\\
    &=0
\end{align}
where we used $\delta(x)=\int dk e^{-2\pi i kx}$ or $\delta^d(x)=\int \frac{d^dk}{(2\pi)^d} e^{-ikx}$.

\item $[\hat\Pi(t,\vec{x}),\hat\Pi(t,\vec{y})]$
\textcolor{red}{Not done yet}
\item $[\hat\phi(t,\vec{x}),\hat\Pi(t,\vec{y})]$
\textcolor{red}{Not done yet}
\end{itemize}

\subsection{Problem 1.3.2 - Lorentz invariant integration measure}
We use the property of the $\delta$-function $\delta(f(x))=\sum_i\frac{\delta(x-a_i)}{]f'(a_i)|}$ where ${a_i}$ are the zeros of $f(x)$ and $\omega_k=\sqrt{\vec{k}^2+m^2}$. With $\int d^dk$ being manifestly Lorentz invariant 
\begin{align}
    dk'^\mu = \Lambda^\mu_\nu dk^\nu\quad\rightarrow\quad \frac{dk'^\mu}{dk^\nu}=\Lambda^\mu_\nu\quad\rightarrow\quad\int d^d k'=|\text{det}(\Lambda^\mu_\nu)|\int d^dk=\int d^dk
\end{align}
$\delta^d[k^2+m^2]$ being invariant and with $k^0=\sqrt{\vec{k}^2+m^2}$ we see that $k$ is inside the forward light cone and remains there under orthochrone transformation ($\Theta(k^0)$ is invariant for relevant $k$) we are convinced that the starting expression is Lorentz invariant (integration over the upper mass shell)
\begin{align}
    \int d^{d}\vec{k} \delta^d[k^2+m^2]\Theta(k^0) &=\int d^{d-1}\vec{k}\int dk^0\delta^d[k^2+m^2]\Theta(k^0)\\
    &=\int d^{d-1}\vec{k}\int dk^0\delta^d[-(k^0)^2+\vec{k}^2+m^2]\Theta(k^0)\\
    &=\int d^{d-1}\vec{k}\int dk^0\delta^d[\omega_k^2-(k^0)^2]\Theta(k^0)\\
    &=\int d^{d-1}\vec{k}\int dk^0\left( \frac{\delta(k^0-\omega_k)}{2\omega_k}+\frac{\delta(k^0+\omega_k)}{2\omega_k} \right)\Theta(k^0)\\
    &=\int \frac{d^{d-1}\vec{k}}{2\omega_k}\int dk^0\delta(k^0-\omega_k)\\
    &=\int\frac{d^{d-1}\vec{k}}{2\omega_k}.
\end{align}
As we started with a Lorentz invariant expression the derived measure is also invariant.

\subsection{Problem 1.3.3 - Retarded Green function}
\begin{align}
    \Delta_\text{F}&=\int\frac{d^dk}{(2\pi)^d}\frac{e^{ik(x-y)}}{k^2+m^2-i\epsilon}\\
    G_\text{R}&=\int\frac{d^dk}{(2\pi)^d}\frac{e^{ik(x-y)}}{-(k^0+i\epsilon)^2+\vec{k}^2+m^2}
\end{align}
For the poles of $G_\text{R}$ we have
\begin{align}
    -(k^0+i&\epsilon)^2+\vec{k}^2+m^2 = 0\\
    k^0&=-i\epsilon\pm\sqrt{\vec{k}^2+m^2}\\
    &=-i\epsilon\pm\omega_k
\end{align}
while we the poles of $\Delta_\text{F}$ are given by
\begin{align}
    -(k^0)^2+&\vec{k^2}+m^2-i\epsilon=0\\
    k^0&=\pm\sqrt{\vec{k^2}+m^2-i\epsilon}\\
    &=\pm\sqrt{\omega_k^2-i\epsilon}
\end{align}
\begin{figure}[h]
\centering
\resizebox{5cm}{!}{%
    \begin{tikzpicture}
        \draw (-4,0) -- (4,0);
        \node[mark size=5pt,color=blue] at (+2,-0.5) {\pgfuseplotmark{triangle}};
        \node[mark size=5pt,color=blue] at (-2,+0.5) {\pgfuseplotmark{triangle}};
        \draw[red,thick] (-2,-0.5) circle (0.1cm);
        \draw[red,thick] (+2,-0.5) circle (0.1cm);
        \draw[black,thick] (-2,0) circle (0.05cm);
        \draw[black,thick] (+2,0) circle (0.05cm);
        \node[label=left:{$-\omega_k$}] () at (-1.4,0.25) {};
        \node[label=left:{$+\omega_k$}] () at (2.6,0.25) {};
        \node[label=left:{$\text{Re} k^0$}] () at (5.4,0.05) {};
    \end{tikzpicture}%
}
\caption{Poles of $G_\text{R}$ (circle) and $\Delta_\text{F}$ (triangle)}
\end{figure}

With $|\vec{k}\rangle=a^\dagger(\vec{k})|0\rangle$ and
\begin{align}
    \hat\phi(x)&=\frac{1}{(2\pi)^{d-1}}\int \frac{d^{d-1}\vec{k}}{2\omega_k}\left[\hat a(\vec{k})e^{ikx} + \hat a^\dagger(\vec{k})e^{-ikx}\right]_{k^0=\omega_k}
\end{align}
we obtain
\begin{align}
    \hat\phi(x)\hat\phi(y) &\sim \left(\hat a(\vec{k})e^{ikx} + \hat a^\dagger(\vec{k})e^{-ikx}\right)\left(\hat a(\vec{q})e^{iqy} + \hat a^\dagger(\vec{q})e^{-iqy}\right)\\
    &=\hat a(\vec{k})\hat a(\vec{q})e^{i(kx+qy)} 
    + \hat a(\vec{k})\hat a^\dagger(\vec{q})e^{-i(-kx+qy)} 
    +\hat a^\dagger(\vec{k})\hat a(\vec{q})e^{i(-kx+qy)} 
    + \hat a^\dagger(\vec{k})\hat a^\dagger(\vec{q})e^{-i(kx+qy)}\\
    &=\hat a(\vec{k})\hat a(\vec{q})e^{i(kx+qy)} 
    + \hat a(\vec{k})\hat a^\dagger(\vec{q})e^{-i(-kx+qy)} 
    + \hat a^\dagger(\vec{k})\hat a^\dagger(\vec{q})e^{-i(kx+qy)}\\
    &\quad+\left(\hat a(\vec{q})\hat a^\dagger(\vec{k})-2\omega_k(2\pi)^{d-1}\delta^{d-1}(\vec{q}-\vec{k})\right)e^{i(-kx+qy)}
\end{align}
and therefore
\begin{align}
    \langle0|\hat\phi(x)\hat\phi(y)|0\rangle
    &=\frac{1}{(2\pi)^{2(d-1)}}\int \frac{d^{d-1}\vec{k}}{2\omega_k}\frac{d^{d-1}\vec{q}}{2\omega_q} \langle0|\hat a(\vec{k})\hat a(\vec{q})|0\rangle e^{i(kx+qy)} 
    + \langle0|\hat a(\vec{k})\hat a^\dagger(\vec{q})|0\rangle e^{-i(-kx+qy)}\\
    &\quad 
    + \langle0|\hat a^\dagger(\vec{k})\hat a^\dagger(\vec{q})|0\rangle e^{-i(kx+qy)}+\left(\langle0|\hat a(\vec{q})\hat a^\dagger(\vec{k})|0\rangle-2\omega_k(2\pi)^{d-1}\delta^{d-1}(\vec{q}-\vec{k})\right)e^{i(-kx+qy)}\\
    &=\frac{1}{(2\pi)^{2(d-1)}}\int \frac{d^{d-1}\vec{k}}{2\omega_k}\frac{d^{d-1}\vec{q}}{2\omega_q}
    \langle\vec{k}|\vec{q}\rangle e^{-i(-kx+qy)}+\left(\langle\vec{q}|\vec{k}\rangle-2\omega_k(2\pi)^{d-1}\delta^{d-1}(\vec{q}-\vec{k})\right)e^{i(-kx+qy)}\\
\end{align}

\textcolor{red}{Not done yet}

\subsection{Problem 1.3.4 - Feynman rules of \texorpdfstring{$\phi^4$}{Lg} theory}
\textcolor{red}{Not done yet}

\subsection{Problem 1.3.5 - Convergence of perturbative expansion}
\textcolor{red}{Not done yet}

\subsection{Problem 1.3.6}
\textcolor{red}{Not done yet}

\subsection{Problem 1.3.7}
\textcolor{red}{Not done yet}

\subsection{Problem 1.3.8}
\textcolor{red}{Not done yet}



\end{document}