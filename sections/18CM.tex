\documentclass[../main.tex]{subfiles}

%\graphicspath{{\subfix{../images/}}}

\begin{document}
\section{{\sc Straumann} -  Mechanik 2015}

\subsection{Exercise 1.2 - Free vertical fall with friction - NOT DONE YET}
Equation of motion
\begin{align}
m\ddot{y}+mg-\alpha\dot{y}^2&=0\\
\ddot{y}+g-\beta\dot{y}^2&=0
\end{align}
Now we can substitude $v=\dot{y}$ and obtain
\begin{align}
\dot{v}+g-\beta v^2&=0\\
\int\frac{dv}{v^2-g/\beta}&=\beta (t+c)\\
\int\frac{dv}{v-\sqrt{g/\beta}}-\int\frac{dv}{v+\sqrt{g/\beta}}&=2\sqrt{\frac{g}{\beta}}\beta (t+c)\\
\log\left(v-\sqrt{g/\beta}\right)-\log\left(v+\sqrt{g/\beta}\right)
&=2\sqrt{g\beta} (t+c)\\
\frac{1}{2}\log\frac{v+\sqrt{g/\beta}}{v-\sqrt{g/\beta}}&=-\sqrt{g\beta}(t+c)\\
\frac{1}{2}\log\frac{\sqrt{\beta/g} v+1}{\sqrt{\beta/g}v-1}&=-\sqrt{g\beta}(t+c)\\
\text{arctanh}\sqrt{\frac{\beta}{g}}v+\frac{1}{2}\log(-1)&=-\sqrt{g\beta}(t+c)
\end{align}

Limit velocity ($\ddot{y}=0$)
\begin{align}
v_\infty=\sqrt{\frac{mg}{\alpha}}
\end{align}

\section{{\sc Goldstein, Poole, Safko} - Classical Mechanics 3rd ed}
\subsection{Exercise 9.1 - Canonical Coordinates}
Try the generalized transformation where ($\alpha=1=\beta$) is the original trafo
\begin{align}
Q&=\alpha(q+ip)\qquad q=\frac{1}{2\alpha}Q+\frac{1}{2\beta}P\\
P&=\beta(q-ip)\qquad p=\frac{1}{2i\alpha}Q-\frac{1}{2i\beta}P
\end{align}
then
\begin{align}
\dot{Q}&=\frac{\partial Q}{\partial p}\frac{\partial p}{\partial t}+\frac{\partial Q}{\partial q}\frac{\partial q}{\partial t}=-i\alpha\frac{\partial H}{\partial q}+\alpha\frac{\partial H}{\partial p}\\
\dot{P}&=\frac{\partial P}{\partial p}\frac{\partial p}{\partial t}+\frac{\partial P}{\partial q}\frac{\partial q}{\partial t}=+i\beta\frac{\partial H}{\partial q}+\beta\frac{\partial H}{\partial p}
\end{align}
and also
\begin{align}
\frac{\partial H(q(Q,P),p(q,P))}{\partial Q}&=\frac{\partial H}{\partial q}\frac{\partial q}{\partial Q}+\frac{\partial H}{\partial p}\frac{\partial p}{\partial Q}\\
&=\frac{\partial H}{\partial q}\frac{1}{2\alpha}+\frac{\partial H}{\partial p}\frac{1}{2i\alpha}\\
\frac{\partial H(q(Q,P),p(q,P))}{\partial P}&=\frac{\partial H}{\partial q}\frac{\partial q}{\partial P}+\frac{\partial H}{\partial p}\frac{\partial p}{\partial P}\\
&=\frac{\partial H}{\partial q}\frac{1}{2\beta}+\frac{\partial H}{\partial p}\frac{i}{2\beta}
\end{align}
which implies
\begin{align}
\frac{\partial H}{\partial q}&=\alpha\frac{\partial H}{\partial Q}+\beta\frac{\partial H}{\partial P}\\
\frac{\partial H}{\partial p}&=\frac{1}{i}\left(\beta\frac{\partial H}{\partial P}-\alpha\frac{\partial H}{\partial Q}\right)
\end{align}
which finally results in 
\begin{align}
\dot{Q}&=-i\alpha\left(\alpha\frac{\partial H}{\partial Q}+\beta\frac{\partial H}{\partial P}\right)+\alpha\frac{1}{i}\left(\beta\frac{\partial H}{\partial P}-\alpha\frac{\partial H}{\partial Q}\right)\\
&=-2i\alpha\beta\frac{\partial H}{\partial P}\\
\dot{P}&=2i\alpha\beta\frac{\partial H}{\partial Q}
\end{align}
So we see
\begin{itemize}
\item for $\alpha=1\beta$ the equations are not canonical
\item for $\alpha=\frac{i}{2}$  and $\beta=1$ the equations are  canonical
\end{itemize}


\subsection{Exercise 12.5 - Anharmonic oscillator - NOT DONE YET}

\begin{align}
L=\frac{1}{2}m\dot{x}^2-2\cdot\frac{1}{2}k[\sqrt{a^2+x^2}-b]^2
\end{align}

\end{document}