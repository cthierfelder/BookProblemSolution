\documentclass[../main.tex]{subfiles}

%\graphicspath{{\subfix{../images/}}}

\begin{document}

\section{{\sc Schmueser} - Feynman-Graphen und Eichtheorie}
\subsection{Problem 1.1}
\begin{align}
\mathbf{\sigma}\cdot\mathbf{a}&=
\left(\begin{matrix}
a_3 & a_1-ia_2\\
a_1+ia_2 & -a_3\\
\end{matrix}\right)\\
(\mathbf{\sigma}\cdot\mathbf{a})(\mathbf{\sigma}\cdot\mathbf{b})&=
\left(\begin{matrix}
a_3 & a_1-ia_2\\
a_1+ia_2 & -a_3\\
\end{matrix}\right)\left(\begin{matrix}
b_3 & b_1-ib_2\\
b_1+ib_2 & -b_3\\
\end{matrix}\right)\\
&=\mathbf{a}\cdot\mathbf{b}\,1_2+i\mathbf{\sigma}\cdot(\mathbf{a}\times\mathbf{b})
\end{align}

\subsection{Problem 1.2}
\begin{align}
\mathbf{P}\times\mathbf{P}
&=(-i\hbar)^2\underbrace{(\nabla\times\nabla)}_{=0}+e^2\underbrace{(\mathbf{A}\times\mathbf{A})}_{=0}-i\hbar e(\nabla\times\mathbf{A}+\mathbf{A}\times\nabla)\\
&=-i\hbar e(\nabla\times\mathbf{A}+\mathbf{A}\times\nabla)\\
&=-i\hbar e\left(\begin{matrix}
\partial_yA_z-\partial_zA_y+A_y\partial_z-A_z\partial_y\\
...\\
...
\end{matrix}\right)\\
&=-i\hbar e\left(\begin{matrix}
(\partial_yA_z+A_z\partial_y)-(\partial_zA_y+A_y\partial_z)+A_y\partial_z-A_z\partial_y\\
...\\
...
\end{matrix}\right)\\
&=-i\hbar e\left(\begin{matrix}
\partial_yA_z-\partial_zA_y\\
...\\
...
\end{matrix}\right)\\
&=-i\hbar e\mathbf{B}
\end{align}
and therefore
\begin{align}
(\mathbf{\sigma}\cdot\mathbf{P})(\mathbf{\sigma}\cdot\mathbf{P})
&=\mathbf{P}^2\,1_2+e\hbar\mathbf{\sigma}\cdot\mathbf{B}
\end{align}

\section{{\sc Lancaster, Blundell} - Quantum Field Theory for the gifted amateur}
\subsection*{Exercise 1.1 - Snell's law via Fermat's principle}
The light travels from point $A$ in medium 1 to point $B$ in medium 2. We assume a vertical medium boundary at $x_0$ and that the light travels within a medium in the straight line. This makes $y_0$ the free parameter and the the travel time is given by
\begin{align}
    t&=\frac{s_{A0}}{c/n_1}+\frac{s_{0B}}{c/n_2}\\
    &=\sqrt{\frac{(x_A-x_0)^2+(y_A-y_0)^2}{c/n_1}}+\sqrt{\frac{(x_0-x_B)^2+(y_0-y_B)^2}{c/n_2}}
\end{align}
The local extrema of the travel time is given by
\begin{align}
    0&=\frac{dt}{dy_0}\\
    &=\frac{y_A-y_0}{s_{A0}c/n_1}+\frac{y_0-y_B}{s_{0B}c/n_2}\\
    &=\frac{\sin\alpha}{c/n_1}-\frac{\sin\beta}{c/n_2}
\end{align}
and therefore
\begin{align}
n_1\sin\alpha=n_2\sin\beta.
\end{align}

\subsection*{Exercise 1.2 - Functional derivatives I}
\begin{itemize}
\item $H[f]=\int G(x,y)f(y)dy$ 
\begin{align}
\frac{\delta H[f]}{\delta f(z)}
&=\lim_{\epsilon\rightarrow0}\frac{1}{\epsilon}\left[\int G(x,y)(f(y)+\epsilon\delta(z-y))dy-\int G(x,y)f(y)dy\right]\\
&=\int G(x,y)\delta(z-y))dy\\
&=G(x,z)
\end{align}
\item $I[f]=\int_{-1}^1f(x)dx$
\begin{align}
\frac{\delta^2I[f^3]}{\delta f(x_0)\delta f(x_1)}
&=\frac{\delta}{\delta f(x_0)}\frac{\delta I[f^3]}{\delta f(x_1)}\\
&=\frac{\delta}{\delta f(x_0)}\frac{\delta}{\delta f(x_1)}\int_{-1}^1f(x)^3dx\\
&=\frac{\delta}{\delta f(x_0)}\frac{1}{\epsilon}\int_{-1}^1(f(x)+\epsilon\delta(x_1-x))^3-f(x)^3dx\\
&=\frac{\delta}{\delta f(x_0)}\frac{1}{\epsilon}\int_{-1}^1(f(x)^3+3\epsilon f(x)^2\delta(x_1-x)+\mathcal{O}(\epsilon^2)-f(x)^3dx\\
&=\frac{\delta}{\delta f(x_0)}
\left\{
\begin{matrix}
 3f(x_1)^2 & x_1\in[-1,1]\\
 0         & \text{else}
\end{matrix}
\right.\\
&=
\left\{
\begin{matrix}
 3\frac{1}{\epsilon}[(f(x_1)-\epsilon\delta(x_0-x_1))^2-f(x_1)^2] & x_1\in[-1,1]\\
 0         & \text{else}
\end{matrix}
\right.\\
&=
\left\{
\begin{matrix}
 6f(x_1)\delta(x_0-x_1) & x_1\in[-1,1]\\
 0         & \text{else}
\end{matrix}
\right.\\
\end{align}
\item $J[f]=\int\left(\frac{\partial f}{\partial y}
\right)^2dy$
\begin{align}
\frac{\delta J[f]}{\delta f(x)}
&=\lim_{\epsilon\rightarrow0}\frac{1}{\epsilon}\left[\int\left(\frac{\partial (f+\epsilon\delta(x-y))}{\partial y}
\right)^2dy-\int\left(\frac{\partial f}{\partial y}
\right)^2dy\right]\\
&=\lim_{\epsilon\rightarrow0}\frac{1}{\epsilon}\left[\int\left(\frac{\partial f}{\partial y}+\epsilon\frac{\partial\delta(x-y)}{\partial y}
\right)^2dy-\int\left(\frac{\partial f}{\partial y}
\right)^2dy\right]\\
&=\lim_{\epsilon\rightarrow0}\frac{1}{\epsilon}\left[\int\left(\frac{\partial f}{\partial y}\right)^2+2\epsilon\frac{\partial f}{\partial y}\frac{\partial\delta(x-y)}{\partial y}
+\mathcal{O}(\epsilon^2)-\left(\frac{\partial f}{\partial y}
\right)^2dy\right]\\
&=2\int\frac{\partial f}{\partial y}\frac{\partial\delta(x-y)}{\partial y}dy\\
&=\text{boundary terms}-2\int\frac{\partial^2 f}{\partial y^2}\delta(x-y)dy\\
&=-2\int\frac{\partial^2 f}{\partial x^2}
\end{align}
\end{itemize}

\subsection*{Exercise 1.3 - Functional derivatives II}
\begin{itemize}
\item
\begin{align}
\frac{\delta G[f]}{\delta f(x)}
&=\lim_{\epsilon\rightarrow0}\frac{1}{\epsilon}\int g(y,f+\epsilon\delta(x-y))-g(y,f)dy\\
&=\lim_{\epsilon\rightarrow0}\frac{1}{\epsilon}\int g(y,f)+\epsilon\frac{\partial g(y,f)}{\partial f}\delta(x-y))-g(y,f)dy\\
&=\frac{\partial g(x,f)}{\partial f}
\end{align}

\item
\begin{align}
\frac{\delta H[f]}{\delta f(x)}
&=\lim_{\epsilon\rightarrow0}\frac{1}{\epsilon}\int g(y,f+\epsilon\delta(x-y),f'+\epsilon\partial_y\delta(x-y))-g(y,f,f')dy\\
&=\lim_{\epsilon\rightarrow0}\frac{1}{\epsilon}\int g(y,f,f')+\epsilon\frac{\partial g(y,f,f')}{\partial f}\delta(x-y)+\epsilon\frac{\partial g(y,f,f')}{\partial f'}\partial_y\delta(x-y))-g(y,f,f')dy\\
&=\int \frac{\partial g(y,f,f')}{\partial f}\delta(x-y)+\frac{\partial g(y,f,f')}{\partial f'}\partial_y\delta(x-y))dy\\
&=\frac{\partial g(x,f,f')}{\partial f}+\int \frac{\partial g(y,f,f')}{\partial f'}\partial_y\delta(x-y))dy\\
&=\frac{\partial g(x,f,f')}{\partial f}-\int \partial_y\frac{\partial g(y,f,f')}{\partial f'}\delta(x-y))dy\\
&=\frac{\partial g(x,f,f')}{\partial f}- \partial_x\frac{\partial g(x,f,f')}{\partial f'}
\end{align}

\item Same as above but two times integration by parts is needed. Therefore $(-1)^2=1$ giving the term a final $+$ sign.
\end{itemize}

\subsection*{Exercise 1.4 - Functional derivatives III}
\begin{itemize}
\item
\begin{align}
\frac{\delta\phi(x)}{\delta\phi(y)}
&=\lim_{\epsilon\rightarrow0}\frac{1}{\epsilon}\left(\phi(x)+\epsilon\delta(x-y)-\phi(x)\right)\\
&=\delta(x-y)
\end{align}
\item
\begin{align}
\frac{\delta\dot\phi(t)}{\delta\phi(t_0)}
&=\lim_{\epsilon\rightarrow0}\frac{1}{\epsilon}\left(\dot\phi(t)+\epsilon\partial_t\delta(t-t_0)-\dot\phi(t)\right)\\
&=\frac{d}{dt}\delta(t-t_0)
\end{align}
\end{itemize}

\subsection*{Exercise 1.5 - Euler-Langrange equations for elastic medium}
\begin{align}
\mathcal{L}=T-V\\
\frac{\partial\mathcal{L}}{\partial\psi}-\partial_\mu\left(\frac{\partial\mathcal{L}}{\partial(\partial_\mu\psi)}\right)=0
\end{align}
then
\begin{align}
\frac{\partial\mathcal{L}}{\partial\psi}&=0\\
\frac{\partial\mathcal{L}}{\partial(\partial_0\psi)}&=\frac{\rho}{2}\int d^3x 2\frac{\partial\psi}{\partial t}\\
\frac{\partial\mathcal{L}}{\partial(\partial_k\psi)}&=-\frac{\mathcal{T}}{2}\int d^3x 2\frac{\partial\psi}{\partial x^k}\\
\rightarrow&-\left(\int d^3x[\rho\ddot\psi-\mathcal{T}\nabla^2\psi]\right)=0\\
\rightarrow&\frac{\rho}{\mathcal{T}}\ddot\psi=\nabla^2\psi
\end{align}

\subsection*{Exercise 1.6 - Functional derivatives IV}
\begin{align}
\frac{\delta Z_0[J]}{\delta J(z_1)}
&=\lim_{\epsilon\rightarrow0}\frac{1}{\epsilon}\exp\left(-\frac{1}{2}\int d^4xd^4y(J(x)+\epsilon\delta(x-z_1))\Delta(x-y)(J(y)+\epsilon\delta(y-z_1))\right)\\
&\qquad\qquad-\exp\left(-\frac{1}{2}\int d^4xd^4yJ(x)\Delta(x-y)J(y)\right)\\
&=Z_0[J]\lim_{\epsilon\rightarrow0}\frac{1}{\epsilon}\left(\exp\left(-\frac{\epsilon}{2}\int d^4xd^4y\, J(x)\Delta(x-y)\delta(y-z_1)+\delta(x-z_1)\Delta(x-y)J(y)\right)-1\right)\\
&=Z_0[J]\lim_{\epsilon\rightarrow0}\frac{1}{\epsilon}\left(1-\frac{\epsilon}{2}\int d^4xd^4y\, J(x)\Delta(x-y)\delta(y-z_1)+\delta(x-z_1)\Delta(x-y)J(y)-1\right)\\
&=-\frac{1}{2}Z_0[J]\int d^4xd^4y\, J(x)\Delta(x-y)\delta(y-z_1)+\delta(x-z_1)\Delta(x-y)J(y)\\
&=-\frac{1}{2}Z_0[J]\left(\int d^4x\, J(x)\Delta(x-z_1)+\int d^4y\,\Delta(z_1-y)J(y)\right)\\
&=-Z_0[J]\int d^4y\,\Delta(z_1-y)J(y)
\end{align}

\subsection*{Exercise 2.1 - Commutators of creation and annihilation operators}
With $[\hat{x},\hat{p}]=\hat{x}\hat{p}-\hat{p}\hat{x}=i\hbar$
\begin{align}
[\hat{a},\hat{a}]
&=\frac{m\omega}{2\hbar}\left(\hat{x}\hat{x}+\frac{i}{m\omega}(\hat{x}\hat{p}+\hat{p}\hat{x})+\frac{i^2}{m^2\omega^2}\hat{p}\hat{p}\right)-\frac{m\omega}{2\hbar}\left(\hat{x}\hat{x}+\frac{i}{m\omega}(\hat{x}\hat{p}+\hat{p}\hat{x})+\frac{i^2}{m^2\omega^2}\hat{p}\hat{p}\right)\\
&=0\\
[\hat{a}^\dagger,\hat{a}^\dagger]
&=...=0
\end{align}
\begin{align}
[\hat{a},\hat{a}^\dagger]
&=\frac{m\omega}{2\hbar}\left(\hat{x}\hat{x}+\frac{i}{m\omega}(-\hat{x}\hat{p}+\hat{p}\hat{x})-\frac{i^2}{m^2\omega^2}\hat{p}\hat{p}\right)-\frac{m\omega}{2\hbar}\left(\hat{x}\hat{x}+\frac{i}{m\omega}(\hat{x}\hat{p}-\hat{p}\hat{x})-\frac{i^2}{m^2\omega^2}\hat{p}\hat{p}\right)\\
&=\frac{m\omega}{2\hbar}\frac{i}{m\omega}2(-\hat{x}\hat{p}+\hat{p}\hat{x})\\
&=\frac{i}{\hbar}(-\hat{p}\hat{x}-i\hbar+\hat{p}\hat{x})\\
&=1
\end{align}
Now the Hamiltonian
\begin{align}
\hat{a}^\dagger\hat{a}
&=\frac{m\omega}{2\hbar}\left(\hat{x}\hat{x}+\frac{i}{m\omega}(\hat{x}\hat{p}-\hat{p}\hat{x})-\frac{i^2}{m^2\omega^2}\hat{p}\hat{p}\right)\\
&=\frac{m\omega}{2\hbar}\left(\hat{x}\hat{x}+\frac{i}{m\omega}i\hbar-\frac{i^2}{m^2\omega^2}\hat{p}\hat{p}\right)\\
&=\frac{1}{2m\omega\hbar}\hat{p}^2+\frac{m\omega}{2\hbar}\hat{x}^2-\frac{1}{2}\\
\hat{a}^\dagger\hat{a}+\frac{1}{2}&=\frac{1}{2m\omega\hbar}\hat{p}^2+\frac{m\omega}{2\hbar}\hat{x}^2\\
\hbar\omega\left(\hat{a}^\dagger\hat{a}+\frac{1}{2}\right)&=\frac{1}{2m}\hat{p}^2+\frac{m\omega^2}{2}\hat{x}^2=\hat{H}
\end{align}

\subsection*{Exercise 2.2 - Perturbed harmonic oscillator}
We see
\begin{align}
a+a^\dagger&=\sqrt{\frac{2m\omega}{\hbar}}x\\
(a+a^\dagger)^2&=\frac{2m\omega}{\hbar}x^2\\
x^2&=\frac{\hbar}{2m\omega}(a+a^\dagger)^2\\
x^4&=(a+a^\dagger)^2\frac{\hbar}{2m\omega}\cdot\frac{\hbar}{2m\omega}(a+a^\dagger)^2
\end{align}
The first order energy perturbation is given by
\begin{align}
E^{(1)}_n
&=\langle n|H_1|n \rangle\\
&=\langle n|x^4|n \rangle\\
&=\langle n|x^2\cdot x^2|n \rangle.
\end{align}
By splitting $H_1$ the calculation gets a bit shorter. Using
\begin{align}
a|n\rangle\sqrt{n}|n\rangle \qquad a^\dagger|n\rangle\sqrt{n+1}|n+1\rangle
\end{align}
we obtain
\begin{align}
x^2|n \rangle
&=\frac{\hbar}{2m\omega}(a+a^\dagger)^2|n \rangle\\
&=\frac{\hbar}{2m\omega}(aa^\dagger+a^\dagger a+(a^\dagger)^2+a^2)|n \rangle\\
&=\frac{\hbar}{2m\omega}\left((n+1)|n\rangle +n|n\rangle+\sqrt{n(n-1)}|n-2\rangle+\sqrt{(n+1)(n+2)}|n+2\rangle\right)\\
&=\frac{\hbar}{2m\omega}\left((2n+1)|n\rangle +\sqrt{n(n-1)}|n-2\rangle+\sqrt{(n+1)(n+2)}|n+2\rangle\right)\\
\langle n|x^2&=(x^2|n \rangle)^\dagger\\
&=\frac{\hbar}{2m\omega}\left((2n+1)|n\rangle +\sqrt{n(n-1)}|n-2\rangle+\sqrt{(n+1)(n+2)}|n+2\rangle\right)
\end{align}
Using the orthogonality of the unperturbed states (eigenstates of the Hamiltonian which is hermitian) we obtain
\begin{align}
E^{(1)}_n
&=\langle n|x^2\cdot x^2|n \rangle\\
&=\frac{\hbar^2}{4m^2\omega^2}\left((2n+1)^2+n(n-1)+(n+1)(n+2)\right)\\
&=\frac{\hbar^2}{4m^2\omega^2}\left(4n^2+4n+1+n^2-n+n^2+3n+2\right)\\
&=\frac{\hbar^2}{4m^2\omega^2}\left(6n^2+6n+3\right)\\
&=\frac{3}{4}\frac{\hbar^2}{m^2\omega^2}\left(2n^2+2n+1\right)
\end{align}
which gives the desired result using $E_n=E_n^{(0)}+\lambda E_n^{(1)}$

\subsection*{Exercise 2.3 - ...}
Odd notation $\tilde{x}=\hat{x}$
\begin{align}
\hat{x}_j&=\sqrt{\frac{\hbar}{2\omega_j m}}(\hat{a}_j+\hat{a}_{-j}^\dagger)\\
x_j
&=\frac{1}{\sqrt{N}}\sum_k\tilde{x}_ke^{ikja}\\
&=\frac{1}{\sqrt{N}}\sqrt{\frac{\hbar}{m}}\sum_k\frac{1}{\sqrt{2\omega_k}}(\hat{a}_k+\hat{a}_{-k}^\dagger)e^{ikja}\\
&=\frac{1}{\sqrt{N}}\sqrt{\frac{\hbar}{m}}\sum_k\frac{1}{\sqrt{2\omega_k}}(\hat{a}_k e^{ikja}+\hat{a}_{k}^\dagger e^{-ikja})
\end{align}


\subsection*{Exercise 2.4 - Wavefunction in space representation}
\begin{align}
\hat{a}&=\sqrt{\frac{2\hbar}{m\omega}}\left(\hat{x}+\frac{i}{m\omega}\hat{p}\right), \qquad \hat{a}|0\rangle=0\\
&\rightarrow\sqrt{\frac{2\hbar}{m\omega}}\langle x|\left(\hat{x}+\frac{i}{m\omega}\hat{p}\right)|0\rangle=0\\
&\rightarrow\sqrt{\frac{2\hbar}{m\omega}}\left(\langle x|\hat{x}|0\rangle+\frac{i}{m\omega}\langle x|\hat{p}|0\rangle\right)=0\\
&\rightarrow\sqrt{\frac{2\hbar}{m\omega}}\left(x\langle x|0\rangle+\frac{i}{m\omega}(-i\hbar)\frac{d}{dx}\langle x|0\rangle\right)=0\\
&\rightarrow\sqrt{\frac{2\hbar}{m\omega}}\left(x+\frac{\hbar}{m\omega}\frac{d}{dx}\right)\langle x|0\rangle=0
\end{align}
Now we can solve the ODE ($\psi_0(x)=\langle x|0\rangle$)
\begin{align}
\left(x+\frac{\hbar}{m\omega}\frac{d}{dx}\right)\psi_0&=0\\
\int dx\,\psi_0'+\int dx\,\frac{m\omega}{\hbar} x\psi_0&=0\\
\frac{\psi_0'}{\psi_0}&=-\frac{m\omega}{\hbar}x\\
\log\psi_0&=-\frac{m\omega}{2\hbar}x^2+c\\
\psi_0&=Ce^{-m\omega x^2/2\hbar}
\end{align}
Normalization
\begin{align}
\int dx\,\psi_0^*\,\psi_0&=1\\
C^*C\int dx\,e^{-m\omega x^2/\hbar}&=1\\
|C|^2\sqrt{\frac{\pi\hbar}{m\omega}}&=1\quad\rightarrow\quad C=\left(\frac{m\omega}{\pi\hbar}\right)^{1/4}
\end{align}


\subsection*{Exercise 3.1 - Commutator Fourier Transformation}
Bosons - commutator
\begin{align}
\frac{1}{\mathcal{V}}\sum_{\mathbf{p,q}}e^{i(\mathbf{p}\cdot\mathbf{x}-\mathbf{q}\cdot\mathbf{y})}[a_\mathbf{p},a^\dagger_\mathbf{q}]
&=\frac{1}{\mathcal{V}}\sum_{\mathbf{p,q}}e^{i(\mathbf{p}\cdot\mathbf{x}-\mathbf{q}\cdot\mathbf{y})}\delta_{\mathbf{pq}}\\
&=\frac{1}{\mathcal{V}}\sum_{\mathbf{p}}e^{i\mathbf{p}\cdot(\mathbf{x}-\mathbf{y})}\\
&=\frac{1}{L_xL_yL_z}\sum_{n_1=-N/2}^{N/2}e^{i\frac{2\pi n_1}{Na_x}(x_1-y_1)}\cdot\sum_{n_2=-N/2}^{N/2}e^{i\frac{2\pi n_2}{Na_y}(x_2-y_2)}\cdot\sum_{n_3=-N/2}^{N/2}e^{i\frac{2\pi n_3}{Na_z}(x_3-y_3)}\\
&=\left(\frac{1}{L}\sum_{n=-N/2}^{N/2}e^{i\frac{2\pi n}{Na}(x-y)}\right)^3\qquad\text{with}\;Na\equiv L\\
&=\left(\frac{1}{L}\frac{Na}{2\pi}\sum_{p_n=-\pi/a}^{\pi/a}e^{ip_n(x-y)}\frac{2\pi}{Na}\right)^3\qquad\text{with}\;\sum_{q_n}f(p_n)\Delta p=\int f(p)dp\\
&=\left(\frac{1}{2\pi}\int_{-\infty}^{\infty}e^{ip(x-y)}dp\right)^3\qquad\text{with}\;N\rightarrow\infty,\,a\rightarrow0\\
&=\left(\delta(x-y)\right)^3\\
&=\delta^{(3)}(\mathbf{x}-\mathbf{y})
\end{align}
with the discretization of the momentum-space $p_j=\left\{\frac{2\pi j}{Na}\right\}_{-N/2}^{N/2}$ and $\Delta p=\frac{2\pi}{Na}$.

Fermions - anticommutator
\begin{align}
\{c_\mathbf{p},c^\dagger_\mathbf{q}\}=\delta_{\mathbf{p}\mathbf{q}}
\end{align}
yields same result.

\subsection*{Exercise 3.2 - Harmonic oscillator relations}
With
\begin{align}
[\hat{a},\hat{a}^\dagger]&=1\\
\hat{a}^\dagger\hat{a}&=\hat{n}\\
\frac{(a^\dagger)^n}{\sqrt{n!}}|0\rangle&=|n\rangle
\end{align}
Then
\begin{enumerate}[(a)]
\item $[\hat{a},(\hat{a}^\dagger)^n]$
\begin{align}
\hat{a}(\hat{a}^\dagger)^n
&=(a a^\dagger)(a^\dagger)^{n-1}\\
&=(a^\dagger a+1)(a^\dagger)^{n-1}\\
&=a^\dagger a(a^\dagger)^{n-1}+(a^\dagger)^{n-1}\\
&=a^\dagger aa^\dagger(a^\dagger)^{n-2}+(a^\dagger)^{n-1}\\
&=a^\dagger (a^\dagger a+1)(a^\dagger)^{n-2}+(a^\dagger)^{n-1}\\
&=(a^\dagger)^2 a(a^\dagger)^{n-2}+2(a^\dagger)^{n-1}\\
&=...\\
&=(a^\dagger)^{n}a+n(a^\dagger)^{n-1}\\
\rightarrow[\hat{a},(\hat{a}^\dagger)^n]&=n(a^\dagger)^{n-1}
\end{align}

\item $\langle0|a^n(a^\dagger)^m|0\rangle$

If $n<m$ (similar for $n>m$) we get zero 
\begin{align}
\langle0|a^n(a^\dagger)^m|0\rangle
&\sim\langle1|a^{n-1}(a^\dagger)^{m-1}|1\rangle\\
&\sim\langle2|a^{n-2}(a^\dagger)^{m-2}|2\rangle\\
&...\\
&\sim\langle k|(a^\dagger)^{m-k}|k\rangle\\
&=0\qquad (\langle k|a^\dagger=0).
\end{align}
For $n=m$ we have with the definition
\begin{align}
\frac{(a^\dagger)^n}{\sqrt{n!}}|0\rangle&=|n\rangle\\
(a^\dagger)^n|0\rangle&=\sqrt{n!}|n\rangle\\
\langle0|a^n(a^\dagger)^m|0\rangle&=\sqrt{n!}^2\langle n|n|\rangle\\
&=n!
\end{align}
Therefore  $\langle0|a^n(a^\dagger)^m|0\rangle=n!\delta_{nm}$

\item $\langle m|a^\dagger|n\rangle$

\begin{align}
\frac{(a^\dagger)^n}{\sqrt{n!}}|0\rangle&=|n\rangle\\
a^\dagger\frac{(a^\dagger)^n}{\sqrt{n!}}|0\rangle&=a^\dagger|n\rangle\\
\frac{1}{\sqrt{n+1}}a^\dagger\frac{(a^\dagger)^n}{\sqrt{n!}}|0\rangle&=\frac{1}{\sqrt{n+1}}a^\dagger|n\rangle=|n+1\rangle
\end{align}
then
\begin{align}
\langle m|a^\dagger|n\rangle
&=\sqrt{n+1}\langle m|n+1\rangle\\
&=\sqrt{n+1}\delta_{m,n+1}
\end{align}

\item $\langle m|a|n\rangle$

\begin{align}
(\langle m|a)^\dagger
&=a^\dagger|m\rangle\\
&=\sqrt{m+1}|m+1\rangle
\end{align}
then
\begin{align}
\langle m|a|n\rangle
&=\sqrt{m+1}\delta_{m+1,n}\\
&=\sqrt{n}\delta_{m+1,n}
\end{align}
\end{enumerate}

\subsection*{Exercise 3.2 - 3d Harmonic oscillator}
Rewriting the Hamiltonian
\begin{align}
H&=H_1+H_2+H_3\\
H_i&=\frac{p_i^2}{2m}+\frac{1}{2}m\omega^2 x_i^2
\end{align}
the we can reutilise the know ladder operators
\begin{align}
a_i&=\sqrt{\frac{m\omega}{2\hbar}}\left(x_i+\frac{i}{m\omega}p_i\right)\\
a_i^\dagger&=\sqrt{\frac{m\omega}{2\hbar}}\left(x_i-\frac{i}{m\omega}p_i\right)
\end{align}
and the Hamiltonian can be obviously written as the sum
\begin{align}
H&=\hbar\omega\sum_k\left(a_k^\dagger a_k+\frac{1}{2}\right).
\end{align}
With the classic definition $\vec{L}=\vec{x}\times\vec{p}$ we see (inverting $a$ and $a^\dagger$ to get $x$ and $p$)
\begin{align}
L_i&=\varepsilon_{ijk} x_j p_k\\
&=-i\varepsilon_{ijk}\sqrt{\frac{\hbar}{2m\omega}}\sqrt{\frac{\hbar m\omega}{2}}(a_j+a_j^\dagger)(a_k-a_k^\dagger)\\
&=-\frac{i\hbar}{2}\varepsilon_{ijk}(a_ja_k+a_j^\dagger a_k -a_ja_k^\dagger-a_j^\dagger a_k^\dagger)\\
&=-\frac{i\hbar}{2}\varepsilon_{ijk}(a_j^\dagger a_k -\delta_{jk}-a_k^\dagger a_j)\qquad[a_j,a_k^\dagger]=\delta_{jk},\,a_j|0\rangle=0,\,\langle0|a_k=0\\
&=-\frac{i\hbar}{2}(\varepsilon_{ijk}a_j^\dagger a_k -\varepsilon_{ijk}\delta_{jk}-\varepsilon_{ijk}a_k^\dagger a_j)\\
&=-\frac{i\hbar}{2}(\varepsilon_{ijk}a_j^\dagger a_k -\varepsilon_{ikk}-\varepsilon_{ikj}a_j^\dagger a_k)\qquad\text{reindexing}\\
&=-\frac{i\hbar}{2}(\varepsilon_{ijk}a_j^\dagger a_k +\varepsilon_{ijk}a_j^\dagger a_k)\qquad\varepsilon_{ikk}=0\\
&=-i\hbar\varepsilon_{ijk}a_j^\dagger a_k
\end{align}
Now the new commutation relations
\begin{align}
[b_0,b_0^\dagger]&=[a_3,a_3^\dagger]=1=\delta_{00}\\
[b_0,b_1^\dagger]
&=-\frac{1}{\sqrt{2}}(a_3(a_1^\dagger+ia_2^\dagger)-(a_1^\dagger+ia_2^\dagger)a_3 )\\
&=-\frac{1}{\sqrt{2}}(a_3a_1^\dagger+ia_3a_2^\dagger-a_1^\dagger a_3-ia_2^\dagger a_3 )\\
&=-\frac{1}{\sqrt{2}}(\delta_{12}+i\delta_{23} )\\
&=0\\
[b_{-1},b_1^\dagger]&=-\frac{1}{2}((a_1-ia_2)(a_1^\dagger-ia_2^\dagger) - (a_1^\dagger-ia_2^\dagger)(a_1-ia_2) )\\
&=-\frac{1}{2}(a_1a_1^\dagger-ia_2a_1^\dagger-ia_1a_2^\dagger-a_2a_2^\dagger-a_1^\dagger a_1+ia_1^\dagger a_2+ia_2^\dagger a_1+a_2^\dagger a_2)\\
&=-\frac{1}{2}(1-i\cdot 0-i\cdot 0-1)\\
&=0\\
&=\delta_{-1,1}\\
...
\end{align}
Now the Hamiltonian with
\begin{align}
b_{-1}^\dagger b_{-1}+b_{1}^\dagger b_{1}
&=\frac{1}{2}(a_1^\dagger-ia_2^\dagger)(a_1+ia_2)+\frac{1}{2}(a_1^\dagger+ia_2^\dagger)(a_1-ia_2)\\
&=\frac{1}{2}(a_1^\dagger a_1-ia_2^\dagger a_1+ia_1^\dagger a_2+a_2^\dagger a_2)+\frac{1}{2}(a_1^\dagger a_1+ia_2^\dagger a_1-ia_1^\dagger a_2+a_2^\dagger a_2)\\
&=a_1^\dagger a_1+a_2^\dagger a_2
\end{align}
and $b_0^\dagger b_0=a_3^\dagger a_3$ we have $H=\hbar\omega\sum(1/2+b_m^\dagger b_m)$. While
\begin{align}
-b_{-1}^\dagger b_{-1}+b_{1}^\dagger b_{1}
&=-\frac{1}{2}(a_1^\dagger-ia_2^\dagger)(a_1+ia_2)+\frac{1}{2}(a_1^\dagger+ia_2^\dagger)(a_1-ia_2)\\
&=-\frac{1}{2}(a_1^\dagger a_1-ia_2^\dagger a_1+ia_1^\dagger a_2+a_2^\dagger a_2)+\frac{1}{2}(a_1^\dagger a_1+ia_2^\dagger a_1-ia_1^\dagger a_2+a_2^\dagger a_2)\\
&=ia_2^\dagger a_1-ia_1^\dagger a_2\\
&=-i(-a_2^\dagger a_1+a_1^\dagger a_2)
\end{align}
gives $L^3=\hbar\sum_m mb_m^\dagger b_m$.

\subsection*{Exercise 5.1 - Time derivative of Lagrangian}
With $\frac{\partial L}{\partial q}=\frac{d}{dt}\left(\frac{\partial L}{\partial \dot{q}}\right)$ we have 
\begin{align}
\frac{dL}{dt}
&=\frac{\partial L}{\partial t}+\frac{\partial L}{\partial q}\dot{q}+\frac{\partial L}{\partial \dot{q}}\ddot{q}\\
&=\frac{\partial L}{\partial t}+\frac{d}{dt}\left(\frac{\partial L}{\partial \dot{q}}\right)\dot{q}+\frac{\partial L}{\partial \dot{q}}\ddot{q}\\
&=\frac{\partial L}{\partial t}+\frac{d}{dt}\left(\frac{\partial L}{\partial \dot{q}}\dot{q}\right)\\
&=\frac{\partial L}{\partial t}+\frac{d}{dt}\left(p\dot{q}\right)
\end{align}
then
\begin{align}
0&=\frac{\partial L}{\partial t}+\frac{d}{dt}\left(p\dot{q}-L\right)
\end{align}
and
\begin{align}
\frac{\partial L}{\partial t}=-\frac{dH}{dt}
\end{align}

\subsection*{Exercise 5.3 - Commutator of Hermitian operators}
In general we have
\begin{align}
[A,B]^\dagger
&=(AB-BA)^\dagger\\
&=B^\dagger A^\dagger-A^\dagger B^\dagger\\
&=[B^\dagger,A^\dagger]\\
&=-[A^\dagger,B^\dagger]
\end{align}
now using $A=A^\dagger$ and $B=B^\dagger$ we obtain
\begin{align}
[A,B]^\dagger
&=-[A,B]
\end{align}

\subsection*{Exercise 5.4 - Relativistic free particle}
Taylor series expansion of the square root gives
\begin{align}
L&=-mc^2\sqrt{1-\frac{v^2}{c^2}}\\
&\simeq -mc^2-\frac{1}{2}mv^2-\frac{3}{8}mv^2\frac{1}{c^2}+...\\
&\simeq -mc^2-\frac{1}{2}mv^2+...
\end{align}
Conjugated momentum
\begin{align}
p&=\frac{\partial L}{\partial v}=\frac{mv}{\sqrt{1-\frac{v^2}{c^2}}}=\gamma mv\simeq mv
\end{align}
Lets solve for $v$ to get exact expression for $H$
\begin{align}
v=\frac{cp}{m^2c^2+p^2}
\end{align}
Then
\begin{align}
H&=pv-L\\
&=p\frac{cp}{m^2c^2+p^2}+mc^2\sqrt{1-\frac{v^2}{c^2}}\\
&=c\frac{m^2c^2+p^2}{\sqrt{p^2+m^2c^2}}=\sqrt{m^2c^4+p^2c^2}\\
&\simeq mc^2+\frac{mv^2}{2}
\end{align}

\subsection*{Exercise 5.6 - Relativistic free particle in EM field}
Euler-Lagrange equations:
\begin{align}
\frac{\partial L}{\partial x_i}=\frac{d}{dt}\frac{\partial L}{\partial v_i}
\end{align}
Definition of the EM potentials
\begin{align}
\mathbf{E}=-\nabla V-\frac{d\mathbf{A}}{dt}\\
\mathbf{B}=\nabla\times\mathbf{A}
\end{align}
From Problem 5.4
\begin{align}
\frac{d}{dt}\frac{\partial L}{\partial v_i}
&=\frac{d}{dt}(\gamma mv_i)+q\frac{d}{dt}A_i(x,t)
\end{align}
Lets proof the identity by calculating the single terms
\begin{align}
\nabla(\mathbf{a}\cdot\mathbf{b})&=[(\partial_ka_i)b_i+(\partial_kb_i)a_i]\mathbf{e}_k\\
(\mathbf{a}\cdot\nabla)\mathbf{b}&=a_i(\partial_i b_k)\mathbf{e}_k\\
(\mathbf{b}\cdot\nabla)\mathbf{a}&=b_i(\partial_i a_k)\mathbf{e}_k\\
\mathbf{b}\times(\nabla\times\mathbf{a})
&=\epsilon_{kja}\epsilon_{bca}b_j(\partial_ba_c)\mathbf{e}_k\\
&=(\delta_{kb}\delta_{jc}-\delta_{kc}\delta_{jb})b_j(\partial_ba_c)\mathbf{e}_k\\
&=[b_c(\partial_k a_c)-b_c(\partial_c a_k)]\mathbf{e}_k\\
\mathbf{a}\times(\nabla\times\mathbf{b})
&=[a_c(\partial_k b_c)-a_c(\partial_c b_k)]\mathbf{e}_k
\end{align}
by adding up we see
\begin{align}
\nabla(\mathbf{a}\cdot\mathbf{b})
&=(\mathbf{a}\cdot\nabla)\mathbf{b}
+(\mathbf{b}\cdot\nabla)\mathbf{a}
+\mathbf{b}\times(\nabla\times\mathbf{a})
+\mathbf{a}\times(\nabla\times\mathbf{b})
\end{align}
Now we calculate
\begin{align}
\frac{\partial L}{\partial x_i}
&=q\frac{\partial }{\partial x_i}[\mathbf{A}\cdot\mathbf{v}-V]\\
&=-q\partial_i V(x,t)+q[\nabla(\mathbf{A}(x,t)\cdot\mathbf{v})]_i\\
&=-q[\nabla V]_i+q[(\mathbf{v}\cdot\nabla)\mathbf{A}+\mathbf{v}\times(\nabla\times\mathbf{A})]_i
\end{align}
then (combining all vector components)
\begin{align}
\frac{d}{dt}(\gamma m\mathbf{v})+q\frac{d}{dt}\mathbf{A}&=q(\mathbf{v}\cdot\nabla)\mathbf{A}+q\mathbf{v}\times(\nabla\times\mathbf{A})-q\nabla V\\
\frac{d}{dt}(\gamma m\mathbf{v})
&=q\mathbf{v}\times(\nabla\times\mathbf{A})-q\nabla V-q\left(\frac{d}{dt}\mathbf{A}-(\mathbf{v}\cdot\nabla)\mathbf{A}\right)\\
&=q\mathbf{v}\times\mathbf{B}-q\nabla V-q\frac{\partial\mathbf{A}}{\partial t}\\
&=q[\mathbf{v}\times\mathbf{B}+\mathbf{E}]
\end{align}
where we used
\begin{align}
\frac{d\mathbf{A}}{dt}
&=\frac{\partial\mathbf{A}}{\partial t}+\frac{\partial\mathbf{A}}{\partial x_i}\frac{\partial x_i}{\partial t}\\
&=\frac{\partial\mathbf{A}}{\partial t}+(\mathbf{v}\cdot\nabla)\mathbf{A}
\end{align}

\subsection*{Exercise 5.6 - Non-relativistic free particle in EM field}
From Problem 5.4/5.5
\begin{align}
p_i&=\frac{\partial L}{\partial v_i}
=\gamma mv_i+qA_i(x,t)\\
\mathbf{p}&=\gamma m\mathbf{v}+q\mathbf{A}\\
&\simeq m\mathbf{v}+q\mathbf{A}
\end{align}
also
\begin{align}
\gamma m\mathbf{v}&=\mathbf{p}-q\mathbf{A}\\
\mathbf{v}&=\frac{\mathbf{p}-q\mathbf{A}}{\gamma m}\\
v^2&=\frac{(\mathbf{p}-q\mathbf{A})^2}{\gamma^2 m^2}\\
&=\frac{(\mathbf{p}-q\mathbf{A})^2c^2}{m^2c^2+(\mathbf{p}-q\mathbf{A})^2}\\
\sqrt{1-\frac{v^2}{c^2}}&=\frac{mc}{\sqrt{(\mathbf{p}-q\mathbf{A})^2+m^2c^2}}
\end{align}
then
\begin{align}
E=H&=\mathbf{p}\cdot\dot{\mathbf{q}}-L\\
&=\mathbf{p}\cdot\mathbf{v}-L\\
&=(\gamma m \mathbf{v})\cdot\mathbf{v}+q\mathbf{A}\cdot\mathbf{v}-\left(-\frac{mc^2}{\gamma}+q\mathbf{A}\cdot\mathbf{v}-qV\right)\\
&=(\gamma m \mathbf{v})\cdot\mathbf{v}+\frac{mc^2}{\gamma}+qV\\
&=(\mathbf{p}-q\mathbf{A})\cdot\mathbf{v}+\frac{mc^2}{\gamma}+qV\\
&=(\mathbf{p}-q\mathbf{A})\cdot\frac{\mathbf{p}-q\mathbf{A}}{m\gamma}+\frac{mc^2}{\gamma}+qV\\
&=\left(\frac{(\mathbf{p}-q\mathbf{A})^2}{m}+mc^2\right)\sqrt{1-\frac{v^2}{c^2}}+qV\\
&=\frac{1}{m}\left((\mathbf{p}-q\mathbf{A})^2+m^2c^2\right)\frac{mc}{\sqrt{(\mathbf{p}-q\mathbf{A})^2+m^2c^2}}+qV\\
&=\sqrt{(\mathbf{p}-q\mathbf{A})^2c^2+m^2c^4}+qV\\
&=mc^2\sqrt{1+\frac{(\mathbf{p}-q\mathbf{A})^2c^2}{m^2c^4}}+qV\\
&\simeq mc^2\left(1+\frac{(\mathbf{p}-q\mathbf{A})^2}{2m^2c^2}+...\right)+qV\\
&\simeq mc^2+\frac{(\mathbf{p}-q\mathbf{A})^2}{2m}+qV
\end{align}

\subsection*{Exercise 6.1 - Klein-Gordon}
\begin{align}
\frac{\partial\mathcal{L}}{\partial\phi}&=-m^2\phi\\
\frac{\partial\mathcal{L}}{\partial(\partial_\mu\phi)}&=\frac{1}{2}g^{\alpha\beta}(\partial_\alpha\phi)(\partial_\beta\phi)\\
&=\frac{1}{2}g^{\alpha\beta}\left[\delta^\mu_\alpha(\partial_\beta\phi)+\delta^\mu_\beta(\partial_\alpha\phi)\right]\\
&=g^{\alpha\mu}\partial_\alpha\phi\\
&=\partial^\mu\phi\\
\partial_\mu\frac{\partial\mathcal{L}}{\partial(\partial_\mu\phi)}&=\partial_\mu\partial^\mu\phi
\end{align}
Euler Lagrange
\begin{align}
(\partial_\mu\partial^\mu+m^2)\phi=0
\end{align}
Canonical momentum
\begin{align}
\pi&=\frac{\partial\mathcal{L}}{\partial(\partial_0\phi)}=\partial^0\phi=\dot{\phi}
\end{align}
Hamiltonian
\begin{align}
\mathcal{H}&=\pi\dot{\phi}-\mathcal{L}\\
&=\pi^2-\left(\frac{1}{2}\pi^2-\frac{1}{2}(\nabla\phi)^2-\frac{m^2}{2}\phi^2\right)\\
&=\frac{1}{2}\pi^2+\frac{1}{2}(\nabla\phi)^2+\frac{m^2}{2}\phi^2\\
\end{align}

\subsection*{Exercise 7.1 - Klein-Gordon plus higher orders}
\begin{align}
\frac{\partial\mathcal{L}}{\partial\phi}
&=-m^2\phi-\sum_{n=1}(2n+2)\lambda_n\phi^{2n+1}\\
\frac{\partial\mathcal{L}}{\partial(\partial_\alpha\phi)}
&=\frac{1}{2}\frac{\partial(\eta^{\mu\nu}\partial_\mu\phi\partial_\nu\phi)}{\partial(\partial_\alpha\phi)}
=\frac{1}{2}(\eta^{\mu\nu}\partial_\nu\phi\delta_\mu^\alpha+\eta^{\mu\nu}\partial_\mu\phi\delta_\nu^\alpha)\\
&=\frac{1}{2}(\eta^{\alpha\nu}\partial_\nu\phi+\eta^{\mu\alpha}\partial_\mu\phi)=\partial^\alpha\phi\\
&\rightarrow\partial_\alpha\partial^\alpha\phi+m^2\phi+\sum_{n=1}(2n+2)\lambda_n\phi^{2n+1}=0
\end{align}

\subsection*{Exercise 7.2 - Klein-Gordon plus source}
\begin{align}
\frac{\partial\mathcal{L}}{\partial\phi}
&=-m^2\phi+J(x)\\
\frac{\partial\mathcal{L}}{\partial(\partial_\alpha\phi)}
&=\frac{1}{2}\frac{\partial(\eta^{\mu\nu}\partial_\mu\phi\partial_\nu\phi)}{\partial(\partial_\alpha\phi)}
=\frac{1}{2}(\eta^{\mu\nu}\partial_\nu\phi\delta_\mu^\alpha+\eta^{\mu\nu}\partial_\mu\phi\delta_\nu^\alpha)\\
&=\frac{1}{2}(\eta^{\alpha\nu}\partial_\nu\phi+\eta^{\mu\alpha}\partial_\mu\phi)=\partial^\alpha\phi\\
&\rightarrow\partial_\alpha\partial^\alpha\phi+m^2\phi-J(x)=0
\end{align}

\subsection*{Exercise 7.3 - Two interacting Klein-Gordon fields}
\begin{align}
\frac{\partial\mathcal{L}}{\partial\phi_i}
&=-m^2\phi_i-2g(\phi_i^2+\phi_k^2)2\phi_i\\
\frac{\partial\mathcal{L}}{\partial(\partial_\alpha\phi_i)}
&=\frac{1}{2}\frac{\partial(\eta^{\mu\nu}\partial_\mu\phi_i\partial_\nu\phi_i)}{\partial(\partial_\alpha\phi_i)}
=\frac{1}{2}(\eta^{\mu\nu}\partial_\nu\phi_i\delta_\mu^\alpha+\eta^{\mu\nu}\partial_\mu\phi_i\delta_\nu^\alpha)\\
&=\frac{1}{2}(\eta^{\alpha\nu}\partial_\nu\phi_i+\eta^{\mu\alpha}\partial_\mu\phi_i)=\partial^\alpha\phi_i\\
&\rightarrow\partial_\alpha\partial^\alpha\phi_i+m^2\phi_i+4g\phi_i(\phi_1^2+\phi_2^2)=0
\end{align}

\subsection*{Exercise 7.4 - Klein-Gordon again}
Same calculation as in 6.1

\subsection*{Exercise 8.1 - Time evolution operator - NOT DONE YET}
With
\begin{align}
U(t_2,t_1)=e^{-iH(t_2-t_1)}
\end{align}
Then
\begin{enumerate}[(1)]
\item $U(t_1,t_1)=e^{-iH(t_1-t_1)}=e^0=1$
\item $U(t_3,t_2)U(t_2,t_1)=e^{-iH(t_3-t_2)}e^{-iH(t_2-t_1)}=e^{-iH(t_3-t_2+t_2-t_1)}=e^{-iH(t_3-t_1)}=U(t_3,t_1)$
\item $U(t_2,t_1)^{-1}$
\item
\item
\end{enumerate}

\subsection*{Exercise 8.2 - Heisenberg equations of motions for ladder operators}
With $[a_k,a_q^\dagger]=\delta_{kq}$ we have 
\begin{align}
\frac{d}{dt}a_k^\dagger
&=\frac{1}{i\hbar}[a_k^\dagger,H]
=\frac{1}{i\hbar}\sum_nE_n[a_k^\dagger,a_n^\dagger a_n]
=\frac{1}{i\hbar}\sum_nE_n(a_k^\dagger a_n^\dagger a_n-a_n^\dagger a_na_k^\dagger)\\
&=\frac{1}{i\hbar}E_k(a_k^\dagger a_k^\dagger a_k-a_k^\dagger a_ka_k^\dagger)
=\frac{1}{i\hbar}E_k(a_k^\dagger a_k^\dagger a_k-a_k^\dagger(1+a_k^\dagger a_k))
=-\frac{1}{i\hbar}E_k a^\dagger_k
\end{align}
then
\begin{align}
a^\dagger_k=c\cdot e^{-E_kt/i\hbar}=a^\dagger_k(0)\cdot e^{-E_kt/i\hbar}
\end{align}
And similar
\begin{align}
\frac{d}{dt}a_k
&=\frac{1}{i\hbar}[a_k,H]
=\frac{1}{i\hbar}\sum_nE_n[a_k,a_n^\dagger a_n]
=\frac{1}{i\hbar}\sum_nE_n(a_k a^\dagger_n a_n-a_n^\dagger a_na_k)\\
&=\frac{1}{i\hbar}E_k(a_k a_k^\dagger a_k-a_k^\dagger a_ka_k)
=\frac{1}{i\hbar}E_k(a_k a_k^\dagger a_k -(a_k a_k^\dagger-1)a_k)
=\frac{1}{i\hbar}E_k a_k
\end{align}
then
\begin{align}
a_k=c\cdot e^{E_kt/i\hbar}=a_k(0)\cdot e^{E_kt/i\hbar}
\end{align}

\subsection*{Exercise 10.1 - Commutator of field and energy momentum tensor}
\begin{align}
\pi(x)=\frac{\partial\mathcal{L}}{\partial(\partial_0\phi)}
\end{align}
\begin{align}
[\phi(x),P^\alpha]
&=\left[\phi(x),\int d^3y\, \pi(y)\partial^\alpha\phi(y)-\delta_{0\alpha}\mathcal{L}\right]\\
&=\int d^3y\, \left[\phi(x),\pi(y)\partial^\alpha\phi(y)\right]-\left[\phi(x),\delta_{0\alpha}\mathcal{L}\right]\\
&=\int d^3y\, \left[\phi(x)\pi(y)\partial^\alpha\phi(y)-\pi(y)\partial^\alpha\phi(y)\phi(x)\right]-\left[\phi(x),\delta_{0\alpha}\mathcal{L}\right]\\
&=\int d^3y\, \left[\underbrace{\phi(x)\pi(y)}_{=i\delta(x-y)+\pi(y)\phi(x)}\partial^\alpha\phi(y)-\pi(y)(\partial^\alpha(\phi(y)\phi(x))-\phi(y)\underbrace{\partial^\alpha\phi(x)}_{=\frac{\partial}{\partial y^\alpha}\phi(x)=0})\right]-\left[\phi(x),\delta_{0\alpha}\mathcal{L}\right]\\
&=i\partial^\alpha\phi(x)+\int d^3y\,\pi(y)\phi(x)\partial^\alpha\phi(y)-\pi(y)\underbrace{\partial^\alpha(\phi(x)\phi(y))}_{=\phi(x)\partial^\alpha\phi(y)}-\delta_{0\alpha}[\phi(x),\mathcal{L}]\\
&=i\partial^\alpha\phi(x)-\delta_{0\alpha}[\phi(x),\mathcal{L}]
\end{align}

\subsection*{Exercise 10.3 - Energy momentum tensor for scalar field}
With
\begin{align}
\mathcal{L}&=\frac{1}{2}(\partial_\mu\phi)^2-\frac{1}{2}m^2\phi^2\\
\Pi^\mu&=\frac{\partial\mathcal{L}}{\partial(\partial_\mu\phi)}\\
&=\partial^\mu\phi\\
T^{\mu\nu}&=\Pi^\mu\partial^\nu\phi-g^{\mu\nu}\mathcal{L}\\
&=\partial^\mu\phi\partial^\nu\phi-\frac{1}{2}g^{\mu\nu}((\partial_\alpha\phi)^2-m^2\phi^2)
\end{align}
Now
\begin{align}
\partial_\mu T^{\mu\nu}&=\Box\phi\partial^\nu\phi+\partial^\mu\phi\partial^\nu_{\,\mu}\phi-g^{\mu\nu}\left[(\partial_\alpha\phi)\partial_{\alpha\mu}\phi-m^2\phi\partial_\mu\phi\right]\\
&=(\Box\phi+m^2\phi)\partial^\nu\phi\\
&=0
\end{align}
then with $g^{00}=1$ and $g^{0i}=0$
\begin{align}
T^{00}
&=(\partial^0\phi)^2-\frac{1}{2}(\partial_\alpha\phi)^2+\frac{1}{2}m^2\phi^2\\
&=\frac{1}{2}(\partial^0\phi)^2+\frac{1}{2}(\partial_k\phi)^2+\frac{1}{2}m^2\phi^2\\
&=\mathcal{H}\\
T^{0i}
&=\partial^0\phi\partial^i\phi
\end{align}
and
\begin{align}
P^0&=\int d^3xT^{00}=\int d^3x\mathcal{H}\\
P^k&=\int d^3xT^{0k}=\int d^3x\partial^0\phi\partial^i\phi
\end{align}

\subsection*{Exercise 11.1 - Commutator of field operators}
\begin{align}
[\hat{\phi}(x),\hat{\phi}(y)]
&=\left[\int\frac{d^3p}{(2\pi)^{3/2}}\frac{1}{\sqrt{2E_p}}(\hat{a}_\mathbf{p}e^{-ipx}+\hat{a}^\dagger_\mathbf{p}e^{ipx}),\int\frac{d^3q}{(2\pi)^{3/2}}\frac{1}{\sqrt{2E_q}}(\hat{a}_\mathbf{q}e^{-iqy}+\hat{a}^\dagger_\mathbf{q}e^{iqy})\right]\\
&=\iint\frac{d^3p}{(2\pi)^{3/2}}\frac{d^3q}{(2\pi)^{3/2}}\frac{1}{\sqrt{2E_q}}\frac{1}{\sqrt{2E_p}}
([\hat{a}_\mathbf{p},\hat{a}_\mathbf{q}]e^{-i(xp+yq)}
+[\hat{a}^\dagger_\mathbf{p},\hat{a}_\mathbf{q}]e^{i(px-qy)}
+[\hat{a}_\mathbf{p},\hat{a}^\dagger_\mathbf{q}]e^{i(-px+qy)}
+[\hat{a}^\dagger_\mathbf{p},\hat{a}^\dagger_\mathbf{q}]e^{i(px+qy)})\\
&=\iint\frac{d^3p}{(2\pi)^{3/2}}\frac{d^3q}{(2\pi)^{3/2}}\frac{1}{\sqrt{2E_q}}\frac{1}{\sqrt{2E_p}}
(-\delta^{(3)}(\mathbf{p}-\mathbf{q})e^{i(px-qy)}
+\delta^{(3)}(\mathbf{p}-\mathbf{q})e^{-i(px-qy)})
\end{align}
$\delta^{(3)}(\mathbf{p}-\mathbf{q}) \rightarrow \mathbf{p}=\mathbf{q}, E_p=E_q$ meaning $p=q$
\begin{align}
[\hat{\phi}(x),\hat{\phi}(y)]
&=\int\frac{d^3p}{(2\pi)^{3}}\frac{1}{2E_p}
(-e^{ip(x-y)}+e^{-ip(x-y)})\\
\end{align}

\section{{\sc van Baal} - A Course in Field Theory}
\subsection{Problem 1. Violation of causality in 1+1 dimensions}
\begin{enumerate}[(a)]
\item With $H^2=m^2c^4+p^2c^2$ and $p=-i\hbar\partial_x$
\begin{align}
H\psi(x,t)&=i\hbar\partial_t\psi(x,t)\\
H^2\psi(x,t)&=-\hbar^2\partial_{tt}\psi(x,t)\\
\left(\partial_{xx}-\frac{1}{c^2}\partial_{tt}-\frac{m^2c^2}{\hbar^2}\right)\psi(x,t)&=0\\
\left(\Box_{x}-\frac{m^2c^2}{\hbar^2}\right)\psi(x,t)&=0
\end{align}
then we try the plane wave ansatz $\psi_k(x,t)=e^{-i(\omega_k t-kx)}$ and see
\begin{align}
-k^2+\frac{1}{c^2}\omega_k^2-\frac{m^2c^2}{\hbar^2}=0\\
\rightarrow\omega_k^2=k^2c^2+\frac{m^2c^4}{\hbar^2}
\rightarrow\omega_k=\sqrt{k^2c^2+\frac{m^2c^4}{\hbar^2}}.
\end{align}
Therefore the general solution is a superposition
\begin{align}
\psi(x,t)=\int dk f(k)e^{-i(\omega_k t-kx)} + g(k) e^{-i(-\omega_k t-kx)}
\end{align}


\item Assume $\psi_0(x,t)$ is a solution then $\psi_0(x-y,t)$ is also a solution
\begin{align}
&\left(\Box_x-\frac{m^2c^2}{\hbar^2}\right)\psi_0(x,t)=0\\
\rightarrow&\left(\Box_x-\frac{m^2c^2}{\hbar^2}\right)\psi_0(x-y,t)=0
\end{align}
then with $\psi(x,t)=\int dy\,f(y)\psi_0(x-y,t)$
\begin{align}
\left(\Box_x-\frac{m^2c^2}{\hbar^2}\right)\psi(x,t)
&=\int dy\,f(y)\left(\Box_x-\frac{m^2c^2}{\hbar^2}\right)\psi_0(x-y,t)\\
&=0
\end{align}
and 
\begin{align}
\psi(x,0)&=\lim_{t\rightarrow0}\int dy\,f(y)\psi_0(x-y,t)\\
&=\int dy\,f(y)\delta(x-y)\\
&=f(x)
\end{align}
Now we can use the time propagation operator
\begin{align}
\psi_0(x,t)
&=e^{-iHt/\hbar}\psi(x,0)\\
&=e^{-it\sqrt{p^2c^2+m^2c^4}/hbar}\delta(x)\\
&=\frac{1}{2\pi\hbar}\int dp\, e^{-it\frac{mc^2}{\hbar}\sqrt{\frac{p^2}{m^2c^2}+1}}e^{ipx/\hbar}
\end{align}
and use $\cosh^2u-\sinh^2u=1$ and
\begin{align}
p&=mc\sinh u\\
dp&=mc\cosh u\;du
\end{align}
then
\begin{align}
\psi_0(x,t)
&=\frac{mc}{2\pi\hbar}\int du\, e^{-it\frac{mc^2}{\hbar}\sqrt{\sinh^2 u+1}}e^{i\frac{mc}{\hbar}x\sinh u}\cosh u\\
&=\frac{mc}{2\pi\hbar}\int du\, e^{-it\frac{mc^2}{\hbar}\cosh u}e^{i\frac{mc}{\hbar}x\sinh u}\cosh u\\
&=\frac{mc}{2\pi\hbar}\int du\,e^{i\frac{mc}{\hbar}(x\sinh u-ct\cosh u)}\cosh u\\
&=\frac{i}{2\pi c}\partial_t\int du\,e^{i\frac{mc}{\hbar}(x\sinh u-ct\cosh u)}.
\end{align}
Now we replace $x, t$ by new coordinates $v$ and $z$
\begin{align}
x&=\frac{\hbar}{mc}z\cosh v\\
ct&=\frac{\hbar}{mc}z\sinh v\\
&\rightarrow x^2-c^2t^2=\frac{\hbar^2}{m^2c^2}z^2
\end{align}
then we obtain with $y=u-v$
\begin{align}
\psi_0(x,t)
&=\frac{i}{2\pi c}\partial_t\int du\,e^{iz(\cosh v\sinh u-\sinh v\cosh u)}\\
&=\frac{i}{2\pi c}\partial_t\int du\,e^{iz\sinh (u-v)}\\
&=\frac{i}{2\pi c}\partial_t\int du\,\left[\cos(z\sinh (u-v))+i\sin(z\sinh (u-v))\right]\\
&=\frac{i}{2\pi c}\partial_t\int dy\,\left[\cos(z\sinh y)+i\sin(z\sinh y)\right]\\
&=\frac{i}{2\pi c}\partial_t\int_{-\infty}^\infty dy\,\cos(z\sinh y)\\
&=\frac{i}{\pi c}\partial_t\int_0^\infty dy\,\cos(z\sinh y)
\end{align}



\item

\item

\end{enumerate}

\section{{\sc Nastase} - Introduction to Quantum Field Theory} 

\subsection{Exercise 1.4 Scalar Dirac–Born–Infeld equations of motion}
With
\begin{align}
\frac{\partial(\partial_\mu\phi)^2}{\partial_\nu\phi}
&=\frac{\partial(\partial_\mu\phi\partial^\mu\phi)}{\partial(\partial_\nu\phi)}\\
&=\frac{\partial(\eta^{\mu\alpha}\partial_\mu\phi\partial_\alpha\phi)}{\partial(\partial_\nu\phi)}\\
&=\eta^{\mu\alpha}\frac{\partial(\partial_\mu\phi\partial_\alpha\phi)}{\partial(\partial_\nu\phi)}\\
&=\eta^{\mu\alpha}(\delta_{\mu\nu}\partial_\alpha\phi+\partial_\mu\phi\delta_{\alpha\nu})\\
&=\eta^{\mu\alpha}\delta_{\mu\nu}\partial_\alpha\phi+\eta^{\mu\alpha}\delta_{\alpha\nu}\partial_\mu\phi\\
&=\delta_\nu^\alpha\partial_\alpha\phi+\delta_\nu^\mu\partial_\mu\phi\\
&=2\partial_\nu\phi
\end{align}
we can calculate the parts for the Euler-Lagrange equations
\begin{align}
\frac{\partial\mathcal{L}}{\partial\phi}
&=-\frac{1}{L^4}\frac{L^4\left[\frac{\partial g}{\partial\phi}(\partial_\mu\phi)^2+2m^2\phi\right]}{2\sqrt{1+L^4[g(\partial_\mu\phi)^2+m^2\phi^2]}}\\
&=-\frac{\left[\frac{\partial g}{\partial\phi}(\partial_\mu\phi)^2+2m^2\phi\right]}{2\sqrt{1+L^4[g(\partial_\mu\phi)^2+m^2\phi^2]}}\\
%
\frac{\partial\mathcal{L}}{\partial(\partial_\nu\phi)}
&=-\frac{1}{L^4}\frac{L^4\left[2g(\partial_\mu\phi)\delta^\mu_\nu\right]}{2\sqrt{1+L^4[g(\partial_\mu\phi)^2+m^2\phi^2]}}\\
&=-\frac{g(\partial_\nu\phi)}{\sqrt{1+L^4[g(\partial_\mu\phi)^2+m^2\phi^2]}}\\
%
\partial_\nu\frac{\partial\mathcal{L}}{\partial(\partial_\nu\phi)}&=-\frac{g(\partial_\nu\partial_\nu\phi)\sqrt{1+L^4[g(\partial_\mu\phi)^2+m^2\phi^2]}-g(\partial_\nu\phi)\frac{L^4[2g(\partial_\mu\phi)(\partial_\nu\partial_\mu\phi)+2m^2\phi\partial_\nu\phi]}{2\sqrt{1+L^4[g(\partial_\mu\phi)^2+m^2\phi^2]}}}{1+L^4[g(\partial_\mu\phi)^2+m^2\phi^2]}
\end{align}
Multiplying the Euler-Lagrange equations 
\begin{align}
\frac{\partial\mathcal{L}}{\partial\phi}-\partial_\nu\frac{\partial\mathcal{L}}{\partial(\partial_\nu\phi)}=0
\end{align}
by $\sqrt{1+L^4[g(\partial_\mu\phi)^2+m^2\phi^2]}$ we obtain
\begin{align}
-\frac{1}{2}\left[\frac{\partial g}{\partial\phi}(\partial_\mu\phi)^2+2m^2\phi\right]+g(\partial_\nu\partial_\nu\phi)+\frac{1}{2}g(\partial_\nu\phi)\frac{L^4[2g(\partial_\mu\phi)(\partial_\nu\partial_\mu\phi)+2m^2\phi\partial_\nu\phi]}{1+L^4[g(\partial_\mu\phi)^2+m^2\phi^2]}&=0\\
%
g(\Box\phi-m^2\phi)-\frac{1}{2}\frac{\partial g}{\partial\phi}(\partial_\mu\phi)^2+gL^4\frac{g(\partial_\nu\phi)(\partial_\mu\phi)(\partial_\nu\partial_\mu\phi)+m^2\phi(\partial_\nu\phi)^2}{1+L^4[g(\partial_\mu\phi)^2+m^2\phi^2]}&=0
\end{align}

\subsection{Exercise 2.1 Equations of motion for an anharmonic}
With
\begin{align}
p
&=\frac{\partial L}{\partial \dot{q}}=\dot{q}\\
H
&=p\dot{q}-L\\
&=p^2-\frac{p^2}{2}+\frac{\lambda}{4!}q^4\\
&=\frac{p^2}{2}+\frac{\lambda}{4!}q^4\\
\end{align}
then
\begin{align}
\dot{p}&=-\frac{\partial H}{\partial q}=-\frac{\lambda}{3!}q^3\\
\dot{q}&=\frac{\partial H}{\partial p}=p
\end{align}
Phase space path integral
\begin{align}
M(q',t';q,t)
&=\mathcal{D}p(t)\mathcal{D}q(t)\exp\left\{i\int_t^{t'}dt[p(t)\dot{q}(t)-H(p(t),q(t))]\right\}\\
&=\mathcal{D}p(t)\mathcal{D}q(t)\exp\left\{i\int_t^{t'}dt[p(t)\dot{q}(t)-\frac{p(t)^2}{2}-\frac{\lambda}{4!}q(t)^4]\right\}
\end{align}

\section{{\sc Mandl, Shaw} - Quantum Field Theory 2e}
\subsection{Problem 1.1. Radiation field in a cube - NOT DONE YET}
First checking orthogonality
\begin{align}
a(a^\dagger)^n
&=(1+a^\dagger a)(a^\dagger)^{n-1}\\
&=(a^\dagger)^{n-1}+a^\dagger a(a^\dagger)^{n-1}\\
&=(a^\dagger)^{n-1}+(a^\dagger)(1+a^\dagger a)(a^\dagger)^{n-2}\\
&=2(a^\dagger)^{n-1}+(a^\dagger)^2a^\dagger a(a^\dagger)^{n-2}\\
&=n(a^\dagger)^{n-1}+(a^\dagger)^{n}a
\end{align}
then iteratively
\begin{align}
a^2(a^\dagger)^n&=n(n-1)(a^\dagger)^{n-2}+n(a^\dagger)^{n-1}a+(a^\dagger)^na^2\\
...&\\
a^n(a^\dagger)^n&=n!+....a+...a^2+...
\end{align}
so only the first term survives because of $a|0\rangle=0$ 
\begin{align}
\langle k|n\rangle=\langle 0|\frac{a^k}{\sqrt{k!}}\frac{(a^\dagger)^n}{\sqrt{n!}}0\rangle=\delta_{kn}.
\end{align}

\begin{enumerate}[(i)]
\item
\begin{align}
\langle c|c\rangle
&=e^{|c|^2}\sum_{n,k}\frac{(c^*)^kc^n}{\sqrt{k!n!}}\underbrace{\langle k|n\rangle}_{\delta_{kn}}\\
&=e^{-|c|^2}\sum_{n}\frac{|c|^{2n}}{n!}\\
&=e^{-|c|^2}\sum_{n}\frac{(|c|^2)^n}{n!}\\
&=e^{-|c|^2}e^{|c|^2}\\
&=1
\end{align}
\item With
\begin{align}
a_r(\mathbf{k})|...n_r(\mathbf{k})...\rangle
=\sqrt{n_r(\mathbf{k})}|...n_r(\mathbf{k})-1...\rangle
\end{align}
then
\begin{align}
a_r(\mathbf{k})|c\rangle
&=a_r(\mathbf{k})e^{|c|^2}\sum_{n=0}^\infty\frac{c^n}{\sqrt{n!}}|n\rangle\\
&=e^{|c|^2}\sum_{n=0}^\infty\frac{c^n}{\sqrt{n!}}a_r(\mathbf{k})|n\rangle\\
&=e^{|c|^2}\sum_{n=0}^\infty\frac{c^n}{\sqrt{n!}}\sqrt{n}|n-1\rangle\\
&=c\,e^{|c|^2}\sum_{n=0}^\infty\frac{c^{n-1}}{\sqrt{n!}}\sqrt{n}|n-1\rangle\\
&=x|c\rangle
\end{align}
\item
\begin{align}
\langle c|N|c\rangle
&=\langle c|a^\dagger a|c\rangle\\
&=\langle c|c^*c|c\rangle\\
&=c^*c\langle c||c\rangle\\
&=|c|^2
\end{align}

\item
\begin{align}
\langle c|N^2|c\rangle
&=\langle c|a^\dagger a a^\dagger a|c\rangle\\
&=|c|^2\langle c|a a^\dagger|c\rangle\\
\end{align}

\end{enumerate}

\subsection{Problem 1.2. Lagrangian of point particle in EM potential - NOT DONE YET}
\begin{enumerate}[(i)]
\item
\begin{align}
\frac{dL}{d\dot{\mathbf{x}}}&=m\dot{\mathbf{x}}+\frac{q}{c}\mathbf{A}\\
\frac{\partial}{\partial t}\frac{dL}{d\dot{\mathbf{x}}}&=m\ddot{\mathbf{x}}+\frac{q}{c}\dot{\mathbf{A}}\\
\frac{dL}{d\mathbf{x}}&=\frac{q}{c}\nabla(\mathbf{A}\cdot\dot{\mathbf{x}})-q\nabla\phi\\
&=\frac{q}{c}\left[
\mathbf{A}\times(\nabla\times\dot{\mathbf{x}})
+\dot{\mathbf{x}}\times(\nabla\times\mathbf{A})
+(\mathbf{A}\cdot\nabla)\dot{\mathbf{x}}
+(\dot{\mathbf{x}}\cdot\nabla)\mathbf{A}
\right]
-q\nabla\phi\\
&=\frac{q}{c}\left[
0
+\dot{\mathbf{x}}\times\mathbf{B}
+0
+(\dot{\mathbf{x}}\cdot\nabla)\mathbf{A}
\right]
-q\nabla\phi\\
\rightarrow&m\ddot{\mathbf{x}}=q\left(+\nabla\phi-\frac{1}{c}\frac{\partial}{\partial t}\dot{\mathbf{A}}\right)
-\frac{q}{c}\dot{\mathbf{x}}\times\mathbf{B}
-\frac{q}{c}(\dot{\mathbf{x}}\cdot\nabla)\mathbf{A}
\end{align}
\item
\end{enumerate}




\subsection{Problem 2.1 - NOT DONE YET}
\begin{align}
\delta S
&=\int d^4x\,\delta(\mathcal{L}+\partial_\alpha\Lambda^\alpha)\\
&=\int d^4x\,\delta\mathcal{L}+\delta\int d^3\sigma_\alpha\Lambda^\alpha\\
&=\int d^4x\,\frac{\partial\mathcal{L}}{\partial\phi}\delta\phi+\frac{\partial\mathcal{L}}{\partial\phi_{,\beta}}\delta\phi_{,\beta}+\int d^3\sigma_\alpha\frac{\partial\Lambda^\alpha}{\partial\phi}\delta\phi\\
&=\int d^4x\,\frac{\partial\mathcal{L}}{\partial\phi}\delta\phi-\frac{\partial}{\partial x^\beta}\left(\frac{\partial\mathcal{L}}{\partial\phi_{,\beta}}\right)\delta\phi+\int d^4\frac{\partial}{\partial x^\beta}\left(\frac{\partial\mathcal{L}}{\partial\phi_{,\beta}}\delta\phi\right)+\int d^3\sigma_\alpha\frac{\partial\Lambda^\alpha}{\partial\phi}\delta\phi\\
&=\int_\Omega d^4x\,\frac{\partial\mathcal{L}}{\partial\phi}\delta\phi-\frac{\partial}{\partial x^\beta}\left(\frac{\partial\mathcal{L}}{\partial\phi_{,\beta}}\right)\delta\phi+\int_{\partial\Omega} d^3\sigma_\beta\left(\frac{\partial\mathcal{L}}{\partial\phi_{,\beta}}\delta\phi\right)+\int_{\partial\Omega} d^3\sigma_\alpha\frac{\partial\Lambda^\alpha}{\partial\phi}\delta\phi
\end{align}
as $\delta\phi$ vanishes on the boundary $\partial\Omega$ the $\Lambda^alpha$ does not change the equation of motion.


\section{{\sc Straumann} - Relativistische Quantentheorie} 
\subsection{Problem 1.11.1. Momentum and angular momentum of the radiation field}

\begin{align}
\mathbf{P}&=\frac{1}{4\pi c}\int_V\mathbf{E}\times\mathbf{B}\,d^3x\\
\mathbf{J}&=\frac{1}{4\pi c}\int_V[\mathbf{x}\times(\mathbf{E}\times\mathbf{B})]\,d^3x
\end{align}
In Coulomb gauge we have
\begin{align}
\mathbf{E}&=-\frac{1}{c}\partial_t\mathbf{A}=-\frac{1}{c}\dot{A}_l\mathbf{e}_l\\
\mathbf{B}&=\nabla\times\mathbf{A}=\varepsilon_{ijk}(\partial_jA_k)\mathbf{e}_i\\
\mathbf{E}\times\mathbf{B}
&=-\frac{1}{c}\varepsilon_{nli}\mathbf{e}_n(\dot{A}_l\mathbf{e}_l)(\varepsilon_{ijk}(\partial_jA_k)\mathbf{e}_i)\\
&=-\frac{1}{c}\varepsilon_{nli}\mathbf{e}_n\dot{A}_l\varepsilon_{ijk}(\partial_jA_k)\mathbf{e}_i\mathbf{e}_l\\
&=-\frac{1}{c}\varepsilon_{nli}\mathbf{e}_n\dot{A}_l\varepsilon_{ijk}(\partial_jA_k)\delta_{il}\\
&=-\frac{1}{c}\varepsilon_{nli}\varepsilon_{ijk}(\partial_jA_k)\dot{A}_i\mathbf{e}_n\\
&=-\frac{1}{c}(\delta_{nj}\delta_{lk}-\delta_{nk}\delta_{lj})(\partial_jA_k)\dot{A}_l\mathbf{e}_n\\
&=-\frac{1}{c}((\partial_jA_k)\dot{A}_k\mathbf{e}_j-(\partial_jA_k)\dot{A}_j\mathbf{e}_k)\\
&=-\frac{1}{c}((\mathbf{e}_j\partial_jA_k)\dot{A}_k-\dot{A}_j(\partial_jA_k)\mathbf{e}_k)\\
&=-\frac{1}{c}[\nabla(\mathbf{A}\cdot\dot{\mathbf{A}})-(\dot{\mathbf{A}}\cdot\nabla)\mathbf{A}]
\end{align}
And from (1.44) and (1.33)
\begin{align}
\mathbf{A}(x,t)&=\frac{1}{\sqrt{V}}\sum_{\mathbf{k},\lambda}\sqrt{\frac{2\pi\hbar c^3}{\omega_k}}\left[
a_{\mathbf{k},\lambda}\boldsymbol{\varepsilon}(k,\lambda)e^{i\mathbf{k}\cdot\mathbf{x}}
+a^*_{\mathbf{k},\lambda}\boldsymbol{\varepsilon}(k,\lambda)^*e^{-i\mathbf{k}\cdot\mathbf{x}}\right]\\
&=\sum_{\mathbf{k},\lambda}\sqrt{\frac{2\pi\hbar c^3}{\omega_k}}\left[
a_{\mathbf{k},\lambda}\mathbf{u}_{\mathbf{k},\lambda}(\mathbf{x})
+a^*_{\mathbf{k},\lambda}\mathbf{u}^*_{\mathbf{k},\lambda}(\mathbf{x})\right]\\
\end{align}

\subsection{Problem 4.5.1. Approximation for polarization potential}

\begin{align}
\Phi^\text{Pol}(\mathbf{x})
&=\frac{e}{(2\pi)^3}\int d^3k e^{i\mathbf{k}\cdot\mathbf{x}}\int_{4m^2}^\infty d\kappa^2\frac{\Pi(x^2)}{\kappa^2(\kappa^2+\mathbf{k}^2)}
\end{align}


\section{{\sc Ramond} - Field Theory - A modern primer} 
\subsection{Problem 1.1 A}
\begin{enumerate}[(i)]
\item With
\begin{align}
\left(\frac{d(x+\delta x)}{dt}\right)^2
&=\left(\frac{dx}{dt}+\delta\frac{dx}{dt}\right)\left(\frac{dx}{dt}+\delta\frac{dx}{dt}\right)\\
&=\left(\frac{dx}{dt}\right)^2+2\frac{dx}{dt}\cdot\delta\frac{dx}{dt}+\left(\delta\frac{dx}{dt}\right)^2\\
&=\left(\frac{dx}{dt}\right)^2+\frac{d}{dt}\left(\frac{dx}{dt}\delta x\right)-2\frac{d^2x}{dt^2}\delta x+\left(\delta\frac{dx}{dt}\right)^2
\end{align}
where we integrates the second term by parts. Now we can expand the action
\begin{align}
S
&=\int dt\frac{1}{2}m\left(\frac{dx}{dt}\right)^2\\
S[x+\delta x]
&=\int dt\frac{1}{2}m\left(\frac{d(x+\delta x)}{dt}\right)^2\\
\delta S&=-\frac{1}{2}m\int_{t_1}^{t_2} dt\,2\frac{dx}{dt}\frac{d\delta(x)}{dt}\\\
&=-\frac{1}{2}m\int_{t_1}^{t_2} dt\delta x\left(2\frac{d^2x}{dt^2}\right)+\left.\frac{1}{2}m\frac{dx}{dt}\delta x\right|_{t_1}^{t_2}
\end{align}
Assuming the equations of motion hold $\ddot{x}=0$ an forcing the surface term to vanish (we CAN'T force $\delta x=0$) we have
\begin{align}
\frac{d}{dt}\left(\frac{dx}{dt}\right)=0
\end{align}

\item
We could assume a velocity dependent potential is considered 
\begin{align}\
V=\sqrt{\dot{x}_1^2+\dot{x}_2^2+\dot{x}_3^2}\left(1-\cos\frac{\sqrt{x_1^2+x_2^2+x_3^2}}{a}\right)
\end{align}
but then units would be off - so we assume $v$ to be a constant. The
\begin{align}
\delta S_V
&=\frac{\partial V}{\partial x_i}\delta x_i\\
&=\frac{vx_i}{ar}\sin\frac{r}{a}\delta x_i\\
&\rightarrow m\ddot{x}_i=\frac{vx_i}{ar}\sin\frac{r}{a}\delta x_i
\end{align} 
Surface term
\begin{align}
\left(\frac{\partial L}{\partial \dot{x}_i}\delta x_i\right)_{t_1}^{t_2}
\end{align}
\begin{align}
\frac{d}{dt}\frac{\partial L}{\partial \dot{x}_i}=
\frac{d}{dt}\frac{\partial L}{\partial p_i}=\frac{\partial L}{\partial x_i}=\frac{vx_i}{ar}\sin\frac{r}{a}\delta x_i
\end{align}
\end{enumerate}

\section{{\sc Muenster} - Von der Quantenfeldtheorie zum Standardmodell} 
\subsection{Problem 2.1 - 1}
\begin{enumerate}[(a)]
\item The Klein-Gordon equations is given by
\begin{align}
    \left(\partial_\mu\partial^\mu+\frac{m^2c^2}{\hbar^2}\right)\varphi&=0\\
    \left(c^2\partial_{tt}-\triangle+\frac{m^2c^2}{\hbar^2}\right)\varphi&=0
\end{align}
We make the ansatz
\begin{align}
    \varphi&=\phi_1+\phi_2\\
    \phi_1&=\frac{1}{2}\varphi-\alpha\partial_t\varphi\\
    \phi_2&=\frac{1}{2}\varphi+\alpha\partial_t\varphi
\end{align}
Then we get expressions for the time derivatives 
\begin{align}
    \phi_2-\phi_1&=2\alpha\partial_t\varphi\\
    \rightarrow\partial_t\varphi&=\frac{1}{2\alpha}(\phi_2-\phi_1)
\end{align}
and
\begin{align}
    \partial_{tt}\varphi&=c^2\left(\triangle-\frac{m^2c^2}{\hbar^2}\right)\varphi\\
    &=c^2\left(\triangle-\frac{m^2c^2}{\hbar^2}\right)(\phi_1+\phi_2)
\end{align}
Therefore we get for $\phi_{1,2}$
\begin{align}
    \partial_t\phi_1
    &=\frac{1}{2}\partial_t\varphi-\alpha\partial_{tt}\varphi\\
    &=\frac{1}{2\alpha}(\phi_2-\phi_1)-\alpha c^2\left(\triangle-\frac{m^2c^2}{\hbar^2}\right)(\phi_1+\phi_2)\\
    \partial_t\phi_2
    &=\frac{1}{2}\partial_t\varphi+\alpha\partial_{tt}\varphi\\
    &=\frac{1}{2\alpha}(\phi_2-\phi_1)+\alpha c^2\left(\triangle-\frac{m^2c^2}{\hbar^2}\right)(\phi_1+\phi_2)
\end{align}
which we can write in the form
\begin{align}
i\hbar\partial_t\begin{pmatrix}
\phi_1 \\
\phi_2 
\end{pmatrix}=-i\hbar\alpha c^2\left(\triangle-\frac{m^2c^2}{\hbar^2}\right)
\begin{pmatrix}
 1 &  1 \\
-1 & -1 
\end{pmatrix}\begin{pmatrix}
\phi_1 \\
\phi_2 
\end{pmatrix}
+\frac{i\hbar}{2\alpha}
\begin{pmatrix}
-1 &  1 \\
-1 &  1 
\end{pmatrix}
\begin{pmatrix}
\phi_1 \\
\phi_2 
\end{pmatrix}\\
=i\hbar\begin{pmatrix}
-\alpha c^2\left(\triangle-\frac{m^2c^2}{\hbar^2}\right)-\frac{1}{2\alpha} &  -\alpha c^2\left(\triangle-\frac{m^2c^2}{\hbar^2}\right)+\frac{1}{2\alpha} \\
\alpha c^2\left(\triangle-\frac{m^2c^2}{\hbar^2}\right)-\frac{1}{2\alpha} &  \alpha c^2\left(\triangle-\frac{m^2c^2}{\hbar^2}\right)+\frac{1}{2\alpha}
\end{pmatrix}
\begin{pmatrix}
\phi_1  \\
\phi_2  
\end{pmatrix}
\end{align}
\item Diagonalization gives
\begin{align}
    i\hbar\partial_t\phi&=\hat{H}\phi\\
    \rightarrow i\hbar\partial_tS^{-1}\phi&=\underbrace{S^{-1}\hat{H}S}_{=h}S^{-1}\phi\\
    \lambda_\pm&=\pm\sqrt{2}c\hbar\sqrt{\Delta-\frac{m^2c^2}{\hbar^2}}\\
    &=\mp\sqrt{2}mc^2\sqrt{1-\frac{\hbar^2}{m^2c^2}\triangle}
\end{align}
A semi-canonical choice for the parameter $\alpha$ is to make the $\triangle$ look like a momentum operator
\begin{align}
    i\hbar\alpha c^2=-\frac{\hbar^2}{2m}\quad\rightarrow\quad\alpha=\frac{i\hbar}{2mc^2}
\end{align}
\end{enumerate}

\section{{\sc Peskin, Schroeder} - An Introduction to Quantum Field Theory}
\subsection{Problem 2.1 - Maxwell equations}
\begin{enumerate}[(a)]
\item
\begin{align}
\mathcal{L}&=-\frac{1}{4}F_{\mu\nu}F^{\mu\nu}=-\frac{1}{4}\eta^{\alpha\mu}\eta^{\beta\nu}F_{\mu\nu}F_{\alpha\beta}\\
&=-\frac{1}{4}\eta^{\alpha\mu}\eta^{\beta\nu}(\partial_\mu A_\nu-\partial_\nu A_\mu)(\partial_\alpha A_\beta-\partial_\beta A_\alpha)
\end{align}
With
\begin{align}
\frac{\partial\mathcal{L}}{\partial A_\gamma}-\partial_\sigma\frac{\mathcal{L}}{\partial(\partial_\sigma A_\gamma)}=0
\end{align}
then
\begin{align}
\frac{\mathcal{L}}{\partial(\partial_\sigma A_\gamma)}
&=-\frac{1}{4}\eta^{\alpha\mu}\eta^{\beta\nu}(\partial_\mu A_\nu-\partial_\nu A_\mu)(\partial_\alpha A_\beta-\partial_\beta A_\alpha)
-\frac{1}{4}\eta^{\alpha\mu}\eta^{\beta\nu}(\partial_\mu A_\nu-\partial_\nu A_\mu)(\partial_\alpha A_\beta-\partial_\beta A_\alpha)\\
&=-\frac{1}{4}\eta^{\alpha\mu}\eta^{\beta\nu}(\delta_\mu^\sigma\delta_\nu^\gamma-\delta_\nu^\sigma\delta_\mu^\gamma)(\partial_\alpha A_\beta-\partial_\beta A_\alpha)
-\frac{1}{4}\eta^{\alpha\mu}\eta^{\beta\nu}(\partial_\mu A_\nu-\partial_\nu A_\mu)(\partial_\alpha A_\beta-\partial_\beta A_\alpha)\\
&=-\frac{1}{4}(\delta^{\alpha\sigma}\delta^{\beta\gamma}-\delta^{\beta\sigma}\delta^{\alpha\gamma})(\partial_\alpha A_\beta-\partial_\beta A_\alpha)-...\\
&=-\frac{1}{4}(\partial^\sigma A^\gamma-\partial^\gamma A^\sigma-\partial^\gamma A^\sigma+\partial^\sigma A^\gamma)-...\\
&=-\frac{1}{4}2F^{\sigma\gamma}-...\\
&=-F^{\sigma\gamma}
\end{align}
and therefore
\begin{align}
\partial_\sigma F^{\sigma\gamma}=0
\end{align}
Rewriting into the common form
\begin{align}
\gamma=0
&\quad\rightarrow\quad\partial_0F^{00}+\sum_i\partial_iF^{i0}=0\\
&\quad\rightarrow\qquad\qquad\sum_i\partial_i(-F^{0i})=0\\
&\quad\rightarrow\qquad\qquad\sum_i\partial_iE^i=0\\
&\quad\rightarrow\qquad\qquad\nabla\cdot\mathbf{E}=0
\end{align}
\begin{align}
\gamma=k
&\quad\rightarrow\quad\partial_0F^{0k}+\sum_i\partial_iF^{ik}=0\\
&\quad\rightarrow\qquad\partial_0(-E^k)+\sum_i\partial_iF^{ik}=0\\
&\quad\rightarrow\qquad\partial_0(-E^k)+\sum_i\partial_i(-\epsilon_{ikm}B^m)=0\\
&\quad\rightarrow\qquad\dot{\mathbf{E}}=\nabla\times\mathbf{B}
\end{align}
The other two equations come from
\begin{align}
F_{\mu\nu}=\partial_\mu A_\nu-\partial_\nu A_\mu\\
\rightarrow\quad \partial_\lambda F_{\mu\nu}+\partial_\mu F_{\nu\lambda}+\partial_\nu F_{\lambda\mu}=0
\end{align}

\item With the definition (2.17)
\begin{align}
T^\mu_{\;\nu}&=\frac{\partial\mathcal{L}}{\partial(\partial_\mu A_\lambda)}\partial_\nu A_\lambda-\mathcal{L}\delta^\mu_{\;\nu}\\
&=-F^{\mu\lambda}\partial_\nu A_\lambda+\frac{1}{4}F_{\alpha\beta}F^{\alpha\beta}\delta^\mu_{\;\nu}
\end{align}
we rewrite
\begin{align}
T^{\mu\nu}
&=-F^{\mu\lambda}\partial^\nu A_\lambda+\frac{1}{4}F_{\alpha\beta}F^{\alpha\beta}\eta^{\mu\nu}\\
\widehat{T}^{\mu\nu}
&=-F^{\mu\lambda}\partial^\nu A_\lambda+\frac{1}{4}F_{\alpha\beta}F^{\alpha\beta}\eta^{\mu\nu}+
\partial_\lambda(F^{\mu\lambda}A^\nu)\\
&=-F^{\mu\lambda}\partial^\nu A_\lambda+\frac{1}{4}F_{\alpha\beta}F^{\alpha\beta}\eta^{\mu\nu}+
\underbrace{(\partial_\lambda F^{\mu\lambda})}_{=0\text{\;(Maxwell)}}A^\nu+F^{\mu\lambda}(\partial_\lambda A^\nu)\\
&=F^{\mu\lambda}F_\lambda^{\,\nu}+\frac{1}{4}F_{\alpha\beta}F^{\alpha\beta}\eta^{\mu\nu}\\
&=F^{\mu\lambda}F_{\lambda\sigma}\eta^{\sigma\nu}+\frac{1}{4}F_{\alpha\beta}F^{\alpha\beta}\eta^{\mu\nu}\\
&=F^{\uparrow\uparrow}F_{\downarrow\downarrow}\eta+\frac{1}{4}\text{tr}(-F^{\uparrow\uparrow}F_{\downarrow\downarrow})\eta
\end{align}
and with
\begin{align}
F_{\mu\nu}&=F_{\downarrow\downarrow}=\begin{pmatrix}
0    &  E_x & E_y  & E_z\\
-E_x &  0   & -B_z & B_y\\
-E_y &  B_z & 0    & -B_x\\
-E_z & -B_y & B_x  & 0
\end{pmatrix}
\qquad
F^{\mu\nu}=F_{\uparrow\uparrow}=\eta F_{\downarrow\downarrow}\eta^T=\begin{pmatrix}
0    &  -E_x & -E_y  & -E_z\\
E_x &  0   & -B_z & B_y\\
E_y &  B_z & 0    & -B_x\\
E_z & -B_y & B_x  & 0
\end{pmatrix}\\
F_{\mu\nu}F^{\mu\nu}&=-\text{tr}(F_{\downarrow\downarrow}.F_{\uparrow\uparrow})=2(\mathbf{B}^2-\mathbf{E}^2)
\qquad
F^{\mu\lambda}F_{\lambda\nu}=...
\end{align}
we obtain
\begin{align}
\widehat{T}^{\mu\nu}
&=\begin{pmatrix}
\mathcal{E} & \mathbf{S} \\
\mathbf{S}  & ...
\end{pmatrix}
\end{align}
which looks symmetric.
\end{enumerate}

\subsection{Problem 2.2 - The complex scalar field}
\begin{enumerate}[(a)]
\item Using $\partial_\mu\phi^*\partial^\mu\phi=\partial_\mu\phi^*\eta^{\mu\nu}\partial_\nu\phi=\partial^\mu\phi^*\partial_\mu\phi$ and $\partial^\mu=\eta^{\mu\nu}\partial_\nu=(\partial_0,-\partial_i)$ we find 
\begin{align}
\pi
&=\frac{\partial\mathcal{L}}{\partial(\partial\dot\phi)}=\frac{\partial\mathcal{L}}{\partial(\partial_0\phi)}=\partial^0\phi^*=\partial_0\phi^*=\dot\phi^*\\
\pi^*
&=\frac{\partial\mathcal{L}}{\partial(\partial\dot\phi^*)}=\frac{\partial\mathcal{L}}{\partial(\partial_0\phi^*)}=\partial^0\phi=\partial_0\phi=\dot\phi
\end{align}
then
\begin{align}
H&=\int d^3x[\pi\dot\phi+\pi^*\dot\phi^*-\mathcal{L}]\\
&=\int d^3x[\pi\pi^*+\pi^*\pi-\partial_\mu\phi^*\eta^{\mu\nu}\partial_\nu\phi+m^2\phi^*\phi]\\
&=\int d^3x[\pi\pi^*+\pi^*\pi-(\underbrace{\dot\phi^*\dot\phi}_{=\pi\pi^*}-\nabla\phi^*\cdot\nabla\phi)+m^2\phi^*\phi]\\
&=\int d^3x[\pi^*\pi+\nabla\phi^*\cdot\nabla\phi+m^2\phi^*\phi]
\end{align}
Let's rewrite the Lagrangian with $\phi=\frac{1}{\sqrt{2}}(\phi_1+i\phi_2)$
\begin{align}
\mathcal{L}&=\partial_\mu\phi^*\partial^\mu\phi-m^2\phi^*\phi\\
&=\frac{1}{2}\partial_\mu(\phi_1-i\phi_2)\partial^\mu(\phi_1+i\phi_2)-\frac{1}{2}m^2(\phi_1-i\phi_2)(\phi_1+i\phi_2)\\
&=\frac{1}{2}(\partial_\mu\phi_1\partial^\mu\phi_1-m^2\phi_1^2)+i\frac{1}{2}(\partial_\mu\phi_2\partial^\mu\phi_2-m^2\phi_2^2)
\end{align}
So we use the results for the scalar field
\begin{align}
\phi_1(\mathbf{x})&=\int\frac{d^3p}{(2\pi)^3\sqrt{2\omega_\mathbf{p}}}\left(a_\mathbf{p}e^{i\mathbf{p}\cdot\mathbf{x}}+a^\dagger_\mathbf{p}e^{-i\mathbf{p}\cdot\mathbf{x}}\right)\\
\pi_1(\mathbf{x})&=-i\int\frac{d^3p}{(2\pi)^3\sqrt{2}}\sqrt{\omega_\mathbf{p}}\left(a_\mathbf{p}e^{i\mathbf{p}\cdot\mathbf{x}}+a^\dagger_\mathbf{p}e^{-i\mathbf{p}\cdot\mathbf{x}}\right)\\
\phi_2(\mathbf{x})&=\int\frac{d^3p}{(2\pi)^3\sqrt{2\omega_\mathbf{p}}}\left(b_\mathbf{p}e^{i\mathbf{p}\cdot\mathbf{x}}+b^\dagger_\mathbf{p}e^{-i\mathbf{p}\cdot\mathbf{x}}\right)\\
\pi_2(\mathbf{x})&=-i\int\frac{d^3p}{(2\pi)^3\sqrt{2}}\sqrt{\omega_\mathbf{p}}\left(b_\mathbf{p}e^{i\mathbf{p}\cdot\mathbf{x}}+b^\dagger_\mathbf{p}e^{-i\mathbf{p}\cdot\mathbf{x}}\right)\\
[a_\mathbf{p},a^\dagger_\mathbf{q}]&=(2\pi)^3\delta^{(3)}(\mathbf{p}-\mathbf{q})\\
[b_\mathbf{p},b^\dagger_\mathbf{q}]&=(2\pi)^3\delta^{(3)}(\mathbf{p}-\mathbf{q})
\end{align}
then
\begin{align}
\phi(\mathbf{x})&=\frac{1}{\sqrt{2}}\int\frac{d^3p}{(2\pi)^3\sqrt{2\omega_\mathbf{p}}}\left((a_\mathbf{p}+ib_\mathbf{p})e^{i\mathbf{p}\cdot\mathbf{x}}+(a^\dagger_\mathbf{p}+ib^\dagger_\mathbf{p})e^{-i\mathbf{p}\cdot\mathbf{x}}\right)\\
&\equiv\int\frac{d^3p}{(2\pi)^3\sqrt{2\omega_\mathbf{p}}}\left(\alpha_\mathbf{p}e^{i\mathbf{p}\cdot\mathbf{x}}+\beta^\dagger_\mathbf{p} e^{-i\mathbf{p}\cdot\mathbf{x}}\right)\\
\phi^\dagger(\mathbf{x})&=\frac{1}{\sqrt{2}\int\frac{d^3p}{(2\pi)^3}\sqrt{2\omega_\mathbf{p}}}\left((a^\dagger_\mathbf{p}-ib^\dagger_\mathbf{p})e^{-i\mathbf{p}\cdot\mathbf{x}}+(a_\mathbf{p}-ib_\mathbf{p})e^{i\mathbf{p}\cdot\mathbf{x}}\right)\\
&\equiv\int\frac{d^3p}{(2\pi)^3\sqrt{2\omega_\mathbf{p}}}\left(\alpha^\dagger_\mathbf{p}e^{-i\mathbf{p}\cdot\mathbf{x}}+\beta_\mathbf{p}e^{i\mathbf{p}\cdot\mathbf{x}}\right)
\end{align}
With the new defines creation/annihilation operators
\begin{align}
\alpha_\mathbf{p}&=\frac{1}{\sqrt{2}}(a_\mathbf{p}+ib_\mathbf{p})\quad\rightarrow\quad \alpha^\dagger_\mathbf{p}=\frac{1}{\sqrt{2}}(a^\dagger_\mathbf{p}-ib^\dagger_\mathbf{p})\\
\beta_\mathbf{p}&=\frac{1}{\sqrt{2}}(a_\mathbf{p}-ib_\mathbf{p})\quad\rightarrow\quad \beta^\dagger_\mathbf{p}=\frac{1}{\sqrt{2}}(a^\dagger_\mathbf{p}+ib^\dagger_\mathbf{p})
\end{align}
we can calculate their commutation relations ({\bf assuming all the cross commutators between $a,a^\dagger$ and $b, b^\dagger$ are zero})
\begin{align}
[\alpha_\mathbf{p},\alpha_\mathbf{q}]
&=\frac{1}{2}[a_\mathbf{p}+ib_\mathbf{p},a_\mathbf{q}+ib_\mathbf{q}]\\
&=\frac{1}{2}([a_\mathbf{p},a_\mathbf{q}]+i[b_\mathbf{p},a_\mathbf{q}]+i[a_\mathbf{p},b_\mathbf{q}]-[b_\mathbf{p},b_\mathbf{q}])\\
&=\frac{1}{2}i([b_\mathbf{p},a_\mathbf{q}]+[a_\mathbf{p},b_\mathbf{q}])\\
&=0
\end{align}
\begin{align}
[\alpha^\dagger_\mathbf{p},\alpha^\dagger_\mathbf{q}]
&=\frac{1}{2}([a^\dagger_\mathbf{p}-ib^\dagger_\mathbf{p},a^\dagger_\mathbf{q}-ib^\dagger_\mathbf{q}])\\
&=\frac{1}{2}([a^\dagger_\mathbf{p},a^\dagger_\mathbf{q}]-i[b^\dagger_\mathbf{p},a^\dagger_\mathbf{q}]-i[a^\dagger_\mathbf{p},b^\dagger_\mathbf{q}]-[b^\dagger_\mathbf{p},b^\dagger_\mathbf{q}])\\
&=\frac{1}{2}(-i[b^\dagger_\mathbf{p},a^\dagger_\mathbf{q}]-i[a^\dagger_\mathbf{p},b^\dagger_\mathbf{q}])\\
&=0
\end{align}
\begin{align}
[\alpha_\mathbf{p},\alpha^\dagger_\mathbf{q}]
&=\frac{1}{2}[a_\mathbf{p}+ib_\mathbf{p},a^\dagger_\mathbf{q}-ib^\dagger_\mathbf{q}]\\
&=\frac{1}{2}([a_\mathbf{p},a^\dagger_\mathbf{q}]+i[b_\mathbf{p},a^\dagger_\mathbf{q}]-i[a_\mathbf{p},b^\dagger_\mathbf{q}]+[b_\mathbf{p},b^\dagger_\mathbf{q}])\\
&=(2\pi)^3\delta^{(3)}(\mathbf{p}-\mathbf{q})+i[b_\mathbf{p},a^\dagger_\mathbf{q}]-i[a_\mathbf{p},b^\dagger_\mathbf{q}]\\
&=(2\pi)^3\delta^{(3)}(\mathbf{p}-\mathbf{q})
\end{align}
\begin{align}
[\beta_\mathbf{p},\beta^\dagger_\mathbf{q}]
&=\frac{1}{2}[a_\mathbf{p}-ib_\mathbf{p},a^\dagger_\mathbf{q}+ib^\dagger_\mathbf{q}]\\
&=\frac{1}{2}([a_\mathbf{p},a^\dagger_\mathbf{q}]-i[b_\mathbf{p},a^\dagger_\mathbf{q}]+i[a_\mathbf{p},b^\dagger_\mathbf{q}]+[b_\mathbf{p},b^\dagger_\mathbf{q}])\\
&=(2\pi)^3\delta^{(3)}(\mathbf{p}-\mathbf{q})
\end{align}
\begin{align}
[\alpha_\mathbf{p},\beta_\mathbf{q}]
&=\frac{1}{2}[a_\mathbf{p}+ib_\mathbf{p},a_\mathbf{q}-ib_\mathbf{q}]\\
&=\frac{1}{2}([a_\mathbf{p},a_\mathbf{q}]+i[a_\mathbf{p},b_\mathbf{q}]+i[b_\mathbf{p},a_\mathbf{q}]-[b_\mathbf{p},b_\mathbf{q}])\\
&=0
\end{align}
\begin{align}
[\alpha_\mathbf{p},\beta^\dagger_\mathbf{q}]
&=\frac{1}{2}[a_\mathbf{p}+ib_\mathbf{p},a^\dagger_\mathbf{q}-ib^\dagger_\mathbf{q}]\\
&=\frac{1}{2}([a_\mathbf{p},a^\dagger_\mathbf{q}]+i[a_\mathbf{p},b^\dagger_\mathbf{q}]+i[b_\mathbf{p},a^\dagger_\mathbf{q}]-[b_\mathbf{p},b^\dagger_\mathbf{q}])\\
&=(2\pi)^3\delta^{(3)}(\mathbf{p}-\mathbf{q})-(2\pi)^3\delta^{(3)}(\mathbf{p}-\mathbf{q})\\
&=0\\
[\alpha^\dagger_\mathbf{p},\beta^\dagger_\mathbf{q}]&=0
\end{align}
As the $\phi{\mathbf{x}}$ is in the Schroedinger picture there is not time dependency and we can not calculate $\pi(\mathbf{x})$ - therefore we need to transform to the Heisenberg picture. To make it simple we do this first for $\phi_1$ and $\phi_2$ using $p\cdot x=E_pt-\mathbf{p}\cdot\mathbf{x}$ and $p^2=E_\mathbf{p}^2-\mathbf{p}^2=m^2$ (meaning $p^0\equiv E_\mathbf{p}=\sqrt{\mathbf{p}^2+m^2}$
\begin{align}
\phi_1(x)
&=e^{iHt}\phi(\mathbf{x})e^{-iHt}\\
&=...\\
&=\int\frac{d^3p}{(2\pi)^3}\frac{1}{\sqrt{2E_\mathbf{p}}}(a_\mathbf{p}e^{-ipx}+a^\dagger_\mathbf{p}e^{ipx})\\
\phi_2(x)
&=\int\frac{d^3p}{(2\pi)^3}\frac{1}{\sqrt{2E_\mathbf{p}}}(b_\mathbf{p}e^{-ipx}+b^\dagger_\mathbf{p}e^{ipx})\\
\end{align}
{\it Here we cheated a bit - we used the result from the scalar Lagrangian - meaning using the scalar Hamiltonian.}
Then
\begin{align}
\phi(x)
&=\int\frac{d^3p}{(2\pi)^3\sqrt{2E_\mathbf{p}}}\left(\alpha_\mathbf{p}e^{-ipx}+\beta^\dagger_\mathbf{p} e^{ipx}\right)\\
\phi^\dagger(x)
&=\int\frac{d^3p}{(2\pi)^3\sqrt{2E_\mathbf{p}}}\left(\alpha^\dagger_\mathbf{p}e^{ipx}+\beta_\mathbf{p}e^{-ipx}\right)
\end{align}
and
\begin{align}
\rightarrow\quad\pi^*(x)=\dot{\phi}(x)
&=i\int\frac{d^3p}{(2\pi)^3\sqrt{2}}\sqrt{E_\mathbf{p}}\left(-\alpha_\mathbf{p}e^{-ipx}+\beta^\dagger_\mathbf{p} e^{ipx}\right)\\
\rightarrow\quad\pi(x)=\dot{\phi}^\dagger(x)
&=i\int\frac{d^3p}{(2\pi)^3\sqrt{2}}\sqrt{E_\mathbf{p}}\left(\alpha^\dagger_\mathbf{p}e^{ipx}-\beta_\mathbf{p} e^{-ipx}\right)
\end{align}
The only non-vanishing commutator relations for field and momentum operators are
\begin{align}
[\phi(\mathbf{x},t),\pi(\mathbf{y},t)]&=i\int\frac{d^3p}{(2\pi)^3\sqrt{2E_\mathbf{p}}}\int\frac{d^3q}{(2\pi)^3\sqrt{2}}\sqrt{E_\mathbf{q}}[\alpha_\mathbf{p}e^{-ipx}+\beta^\dagger_\mathbf{p} e^{ipx},\alpha^\dagger_\mathbf{q}e^{iqy}-\beta_\mathbf{q} e^{-iqy}]\\
&=i\int\frac{d^3p\;d^3q}{(2\pi)^6}\frac{1}{2}\sqrt{\frac{E_\mathbf{q}}{E_\mathbf{p}}}([\alpha_\mathbf{p},\alpha^\dagger_\mathbf{q}]e^{-ipx+iqy}-[\beta^\dagger_\mathbf{p},\beta_\mathbf{q}]e^{ipx-iqy})\\
&=i\int\frac{d^3p\;d^3q}{(2\pi)^6}\frac{1}{2}\sqrt{\frac{E_\mathbf{q}}{E_\mathbf{p}}}(e^{-ipx+iqy}+e^{ipx-iqy})(2\pi)^3\delta^{(3)}(\mathbf{p}-\mathbf{q})\\
&=i\int\frac{d^3p}{(2\pi)^3}\frac{1}{2}(e^{-ip(x-y)}+e^{ip(x-y)})\\
&=i\delta^{(3)}(\mathbf{x}-\mathbf{y})\\
[\phi^\dagger(\mathbf{x},t),\pi^\dagger(\mathbf{y},t)]
&=i\delta^{(3)}(\mathbf{x}-\mathbf{y})
\end{align}
To calculate the Heisenberg equations of motion we start with
\begin{align}
\nabla\phi(x)
&=i\int\frac{d^3p}{(2\pi)^3\sqrt{2E_\mathbf{p}}}\mathbf{p}\left(\alpha_\mathbf{p}e^{-ipx}-\beta^\dagger_\mathbf{p} e^{ipx}\right)\\
\nabla\phi^\dagger(x)
&=i\int\frac{d^3p}{(2\pi)^3\sqrt{2E_\mathbf{p}}}\mathbf{p}\left(-\alpha^\dagger_\mathbf{p}e^{ipx}+\beta_\mathbf{p}e^{-ipx}\right)
\end{align}
and then
\begin{align}
i\dot{\phi}(x)=[\phi(x),H]
&=\left[\phi(x),\int d^3y(\pi^\dagger\pi+\nabla\phi^\dagger\cdot\nabla\phi+m^2\phi^\dagger\phi)\right]\\
&=\int d^3y\pi^\dagger(y)[\phi(x),\pi(y)]\\
&=i\pi^\dagger(x)\\
i\dot{\phi}^\dagger(x)=[\phi^\dagger(x),H]
&=\left[\phi(x),\int d^3y(\pi^\dagger\pi+\nabla\phi^\dagger\cdot\nabla\phi+m^2\phi^\dagger\phi)\right]\\
&=\int d^3y[\phi^\dagger(x),\pi^\dagger(y)]\pi(y)\\
&=i\pi(x)
\end{align}
and 
\begin{align}
i\dot{\pi}(x)=[\pi(x),H]
&=\left[\pi(x),\int d^3y(\pi^\dagger\pi+\nabla\phi^\dagger\cdot\nabla\phi+m^2\phi^\dagger\phi)\right]\\
&=\left[\pi(x),\int d^3y(\pi^\dagger\pi-\triangle\phi^\dagger\cdot\phi+m^2\phi^\dagger\phi)\right]\\
&=\int d^3y(-\triangle\phi^\dagger+m^2\phi^\dagger)[\pi(x),\phi(y)]\\
&=i(\triangle_x-m^2)\phi^\dagger(x)\\
i\dot{\pi}^\dagger(x)=[\pi^\dagger(x),H]
&=\left[\pi^\dagger(x),\int d^3y(\pi^\dagger\pi+\nabla\phi^\dagger\cdot\nabla\phi+m^2\phi^\dagger\phi)\right]\\
&=\left[\pi^\dagger(x),\int d^3y(\pi^\dagger\pi-\phi^\dagger\cdot\triangle\phi+m^2\phi^\dagger\phi)\right]\\
&=\int d^3y[\pi^\dagger(x),\phi^\dagger(y)](-\triangle\phi+m^2\phi)\\
&=i(\triangle_x-m^2)\phi(x)
\end{align}
resulting in
\begin{align}
i\dot\pi(x)\quad&\rightarrow\quad\ddot\phi^\dagger=(\triangle-m^2)\phi^\dagger\\
&\rightarrow\quad(\Box+m^2)\phi^\dagger=0\\
i\dot\pi^\dagger(x)\quad&\rightarrow\quad\ddot\phi=(\triangle-m^2)\phi\\
&\rightarrow\quad(\Box+m^2)\phi=0
\end{align}
\item
\item
\item
\end{enumerate}

\subsection{Problem 2.3 - Calculating $D(x-y)$}
As we are calculation the vacuum expectation value we need to get the $a^\dagger$'s to the right and the $a$'s to the left
\begin{align}
\phi(x)\phi(y)
&=\int\frac{d^3p}{(2\pi)^3}\frac{1}{\sqrt{2E_\mathbf{p}}}(a_\mathbf{p}e^{-ipx}+a^\dagger_\mathbf{p}e^{ipx})\int\frac{d^3q}{(2\pi)^3}\frac{1}{\sqrt{2E_\mathbf{q}}}(a_\mathbf{q}e^{-iqy}+a^\dagger_\mathbf{q}e^{iqy})\\
&=\iint\frac{d^3p}{(2\pi)^3}\frac{d^3q}{(2\pi)^3}\frac{1}{\sqrt{2E_\mathbf{q}}}\frac{1}{\sqrt{2E_\mathbf{p}}}(a_\mathbf{p}e^{-ipx}+a^\dagger_\mathbf{p}e^{ipx})(a_\mathbf{q}e^{-iqy}+a^\dagger_\mathbf{q}e^{iqy})\\
&=\iint\frac{d^3p}{(2\pi)^3}\frac{d^3q}{(2\pi)^3}\frac{1}{\sqrt{2E_\mathbf{q}}}\frac{1}{\sqrt{2E_\mathbf{p}}}(a_\mathbf{p}a_\mathbf{q}e^{-ipx-iqy}+a^\dagger_\mathbf{p}a_\mathbf{q}e^{ipx-iqy}+a_\mathbf{p}a^\dagger_\mathbf{q}e^{-ipx+iqy}+a^\dagger_\mathbf{p}a^\dagger_\mathbf{q}e^{ipx+iqy})\\
&=\iint\frac{d^3p\,d^3q}{(2\pi)^6}\frac{1}{\sqrt{4E_\mathbf{q}E_\mathbf{p}}}(a_\mathbf{p}a_\mathbf{q}e^{-ipx-iqy}+(a_\mathbf{q}a^\dagger_\mathbf{p}-(2\pi)^3\delta(\mathbf{q}-\mathbf{p}))e^{ipx-iqy}+a_\mathbf{p}a^\dagger_\mathbf{q}e^{-ipx+iqy}+a^\dagger_\mathbf{p}a^\dagger_\mathbf{q}e^{ipx+iqy})
\end{align}
then with $a^\dagger|0\rangle=0$ and $\langle0|a=0$
\begin{align}
\langle0|\phi(x)\phi(y)|\rangle
&=\iint\frac{d^3p\,d^3q}{(2\pi)^6}\frac{1}{\sqrt{4E_\mathbf{q}E_\mathbf{p}}}((\langle0|a_\mathbf{q}a^\dagger_\mathbf{p}|0\rangle-\langle0|0\rangle(2\pi)^3\delta(\mathbf{q}-\mathbf{p}))e^{ipx-iqy}+\langle0|a_\mathbf{p}a^\dagger_\mathbf{q}|0\rangle e^{-ipx+iqy})\\
&=\iint\frac{d^3p\,d^3q}{(2\pi)^6}\frac{1}{\sqrt{4E_\mathbf{q}E_\mathbf{p}}}\left(\left(\frac{\langle\mathbf{q}|\mathbf{p}\rangle}{\sqrt{4E_\mathbf{p}E_\mathbf{q}}}-(2\pi)^3\delta(\mathbf{q}-\mathbf{p})\right)e^{ipx-iqy}+\frac{\langle\mathbf{p}|\mathbf{q}\rangle}{\sqrt{4E_\mathbf{p}E_\mathbf{q}}} e^{-ipx+iqy}\right)\\
&=\iint\frac{d^3p\,d^3q}{(2\pi)^6}\frac{1}{\sqrt{4E_\mathbf{q}E_\mathbf{p}}}\left(\left(\frac{2E_\mathbf{p}(2\pi)^3\delta^{(3)}(\mathbf{q}-\mathbf{p})}{\sqrt{4E_\mathbf{p}E_\mathbf{q}}}-(2\pi)^3\delta(\mathbf{q}-\mathbf{p})\right)e^{ipx-iqy}+\frac{2E_\mathbf{p}(2\pi)^3\delta^{(3)}(\mathbf{q}-\mathbf{p})}{\sqrt{4E_\mathbf{p}E_\mathbf{q}}} e^{-ipx+iqy}\right)\\
&=\int\frac{d^3p}{(2\pi)^3}\frac{1}{\sqrt{4E^2_\mathbf{p}}}\left(
\underbrace{\left(\frac{2E_\mathbf{p}}{\sqrt{4E^2_\mathbf{p}}}-1\right)}_{=0}e^{ipx-ipy}+\frac{2E_\mathbf{p}}{\sqrt{4E^2_\mathbf{p}}} e^{-ipx+ipy}\right)\\
&=\int\frac{d^3p}{(2\pi)^3}\frac{1}{2E_\mathbf{p}}e^{-ip(x-y)}
\end{align}
Now we can calculate with $x^0-y^0=0$ and $\mathbf{x}-\mathbf{y}=\mathbf{r}$
\begin{align}
D(x-y)
&=\int\frac{d^3p}{(2\pi)^3}\frac{1}{2E_\mathbf{p}}e^{-ip(x-y)}\\
&=\int\frac{d^3p}{(2\pi)^3}\frac{1}{2E_\mathbf{p}}e^{-i(E_p(x^0-y^0)-\mathbf{p}\cdot(\mathbf{x}-\mathbf{y}))}\\
&=\int\frac{d^3p}{(2\pi)^3}\frac{1}{2E_\mathbf{p}}e^{i\mathbf{p}\cdot(\mathbf{x}-\mathbf{y})}
\end{align}
transforming to spherical coordinates
\begin{align}
D(x-y)
&=2\pi\int_0^\infty\frac{p^2dp}{(2\pi)^3}\frac{1}{2\sqrt{p^2+m^2}}\int\sin\theta \,e^{ipr\cos\theta}d\theta\\
&=2\pi\int_0^\infty\frac{p^2dp}{(2\pi)^3}\frac{1}{2\sqrt{p^2+m^2}}\left[\frac{1}{(-ipr)}e^{ipr\cos\theta}\right]_0^\pi\\
&=2\pi\int_0^\infty\frac{p^2dp}{(2\pi)^3}\frac{1}{2\sqrt{p^2+m^2}}\frac{1}{(-ipr)}(e^{-ipr}-e^{ipr})\\
&=\frac{i}{2(2\pi)^2r}\int_0^\infty\frac{p\,dp}{\sqrt{p^2+m^2}}(e^{-ipr}-e^{ipr})\\
&=\frac{i}{2(2\pi)^2r}\left(\int_0^\infty\frac{p\,dp}{\sqrt{p^2+m^2}}e^{-ipr}-\int_0^\infty\frac{p\,dp}{\sqrt{p^2+m^2}}e^{ipr}\right)\\
&=\frac{i}{2(2\pi)^2r}\left(\int_0^\infty\frac{p\,dp}{\sqrt{p^2+m^2}}e^{-ipr}-\int_0^{-\infty}\frac{(-p)\,(-dp)}{\sqrt{(-p)^2+m^2}}e^{i(-p)r}\right)\\
&=\frac{i}{2(2\pi)^2r}\int_{-\infty}^\infty\frac{p\,dp}{\sqrt{p^2+m^2}}e^{-ipr}\\
&=\frac{-i}{2(2\pi)^2r}\int_{-\infty}^\infty\frac{p\,dp}{\sqrt{p^2+m^2}}e^{ipr}\qquad (r\rightarrow-r)
\end{align}
Let's use contour integration (closing the contour above - $\lim_{p\rightarrow i\infty}e^{ipr}=e^{-\infty r}=0$ so the upper half circle integral vanishes). Furthermore we see that the square root becomes zero at $\pm im$. 


\subsection{Problem 3.1 - Lorentz group}
With the Lie algebra for the six generators ($J^{01}, J^{02}, J^{03}, J^{12}, J^{13}, J^{12}$ - three boosts and three rotations) are  given by
\begin{align}
[J^{\mu\nu},J^{\rho\sigma}]=i(g^{\nu\rho}J^{\mu\sigma}-g^{\mu\rho}J^{\nu\sigma}-g^{\nu\sigma}J^{\mu\rho}+g^{\mu\sigma}J^{\nu\rho})
\end{align}
and
\begin{align}
L^i=\frac{1}{2}\epsilon^{ijk}J^{jk},\qquad K^i=J^{0i}
\end{align}
\begin{enumerate}[(a)]
\item We start with calculating $[L^a,L^b]$, $[K^a,K^b]$ and $[L^a,K^b]$.
Using $g^{kl}=-\delta^{kl}$ where $k=1,2,3$
\begin{align}
[L^a,L^b]&=\frac{1}{4}[\epsilon^{ajk}J^{jk},\epsilon^{blm}J^{lm}]\\
&=\frac{1}{4}\epsilon^{ajk}\epsilon^{blm}[J^{jk},J^{lm}]\\
&=\frac{i}{4}\epsilon^{ajk}\epsilon^{blm}(g^{kl}J^{jm}-g^{jl}J^{km}-g^{km}J^{jl}+g^{jm}J^{kl})\\
&=-\frac{i}{4}(\epsilon^{ajk}\epsilon^{bkm}J^{jm}-\epsilon^{ajk}\epsilon^{bjm}J^{km}-\epsilon^{ajk}\epsilon^{blk}J^{jl}+\epsilon^{ajk}\epsilon^{blj}J^{kl})\\
&=-\frac{i}{4}(-\epsilon^{ajk}\epsilon^{bmk}J^{jm}-\epsilon^{akj}\epsilon^{bmj}J^{km}-\epsilon^{ajk}\epsilon^{blk}J^{jl}-\epsilon^{akj}\epsilon^{blj}J^{kl})
\end{align}
and use $\epsilon_{abk}\epsilon^{cdk}=\delta^c_a\delta^d_b-\delta^d_a\delta^c_b$
\begin{align}
[L^a,L^b]
&=-\frac{i}{4}[
-(\delta_{ab}\delta_{jm}-\delta_{am}\delta_{jb})J^{jm}
-(\delta_{ab}\delta_{km}-\delta_{am}\delta_{kb})J^{km}
-(\delta_{ab}\delta_{jl}-\delta_{al}\delta_{jb})J^{jl}
-(\delta_{ab}\delta_{kl}-\delta_{al}\delta_{kb})J^{kl})\\
&=-\frac{i}{4}\left[
-(\delta_{ab}J^{mm}-J^{ba})
-(\delta_{ab}J^{mm}-J^{ba})
-(\delta_{ab}J^{ll}-J^{ba})
-(\delta_{ab}J^{ll}-J^{ba})\right]
\end{align}
as the diagonal elements of $J$ are zero the trace $J^{mm}$ vanishes as well and we obtain
\begin{align}
[L^a,L^b]
&=-iJ^{ba}=iJ^{ab}=i\frac{1}{2}(J^{ab}-J^{ba})\\
&=\frac{i}{2}(\delta_{am}\delta_{bn}-\delta_{an}\delta_{bm})J^{mn}\\
&=\frac{i}{2}\epsilon_{abk}\epsilon^{mnk}J^{mn}\\
&=\frac{i}{2}\epsilon_{abk}\epsilon^{mnk}J^{mn}\\
&=i\epsilon_{abk}\frac{1}{2}\epsilon^{mnk}J^{mn}\\
&=i\epsilon_{abk}\frac{1}{2}\epsilon^{kmn}J^{mn}\\
&=i\epsilon_{abk}L^k.
\end{align}
Now with $a,b=1,2,3$
\begin{align}
[K^a,K^b]
&=[J^{0a},J^{0b}]\\
&=i(g^{a0}J^{0b}-g^{00}J^{ab}-g^{ab}J^{00}+g^{0b}J^{a0})\\
&=i(0\cdot J^{0b}-1\cdot J^{ab}-0\cdot J^{00}+0\cdot J^{a0})\\
&=-iJ^{ab}\\
&=...\qquad\text{(same as last calculation above)}\\
&=-i\epsilon_{abk}L^k
\end{align}
And
\begin{align}
[L^a,K^b]
&=\frac{1}{2}\epsilon^{ajk}[J^{jk},J^{0b}]\\
&=\frac{i}{2}\epsilon^{ajk}(g^{k0}J^{jb}-g^{j0}J^{kb}-g^{kb}J^{j0}+g^{jb}J^{k0})\\
&=\frac{i}{2}\epsilon^{ajk}\left(0\cdot J^{jb}-0\cdot J^{kb}-g^{kb}\cdot(-K^j)+g^{jb}\cdot(-K^k)\right)\\
&=\frac{i}{2}\left(+\epsilon^{ajb}(-1)K^j-\epsilon^{abk}(-1)K^k\right)\\
&=\frac{i}{2}\left(-\epsilon^{abj}(-1)K^j-\epsilon^{abk}(-1)K^k\right)\\
&=i\epsilon^{abj}K^j
\end{align}
Now we can finally calculate 
\begin{align}
[J_+^a,J_+^b]&=\frac{1}{4}\left([L^a,L^b]+i[L^a,K^b]+i[K^a,L^b]+i^2[K^a,K^b]\right)\\
&=\frac{1}{4}\left(i\epsilon^{abk}L^k+i\cdot i\epsilon^{abj}K^j+i\cdot i\epsilon^{abj}K^j-(-1)i\epsilon^{abk}L^k\right)\\
&=\frac{1}{4}\left(i\epsilon^{abk}L^k-\epsilon^{abj}K^j-\epsilon^{abj}K^j+i\epsilon^{abk}L^k\right)\\
&=\frac{1}{2}i\epsilon^{abk}(L^k+iK^k)\\
&=i\epsilon^{abk}J_+^k
\end{align}
and
\begin{align}
[J_-^a,J_-^b]&=\frac{1}{4}\left([L^a,L^b]-i[L^a,K^b]-i[K^a,L^b]+i^2[K^a,K^b]\right)\\
&=\\
[J_-^a,J_+^b]&=\frac{1}{4}\left([L^a,L^b]-i[L^a,K^b]-i[K^a,L^b]-i^2[K^a,K^b]\right)\\
&=
\end{align}

\end{enumerate}


\section{{\sc Schwartz} - Quantum Field Theory and the Standard Model}
\subsection{Problem 2.2 Special relativity and colliders}
\begin{enumerate}
    \item Quick special relativity recap
    \begin{align}
        p'^\mu&=\Lambda^\mu_\nu p^\nu\quad p^\mu p_\mu=m^2c^2
    \end{align}
    At rest
    \begin{align}
        p^\mu p_\mu&=(p^0)^2-\vec{p}^2=(p^0)^2=m^2c^2
    \end{align}
    After Lorentz trafo in $x$ direction
    \begin{align}
        \Lambda=\begin{pmatrix}
        \gamma & -\beta\gamma & 0 & 0\\
        -\beta\gamma & \gamma & 0 & 0\\
        0 & 0 & 1 & 0\\
        0 & 0 & 0 & 1\\
        \end{pmatrix}
    \end{align}
    \begin{align}
        p'^\mu&=(\gamma p^0,-\beta\gamma p^0,0,0)\\
        &\equiv\left(\frac{E}{c},\vec{p}\right)
    \end{align}
    with $p^\mu p_\mu=m^2c^2$ we have $E^2/c^2+\vec{p}^2=m^2c^2$.
    
    Now we can solve the problem
    \begin{align}
        \frac{E_{cm}}{2}&=\sqrt{m_p^2c^4+p^2c^2}\\
        &\rightarrow p = \frac{1}{c}\sqrt{\frac{E_{cm}^2}{4}-m_p^2c^4}\equiv\beta\gamma m_pc\\
        &\rightarrow \frac{E_{cm}^2}{4}=m_p^2c^4(\beta^2\gamma^2+1)\\
        &\rightarrow \gamma=\frac{E_{cm}}{2m_pc^2}\\
        &\rightarrow\beta=\sqrt{1-\left(\frac{2m_pc}{E_{cm}}\right)^2}\approx1-\frac{1}{2}\left(\frac{2m_pc^2}{E_{cm}}\right)^2\\
        &\rightarrow c-v=2\left(\frac{m_pc^2}{E_{cm}}\right)^2c=2.69\text{m/s}
    \end{align}
    \item Using the velocity addition formula
    \begin{align}
        \Delta v=\frac{2v}{1+\frac{v^2}{c^2}}\approx c\left(1-2\left[\frac{m_pc^2}{E_{cm}}\right]^4\right)
    \end{align}
\end{enumerate}


\subsection{Problem 2.3 GZK bound}
\begin{enumerate}
    \item We are utilizing Plancks law
    \begin{align}
        w_\nu d\nu = \frac{8\pi h\nu^3}{c^3}\frac{d\nu}{e^{h\nu/k_BT}-1}
    \end{align}
    where the spectral energy density $w_\nu$ [J m$^{-3}$ s] gives the spacial energy density per frequency interval $d\nu$. The total radiative energy density is then given by
    \begin{align}
        \rho_\text{rad} &= \frac{8\pi h}{c^3}\int_0\infty \frac{\nu^3d\nu}{e^{h\nu/k_BT}-1}\\
        &=\frac{8\pi h}{c^3}\cdot\frac{(\pi k_B T)^4}{15h^4}\\
        &=\frac{8\pi^5k_B^4 T^4}{15h^3 c^3}=0.26\text{MeV/m}^3.
    \end{align}
    The photon density is given by
    \begin{align}
        n_\text{rad} &=\int_0^\infty\frac{w_\nu}{h\nu}d\nu\\
        &= \frac{8\pi}{c^3}\int \frac{\nu^2d\nu}{e^{h\nu/k_BT}-1}\\
        &=\frac{8\pi}{c^3}\cdot\frac{2\zeta(3) k_B^3 T^3}{h^3}\\
        &=\frac{16\pi\zeta(3) k_B^3 T^3}{h^3c^3}=416\text{cm}^{-3}.
    \end{align}
    The average photon energy is then given by
    \begin{align}
        E_\text{ph}&=\frac{\rho_\text{rad}}{n_\text{rad}}=\frac{\pi^4}{30\zeta(3)}k_BT=0.63\text{meV}\\
        \lambda_\text{ph}&=\frac{hc}{E_\text{ph}}=1.9\text{mm}
    \end{align}
    therefore it is called CM(icrowave)B.
    One obtains slightly other values if the peak of the Planck spectrum is used as definition of the average photon energy.
    \item In the center-of-mass system the total momentum before and after the collision vanishes
    \begin{align}
        \vec{p}_{p^+}^{cm}+\vec{p}_\gamma^{cm}=0=\vec{\hat{p}}_{p^+}^{cm}+\vec{\hat{p}}_{\pi^0}^{cm}.
    \end{align}
    which implies for (Lorentz-invariant) norm the systems 4-momentum $P^{cm}=p_{p^+}^{cm}+p_{\pi^0}^{cm}$
    \begin{align}
        \left(P^{cm}\right)^2&=(E_{p^+}^{cm}+E_\gamma^{cm})^2-c^2(\vec{p}_{p^+}^{cm}+\vec{p}_\gamma^{cm})^2\\
        &=(E_{p^+}^{cm}+E_\gamma^{cm})^2\\
        &=(E^{cm})^2\\
        &\overset{!}{=}(E_{p^+}+E_\gamma)^2-c^2(\vec{p}_{p^+}+\vec{p}_\gamma)^2\\
        &\overset{!}{=}(\hat{E}_{p^+}+\hat{E}_{\pi^0})^2-c^2(\vec{\hat{p}}_{p^+}+\vec{\hat{p}}_{\pi^0})^2
    \end{align}
    with $p^i=\hbar k^i=\hbar(\omega,\vec{k})=\hbar(\omega,\frac{2\pi}{\lambda}\vec{e}_k)=h(\nu,\frac{\nu}{c}\vec{e}_k)$ and the values before
    \begin{align}
        E_{p^+}&=m_{p^+}c^2+T_{p^+}\\
        E_\gamma&=h\nu\\  
        (\vec{p}_{p^+})^2
        &=\frac{1}{c^2}\left[(E_{p^+})^2-(m_{p^+})^2c^4\right]\\
        &=\frac{T_{p^+}}{c^2}\left[T_{p^+}+2m_{p^+}c^2\right]\\
        (\vec{p}_\gamma)^2&=\frac{h^2\nu^2}{c^2}
    \end{align}
    At the threshold the $\pi^0$ is created without any kinetic energy. As the total momentum is vanishing the proton
    also needs to be at rest
    \begin{align}
        (E_{p^+}+E_\gamma)^2-c^2(\vec{p}_{p^+}+\vec{p}_\gamma)^2&=\left(m_{p^+}c^2+m_{\pi^0}c^2\right)^2\\
        E_{p^+}^2+2E_{p^+}E_\gamma+E_\gamma^2-c^2\left(\vec{p}_{p^+}^2+\vec{p}_\gamma^2-2\vec{p}_{p^+}\cdot\vec{p}_\gamma\right)&=\left(m_{p^+}c^2+m_{\pi^0}c^2\right)^2\\
        m_{p^+}^2c^4+2E_{p^+}E_\gamma+2c^2\vec{p}_{p^+}\cdot\vec{p}_\gamma&=\left(m_{p^+}c^2+m_{\pi^0}c^2\right)^2\\
        m_{p^+}^2c^4+2E_{p^+}E_\gamma+2E_\gamma\sqrt{E_{p^+}^2-m_{p^+}^2c^2}\cos{\phi}&=\left(m_{p^+}c^2+m_{\pi^0}c^2\right)^2\\
        E_{p^+}E_\gamma+E_\gamma\sqrt{E_{p^+}^2-m_{p^+}^2c^2}\cos{\phi}&=\left(m_{p^+}+\frac{m_{\pi^0}}{2}\right)m_{\pi^0}c^4
    \end{align}
    Now we can square the equation and solve approximately assuming $E_\gamma\ll m_{p^+}c^2$
    \begin{align}
        E_\gamma\sqrt{E_{p^+}^2-m_{p^+}^2c^2}\cos{\phi}&=\left(m_{p^+}+\frac{m_{\pi^0}}{2}\right)m_{\pi^0}c^4-E_{p^+}E_\gamma\\
        E_\gamma^2\left(E_{p^+}^2-m_{p^+}^2c^2\right)\cos^2{\phi}
        &=\left(m_{p^+}+\frac{m_{\pi^0}}{2}\right)^2m_{\pi^0}^2c^8+(E_{p^+}E_\gamma)^2-2E_{p^+}E_\gamma\left(m_{p^+}+\frac{m_{\pi^0}}{2}\right)m_{\pi^0}c^4\\
        -E_\gamma^2m_{p^+}^2c^2\cos^2{\phi}
        &=\left(m_{p^+}+\frac{m_{\pi^0}}{2}\right)^2m_{\pi^0}^2c^8-2E_{p^+}E_\gamma\left(m_{p^+}+\frac{m_{\pi^0}}{2}\right)m_{\pi^0}c^4\\
        E_{p^+}&\approx\frac{\left(m_{p^+}+m_{\pi^0}/2\right)m_{\pi^0}c^4}{2E_\gamma}\\
        &=10.8\cdot 10^{19}\text{eV}
    \end{align}
    \item By assumption the $p^+$ and the $\pi^0$ would rest in the CM system
    \begin{align}
        (P^\mu)^{cm}&=(p^\mu_{p^+})^{cm}+(p^\mu_{\pi^0})^{cm}\\
                    &=\left([m_{p^+}+m_{\pi^0}]c^2,\vec{0}\right)\\
                    &=\Lambda^\mu_\alpha\left[\hat p^\alpha_{p^+}+\hat p^\alpha_{\pi^0}\right]\\
                    &=\Lambda^\mu_\alpha\left[ p^\alpha_{p^+}+ p^\alpha_{\gamma}\right]\\
    \end{align}
    We can therefore calculate $\gamma$
    \begin{align}
        \mu=1:\quad0&=\underbrace{\Lambda^1_0}_{-\gamma\beta}(E_{p^+}+E_\gamma)+\underbrace{\Lambda^1_1}_{\gamma}c(p^x_{p^+}+p^x_\gamma)\\
        &=-\gamma\beta(E_{p^+}+E_\gamma)+\gamma\left(\sqrt{E_{p^+}^2-m_p^2c^4}+E_\gamma\right)\\
        &\rightarrow\beta=\frac{\sqrt{E_{p^+}^2-m_p^2c^4}+E_\gamma}{E_{p^+}+E_\gamma}\approx\frac{\sqrt{E_{p^+}^2-m_p^2c^4}}{E_{p^+}}\\
        &\rightarrow\gamma=\frac{1}{\sqrt{1-\beta^2}}=\frac{E_{p^+}}{m_{p^+}c^2}
    \end{align}
    which can be used to calculate the pion momentum
    \begin{align}
        c\hat{p}_{\pi^0}&=\Lambda^0_\mu (p_{\pi^0}^\mu)^{cm}\\
        &=\Lambda^0_0 (p_{\pi^0}^0)^{cm}\\
        &=\gamma m_{\pi^0}c^2\\
        &=E_{p^+}\frac{m_{\pi^0}}{m_{p^+}}.
    \end{align}
    The $p+$ energy after the collision is then given by 
    \begin{align}
        E_{p^+}+E_\gamma&=\hat E_{p^+}+\hat E_{\pi^0}\\
        \rightarrow\hat E_{p^+}&=E_{p^+}+E_\gamma - \hat E_{\pi^0}\\
        &=E_{p^+}+E_\gamma - \sqrt{m_{\pi^0}^2c^4+\hat{\vec{p}}_{\pi^0}c^2}\\
        &=E_{p^+}+E_\gamma - \sqrt{m_{\pi^0}^2c^4+E_{p^+}^2\frac{m_{\pi^0}^2}{m_{p^+}^2}}\\
        &=E_{p^+}+E_\gamma - m_{\pi^0}c^2\sqrt{1+\frac{E_{p^+}^2}{m_{p^+}^2c^4}}\\
        &\approx E_{p^+} - m_{\pi^0}c^2\frac{E_{p^+}}{m_{p^+}c^2}\\
        &=E_{p^+}\left(1-\frac{m_{\pi^0}}{m_{p^+}}\right)\\
        &\approx0.85\cdot E_{p^+}.
    \end{align}
\end{enumerate}

\subsection{Problem 2.5 Compton scattering}
\begin{enumerate}
    \item the binding energy of outer(!!!) electrons is in the eV range while typical X-rays energies are in the keV range.
    \item In the nonrelativistic case we have energy and momentum conservation
    \begin{align}
        \frac{hc}{\lambda}&=\frac{hc}{\lambda'}+\frac{1}{2}m_ev^2\\
        \frac{h}{\lambda}&=\frac{h}{\lambda'}\cos\theta+m_ev\cos\phi\\
        0&=\frac{h}{\lambda'}\sin\theta+m_ev\sin\phi
    \end{align}
    then we see
    \begin{align}
        v=\sqrt{\frac{2hc}{m_e}\left(\frac{1}{\lambda}-\frac{1}{\lambda'}\right)}
        =\sqrt{\frac{2hc}{m_e}\frac{\lambda'-\lambda}{\lambda\lambda'}}
    \end{align}
    and
    \begin{align}
        \sin\phi&=-\frac{h}{m_ev}\frac{1}{\lambda'}\sin\theta\\
        \cos\phi&=\frac{h}{m_ev}\frac{1}{\lambda'}\left(\frac{\lambda'}{\lambda}-\cos\theta\right)\\
        \rightarrow1&=\sin^2\phi+\cos^2\phi\\
        &=\frac{h^2}{m_e^2v^2\lambda'^2}\left(\sin^2\theta+\frac{\lambda'^2}{\lambda^2}-2\frac{\lambda'}{\lambda}\cos\theta+\cos^2\theta\right)\\
        &=\frac{h^2}{m_e^2v^2\lambda'^2}\left(1+\frac{\lambda'^2}{\lambda^2}-2\frac{\lambda'}{\lambda}\cos\theta\right)\\
        &=\frac{h\lambda}{2m_ec\lambda'(\lambda'-\lambda)}\left(1+\frac{\lambda'^2}{\lambda^2}-2\frac{\lambda'}{\lambda}\cos\theta\right)\\
        &=\frac{h}{2m_ec(\lambda'-\lambda)}\left(\frac{\lambda}{\lambda'}+\frac{\lambda'}{\lambda}-2\cos\theta\right)\\
        \lambda'-\lambda&\approx\frac{h}{m_ec}\left(1-\cos\theta\right)
    \end{align}
    where we used $\lambda\approx\lambda'$.
    \item 
\end{enumerate}
    
\subsection{Problem 2.6 Lorentz invariance}
\begin{enumerate}
    \item With $\omega_k=\sqrt{\vec{k}^2+m^2}$
    \begin{align}
        \int_{-\infty}^\infty dk^0\delta(k^2-m^2)\theta(k^0)
        &=\int_{-\infty}^\infty dk^0\delta({k^0}^2-[\vec{k}^2+m^2])\theta(k^0)\\
        &=\frac{\theta(\omega_k)}{2\omega_k}+\frac{\theta(-\omega_k)}{2\omega_k}\\
        &=\frac{1}{2\omega_k}
    \end{align}
    \item Under Lorentz transformations we have $k^2-m^2=0$. For orthochronous transformation we have $k^0 ...$
    \item Now we can put it all together
    \begin{align}
        \int d^4k\delta(k^2-m^2)\theta(k^0)
        &=\int d^3k\int dk^0\delta(k^2-m^2)\theta(k^0)\\
        &=\int\frac{d^3k}{2\omega_k}
    \end{align}
\end{enumerate}

\subsection{Problem 2.7 Coherent states}
\begin{enumerate}
\item 
\begin{align}
    \partial_z\left(e^{-za^\dagger} a e^{-za^\dagger}\right)
    &=-e^{-za^\dagger}a^\dagger a e^{-za^\dagger}+e^{-za^\dagger} a a^\dagger e^{-za^\dagger}\\
    &=e^{-za^\dagger} [a,a^\dagger] e^{-za^\dagger}\\
    &=1
\end{align}
\item Rolling the $a$ through the $(a^\dagger)^k$ using the commutator $[a,a^\dagger]=1$
\begin{align}
    a|z\rangle
    &=a e^{za^\dagger}|0\rangle\\
    &=a\sum_{k=0}\frac{1}{k!}z^k(a^\dagger)^k|0\rangle\\
    &=a|0\rangle+\sum_{k=1}\frac{k}{k!}z^k(a^\dagger)^{k-1}|0\rangle\\
    &=z\sum_{n=0}\frac{1}{n!}z^n(a^\dagger)^{n}|0\rangle\\
    &=z|z\rangle
\end{align}
\item With $a^\dagger|n\rangle=\sqrt{n+1}|n+1\rangle$ and using the $|z\rangle$ is an eigenstate of $a$ we have
\begin{align}
    \langle n|z\rangle&=\frac{1}{\sqrt{n!}}\langle0|a^n|z\rangle
    =\frac{z^n}{\sqrt{n!}}\langle0|z\rangle
    =\frac{z^n}{\sqrt{n!}}\langle0|e^{za^\dagger}|0\rangle\\
    &=\frac{z^n}{\sqrt{n!}}\langle0|1+za^\dagger+\frac{1}{2}z^2(a^\dagger)^2+...|0\rangle\\
    &=\frac{z^n}{\sqrt{n!}}\langle0|0\rangle
    =\frac{z^n}{\sqrt{n!}}
\end{align}
where we used $\langle0|a^\dagger=0$.
\item With
\begin{align}
    a+a^\dagger&=\sqrt{\frac{m\omega}{2}}2q\quad\rightarrow\quad q=\frac{1}{\sqrt{2m\omega}}(a+a^\dagger)\\
    a-a^\dagger&=\sqrt{\frac{m\omega}{2}}2\frac{ip}{m\omega}\quad\rightarrow\quad p=-i\frac{\sqrt{m\omega}}{\sqrt{2}}(a-a^\dagger)
\end{align}
and $a|z\rangle=z|z\rangle$ and $\langle z|a^\dagger=\bar{z}\langle z|$ 
\begin{align}
    \langle z|q|z\rangle
    &=\frac{1}{\sqrt{2m\omega}}\langle z|a+a^\dagger|z\rangle
    =\frac{1}{\sqrt{2m\omega}}\langle z|z\rangle(z+\bar{z})\\
    \langle z|p|z\rangle&=-i\frac{\sqrt{m\omega}}{\sqrt{2}}\langle z|a-a^\dagger|z\rangle=
    -i\frac{\sqrt{m\omega}}{\sqrt{2}}\langle z|z\rangle(z-\bar{z})\\
    \langle z|q^2|z\rangle
    &=\frac{1}{2m\omega}\langle z|aa+\underbrace{aa^\dagger}_{=1+a^\dagger a}+a^\dagger a+a^\dagger a^\dagger|z\rangle\\
    &=\frac{1}{2m\omega}\langle z|z\rangle\left(z^2+1+2z\bar{z}+\bar{z}^2\right)\\
    \langle z|p^2|z\rangle
    &=-\frac{m\omega}{2}\langle z|aa-\underbrace{aa^\dagger}_{=1+a^\dagger a}-a^\dagger a+a^\dagger a^\dagger|z\rangle\\
    &=-\frac{m\omega}{2}\langle z|z\rangle\left(z^2-1-2z\bar{z}+\bar{z}^2\right)
\end{align}
Therefore
\begin{align}
    \Delta q^2
    &=\langle q^2\rangle-\langle q\rangle^2\\
    &=\frac{1}{2m\omega}\left(z^2+1+2z\bar{z}+\bar{z}^2\right)-\left(\frac{1}{\sqrt{2m\omega}}(z+\bar{z})\right)^2\\
    &=\frac{1}{2m\omega}
\end{align}
and
\begin{align}
    \Delta p^2
    &=\langle p^2\rangle-\langle p\rangle^2\\
    &=-\frac{m\omega}{2}\left(z^2-1-2z\bar{z}+\bar{z}^2\right)-\left(-i\frac{\sqrt{m\omega}}{\sqrt{2}}(z-\bar{z})\right)^2\\
    &=\frac{m\omega}{2}
\end{align}
which means
\begin{align}
    \Delta p\Delta q=\frac{1}{\sqrt{2m\omega}}\frac{\sqrt{m\omega}}{\sqrt{2}}=\frac{1}{2}.
\end{align}
\item At first let's construct the eigenstate $|w\rangle$ for $a$ manually
\begin{align}
    a|w\rangle=c_w|w\rangle
\end{align}
Expanding the eigenstate with $a^\dagger|n\rangle=\sqrt{n+1}|n+1\rangle$
\begin{align}
    |w\rangle
    &=\sum_n\alpha_n|n\rangle\\
    a|w\rangle
    &=\sum_n\alpha_n\sqrt{n}|n-1\rangle\overset{!}{=}c_w\sum_n\alpha_n|n\rangle=c_w|n\rangle\\
    &\rightarrow\alpha_n\sqrt{n}=c_w\alpha_{n-1}\\
    &\rightarrow\alpha_n=\frac{c_w}{\sqrt{n}}\alpha_{n-1}\\
    |w\rangle
    &=\sum_n\alpha_0\frac{c_w^n}{\sqrt{n!}}|n\rangle
    =\alpha_0\sum_n\frac{c_w^n}{n!}(a^\dagger)^n|0\rangle
    =\alpha_0 e^{c_wa^\dagger}|0\rangle
\end{align}
Now we do the same for $a^\dagger$
\begin{align}
    a^\dagger|v\rangle=c_v|v\rangle
\end{align}
Expanding the eigenstate 
\begin{align}
    |v\rangle
    &=\sum_n\beta_n|n\rangle\\
    a^\dagger|v\rangle
    &=\sum_n\beta_n\sqrt{n+1}|n+1\rangle\overset{!}{=}c_v\sum_n\beta_n|n\rangle=c_v|n\rangle\\
    &\rightarrow\beta_{n}\sqrt{n+1}=c_v\beta_{n+1}\\
    &\rightarrow\beta_{n+1}=\frac{\sqrt{n+1}}{c_v}\beta_{n}\\
    |v\rangle
    &=\sum_n\beta_0\frac{\sqrt{n!}}{c_v^n}|n\rangle=\beta_0\sum_n\frac{1}{c_v^n}(a^\dagger)^n|0\rangle
\end{align}
Now we calculate with $\langle 0|a^\dagger=0$
\begin{align}
    \langle 0|a^\dagger|v\rangle&=\beta_0\sum_n\frac{1}{c_v^n}\langle 0|(a^\dagger)^{n+1}|0\rangle\\
    &=\beta_0\frac{1}{c_v^0}\langle 0|a^\dagger|0\rangle\\
\end{align}
\end{enumerate}


\subsection{Problem 3.1 Higher order Lagrangian}
With the principle of least action
\begin{align}
\delta S=\delta\int\mathcal{L}d^4x=\int\delta\mathcal{L}d^4x
\end{align}
we calculate
\begin{align}
\delta\mathcal{L}
&=\frac{\partial \mathcal{L}}{\partial\phi}\delta\phi+
\frac{\partial \mathcal{L}}{\partial(\partial_\mu\phi)}\delta(\partial_\mu\phi)+
\frac{\partial \mathcal{L}}{\partial(\partial_\nu\partial_\mu\phi)}\delta(\partial_\nu\partial_\mu\phi)+...
\end{align}
Now we can integrate each term
\begin{align}
\delta\mathcal{L}_0&=\int \frac{\partial \mathcal{L}}{\partial\phi}\delta\phi d^4x\\
%
\delta\mathcal{L}_1&=\int \frac{\partial \mathcal{L}}{\partial(\partial_\mu\phi)}\delta(\partial_\mu\phi) d^4x
=\int \frac{\partial \mathcal{L}}{\partial(\partial_\mu\phi)}\partial_\mu\delta\phi d^4x\\
&=\left.\frac{\partial \mathcal{L}}{\partial(\partial_\mu\phi)}\delta\phi\right|_{\partial\Omega}-\int \partial_\mu\frac{\partial \mathcal{L}}{\partial(\partial_\mu\phi)}\delta\phi d^4x\\
%
\delta\mathcal{L}_2&=\int \frac{\partial \mathcal{L}}{\partial(\partial_\nu\partial_\mu\phi)}\delta(\partial_\nu\partial_\mu\phi) d^4x
=\int \frac{\partial \mathcal{L}}{\partial(\partial_\nu\partial_\mu\phi)}\partial_\nu\delta\partial_\mu\phi d^4x\\
&=\left.\frac{\partial \mathcal{L}}{\partial(\partial_\nu\partial_\mu\phi)}\delta\partial_\mu\phi\right|_{\partial\Omega}-\int \partial_\nu\frac{\partial \mathcal{L}}{\partial(\partial_\nu\partial_\mu\phi)}\delta\partial_\mu\phi d^4x\\
&=\left.\frac{\partial \mathcal{L}}{\partial(\partial_\nu\partial_\mu\phi)}\delta\partial_\mu\phi\right|_{\partial\Omega}-\left.\partial_\nu\frac{\partial \mathcal{L}}{\partial(\partial_\nu\partial_\mu\phi)}\delta\phi\right|_{\partial\Omega}+\int \partial_\mu\partial_\nu\frac{\partial \mathcal{L}}{\partial(\partial_\nu\partial_\mu\phi)}\delta\phi d^4x
\end{align}
Requiring that all derivatives vanish at infinity we obtain
\begin{align}
\delta S=\int d^4x\left(
\frac{\partial \mathcal{L}}{\partial\phi}
- \partial_\mu\frac{\partial \mathcal{L}}{\partial(\partial_\mu\phi)}
+\partial_\mu\partial_\nu\frac{\partial \mathcal{L}}{\partial(\partial_\nu\partial_\mu\phi)}-...
\right)\delta\phi
\end{align}
and therefore
\begin{align}
\frac{\partial \mathcal{L}}{\partial\phi}
- \partial_\mu\frac{\partial \mathcal{L}}{\partial(\partial_\mu\phi)}
+\partial_\mu\partial_\nu\frac{\partial \mathcal{L}}{\partial(\partial_\nu\partial_\mu\phi)}-...=0
\end{align}

\subsection{Problem 3.5 Spontaneous symmetry}
\begin{align}
\mathscr{L}=-\frac{1}{2}\phi\Box\phi+\frac{1}{2}m^2\phi^2-\frac{\lambda}{4!}\phi^4\\
    \frac{\partial\mathscr{L}}{\partial \phi}-\partial_\beta\frac{\partial\mathscr{L}}{\partial(\partial_\beta \phi)}+\partial_\mu\partial_\nu\frac{\partial \mathcal{L}}{\partial(\partial_\nu\partial_\mu\phi)}=0\\
    \rightarrow -\Box\phi+m^2\phi-\frac{\lambda}{3!}\phi^3=0
\end{align}
and the Hamiltonian with $-\phi\Box\phi\sim(\partial_\mu\phi)(\partial^\mu\phi)=\eta^{\mu\nu}\partial_\mu\phi\partial_\nu\phi$
\begin{align}
\pi
&=\frac{\partial\mathcal{L}}{\partial\dot\phi}\\
&=\dot\phi\\
\mathcal{H}
&=\pi\dot\phi-\mathcal{L}\\
&=(\dot\phi)^2-\mathcal{L}\\
&=\frac{1}{2}\dot\phi^2+\frac{1}{2}(\nabla\phi)^2-\frac{1}{2}m^2\phi^2+\frac{\lambda}{4!}\phi^4
\end{align}
\begin{enumerate}[label=(\alph*)]
\item 
\begin{align}
m^2\phi-\frac{\lambda}{3!}\phi^3=0\\
(m^2-\frac{\lambda}{3!}\phi^2)\phi&=0\\
\phi_0&=0\quad\rightarrow\quad\mathcal{H}[\phi]=0\\
\phi_{1,2}&=\pm\sqrt{\frac{3!}{\lambda}}m\quad\rightarrow\quad\mathcal{H}[\phi]=-\frac{3m^4}{2\lambda}
\end{align} 

\item
\item   
\end{enumerate}

\subsection{Problem 3.6 Yukawa potential}
\begin{enumerate}[label=(\alph*)]
\item We slit the Lagranian in three parts
\begin{align}
\mathscr{L}&=-\frac{1}{4}F_{\mu\nu}^2+\frac{1}{2}m^2A_\mu^2-A_\mu J_\mu\\
&=\mathscr{L}_F+\mathscr{L}_m+\mathscr{L}_J
\end{align}
with the Euler Lagrange equations
\begin{align}
        \frac{\partial\mathscr{L}}{\partial A_\alpha}-\partial_\beta\frac{\partial\mathscr{L}}{\partial(\partial_\beta A_\alpha)}=0
\end{align}
with
\begin{align}
    \frac{\partial(\partial_\mu A_\nu)}{\partial(\partial_\beta A_\alpha)}=\delta_{\mu\beta}\delta_{\nu\alpha}
\end{align}
we can calculate
\begin{align}
    \frac{\partial\mathscr{L}_m}{\partial A_\alpha}-\partial_\beta\frac{\partial\mathscr{L}_m}{\partial(\partial_\beta A_\alpha)}&=m^2A_\alpha\\
    \frac{\partial\mathscr{L}_J}{\partial A_\alpha}-\partial_\beta\frac{\partial\mathscr{L}_J}{\partial(\partial_\beta A_\alpha)}&=-J_\alpha\\
    \frac{\partial\mathscr{L}_F}{\partial A_\alpha}-\partial_\beta\frac{\partial\mathscr{L}_F}{\partial(\partial_\beta A_\alpha)}&=-\frac{1}{4}\partial_\beta\left(-2F_{\mu\nu}(\delta_{\mu\beta}\delta_{\nu\alpha}-\delta_{\nu\beta}\delta_{\mu\alpha})\right)\\
    &=\frac{1}{4}\partial_\beta\left(2(F_{\beta\alpha}-F_{\alpha\beta})\right)\\
    &=\partial_\beta F_{\beta\alpha}\\
    &=\partial_\beta\partial_\beta A_\alpha-\partial_\beta\partial_\alpha A_\beta
\end{align}
to obtain (the Proca equation)
\begin{align}
    \Box A_\alpha-\partial_\beta\partial_\alpha A_\beta+m^2A_\alpha-J_\alpha=0.
\end{align}
Now we can calculate the divergence of the equations
\begin{align}
    \partial_\alpha\left(\Box A_\alpha-\partial_\beta\partial_\alpha A_\beta+m^2A_\alpha-J_\alpha\right)=0.\\
    \Box \partial_\alpha A_\alpha-\partial_\alpha\partial_\alpha\partial_\beta A_\beta+m^2\partial_\alpha A_\alpha-\underbrace{\partial_\alpha J_\alpha}_{=0}=0
\end{align}
which implies $\partial_\alpha A_\alpha=0$ and therefore
\begin{align}
    \Box A_\alpha+m^2A_\alpha-J_\alpha=0.
\end{align}
\item For $A_0$ we have for a static potential
\begin{align}
    (\partial_{tt}-\triangle)A_0+m^2A_0-e\delta(x)=0\\
    -\triangle A_0+m^2A_0-e\delta(x)=0.
\end{align}
A Fourier transformation of the equation of motion yields
\begin{align}
    -(ik)^2 A_0(k)+m^2A_0(k)-e=0\\
    \rightarrow A_0(k)=\frac{e}{k^2+m^2}
\end{align}
which we can now transform back 
\begin{align}
    A_0
    &=\frac{e}{(2\pi)^3}\int d^3k\frac{e^{ikx}}{k^2+m^2}\\
    &=\frac{e}{4\pi r} e^{-mr}
\end{align}
where we used the integral evaluation from {\sc Kachelriess} Problem 3.5.
\item 
\begin{align}
    \lim_{m\rightarrow0}\frac{e}{4\pi r} e^{-mr}=\frac{e}{4\pi r}
\end{align}
\item Scaling down the Coulomb potential exponentially with a characteristic length of $1/m$.
\item
\item We can expand and the integrate each term by parts to move over the partial derivatives 
\begin{align}
    \mathscr{L}_F
    &=-\frac{1}{4}F_{\mu\nu}^2\\
    &=-\frac{1}{4}(\partial_\mu A_\nu-\partial_\nu A_\mu)(\partial_\mu A_\nu-\partial_\nu A_\mu)\\
    &=-\frac{1}{4}\left(\partial_\mu A_\nu \partial_\mu A_\nu-\partial_\mu A_\nu \partial_\nu A_\mu
        -\partial_\nu A_\mu \partial_\mu A_\nu + \partial_\nu A_\mu\partial_\nu A_\mu\right)\\
    &=-\frac{1}{2}\left(\partial_\mu A_\nu \partial_\mu A_\nu-\partial_\mu A_\nu \partial_\nu A_\mu\right)\\
    &=-\frac{1}{2}\left(-A_\nu\partial_\mu\partial_\mu A_\nu+ A_\nu \partial_\nu \partial_\mu A_\mu\right)\\
    &=\frac{1}{2}\left(A_\mu \Box A_\mu- A_\nu \partial_\nu \underbrace{\partial_\mu A_\mu}_{=0}\right)\\ 
    &=\frac{1}{2}A_\mu \Box A_\mu
\end{align}
We can plug this into the full Lagrangian (renaming the summation index)
\begin{align}
    \mathscr{L}&=\frac{1}{2}A_\mu \Box A_\mu+\frac{1}{2}m^2A_\mu^2-A_\mu J_\mu\\
    &=\frac{1}{2}A_\mu\left(\Box+m^2 \right)A_\mu-A_\mu J_\mu
\end{align} 
then we calculate the derivatives for the Euler-Lagrange equations up to second order (see problem 3.1)
\begin{align}
\frac{\partial\mathcal{L}}{\partial A_\mu}&=\frac{1}{2}\Box A_\mu+m^2A_\mu-J_\mu\\
\frac{\partial\mathcal{L}}{\partial (\partial_\alpha A_\mu)}&=0\\
\frac{\partial\mathcal{L}}{\partial (\partial_\alpha\partial_\alpha A_\mu)}&=\frac{1}{2}A_\mu
\end{align}
and get
\begin{align}
(\Box+m^2)A_\mu=J_\mu
\end{align}
   
\end{enumerate}

\subsection{Problem 3.7 Perihelion shift of Mercury by dimensional analysis - NOT DONE YET}
\begin{enumerate}[label=(\alph*)]
\item Lets summarize the rules of dimensional analysis 

\begin{center}
\begin{tabular}{ lccc } 
 \hline
 varible & SI unit &equation & natural unit \\ 
 \hline\hline
 $c$               & m/s    & -                        & 1        \\
 $\hbar$           & Js     & -                        & 1        \\
 Velocity          & m/s    & -                        & 1      \\
 mass              & kg     & $E=mc^2$                 & $E$      \\
 frequency         & 1/s    & $E=\hbar\omega$          & $E$      \\
 time              & s      & $t=2\pi/\omega$          & $E^{-1}$ \\
 length            & m      & $s=ct$                   & $E^{-1}$ \\
 $\partial_\mu$    & 1/m    & -                        & $E$      \\
 momentum          & kg\,m/s& $E=p^2/2m$               & $E$      \\
 action            & Js     & $S=Et$                   & 1        \\
 $\mathcal{L}$     & J/m$^3$& $S=\int d^4x\mathcal{L}$ & $E^4$    \\
 energy density    & J/m$^3$& $\rho=E/V$               & $E^4$    \\
 $T^{\mu\nu}$      & J/m$^3$& $\rho=E/V$               & $E^4$    \\
 \hline
\end{tabular}
\end{center}
Now we can do a dimensions count for each term
\begin{align}
\underbrace{\mathcal{L}}_{=4}&=-\frac{1}{2}\underbrace{h\Box h}_{2\cdot[h]+2}+ \underbrace{M_\text{Pl}^a h^2\Box h}_{=a+3\cdot[h]+2}-\underbrace{M_\text{Pl}^bhT}_{b+[h]+4}\\
&\rightarrow\quad[h]=1\\
&\rightarrow\quad a=-1\\
&\rightarrow\quad b=-1
\end{align} 
\item Deriving the equations of motions: keeping in mind that the Lagrangian contains second order derivatives with implies and extra term in the Euler-Lagrange equations (see problem 3.1)
\begin{align}
\mathcal{L}&=-\frac{1}{2}h\Box h+\frac{1}{M_\text{Pl}}h^2\Box h-\frac{1}{M_\text{Pl}}hT\\
\frac{\partial\mathcal{L}}{\partial h}&=-\frac{1}{2}\cdot\Box h+2\frac{1}{M_\text{Pl}}h\Box h-\frac{1}{M_\text{Pl}}T\\
\frac{\partial\mathcal{L}}{\partial(\partial h)}&=0\\
\frac{\partial\mathcal{L}}{\partial(\Box h)}&=-\frac{1}{2}h+\frac{1}{M_\text{Pl}}h^2\\
&\rightarrow\Box h=\frac{1}{M_\text{Pl}}\Box(h^2)+\frac{2}{M_\text{Pl}}h\Box h-\frac{1}{M_\text{Pl}}T
\end{align}
which show an extra term. Alternatively we can integrate the Lagrangian by parts (neglecting the boundary terms) and get
\begin{align}
\mathcal{L}&=\frac{1}{2}\partial h\partial h-\frac{1}{M_\text{Pl}}\partial(h^2)\partial h-\frac{1}{M_\text{Pl}}hT\\
\frac{\partial\mathcal{L}}{\partial h}&=-\frac{1}{M_\text{Pl}}T\\
\frac{\partial\mathcal{L}}{\partial(\partial h)}&=\Box h-\frac{1}{M_\text{Pl}}\Box(h^2)\\
&\rightarrow\Box h=\frac{1}{M_\text{Pl}}\Box(h^2)-\frac{1}{M_\text{Pl}}T
\end{align}
We now assume a solution of the form
\begin{align}
h&=h_0+\frac{1}{M_\text{Pl}}h_1+\frac{1}{M_\text{Pl}^2}h_2+...\\
\rightarrow h^2&=h_0^2+\frac{1}{M_\text{Pl}}2h_0h_1+\frac{1}{M_\text{Pl}^2}(2h_0h_2+h_1^2)+\frac{1}{M_\text{Pl}^3}(2h_1h_2+2h_0h_3)+...
\end{align}
and obtain (with the Coulomb solution 3.61 and 3.61)
\begin{align}
k=0:&\quad\Box h_0=0\quad\rightarrow\quad h_0=0\\
k=1:&\quad\Box h_1=\Box h_0^2-m\delta^{(3)}\\
    &\quad\Box h_1=-m\delta^{(3)}\quad\rightarrow\quad h_1=-\frac{m}{\Box}\delta^{(3)}=\frac{m}{\triangle}\delta^{(3)}=-\frac{m}{4\pi r}\\
k=2:&\quad\Box h_2=2\Box h_0h_1\quad\rightarrow\quad h_2=0\\
k=3:&\quad\Box h_3=\Box(2h_0h_2+h_1^2)\\
    &\quad\Box h_3=\Box(h_1^2)\quad\rightarrow\quad h_3=h_1^2=\frac{m^2}{16\pi^2r^2}
\end{align}
and therefore
\begin{align}
h&=-\frac{m}{4\pi r}\frac{1}{M_\text{Pl}}+\frac{m^2}{16\pi^2r^2}\frac{1}{M_\text{Pl}^3}\\
&=-\frac{m}{4\pi r}\sqrt{G_N}+\frac{m^2}{16\pi^2r^2}\sqrt{G_N^3}
\end{align}
\item 
The Newton potential is actually given by (and additional power of $M_\text{Pl}$ is missing and we are dropping the $4\pi$)
\begin{align}
V_N=h_1\frac{1}{M_\text{Pl}}\cdot\frac{1}{M_\text{Pl}}=-\frac{Gm_\text{Sun}}{r}
\end{align}
the virial theorem implies $E_\text{kin}\simeq E_\text{pot}$ and therefore
\begin{align}
\frac{1}{2}J\omega^2&\simeq \frac{G_N m_\text{Sun} m_\text{Mercury}}{R}\\
\frac{1}{2}m_\text{Mercury}R^2\omega^2&\simeq \frac{G_N m_\text{Sun} m_\text{Mercury}}{R}\\
\omega^2&\simeq\frac{G_Nm_\text{Sun}}{R^3}
\end{align}

\item 
\item 
\item 
\item 
\end{enumerate}

\subsection{Problem 3.9 - Photon polarizations}

\begin{enumerate}[label=(\alph*)]
\item Then using the results from problem 3.6 and the corrected sign in the Lagrangian we get
\begin{align}
-\frac{1}{4}(F_{\mu\nu})^2
    &=-\frac{1}{4}(\partial_\mu A_\nu-\partial_\nu A_\mu)(\partial_\mu A_\nu-\partial_\nu A_\mu)\\
    &=-\frac{1}{4}\left(\partial_\mu A_\nu \partial_\mu A_\nu-\partial_\mu A_\nu \partial_\nu A_\mu
        -\partial_\nu A_\mu \partial_\mu A_\nu + \partial_\nu A_\mu\partial_\nu A_\mu\right)\\
    &=-\frac{1}{2}\left(\partial_\mu A_\nu \partial_\mu A_\nu-\partial_\mu A_\nu \partial_\nu A_\mu\right)\\
    &=-\frac{1}{2}\left(-A_\nu\partial_\mu\partial_\mu A_\nu+ A_\nu \partial_\nu \partial_\mu A_\mu\right)\\
    &=\frac{1}{2}\left(A_\mu \Box A_\mu- A_\nu \partial_\nu \underbrace{\partial_\mu A_\mu}_{=0}\right)\\ 
    &=\frac{1}{2}A_\mu \Box A_\mu
\end{align}
and therefore
\begin{align}
\mathcal{L}&=-\frac{1}{4}(F_{\mu\nu})^2-J_\mu A_\mu\\
&=\frac{1}{2}A_\mu\Box A_\mu-J_\mu A_\mu\\
&=\frac{1}{2}A_\mu\Box A_\mu-(\Box A_\mu) A_\mu\\
&=-\frac{1}{2}A_\mu\Box A_\mu
\end{align}
The equations of motion are $\Box A_\mu=J_\mu$ which can be written in momentum space as $k^2A_\mu(k)=J_\mu(k)$. Now let's write the Lagranian in momentum space as well
\begin{align}
\mathcal{L}&=\int d^4k e^{ikx}A_\mu(k)k^2A_\mu(k)\\
&=\int d^4k e^{ikx}\frac{J_\mu(k)}{k^2}k^2\frac{J_\mu(k)}{k^2}\\
&=\int d^4k e^{ikx}J_\mu(k)\frac{1}{k^2}J_\mu(k)
\end{align} 
\item In momentum space charge conservation is given by
\begin{align}
  k_\mu J_\mu&=0\\
  \omega J_0-\kappa J_1&=0\\
  &\rightarrow\quad J_1=\frac{\omega}{\kappa}J_0
\end{align}

\item
\begin{align}
\mathcal{L}&=\int d^4k e^{ikx}J_\mu(k)\frac{1}{k^2}J_\mu(k)\\
&\simeq\frac{J_0^2-J_1^2-J_2^2-J_3^2}{\omega^2-\kappa^2}\\
&\simeq\frac{J_0^2(1-\omega^2/\kappa^2)}{\omega^2-\kappa^2}-\frac{J_2^2+J_3^2}{\omega^2-\kappa^2}\\
&\simeq-\frac{J_0^2}{\kappa^2}-\frac{J_2^2+J_3^2}{\omega^2-\kappa^2}\\
&\simeq\triangle J_0^2-\Box(J_2^2+J_3^2)
\end{align} 

\item A time derivative in the Lagrangian results in a time derivative in time derivative in the equations of motion which means a time-evolution equation. There are two causally propagating degrees of freedom $J_2$ and $J_3$.

\item Hmmmm .... calculate the two point field correlation functions and see if they vanish outside of the light cone.

\end{enumerate}


\subsection{Problem 3.10 - Graviton polarizations - NOT DONE YET}
\begin{enumerate}[label=(\alph*)]
\item With the higher order Euler-Lagrange equations from 3.1
\begin{align}
\frac{\partial \mathcal{L}}{\partial\phi}
- \partial_\mu\frac{\partial \mathcal{L}}{\partial(\partial_\mu\phi)}
+\partial_\mu\partial_\nu\frac{\partial \mathcal{L}}{\partial(\partial_\nu\partial_\mu\phi)}-...=0
\end{align}
we obtain
\begin{align}
-\frac{1}{2}\Box h_{\mu\nu}+\frac{1}{M_\text{Pl}}T_{\mu\nu}-\frac{1}{2}\Box h_{\mu\nu}&=0\\
\rightarrow \Box h_{\mu\nu}&=\frac{1}{M_\text{Pl}}T_{\mu\nu}\\
\rightarrow h_{\mu\nu}&=\frac{1}{M_\text{Pl}}\frac{1}{\Box}T_{\mu\nu}
\end{align}
and
\begin{align}
\mathcal{L}
&=-\frac{1}{2}h_{\mu\nu}\Box h_{\mu\nu}+\frac{1}{M_\text{Pl}}h_{\mu\nu}T_{\mu\nu}\\
&=-\frac{1}{2}\frac{1}{M_\text{Pl}^2}(\frac{1}{\Box} T_{\mu\nu})T_{\mu\nu}+\frac{1}{M_\text{Pl}^2}(\frac{1}{\Box} T_{\mu\nu})T_{\mu\nu}\\
&=\frac{1}{2}\frac{1}{M_\text{Pl}^2}T_{\mu\nu}\frac{1}{\Box} T_{\mu\nu}\\
&\simeq\frac{1}{2}\frac{1}{M_\text{Pl}^2}T_{\mu\nu}\frac{1}{k^2} T_{\mu\nu}\\
\end{align}
\item
\item
\item
\end{enumerate}

\section{{\sc Srednicki} - Quantum Field Theory}
\subsection{Problem 1.2 - Schroedinger equation}
\begin{align}
    H&=\int d^3x a^\dagger(x)\left(-\frac{\hbar^2}{2m}\triangle_x+V(x)\right)a(x)+\frac{1}{2}\int d^3xd^3yV(x-y)a^\dagger(x)a^\dagger(y)a(x)a(y)\\
    |\psi,t\rangle&=\int d^3x_1...d^3x_n\psi(x_1,...,x_n;t)a^\dagger(x_1)...a^\dagger(x_n)|0\rangle
\end{align}
\begin{enumerate}
    \item Bosons:
    With the commutations relation and $a|0\rangle=0$
    \begin{align}
        a(x)a^\dagger(x_1)...a^\dagger(x_n)|0\rangle
        &=\left(\delta^3(x-x_1)-a^\dagger(x_1)a(x)\right)...a^\dagger(x_n)|0\rangle\\
        &=\sum_{k=1}^n(-1)^{k-1} \delta^3(x-x_k)\underbrace{a^\dagger(x_1)...a^\dagger(x_n)}_{(n-1) \times a^\dagger}|0\rangle
    \end{align}
    and similar
    \begin{align}
        a(y)a(x)a^\dagger(x_1)...a^\dagger(x_n)|0\rangle
        &=\sum_{j\neq k}^n \delta^3(x-x_k)\delta^3(y-x_j)\underbrace{a^\dagger(x_1)...a^\dagger(x_n)}_{(n-2) \times a^\dagger}|0\rangle
    \end{align}
    we obtain
	\begin{align}
		i\hbar\frac{\partial}{\partial t}|\psi,t\rangle
		&=\int d^3x_1...d^3x_n\frac{\partial}{\partial t}\psi(x_1,...,x_n;t)a^\dagger(x_1)...a^\dagger(x_n)|0\rangle 
	\end{align}
    and
    \begin{align}
        H|\psi,t\rangle=&\sum_{k=1}^na^\dagger(x_k)\left(-\frac{\hbar^2}{2m}\triangle_{x_k}+V(x_k)\right)\psi(x_1,...,x_n;t)\underbrace{a^\dagger(x_1)...a^\dagger(x_n)}_{(n-1) \times a^\dagger}|0\rangle\\
        &+\frac{1}{2}\sum_{j\neq k}^nV(x_k-x_j)\psi(x_1,...,x_n;t)a^\dagger(x_k)a^\dagger(x_j)\underbrace{a^\dagger(x_1)...a^\dagger(x_n)}_{(n-2) \times a^\dagger}|0\rangle
    \end{align}
    \item Fermions:
\end{enumerate}

\subsection{Problem 1.3 - Commutator of the number operator}
Preliminary calculations (we use the boson commutation relations)
\begin{align}
a^\dagger(z)a(z)a^\dagger(x)&=a^\dagger(z)(\delta(x-z)+a^\dagger(x)a(z))\\
&=a^\dagger(z)\delta^3(x-z)+a^\dagger(z)a^\dagger(x)a(z)\\
&=a^\dagger(z)\delta^3(x-z)+a^\dagger(x)a^\dagger(z)a(z)
\end{align}
and
\begin{align}
a(x)a^\dagger(z)a(z)&=(\delta(x-z)+a^\dagger(z)a(x))a(z)\\
&=\delta^3(x-z)a(z)+a^\dagger(z)a(x)a(z)\\
&=\delta^3(x-z)a(z)+a^\dagger(z)a(z)a(x)
\end{align}
With
\begin{align}
N&=\int d^3z\; a^\dagger(z)a(z)\\
H&=H_1+H_\text{int}\\
&=\int d^3x\; a^\dagger(x)\left(-\frac{\hbar^2}{2m}\triangle_x+U(x)\right)a(x)+\frac{1}{2}\int d^3xd^3y\;V(x-y) a^\dagger(x)a^\dagger(y)a(y)a(x)
\end{align}
We are calculating the commutator in two parts. We start with $[N,H_1]$
\begin{align}
N H_1&=\int d^3xd^3z\left(a^\dagger(z)\delta^3(x-z)+a^\dagger(x)a^\dagger(z)a(z)\right)\left(-\frac{\hbar^2}{2m}\triangle_x+U(x)\right)a(x)\\
&=\int d^3xa^\dagger(x)\left(-\frac{\hbar^2}{2m}\triangle_x+U(x)\right)a(x)+\int d^3xd^3za^\dagger(x)\left(-\frac{\hbar^2}{2m}\triangle_x+U(x)\right)a^\dagger(z)a(z)a(x)
\end{align}
and
\begin{align}
H_1N&=\int d^3x\; a^\dagger(x)\left(-\frac{\hbar^2}{2m}\triangle_x+U(x)\right)(\delta^3(x-z)a(z)+a^\dagger(z)a(z)a(x))\\
&=\int d^3xa^\dagger(x)\left(-\frac{\hbar^2}{2m}\triangle_x+U(x)\right)a(x)+\int d^3xd^3za^\dagger(x)\left(-\frac{\hbar^2}{2m}\triangle_x+U(x)\right)a^\dagger(z)a(z)a(x)
\end{align}
therefore $[N,H_1]=0$.
For the second part $[N,H_\text{int}]$ we calculate
\begin{align}
a^\dagger_za_za^\dagger_xa^\dagger_ya_ya_x
&=a^\dagger_z(\delta^3_{zx}+a^\dagger_xa_z)a^\dagger_ya_ya_x\\
&=\delta^3_{zx}a^\dagger_za^\dagger_ya_ya_x+a^\dagger_za^\dagger_xa_za^\dagger_ya_ya_x\\
&=\delta^3_{zx}a^\dagger_ya^\dagger_za_ya_x+a^\dagger_za^\dagger_x(\delta^3_{zy}+a^\dagger_ya_z)a_ya_x\\
&=\delta^3_{zx}a^\dagger_ya^\dagger_za_ya_x+\delta^3_{zy}a^\dagger_za^\dagger_xa_ya_x+a^\dagger_za^\dagger_xa^\dagger_ya_za_ya_x\\
&=\delta^3_{zx}a^\dagger_ya^\dagger_za_ya_x+\delta^3_{zy}a^\dagger_xa^\dagger_za_ya_x+a^\dagger_xa^\dagger_ya^\dagger_za_za_ya_x\\
&\rightarrow a^\dagger_ya^\dagger_xa_ya_x+a^\dagger_xa^\dagger_ya_ya_x+a^\dagger_xa^\dagger_ya^\dagger_za_za_ya_x
\end{align}
and
\begin{align}
a^\dagger_xa^\dagger_ya_ya_xa^\dagger_za_z
&=a^\dagger_xa^\dagger_ya_y(\delta^3_{xz}+a^\dagger_za_x)a_z\\
&=\delta^3_{xz}a^\dagger_xa^\dagger_ya_ya_z+a^\dagger_xa^\dagger_ya_ya^\dagger_za_xa_z\\
&=\delta^3_{xz}a^\dagger_xa^\dagger_ya_za_y+a^\dagger_xa^\dagger_y(\delta^3_{zy}+a^\dagger_za_y)a_xa_z\\
&=\delta^3_{xz}a^\dagger_xa^\dagger_ya_za_y+\delta^3_{zy}a^\dagger_xa^\dagger_ya_xa_z+a^\dagger_xa^\dagger_ya^\dagger_za_ya_xa_z\\
&=\delta^3_{xz}a^\dagger_xa^\dagger_ya_za_y+\delta^3_{zy}a^\dagger_xa^\dagger_ya_za_x+a^\dagger_xa^\dagger_ya^\dagger_za_za_ya_x\\
&\rightarrow a^\dagger_xa^\dagger_ya_xa_y+a^\dagger_xa^\dagger_ya_ya_x+a^\dagger_xa^\dagger_ya^\dagger_za_za_ya_x
\end{align}
We therefore see that the commutator vanishes as well.

\subsection{Problem 2.1 - Infinitesimal LT}
\begin{align}
g_{\mu\nu}\Lambda^\mu_{\;\rho}\Lambda^\nu_{\;\sigma}&=g_{\rho\sigma}\\
g_{\mu\nu}\left(\delta^\mu_{\;\rho}+\delta\omega^\mu_{\;\rho}\right)\left(\delta^\nu_{\;\sigma}+\delta\omega^\nu_{\;\sigma}\right)&=g_{\rho\sigma}\\
g_{\mu\nu}\left(\delta^\mu_{\;\rho}\delta^\nu_{\;\sigma}+\delta^\nu_{\;\sigma}\cdot\delta\omega^\mu_{\;\rho}+\delta^\mu_{\;\rho}\cdot\delta\omega^\nu_{\;\sigma}+\mathcal{O}(\delta\omega^2)\right)&=g_{\rho\sigma}\\
g_{\rho\sigma}+g_{\mu\sigma}\cdot\delta\omega^\mu_{\;\rho}+g_{\rho\nu}\cdot\delta\omega^\nu_{\;\sigma}&=g_{\rho\sigma}
\end{align}
which implies
\begin{align}
\delta\omega_{\sigma\rho}+\delta\omega_{\rho\sigma}=0
\end{align}

\subsection{Problem 2.2 - Infinitesimal LT II}
Important: each $M^{\mu\nu}$ is an operator and $\delta\omega$ is just a coefficient matrix so $\delta\omega _{\mu\nu}M^{\mu\nu}$ ist a weighted sum of operators.
\begin{align}
U(\Lambda^{-1}\Lambda'\Lambda)&=U(\Lambda^{-1})U(\Lambda')U(\Lambda)\\
U(\Lambda^{-1}(I+\delta\omega')\Lambda)&=U(\Lambda^{-1})\left(I+\frac{i}{2\hbar}\delta\omega'_{\mu\nu}M^{\mu\nu}\right)U(\Lambda)\\
U(I+\Lambda^{-1}\delta\omega'\Lambda)&=I+\frac{i}{2\hbar}\delta\omega'_{\mu\nu}U(\Lambda^{-1})M^{\mu\nu}U(\Lambda)
\end{align}
now we calculate recalling successive LT's $(\Lambda^{-1})^{\varepsilon}_{\;\gamma}\delta\omega'^\gamma_{\;\;\beta}\Lambda^\beta_{\;\alpha}x^\alpha$
\begin{align}
(\Lambda^{-1}\delta\omega'\Lambda)_{\rho\sigma}
&=g_{\varepsilon\rho}(\Lambda^{-1})^{\varepsilon}_{\;\mu}\delta\omega'^\mu_{\;\;\nu}\Lambda^\nu_{\;\sigma}\\
&=g_{\varepsilon\rho}\Lambda^{\;\varepsilon}_{\mu}\delta\omega'^\mu_{\;\;\nu}\Lambda^\nu_{\;\sigma}\\
&=\delta\omega'_{\mu\nu}\Lambda^{\mu}_{\;\rho}\Lambda^\nu_{\;\sigma}
\end{align}
now we can rewrite $U(I+\Lambda^{-1}\delta\omega'\Lambda)$ and therefore
\begin{align}
\delta\omega'_{\mu\nu}\Lambda^{\mu}_{\;\rho}\Lambda^\nu_{\;\sigma}M^{\rho\sigma}&=\delta\omega'_{\mu\nu}U(\Lambda^{-1})M^{\mu\nu}U(\Lambda)
\end{align}
As all $\delta\omega'$ components are basically independent the equation must hold for each pair $\mu,\nu$.

\subsection{Problem 2.3 - Commutators of LT generators I}
LHS:
\begin{align}
U(\Lambda)^{-1}M^{\mu\nu}U(\Lambda)
&\simeq\left(I-\frac{i}{2\hbar}\delta\omega_{\alpha\beta}M^{\alpha\beta}\right)M^{\mu\nu}\left(I+\frac{i}{2\hbar}\delta\omega_{\rho\sigma}M^{\rho\sigma}\right)\\
&\simeq M^{\mu\nu}-\frac{i}{2\hbar}\delta\omega_{\rho\sigma}(M^{\rho\sigma}M^{\mu\nu}-M^{\mu\nu}M^{\rho\sigma})+\mathcal{O}(\delta\omega^2)\\
&= M^{\mu\nu}-\frac{i}{2\hbar}\delta\omega_{\rho\sigma}[M^{\rho\sigma},M^{\mu\nu}]\\
&= M^{\mu\nu}+\frac{i}{2\hbar}\delta\omega_{\rho\sigma}[M^{\mu\nu},M^{\rho\sigma}]
\end{align}
RHS:
\begin{align}
\Lambda^{\mu}_{\;\rho}\Lambda^\nu_{\;\sigma}M^{\rho\sigma}
&\simeq\left(\delta^{\mu}_{\;\rho}+\delta\omega^{\mu}_{\;\rho}\right)\left(\delta^{\nu}_{\;\sigma}+\delta\omega^{\nu}_{\;\sigma}\right)M^{\rho\sigma}\\
&\simeq M^{\mu\nu}+\delta^\mu_{\;\rho}\delta\omega^\nu_{\;\sigma}M^{\rho\sigma}+\delta^\nu_{\;\sigma}\delta\omega^\mu_{\;\rho}M^{\rho\sigma}\\
&\simeq M^{\mu\nu}+\delta\omega^\nu_{\;\sigma}M^{\mu\sigma}+\delta\omega^\mu_{\;\rho}M^{\rho\nu}\\
&\simeq M^{\mu\nu}+\delta\omega_{\alpha\sigma}g^{\alpha\nu}M^{\mu\sigma}+\delta\omega_{\alpha\rho}g^{\alpha\mu}M^{\rho\nu}\\
&\simeq M^{\mu\nu}+\delta\omega_{\alpha\sigma}(g^{\alpha\nu}M^{\mu\sigma}+g^{\alpha\mu}M^{\sigma\nu})\\
&\simeq M^{\mu\nu}+\delta\omega_{\rho\sigma}(g^{\rho\nu}M^{\mu\sigma}+g^{\rho\mu}M^{\sigma\nu})\\
&\simeq M^{\mu\nu}+\frac{1}{2}\delta\omega_{\rho\sigma}\left(g^{\rho\nu}(M^{\mu\sigma}-M^{\sigma\mu})+g^{\rho\mu}(M^{\sigma\nu}-M^{\nu\sigma})\right)\\
&\simeq M^{\mu\nu}+\frac{1}{2}\delta\omega_{\rho\sigma}\left(g^{\rho\nu}M^{\mu\sigma}-g^{\nu\rho}M^{\sigma\mu}+g^{\rho\mu}M^{\sigma\nu}-g^{\mu\rho}M^{\nu\sigma}\right)
\end{align}
Now we use the antisymmetry of $M$ 
\begin{align}
\Lambda^{\mu}_{\;\rho}\Lambda^\nu_{\;\sigma}M^{\rho\sigma}
&\simeq M^{\mu\nu}+\frac{1}{2}\delta\omega_{\rho\sigma}\left(g^{\nu\rho}M^{\mu\sigma}-g^{\nu\rho}M^{\sigma\mu}+g^{\rho\mu}M^{\sigma\nu}-g^{\mu\rho}M^{\nu\sigma}\right)\\
&\simeq M^{\mu\nu}-\frac{1}{2}\delta\omega_{\rho\sigma}\left(-g^{\nu\rho}M^{\mu\sigma}+g^{\nu\rho}M^{\sigma\mu}-g^{\rho\mu}M^{\sigma\nu}+g^{\mu\rho}M^{\nu\sigma}\right)\\
&\simeq M^{\mu\nu}-\frac{1}{2}\delta\omega_{\rho\sigma}\left(g^{\mu\rho}M^{\nu\sigma}-g^{\nu\rho}M^{\mu\sigma}-g^{\rho\mu}M^{\sigma\nu}+g^{\nu\rho}M^{\sigma\mu}\right)\\
&\simeq M^{\mu\nu}-\frac{1}{2}\delta\omega_{\rho\sigma}\left(g^{\mu\rho}M^{\nu\sigma}-g^{\nu\rho}M^{\mu\sigma}\right)-\frac{1}{2}\underbrace{\delta\omega_{\rho\sigma}\left(-g^{\rho\mu}M^{\sigma\nu}+g^{\nu\rho}M^{\sigma\mu}\right)}_{=\underbrace{\delta\omega_{\sigma\rho}\left(-g^{\sigma\mu}M^{\rho\nu}+g^{\nu\sigma}M^{\rho\mu}\right)}_{=-\delta\omega_{\rho\sigma}\left(-g^{\mu\sigma}(-M^{\nu\rho})+g^{\nu\sigma}(-M^{\mu\rho})\right)}}\\
&\simeq M^{\mu\nu}-\frac{1}{2}\delta\omega_{\rho\sigma}\left(g^{\mu\rho}M^{\nu\sigma}-g^{\nu\rho}M^{\mu\sigma}\right)-\frac{1}{2}\delta\omega_{\rho\sigma}\left(-g^{\mu\sigma}M^{\nu\rho}+g^{\nu\sigma}M^{\mu\rho}\right)\\
&\simeq M^{\mu\nu}-\frac{1}{2}\delta\omega_{\rho\sigma}\left(g^{\mu\rho}M^{\nu\sigma}-g^{\nu\rho}M^{\mu\sigma}-g^{\mu\sigma}M^{\nu\rho}+g^{\nu\sigma}M^{\mu\rho}\right)
\end{align}
As the components of $\delta\omega$ (besides the antisymmetry) are independent we get
\begin{align}
[M^{\mu\nu},M^{\rho\sigma}]=i\hbar\left(g^{\mu\rho}M^{\nu\sigma}-g^{\nu\rho}M^{\mu\sigma}-g^{\mu\sigma}M^{\nu\rho}+g^{\nu\sigma}M^{\mu\rho}\right)
\end{align}


\subsection{Problem 2.4 - Commutators of LT generators II}
Preliminary calculations
\begin{align}
\epsilon_{ijk}J_k
&=\varepsilon_{ijk}\frac{1}{2}\varepsilon_{kab}M^{ab}\\
&=-\frac{1}{2}\varepsilon_{kij}\varepsilon_{kab}M^{ab}\\
&=-\frac{1}{2}\left(\delta_{ia}\delta_{jb}-\delta_{ja}\delta_{ib}\right)M^{ab}\\
&=-\frac{1}{2}\left(M^{ij}-M^{ji}\right)\\
&=-M^{ij}
\end{align}
\begin{itemize}
\item  With
\begin{align}
J_1
&=\frac{1}{2}(\varepsilon_{123}M^{23}+\varepsilon_{132}M^{32})\\
&=\varepsilon_{123}M^{23}\\
&=M^{23}
\end{align}
then
\begin{align}
[J_1,J_3]
&=[M^{23},M^{12}]\\
&=i\hbar\left(g^{21}M^{32}-g^{31}M^{22}-g^{22}M^{31}+g^{32}M^{21}\right)\\
&=-i\hbar g^{22}M^{31}\\
&=-i\hbar M^{31}\\
&=-i\hbar J_2
\end{align}
\item analog ...
\item
\begin{align}
[K^i,K^j]
&=[M^{i0},M^{j0}]\\
&=i\hbar\left(g^{ij}M^{00}-g^{0j}M^{i0}-g^{i0}M^{0j}+g^{00}M^{ij}\right)\\
&=i\hbar\left(-\delta^{ij}M^{00}+M^{ij}\right) \\
&=\left\{
\begin{array}{ll} 
i\hbar M^{ij}=-i\hbar\epsilon_{ijk}J_k 	& (i=j)\\
 0 										& (i\neq j)
\end{array}\right.
\end{align}
where we used the result from the preliminary calculation in the last step.
\end{itemize}

\subsection{Problem 2.7 - Translation operator}
The obvious property $T(a)T(b)=T(a+b)$. Then
\begin{align}
T(\delta a+\delta b)&=T(\delta a)T(\delta b)\\
&=\left(1-\frac{i}{\hbar}\delta a_\mu P^\mu\right)\left(1-\frac{i}{\hbar}\delta b_\nu P^\nu\right)\\
&\simeq 1-\frac{i}{\hbar}(\delta a_\mu +\delta b_\mu) P^\mu+\frac{1}{\hbar^2}\delta a_\mu\delta b_\mu P^\mu P^\nu
\end{align}
and 
\begin{align}
T(\delta a+\delta b)&=T(\delta b)T(\delta a)\\
&=\left(1-\frac{i}{\hbar}\delta b_\nu P^\nu\right)\left(1-\frac{i}{\hbar}\delta a_\mu P^\mu\right)\\
&\simeq 1-\frac{i}{\hbar}(\delta a_\mu +\delta b_\mu) P^\mu+\frac{1}{\hbar^2}\delta a_\mu\delta b_\mu P^\nu P^\mu
\end{align}
which implies $P^\mu P^\nu=P^\nu P^\mu$.

\subsection{Problem 2.8 - Transformation of scalar field}
(a) We start with
\begin{align}
U(\Lambda)^{-1}\varphi(x)U(\Lambda)&=\varphi(\Lambda^{-1}x)\\
\left(1-\frac{i}{2\hbar}\delta\omega_{\mu\nu}M^{\mu\nu}\right)\varphi(x)\left(1+\frac{i}{2\hbar}\delta\omega_{\mu\nu}M^{\mu\nu}\right)&=\varphi([\delta^\mu_{\;\nu}-\delta\omega^\mu_{\;\nu}]x^\nu)\\
\varphi(x)-\frac{i}{2\hbar}\delta\omega_{\mu\nu}[M^{\mu\nu},\varphi(x)]&=\varphi(x)-\delta\omega^\mu_{\;\nu}x^\nu\frac{\partial\varphi}{\partial x^\mu}\\
&=\varphi(x)-\delta\omega^\mu_{\;\nu}\frac{1}{2}\left(x^\nu\frac{\partial\varphi}{\partial x^\mu}-x^\mu\frac{\partial\varphi}{\partial x^\nu}\right)\\
&=\varphi(x)-\delta\omega_{\mu\nu}\frac{1}{2}\left(x^\nu\partial^\mu-x^\mu\partial^\nu\right)\varphi
\end{align}
and therefore
\begin{align}
[\varphi,M^{\mu\nu}]=\frac{\hbar}{i}(x^\mu\partial^\nu-x^\nu\partial^\mu)\varphi
\end{align}

(b)
(c)
(d)
(e)
(f)

\subsection{Problem 3.2 - Multiparticle eigenstates of the hamiltonian}
With
\begin{align}
|k_1...k_n\rangle&=a^\dagger_{k_1}...a^\dagger_{k_n}|0\rangle\\
H&=\int\widetilde{dk}\;\omega_k a^\dagger_ka_k\\
[a_k,a_q^\dagger]&=\underbrace{(2\pi)^3 2\omega_k\delta^3(\vec{k}-\vec{q})}_{\delta_{kq}}
\end{align}
we see that the expression which needs calculating is the creation and annihilation operators. The idea is to use the commutation relations to move the $a_k$ to the right end to use $a_k|0\rangle$
\begin{align}
a^\dagger_ka_ka^\dagger_{k_1}...a^\dagger_{k_n}|0\rangle
&=a^\dagger_k(a^\dagger_{k_1}a_k+\delta_{kk_1})a^\dagger_{k_2}...a^\dagger_{k_n}|0\rangle\\
&=\delta_{kk_1}a^\dagger_ka^\dagger_{k_2}...a^\dagger_{k_n}|0\rangle
+a^\dagger_ka^\dagger_{k_1}a_ka^\dagger_{k_2}...a^\dagger_{k_n}|0\rangle\\
&=...\\
&=\sum_j\delta_{kk_j}a^\dagger_k\underbrace{a^\dagger_{k_2}...a^\dagger_{k_n}}_{(n-1)\, \text{times with}\;a_{k_j}\;\text{missing}}|0\rangle+a^\dagger_ka^\dagger_{k_1}...a^\dagger_{k_n}\underbrace{a_k|0\rangle}_{=0}.
\end{align}
Therefore we obtain
\begin{align}
H|k_1...k_n\rangle
&=\int\frac{d^3k}{(2\pi)^3 2\omega_k}\omega_k\sum_j\delta_{kk_j}a^\dagger_ka^\dagger_{k_2}...a^\dagger_{k_n}|0\rangle\\
&=\int\frac{d^3k}{(2\pi)^3 2\omega_k}\omega_k\sum_j(2\pi)^32\omega_k\delta^3(\vec{k}-\vec{k}_j)a^\dagger_ka^\dagger_{k_2}...a^\dagger_{k_n}|0\rangle\\
&=\int d^3k\omega_k\sum_j\delta^3(\vec{k}-\vec{k}_j)a^\dagger_ka^\dagger_{k_2}...a^\dagger_{k_n}|0\rangle
\end{align}
which we can integrate obtaining the desired result
\begin{align}
H|k_1...k_n\rangle
&=\sum_j\omega_{k_j}a^\dagger_{k_j}a^\dagger_{k_2}...a^\dagger_{k_n}|0\rangle\\
&=\left(\sum_j\omega_{k_j}\right)a^\dagger_{k_1}a^\dagger_{k_2}...a^\dagger_{k_n}|0\rangle\\
&=\left(\sum_j\omega_{k_j}\right)|k_1...k_n\rangle.
\end{align}

\subsection{Problem 3.4 - Heisenberg equations of motion for free field}
(a) For the translation operator $T(a)=e^{-iP^\mu a_\mu}$ we expand in first order
\begin{align}
T(a)^{-1}\varphi(a)T(a)
&=\left(1-(-i)P^\mu a_\mu+\mathcal{O}(a^2)\right)\varphi(x)\left(1+(-i)P^\mu a_\mu+\mathcal{O}(a^2)\right)\\
&=\left(1+iP^\mu a_\mu+\mathcal{O}(a^2)\right)\varphi(x)\left(1-iP^\mu a_\mu+\mathcal{O}(a^2)\right)\\
&\simeq\varphi(x)+ia_\mu P^\mu\varphi(x)-ia_\mu\varphi(x)P^\mu\\
&\simeq\varphi(x)+ia_\mu [P^\mu,\varphi(x)]
\end{align}
for the right hand right we get
\begin{align}
\varphi(x-a)\simeq\varphi(x)-\partial^\mu \varphi(x)a_\mu
\end{align}
and therefore
\begin{align}
i[P^\mu,\varphi(x)]=-\partial^\mu\varphi(x)
\end{align}
(b) With $\mu=0$ and $\partial^0=g_{0\nu}\partial_\nu=-\partial_0$ we have
\begin{align}
i[H,\varphi(x)]&=-\partial^0\varphi(x)=+\partial_0\varphi(x)\\
&\rightarrow\quad\dot{\varphi}(x)=i[H,\varphi(x)]
\end{align}
(c) We start with the hamiltonian (3.25)
\begin{align}
H=\int d^3y\frac{1}{2}\Pi^2(y)+\frac{1}{2}(\nabla_y\varphi(y))^2+\frac{1}{2}m^2\varphi(y)^2-\Omega_0
\end{align}
\begin{itemize}
\item Obtaining $\dot{\varphi}(x)=i[H,\varphi(x)]$

We need to calculate (setting $x^0=y^0$ - why can we?)
\begin{align}
[\Pi^2(y),\varphi(x)]
&=\Pi(y)\Pi(y)\varphi(x)-\varphi(x)\Pi(y)\Pi(y)\\
&=\Pi(y)\Pi(y)\varphi(x)-\Pi(y)\varphi(x)\Pi(y)+\Pi(y)\varphi(x)\Pi(y)-\varphi(x)\Pi(y)\Pi(y)\\
&=\Pi(y)[\Pi(y),\varphi(x)]+[\Pi(y),\varphi(x)]\Pi(y)\\
&=2\Pi(y)(-1)i\delta^3(\vec{y}-\vec{x})\\
%
[(\nabla_y\varphi(y))^2,\varphi(x)]
&=\nabla_y\varphi(y)\nabla_y\varphi(y)\varphi(x)-\varphi(x)\nabla_y\varphi(y)\nabla_y\varphi(y)\\
&=\nabla_y\varphi(y)[\nabla_y\varphi(y),\varphi(x)]+[\nabla_y\varphi(y),\varphi(x)]\nabla_y\varphi(y)\\
&=\nabla_y\varphi(y)\nabla_y[\varphi(y),\varphi(x)]+\nabla_y[\varphi(y),\varphi(x)]\nabla_y\varphi(y)\\
&=0\\
%
[\varphi(y)^2,\varphi(x)]
&=\varphi(y)\varphi(y)\varphi(x)-\varphi(x)\varphi(y)\varphi(y)\\
&=\varphi(y)\varphi(y)\varphi(x)-\varphi(y)\varphi(x)\varphi(y)+\varphi(y)\varphi(x)\varphi(y)-\varphi(x)\varphi(y)\varphi(y)\\
&=\varphi(y)[\varphi(y),\varphi(x)]+[\varphi(y),\varphi(x)]\varphi(y)\\
&=0
\end{align}
then
\begin{align}
\int d^3y[\Pi^2(y),\varphi(x)]&=-2i\Pi(x)\\
\int d^3y[(\nabla_y\varphi(y))^2,\varphi(x)]&=\int d^3y\nabla_y\varphi(y)
%
[\nabla_y\varphi(y),\varphi(x)]+[\nabla_y\varphi(y),\varphi(x)]\nabla_y\varphi(y)\\
&=0\\
%
\int d^3y[\varphi(y)^2,\varphi(x)]&=0
\end{align}
and therefore
\begin{align}
\dot{\varphi}(x)
&=i[H,\varphi(x)]\\
&=i\frac{1}{2}(-2i)\Pi(x)\\
&=\Pi(x)
\end{align}

\item Obtaining $\dot{\Pi}(x)=-i[H,\Pi(x)]$ (sign!?!)

Now we need to calculate - by using the results from above we can now shortcut a bit
\begin{align}
[\Pi^2(y),\Pi(x)]
&=0\\
%
[(\nabla_y\varphi(y))^2,\Pi(x)]
&=(\nabla_y\varphi(y))(\nabla_y\varphi(y))\Pi(x)-\Pi(x)(\nabla_y\varphi(y))(\nabla_y\varphi(y))
\\
&=(\nabla_y\varphi(y))[(\nabla_y\varphi(y)),\Pi(x)]-[\Pi(x),(\nabla_y\varphi(y))](\nabla_y\varphi(y))
\\
&=(\nabla_y\varphi(y))\nabla_y[\varphi(y),\Pi(x)]-(\nabla_y[\Pi(x),\varphi(y)])(\nabla_y\varphi(y))
\\
&=(\nabla_y\varphi(y))\nabla_yi\delta^3(\vec{x}-\vec{y})-(\nabla_y(-i)\delta^3(\vec{x}-\vec{y}))(\nabla_y\varphi(y))
\\
&=2i(\nabla_y\delta^3(\vec{x}-\vec{y}))(\nabla_y\varphi(y))
\\
%
[\varphi(y)^2,\Pi(x)]
&=\varphi(y)\varphi(y)\Pi(x)-\Pi(x)\varphi(y)\varphi(y)\\
&=\varphi(y)\varphi(y)\Pi(x)-\varphi(y)\Pi(x)\varphi(y)+\varphi(y)\Pi(x)\varphi(y)-\Pi(x)\varphi(y)\varphi(y)\\
&=\varphi(y)[\varphi(y),\Pi(x)]+[\varphi(y),\Pi(x)]\varphi(y)\\
&=2i\varphi(y)\delta^3(\vec{x}-\vec{y})
\end{align}
then
\begin{align}
\int d^3y [\Pi^2(y),\Pi(x)]
&=0\\
\int d^3y [(\nabla_y\varphi(y))^2,\Pi(x)]
&=2i\int d^3y (\nabla_y\delta^3(\vec{x}-\vec{y}))(\nabla_y\varphi(y))\\
&=-2i\int d^3y \delta^3(\vec{x}-\vec{y})(\nabla_y\nabla_y\varphi(y))\\
&=-2i\triangle_x\varphi(x)\\
\int d^3y[\varphi(y)^2,\Pi(x)]
&=2i\varphi(x)
\end{align}
and therefore
\begin{align}
\dot\Pi(x)&=-i[H,\Pi(x)]\\
&=-i\left(\frac{1}{2}(-2i)\triangle_x\varphi(x)+\frac{1}{2}m^22i\varphi(x)\right)\\
&=-i\left(-i\triangle_x\varphi(x)+m^2i\varphi(x)\right)\\
&=-\triangle_x\varphi(x)+m^2\varphi(x)
\end{align}
which finally leads to (with $\Box=\partial_{tt}-\triangle$)
\begin{align}
\partial^0\partial_0\varphi(x)
&=\partial^0\Pi(x)\\
&=-\partial_0\Pi(x)\\
&=-(-\triangle_x\varphi(x)+m^2\varphi(x))\\
&\rightarrow(\Box_x+m^2)\varphi(x)=0
\end{align}
\end{itemize}
(d) With
\begin{align}
\vec{P}\equiv-\int d^3x\Pi(x)\nabla_x\varphi(x)
\end{align}
we have to calculate
\begin{align}
[\vec{P},\varphi(y)]=-\int d^3x[\Pi(x)\nabla_x\varphi(x),\varphi(y)].
\end{align}
Let's start with
\begin{align}
[\Pi(x)\nabla_x\varphi(x),\varphi(y)]
&=\Pi(x)\nabla_x\varphi(x)\varphi(y)-\varphi(y)\Pi(x)\nabla_x\varphi(x)\\
&=\Pi(x)\nabla_x\varphi(x)\varphi(y)-(\Pi(x)\varphi(y)+i\delta^3(\vec{x}-\vec{y}))\nabla_x\varphi(x)\\
&=\Pi(x)\nabla_x\varphi(x)\varphi(y)-\Pi(x)\varphi(y)\nabla_x\varphi(x)+i\delta^3(\vec{x}-\vec{y})\nabla_x\varphi(x)\\
&=\Pi(x)\nabla_x(\varphi(x)\varphi(y))-\Pi(x)\nabla_x(\varphi(y)\varphi(x))+i\delta^3(\vec{x}-\vec{y})\nabla_x\varphi(x)\\
&=\Pi(x)\nabla_x[\varphi(x),\varphi(y)]+i\delta^3(\vec{x}-\vec{y})\nabla_x\varphi(x)\\
&=i\delta^3(\vec{x}-\vec{y})\nabla_x\varphi(x)
\end{align}
and then
\begin{align}
[\vec{P},\varphi(y)]
&=-i\int d^3x\delta^3(\vec{x}-\vec{y})\nabla_x\varphi(x)\\
&=-i\nabla_y\varphi(y)
\end{align}
(e) With
\begin{align}
\Pi(x)&=
\dot\varphi(x)\\
&=\int\frac{d^3k}{(2\pi)^32\omega_k}(-i\omega_k)(a_ke^{ikx}-a^\dagger_ke^{-ikx})\\
\nabla\varphi(x)
&=\int\frac{d^3q}{(2\pi)^32\omega_k}(i\vec{q})(a_qe^{iqx}-a^\dagger_qe^{-iqx})\\
\end{align}
then
\begin{align}
\vec{P}
&=-\int d^3x\Pi(x)\nabla_x\varphi(x)\\
&=-\iiint d^3x
\frac{d^3k}{(2\pi)^3 2\omega_k}
\frac{d^3q}{(2\pi)^3 2\omega_k}
(-i\omega_k)(i\vec{q})
(a_ke^{ikx}-a^\dagger_ke^{-ikx})
(a_qe^{iqx}-a^\dagger_qe^{-iqx})\\
&=-\iiint d^3x
\frac{d^3k}{(2\pi)^3 2}
\frac{d^3q}{(2\pi)^3 2\omega_k}\vec{q}
(a_ka_qe^{i(k+q)x}-a^\dagger_ka_qe^{-i(k-q)x}-
a_ka^\dagger_q e^{i(k-q)x}+a^\dagger_ka^\dagger_qe^{-i(k+q)x})\\
\end{align}
now we can use the commutation relations and reindex
\begin{align}
&=-\iiint d^3x
\frac{d^3kd^3q}{4\omega_k(2\pi)^6}
\vec{q}
(a_ka_qe^{i(k+q)x}-a^\dagger_ka_qe^{-i(k-q)x}-
(a^\dagger_qa_k+(2\pi)^32\omega_k\delta^3(\vec{k}-\vec{q})) e^{i(k-q)x}+a^\dagger_ka^\dagger_qe^{-i(k+q)x})\\
&=-\iiint d^3x
\frac{d^3kd^3q}{4\omega_k(2\pi)^6}
\vec{q}
(a_ka_qe^{i(k+q)x}+a^\dagger_ka^\dagger_qe^{-i(k+q)x})+\iiint d^3x
\frac{d^3kd^3q}{4\omega_k(2\pi)^6}
\vec{q}
2a^\dagger_ka_qe^{-i(k-q)x}\\
&\quad+\iiint d^3x
\frac{d^3kd^3q}{4\omega_k(2\pi)^6}
\vec{q}
(2\pi)^32\omega_k\delta^3(\vec{k}-\vec{q})e^{i(k-q)x}
\end{align}
Now we can look at the integrals individually and use the asymmetry. The first
\begin{align}
-\iiint d^3x
\frac{d^3kd^3q}{4\omega_k(2\pi)^6}
\vec{q}
(a_ka_qe^{i(k+q)x}+a^\dagger_ka^\dagger_qe^{-i(k+q)x})
&=...\\
&=0
\end{align}
second
\begin{align}
\iiint d^3x
\frac{d^3kd^3q}{4\omega_k(2\pi)^6}
\vec{q}
(2\pi)^32\omega_k\delta^3(\vec{k}-\vec{q})e^{i(k-q)x}
&=\iiint d^3x
\frac{d^3kd^3q}{2(2\pi)^3}
\vec{q}
\delta^3(\vec{k}-\vec{q})e^{i(k-q)x}\\
&=\iiint d^3x
\frac{d^3k}{2(2\pi)^3}
\vec{k}\\
&=0
\end{align}
and third
\begin{align}
\iiint d^3x
\frac{d^3kd^3q}{4\omega_k(2\pi)^6}
\vec{q}
2a^\dagger_ka_qe^{-i(k-q)x}
&=\iint 
\frac{d^3kd^3q}{4\omega_k(2\pi)^6}
\vec{q}
2a^\dagger_ka_q\int d^3x\;e^{-i(k-q)x}\\
&=\iint 
\frac{d^3kd^3q}{4\omega_k(2\pi)^6}
\vec{q}
2a^\dagger_ka_qe^{-i(k-q)x} e^{-i(k^0-q^0)x^0}\int d^3x\;e^{-i(\vec{k}-\vec{q})\vec{x}}\\
&=\iint 
\frac{d^3kd^3q}{4\omega_k(2\pi)^6}
\vec{q}
2a^\dagger_ka_qe^{-i(k-q)x} e^{-i(k^0-q^0)x^0}(2\pi)^3\delta^3(\vec{k}-\vec{q})\\
&=\int 
\frac{d^3k}{2\omega_k(2\pi)^3}
\vec{k}
a^\dagger_ka_k\\
&=\int 
\widetilde{d^3k}\;
\vec{k}
a^\dagger_ka_k
\end{align}
Therefore we obtain
\begin{align}
\vec{P}&=\int 
\frac{d^3k}{2\omega_k(2\pi)^3}
\vec{k}
a^\dagger_ka_k\\
&=\int 
\widetilde{d^3k}\;
\vec{k}
a^\dagger_ka_k
\end{align}

\subsection{Problem 3.5 - Complex scalar field}
(a) Sloppy way - Calculating the Euler-Lagrange equations
\begin{align}
\frac{\partial\mathcal{L}}{\partial\varphi}
&=-m^2\varphi^\dagger\\
\frac{\partial\mathcal{L}}{\partial(\partial_\mu\varphi)}&=-\partial^\mu\varphi^\dagger\\
&\rightarrow -m^2\varphi^\dagger+\partial_\mu\partial^\mu\varphi^\dagger=0\\
&\rightarrow (\partial_\mu\partial^\mu-m^2)\varphi^\dagger=0
\end{align}
Bit more rigorous with
\begin{align}
\frac{\delta\phi(x_1,t_1)}{\delta\phi(x_2,t_2)}&=\delta(x_1-x_2)\times\delta(t_1-t_2)\\
\frac{\delta\partial_\mu\phi(x)}{\delta\phi(y)}&=\frac{\delta}{\delta\phi(y)}\lim_{\epsilon\rightarrow0}\frac{\phi(x_1,x_\mu+\epsilon,...,x_4)-\phi(x_1,x_2,x_3,x_4)}{\epsilon}\\
&=\lim_{\epsilon\rightarrow0}\frac{1}{\epsilon}\left(\delta(x_\mu+\epsilon-y_\mu)-\delta(x_\mu-y_\mu)\right)\times\delta(x_1-y_1)\times...\times\delta(x_4-y_4)\\
&=\frac{\partial}{\partial x^\mu}\delta^4(x-y)
\end{align}
we get
\begin{align}
S[\varphi]
&=\int d^4x\left(-\partial^\mu\varphi^\dagger(x)\partial_\mu\varphi(x)-m^2\varphi^\dagger(x)\varphi(x)\right)\\
\frac{\delta S[\varphi]}{\delta\varphi(y)}
&=\int d^4x\left(-\partial^\mu\varphi^\dagger(x)\partial_\mu\delta^4(x-y)-m^2\varphi^\dagger(y)\delta^4(x-y)\right)\\
&=\int d^4x\left(\partial_\mu\partial^\mu\varphi^\dagger(x)\delta^4(x-y)-m^2\varphi^\dagger(x)\delta^4(x-y)\right)\\
&=(\Box_y-m^2)\varphi^\dagger(y)
\end{align}
(b) With
\begin{align}
\mathcal{L}
&=-\partial^0\varphi^\dagger\partial_0\varphi-\partial^a\varphi^\dagger\partial_a\varphi-m^2\varphi^\dagger\varphi+\Omega_0\\
&=\partial_0\varphi^\dagger\partial_0\varphi-\partial^a\varphi^\dagger\partial_a\varphi-m^2\varphi^\dagger\varphi+\Omega_0\\
\Pi&=\frac{\partial\mathcal{L}}{\partial\dot\varphi}
=\dot\varphi^\dagger\\
\Pi^\dagger&=\frac{\partial\mathcal{L}}{\partial\dot\varphi^\dagger}
=\dot\varphi\\
\rightarrow\mathcal{H}&=\Pi\dot\varphi+\Pi^\dagger\dot\varphi^\dagger-\mathcal{L}\\
&=\dot\varphi^\dagger\dot\varphi+\dot\varphi\dot\varphi^\dagger-\dot\varphi^\dagger\dot\varphi+(\nabla^a\varphi^\dagger)(\nabla_a\varphi)+m^2\varphi^\dagger\varphi-\Omega_0\\
&=\Pi^\dagger\Pi+(\nabla^a\varphi^\dagger)(\nabla_a\varphi)+m^2\varphi^\dagger\varphi-\Omega_0
\end{align}
(c) Considering the plane wave solutions $e^{i\vec{k}\vec{x}\pm i\omega_kt}$ with 
\begin{align}
kx=g_{\mu\nu}k^\mu x^\nu=g_{00}k^0x^0+g_{ik}k^ix^k=-\omega_kt+\vec{k}\vec{x}
\end{align}
we have
\begin{align}
\varphi(\vec{x},t)
&=\int\frac{d^3k}{(2\pi)^32\omega_k}\;a_ke^{ikx}+b^\dagger_ke^{-ikx}\\
&=\int\frac{d^3k}{(2\pi)^32\omega_k}\;a_ke^{i\vec{k}\vec{x}-i\omega_kt}+b^\dagger_ke^{-i\vec{k}\vec{x}+i\omega_kt}\\
e^{-iqx}\varphi(\vec{x},t)&=\int\frac{d^3k}{(2\pi)^32\omega_k}\;a_ke^{i(k-q)x}+b^\dagger_ke^{-i\vec{k}\vec{x}+i\omega_kt}e^{-iqx}\\
\int d^3xe^{-iqx}\varphi(\vec{x},t)
&=\int\frac{d^3k}{(2\pi)^32\omega_k} \;a_k\underbrace{\int d^3xe^{i(k-q)x}}_{(2\pi)^3\delta^3(\vec{k}-\vec{q})e^{-i(\omega_k-\omega_q)t}}+b_{-k}\underbrace{\int d^3xe^{i(\vec{k}-\vec{q})\vec{x}}}_{(2\pi)^3\delta^3(\vec{k}-\vec{q})}e^{i(\omega_k+\omega_q)t}\\
&=\frac{1}{2\omega_q}\left(a_q+b^\dagger_{-q}e^{2i\omega_qt}\right)
\end{align}
and
\begin{align}
\partial_0\varphi(\vec{x},t)&=\int\frac{d^3k\;(-i\omega_k)}{(2\pi)^32\omega_k}\;a_ke^{i\vec{k}\vec{x}-i\omega_kt}-b^\dagger_ke^{-i\vec{k}\vec{x}+i\omega_kt}\\
\int d^3x e^{-iqx}\partial_0\varphi(\vec{x},t)&=-\frac{i}{2}\left(a_q-b^\dagger_{-q}e^{2i\omega_qt}\right)
\end{align}
adding both equations gives with $\partial_0e^{-iqx}=\partial_0e^{-i(-\omega_kt+\vec{k}\vec{x})}=-i\omega_qe^{-iqx}$ and $f\stackrel{\leftrightarrow}{\partial_\mu}g=f(\partial_\mu g)-(\partial_\mu f)g$
\begin{align}
a_q&=\omega_q\int d^3x e^{-iqx}\varphi(\vec{x},t)+i\int d^3x e^{-iqx}\partial_0\varphi(\vec{x},t)\\
&=i\int d^3x e^{-iqx}(-i\omega_q+\partial_0)\varphi(\vec{x},t)\\
&=i\int d^3x e^{-iqx}\stackrel{\leftrightarrow}{\partial_0}\varphi(\vec{x},t)
\end{align}
To get $b_q$ we solve a second set of equations for $\varphi^\dagger$ 
\begin{align}
\varphi(\vec{x},t)
&=\int\frac{d^3k}{(2\pi)^32\omega_k}\;a_ke^{ikx}+b^\dagger_ke^{-ikx}\\
\rightarrow\varphi^\dagger(\vec{x},t)
&=\int\frac{d^3k}{(2\pi)^32\omega_k}\;a^\dagger_ke^{-ikx}+b_ke^{ikx}\\
&=\int\frac{d^3k}{(2\pi)^32\omega_k}\;b_ke^{ikx}+a^\dagger_ke^{-ikx}
\end{align}
Now $b_k$ takes the role of $a_k$ and we can just copy the solution
\begin{align}
b_q&=\omega_q\int d^3x e^{-iqx}\varphi^\dagger(\vec{x},t)+i\int d^3x e^{-iqx}\partial_0\varphi^\dagger(\vec{x},t)\\
&=i\int d^3x e^{-iqx}(-i\omega_q+\partial_0)\varphi^\dagger(\vec{x},t)\\
&=i\int d^3x e^{-iqx}\stackrel{\leftrightarrow}{\partial_0}\varphi^\dagger(\vec{x},t)
\end{align}
(d) Starting with the observation
\begin{align}
[A,B]^\dagger
&=(AB)^\dagger-(BA)^\dagger\\
&=B^\dagger A^\dagger- A^\dagger B^\dagger\\
&=[B^\dagger,A^\dagger]\\
&=-[A^\dagger,B^\dagger]
\end{align}
therefore the relevant commutation relations for the fields are
\begin{align}
[\varphi(\vec{x},t),\varphi(\vec{y},t)]&=0\quad&\rightarrow\quad[\varphi^\dagger(\vec{x},t),\varphi^\dagger(\vec{y},t)]=0\\
%
[\varphi^\dagger(\vec{x},t),\varphi(\vec{y},t)]&=0&\\
%
[\Pi(\vec{x},t),\Pi(\vec{y},t)]&=0\quad&\rightarrow\quad[\Pi^\dagger(\vec{x},t),\Pi^\dagger(\vec{y},t)]=0\\
%
[\Pi^\dagger(\vec{x},t),\Pi(\vec{y},t)]&=0&\\
%
[\varphi(\vec{x},t),\Pi(\vec{y},t)]&=i\delta^3(\vec{x}-\vec{y})\quad&\rightarrow\quad[\varphi^\dagger(\vec{x},t),\Pi^\dagger(\vec{y},t)]=i\delta^3(\vec{x}-\vec{y})\\
%
[\varphi^\dagger(\vec{x},t),\Pi(\vec{y},t)]&=0\quad&\rightarrow\quad[\varphi(\vec{x},t),\Pi^\dagger(\vec{y},t)]=0
\end{align}
with the previous results
\begin{align}
a_q&=i\int d^3x e^{-iqx}(-i\omega_q+\partial_0)\varphi(\vec{x},t)\\
&=i\int d^3x e^{-iqx}(-i\omega_q\varphi(\vec{x},t)+\Pi^\dagger(\vec{x},t))\\
a^\dagger_q&=i\int d^3x e^{iqx}(i\omega_q\varphi^\dagger(\vec{x},t)+\Pi(\vec{x},t))\\
b_q&=i\int d^3x e^{-iqx}(-i\omega_q+\partial_0)\varphi^\dagger(\vec{x},t)\\
&=i\int d^3x e^{-iqx}(-i\omega_q\varphi(\vec{x},t)+\Pi(\vec{x},t))\\
b^\dagger_q&=i\int d^3x e^{iqx}(i\omega_q\varphi^\dagger(\vec{x},t)+\Pi^\dagger(\vec{x},t))
\end{align}
let's calculate each of the commutators
\begin{align}
[a_k,a^\dagger_q]
&=\iint d^3x\,d^3y\,e^{-ikx}e^{iqy}\left(\omega_k\omega_q[\varphi_x,\varphi^\dagger_y]-i\omega_q[\varphi_x,\Pi_y]+i\omega_q[\Pi^\dagger_x,\varphi^\dagger_y]+[\Pi^\dagger_x,\Pi_y]\right)\\
&=\iint d^3x\,d^3y\,e^{-i(kx-qy)}\left(-i\omega_q[\varphi_x,\Pi_y]+i\omega_q[\Pi^\dagger_x,\varphi^\dagger_y]\right)\\
&=\iint d^3x\,d^3y\,e^{-i(kx-qy)}\left(-i\omega_qi\delta^3(\vec{x}-\vec{y})+i\omega_q(-i)\delta^3(\vec{x}-\vec{y})\right)\\
&=\left(\omega_q+\omega_q\right)\iint d^3x\,e^{-i(k-q)x}\\
&=\left(\omega_q+\omega_q\right)(2\pi)^3\delta^3(\vec{k}-\vec{q})\\
&=2\omega_q(2\pi)^3\delta^3(\vec{k}-\vec{q})
\end{align}
and so on
\begin{align}
[b_k,b^\dagger_q]=...=2\omega_q(2\pi)^3\delta^3(\vec{k}-\vec{q})
\end{align}
(e) Now
\begin{align}
H&=\int d^3x\;\Pi^\dagger\Pi+(\nabla^a\varphi^\dagger)(\nabla_a\varphi)+m^2\varphi^\dagger\varphi-\Omega_0\\
%
\Pi^\dagger\Pi&=\dot\varphi\dot\varphi^\dagger\\
&=\int\widetilde{d^3k}\widetilde{d^3q}(i\omega_k)(i\omega_q)\left(a_ke^{ikx}-b^\dagger_ke^{-ikx}\right)\left(a^\dagger_qe^{-iqx}-b_qe^{iqx}\right)\\
&=\int\widetilde{d^3k}\widetilde{d^3q}(-\omega_k\omega_q)\left(a_ka^\dagger_qe^{-iqx}e^{ikx}-b^\dagger_ka^\dagger_qe^{-iqx}e^{-ikx}-a_kb_qe^{iqx}e^{ikx}+b^\dagger_kb_qe^{iqx}e^{-ikx}\right)\\
&=\int\widetilde{d^3k}\widetilde{d^3q}(-\omega_k\omega_q)\left([a^\dagger_qa_k-2\omega_k(2\pi)^3\delta^3(\vec{k}-\vec{q})]e^{-i(q-k)x}-b^\dagger_ka^\dagger_qe^{-i(q+k)x}-a_kb_qe^{i(q+k)x}+b^\dagger_kb_qe^{i(q-k)x}\right)
\end{align}
\begin{align}
(\nabla^a\varphi^\dagger)(\nabla_a\varphi)&=\int\widetilde{d^3k}\widetilde{d^3q}(k^aq_a)\left(-a^\dagger_ke^{-ikx}+b_ke^{ikx}\right)\left(a_qe^{iqx}-b^\dagger_qe^{-iqx}\right)\\
&=\int\widetilde{d^3k}\widetilde{d^3q}(k^aq_a)\left(-a^\dagger_ka_qe^{iqx}e^{-ikx}+b_ka_qe^{iqx}e^{ikx}+a^\dagger_kb^\dagger_qe^{-iqx}e^{-ikx}-b_kb^\dagger_qe^{-iqx}e^{ikx}\right)\\
&=\int\widetilde{d^3k}\widetilde{d^3q}(k^aq_a)\left(-a^\dagger_ka_qe^{i(q-k)x}+a_qb_ke^{i(q+k)x}+a^\dagger_kb^\dagger_qe^{-i(q+k)x}-[b^\dagger_qb_k-2\omega_k(2\pi)^3\delta^3(\vec{k}-\vec{q})]e^{-i(q-k)x}\right)
\end{align}
\begin{align}
\varphi^\dagger\varphi&=\int\widetilde{d^3k}\widetilde{d^3q}\left(a^\dagger_ke^{-ikx}+b_ke^{ikx}\right)\left(a_qe^{iqx}+b^\dagger_qe^{-iqx}\right)\\
&=\int\widetilde{d^3k}\widetilde{d^3q}\left(a^\dagger_ka_qe^{iqx}e^{-ikx}+b_ka_qe^{iqx}e^{ikx}+a^\dagger_kb^\dagger_qe^{-iqx}e^{-ikx}+b_kb^\dagger_qe^{-iqx}e^{ikx}\right)\\
&=\int\widetilde{d^3k}\widetilde{d^3q}\left(a^\dagger_ka_qe^{i(q-k)x}+a_qb_ke^{i(q+k)x}+a^\dagger_kb^\dagger_qe^{-i(q+k)x}+[b^\dagger_qb_k-2\omega_k(2\pi)^3\delta^3(\vec{k}-\vec{q})]e^{-i(q-k)x}\right)
\end{align}
then
\begin{align}
H_{a^\dagger a}&=\int\widetilde{d^3k}\widetilde{d^3q}\int d^3x\left[(-\omega_k\omega_q)[a^\dagger_qa_k-2\omega_k(2\pi)^3\delta^3(\vec{k}-\vec{q})]e^{-i(q-k)x}\right]\\
&\quad+\int\widetilde{d^3k}\widetilde{d^3q}\int d^3x(k^aq_a)\left[-a^\dagger_ka_qe^{i(q-k)x}\right]+m^2a^\dagger_ka_qe^{i(q-k)x}\\
&=\int\widetilde{d^3k}\widetilde{d^3q}\;a^\dagger_ka_q\left[-\omega_k\omega_q-k^aq_a+m^2\right]\int d^3xe^{i(q-k)x}\\
&\quad-\int\widetilde{d^3k}\widetilde{d^3q}\;(-\omega_k\omega_q)2\omega_q(2\pi)^3\delta^3(\vec{q}-\vec{k})\int d^3x\;e^{i(q-k)x}\\
&=\int\widetilde{d^3k}\frac{d^3q}{(2\pi)^32\omega_q}\;a^\dagger_ka_q\left[-\omega_k\omega_q-k^aq_a+m^2\right](2\pi)^3\delta^3(\vec{q}-\vec{k})e^{-i(\omega_q-\omega_k)t}\\
&\quad-\int\frac{d^3k}{(2\pi)^32\omega_k}\frac{1}{(2\pi)^32\omega_k}\;(-\omega_k^2)2\omega_k(2\pi)^3e^{-i(\omega_k-\omega_k)t}\int d^3x \\
&=\int\widetilde{d^3k}\frac{1}{2\omega_k}\;a^\dagger_ka_k\underbrace{\left[-\omega^2_k-\vec{k}^2+m^2\right]}_{2\omega_k^2!?!?!}+\frac{V}{2(2\pi)^3}\int d^3k\;\omega_k\\
&=\int\widetilde{d^3k}\omega_k\;a^\dagger_ka_k+\frac{V}{2(2\pi)^3}\int d^3k\;\omega_k
\end{align}
and similar for $H_{b^\dagger b},H_{ab},H_{a^\dagger b^\dagger}$.
\begin{align}
H=\int\widetilde{d^3k}\omega_k\;(a^\dagger_ka_k+b^\dagger_kb_k)+\frac{V}{2(2\pi)^3}\int d^3k\;\omega_k
\end{align}
\subsection{Problem 4.1 - Commutator non-hermitian field}
With $t=t'$ and $|\vec{x}-\vec{x}'|=r$ we have 
\begin{align}
[\varphi^+(x),\varphi^-(x')]_\pm
&=\int\widetilde{dk}e^{ik(x-x')}\\
&=\int d^3k\frac{1}{(2\pi)^3 2\omega_k}e^{ik(x-x')}\\
&=\frac{1}{2\cdot8\pi^3}\int d^3k\frac{1}{\sqrt{|k|^2+m^2}}e^{i[\vec{k}(\vec{x}-\vec{x}')]}\\
&=\frac{1}{16\pi^3}\int |k|^2dkd\phi d\theta\sin\theta\frac{1}{\sqrt{|k|^2+m^2}}e^{i|k|r\cos\theta}\\
&=\frac{2\pi}{16\pi^3}\int |k|^2dk \underbrace{d\theta\sin\theta}_{-d\cos\theta}\frac{1}{\sqrt{|k|^2+m^2}}e^{i|k|r\cos\theta}\\
&=\frac{2\pi}{16\pi^3}\int |k|^2dk \frac{1}{\sqrt{|k|^2+m^2}}\int_{-1}^1d\cos\theta e^{i|k|r\cos\theta}\\
&=\frac{2\pi}{16\pi^3}\int |k|^2dk \frac{1}{\sqrt{|k|^2+m^2}}2\frac{\sin(|k|r)}{|k|r}\\
&=\frac{1}{4\pi^2r}\int_0^\infty dk \frac{|k|\sin(|k|r)}{\sqrt{|k|^2+m^2}}
\end{align}
With Gradshteyn, Ryzhik 7ed (8.486) - we find for the definition of the modified Bessel function $K_1$ 
\begin{align}
\frac{d}{dz}K_0(z)=-K_1(z)
\end{align}
and Gradshteyn, Ryzhik 7ed (3.754)
\begin{align}
\int_0^\infty dx \frac{\cos(ax)}{\sqrt{\beta^2+x^2}}=K_0(a\beta)
\end{align}
therefore
\begin{align}
\frac{d}{da}K_0(a\beta)&=\int_0^\infty dx \frac{-x\sin(ax)}{\sqrt{\beta^2+x^2}}\\
&=\beta K_0'(a\beta)\\
&=-\beta K_1(a\beta)\\
&\rightarrow K_1(a\beta)=\frac{1}{\beta}\int_0^\infty dx \frac{x\sin(ax)}{\sqrt{\beta^2+x^2}}
\end{align}
which we can use to finish the calculation
\begin{align}
[\varphi^+(x),\varphi^-(x')]_\pm=\frac{1}{4\pi^2r}mK_1(mr)
\end{align}
From \texttt{https://dlmf.nist.gov/10.30} we get
\begin{align}
&\lim_{z\rightarrow0}K_\nu(z)\sim\frac{1}{2}\Gamma(\nu)\left(\frac{1}{2}z\right)^{-\nu}\\
\rightarrow&\lim_{z\rightarrow0}K_1(z)\sim\frac{1}{2}\left(\frac{1}{2}z\right)^{-1}=1/z
\end{align}
and therefore
\begin{align}
[\varphi^+(x),\varphi^-(x')]_\pm=\frac{1}{4\pi^2r^2}.
\end{align}

\subsection{Problem 5.1 - LSZ reduction for complex scalar field}
From Exercise 3.5 we have
\begin{align}
a_q&=i\int d^3x e^{-iqx}\stackrel{\leftrightarrow}{\partial_0}\varphi(\vec{x},t)\\
a_q^\dagger&=-i\int d^3x e^{iqx}\stackrel{\leftrightarrow}{\partial_0}\varphi^\dagger(\vec{x},t)\\
b_q&=i\int d^3x e^{-iqx}\stackrel{\leftrightarrow}{\partial_0}\varphi^\dagger(\vec{x},t)\\
b_q^\dagger&=-i\int d^3x e^{iqx}\stackrel{\leftrightarrow}{\partial_0}\varphi(\vec{x},t)
\end{align}
then
\begin{align}
a^\dagger_1(+\infty)-a^\dagger_1(-\infty)
&=-i\int d^3k f_1(\vec{k})\int d^4x e^{ikx}(-\Box_x+m^2)\varphi^\dagger(x)
\end{align}
rearranging leads to
\begin{align}
a^\dagger_1(-\infty)&=a^\dagger_1(+\infty)+i\int d^3k f_1(\vec{k})\int d^4x e^{ikx}(-\Box_x+m^2)\varphi^\dagger(x)\\
a_1(+\infty)&=a_1(-\infty)+i\int d^3k f_1(\vec{k})\int d^4x e^{-ikx}(-\Box_x+m^2)\varphi(x)\\
b^\dagger_1(-\infty)&=b^\dagger_1(+\infty)+i\int d^3k f_1(\vec{k})\int d^4x e^{ikx}(-\Box_x+m^2)\varphi^\dagger(x)\\
b_1(+\infty)&=b_1(-\infty)+i\int d^3k f_1(\vec{k})\int d^4x e^{-ikx}(-\Box_x+m^2)\varphi(x)
\end{align}
then we get for $a,b$ particle scattering with the time ordering operator $T$ (Later time to the Left)
\begin{align}
\langle f|i\rangle
&=\langle 0|a_{1'}(+\infty)b_{2'}(+\infty)a^\dagger_1(-\infty)b^\dagger_2(-\infty)|0\rangle\\
&=\langle 0|Ta_{1'}(+\infty)b_{2'}(+\infty)a^\dagger_1(-\infty)b^\dagger_2(-\infty)|0\rangle\\
&=\langle 0|T(a_{1'}(-\infty)+i\int)(b_{2'}(-\infty)+i\int)(a^\dagger_{1}(+\infty)+i\int)(b^\dagger_{2}(+\infty)+i\int)|0\rangle\\
&=i^4
\int d^4x'_1e^{-ik'_1x'_1}(-\Box_{x'_1}+m_a^2)
\int d^4x'_2e^{-ik'_2x'_2}(-\Box_{x'_2}+m_b^2)\times\\
&\quad\times
\int d^4x_1e^{-ik_1x_1}(-\Box_{x_1}+m_a^2)
\int d^4x_2e^{-ik_2x_2}(-\Box_{x_2}+m_b^2)\langle0|\phi_{x'_1}\phi_{x'_2}\phi^\dagger_{x_1}\phi^\dagger_{x_2}|0\rangle
\end{align}

\subsection{Problem 6.1 - Path integral in quantum mechanics}
(a) The transition amplitude $\langle q''|e^{-iH(t''-t')}|q'\rangle$ (particle to start at $q',t'$ and ends at position $q''$ at time $t''$) can be written in the Heisenberg picture as
\begin{align}
    \langle q''|e^{-iH(t''-t')}|q'\rangle
    &=\langle q''|e^{-iHt''}e^{iHt''}e^{-iH(t''-t')}e^{-iHt'}e^{iHt'}|q'\rangle\\
    &=\langle q'',t''|e^{iHt''}e^{iH(t''-t')}e^{-iHt'}|q',t'\rangle\\
    &=\langle q'',t''|q',t'\rangle.
\end{align}
Now we can do the standard path integral derivation
\begin{align}
    \langle q'',t''|q',t'\rangle
    &=\int\left(\prod_{j=1}^N dq_j\right) \langle q''|e^{-iH\delta t}|q_N\rangle \langle q_N|e^{-iH\delta t}|q_{N-1}\rangle \dots \langle q_1|e^{-iH\delta t}|q'\rangle\\
    &=\int\left(\prod_{j=1}^N dq_j\right) \int\frac{dp_N}{2\pi}e^{-iH(p_N,q_N)\delta t}e^{ip_N(q'-q_N)} \dots  \int\frac{dp'}{2\pi}e^{-iH(p',q')\delta t}e^{ip'(q_1-q')}\\
    &=\int\left(\prod_{j=1}^N dq_j\right)\left(\prod_{k=0}^{N}\frac{dp_k}{2\pi} e^{ip_k(q_{k+1}-q_k)}e^{-iH(p_k,\textcolor{red}{q_k})\delta t}\right)\quad(q_0=q',q_{N+1}=q'')
\end{align}
which under Weyl ordering (see Greiner, Reinhard - field quantization) has to be replaced by
\begin{align}
    \langle q'',t''|q',t'\rangle
    &=\int\left(\prod_{j=1}^N dq_j\right)\left(\prod_{k=0}^{N}\frac{dp_k}{2\pi} e^{ip_k(q_{k+1}-q_k)}e^{-iH(p_k,\textcolor{red}{\bar{q}_k})\delta t}\right)\quad \bar{q}_k=(q_{k+1}+q_k)/2\\
    &=\int\left(\prod_{j=1}^N dq_j\right)\left(\prod_{k=0}^{N}\frac{dp_k}{2\pi} e^{i[p_k\dot{q}_k-H(p_k,\textcolor{red}{\bar{q}_k})]\delta t}\right)\quad \dot{q}_k=(q_{k+1}-q_k)/{\delta t}\\
    &=\int\left(\prod_{j=1}^N dq_j\right)\left(\prod_{k=0}^{N}\frac{dp_k}{2\pi}\right) \left(e^{i\sum_{n=0}^N[p_n\dot{q}_n-H(p_n,\textcolor{red}{\bar{q}_n})]\delta t}\right)\\
    &=\int\mathcal{D}q\mathcal{D}p\exp\left[i\int_{t'}^{t''}dt\left(p(t)\dot{q}(t)-H(p(t),q(t))\right)\right]
\end{align}
Let's now assume $H(p,q)$ has only a quadratic term in $p$ which is independent of $q$ meaning
\begin{align}
    H(p,q)=\frac{p^2}{2m}+V(q)
\end{align}
then
\begin{align}
\langle q'',t''|q',t'\rangle
        &=\int\left(\prod_{j=1}^N dq_j\right)\left(\prod_{k=0}^{N}\frac{dp_k}{2\pi}\right) \left(e^{i\sum_{n=0}^N[p_n\dot{q}_n-\frac{1}{2m}p_n^2-V(\textcolor{red}{\bar{q}_n})]\delta t}\right)
\end{align}
We can evaluate a single $p$-integral using
\begin{align}
    \int_{-\infty}^\infty dx e^{-ax^2+bx+c}=\sqrt{\frac{\pi}{a}}e^{\frac{b^2}{4a}+c}
\end{align}
and obtain
\begin{align}
        \frac{1}{2\pi}\int_{-\infty}^\infty dp_k \left(e^{i[p_k\dot{q}_k-\frac{1}{2m}p_k^2-V(\textcolor{red}{\bar{q}_k})]\delta t}\right)
        &=\frac{1}{2\pi}e^{-iV(\textcolor{red}{\bar{q}_k})\delta t}\int dp_k \left(e^{i[p_k\dot{q}_k-\frac{1}{2m}p_k^2]\delta t}\right)\\
        &=\frac{1}{2\pi}e^{-iV(\textcolor{red}{\bar{q}_k})\delta t}\sqrt{\frac{\pi}{i\frac{\delta t}{2m}}} e^{\frac{-\dot{q}_k^2\delta t^2}{4\frac{i\delta t}{2m}}}\\
        &=\frac{1}{2\pi}\sqrt{\frac{2\pi m}{i\delta t}} e^{i\left(\frac{m\dot{q}_k^2}{2}-V(\textcolor{red}{\bar{q}_k})\right)\delta t}\\
        &=\sqrt{\frac{m}{2\pi i\delta t}} e^{iL(\bar{q}_k,\dot{q}_k)\delta t}.
\end{align}
As there are $N+1$ $p$-integrals we have
\begin{align}
    \mathcal{D}q=\left(\frac{m}{2\pi i\delta t}\right)^{(N+1)/2}\prod_{j=1}^N dq_j
\end{align}
(b) We now assume $V(q)=0$
\begin{align}
\langle q'',t''|q',t'\rangle
&=\int\mathcal{D}q\,e^{i\int_{t'}^{t''}dt\frac{\dot{q}^2}{2m}}\\
&=\lim_{N\rightarrow\infty}\left(\frac{m}{2\pi i\delta t}\right)^\frac{N+1}{2}\left(\prod_{j=1}^N \int_{-\infty}^{\infty}dq_j\, e^{im\frac{(q_j-q_{j+1})^2}{2\delta t^2}\delta t}\right)e^{im\frac{(q'-q_1)^2}{2\delta t}}e^{im\frac{(q_N-q'')^2}{2\delta t}}\\
&=\lim_{N\rightarrow\infty}\left(\frac{m}{2\pi i\delta t}\right)^\frac{N+1}{2}\left(\prod_{j=3}^N \int_{-\infty}^{\infty}dq_j\, e^{im\frac{(q_j-q_{j+1})^2}{2\delta t}}\right)\int dq_2e^{im\frac{(q_2-q_3)^2}{2\delta t}}\int dq_1e^{im\frac{(q_1-q_2)^2}{2\delta t}}e^{im\frac{(q_0-q_1)^2}{2\delta t}}
\end{align}
now we can simplify the $q_1$-integral
\begin{align}
\int_{-\infty}^{\infty}dq_1\,e^{im\frac{(q_1-q_2)^2}{2\delta t}}e^{im\frac{(q_0-q_1)^2}{2\delta t}}
&=\int_{-\infty}^{\infty}dq_1\,e^{\frac{im}{2\delta t}(q_0^2-2q_0q_1+q_1^2+q_1^2-2q_1q_2+q_2^2)}\\
&=e^{\frac{im}{2\delta t}(q_0^2+q_2^2)}\int_{-\infty}^{\infty}dq_1\,e^{\frac{im}{\delta t}(q^2_1-q_1(q_2+q_0))}\\
&=e^{\frac{im}{2\delta t}(q_0^2+q_2^2)}\sqrt{\frac{\pi\delta t}{m}}e^{\frac{i}{4}\left(\pi-\frac{(q_2+q_0)^2m}{\delta t}\right)}\\
&=e^{\frac{im}{4\delta t}(q_0-q_2)^2}\sqrt{\frac{\pi\delta t}{m}}\sqrt{i}\\
&=e^{\frac{im}{4\delta t}(q_0-q_2)^2}\sqrt{\frac{i\pi\delta t}{m}}
\end{align}
now simplify the $q_2$-integral
\begin{align}
\sqrt{\frac{i\pi\delta t}{m}}\int_{-\infty}^\infty dq_2e^{\frac{im}{2\delta t}(q_2-q_3)^2}e^{\frac{im}{4\delta t}(q_0-q_2)^2}
&=\sqrt{\frac{i\pi\delta t}{m}}\int_{-\infty}^\infty dq_2e^{\frac{im}{4\delta t}(2q_2^2-4q_3q_2+2q_3^2+q_0^2-2q_0q_2+q_2^2)}\\
&=\sqrt{\frac{i\pi\delta t}{m}}\int_{-\infty}^\infty dq_2e^{\frac{im}{4\delta t}(3q_2^2-(4q_3+2q_0)q_2+2q_3^2+q_0^2)}\\
&=\sqrt{\frac{i\pi\delta t}{m}}e^{\frac{im}{4\delta t}(2q_3^2+q_0^2)}\int_{-\infty}^\infty dq_2e^{\frac{im}{4\delta t}(3q_2^2-(4q_3+2q_0)q_2)}\\
&=\sqrt{\frac{i\pi\delta t}{m}}e^{\frac{im}{4\delta t}(2q_3^2+q_0^2)}\sqrt{\frac{\pi4\delta t}{3m}}e^{\frac{i}{4}\left(\pi-\frac{(4q_3+2q_0)^2m}{12\delta t}\right)}\\
&=\sqrt{\frac{i\pi\delta t}{m}}\sqrt{\frac{4i\pi\delta t}{3m}}e^{\frac{im}{6\delta t}(q_3-q_0)^2}
\end{align}
then we can extend the results (without explicitly proving)
\begin{align}
\langle q'',t''|q',t'\rangle
&=\lim_{N\rightarrow\infty}\left(\frac{m}{2\pi i\delta t}\right)^\frac{N+1}{2}\prod_{j=1}^N\sqrt{\frac{2i\pi\delta t}{m}\frac{j}{j+1}}\cdot e^{\frac{im}{2(j+1)\delta t}(q''-q')^2}\\
&=\lim_{N\rightarrow\infty}\sqrt{\frac{m}{2\pi i\delta t}}\sqrt{\frac{1}{N+1}}\cdot e^{\frac{im}{2(N+1)\delta t}(q_{N+1}-q_0)^2}\\
&=\sqrt{\frac{m}{2\pi i(t''-t')}}\cdot e^{\frac{im(q''-q')^2}{2(t''-t')}}.
\end{align}
The exponent has the dimension $\text{kg}\cdot\text{m}^2/s$ which is the same as Js. So we just insert an $\hbar$
\begin{align}
\langle q'',t''|q',t'\rangle
&=\sqrt{\frac{m}{2\pi i\hbar(t''-t')}}\cdot e^{\frac{im(q''-q')^2}{2\hbar(t''-t')}}.
\end{align}
(c) Simple - with $H|k\rangle=\frac{k^2}{2m}|k\rangle$ we get
\begin{align}
\langle q'',t''|q',t'\rangle
&=\langle q''|\exp(-iH(t''-t'))|q'\rangle\\
&=\int dp\int dk\langle q''|p\rangle\langle p|\exp(-iH(t''-t'))|k\rangle\langle k|q'\rangle\\
&=\int dp\int dk\frac{1}{\sqrt{2\pi}}e^{ipq'}\langle p|k\rangle\exp(-i\frac{k^2}{2m}(t''-t'))\frac{1}{\sqrt{2\pi}}e^{-ikq''}\\
&=\int dp\int dk\frac{1}{\sqrt{2\pi}}e^{ipq'}\exp(-i\frac{k^2}{2m}(t''-t'))\delta(k-p)\frac{1}{\sqrt{2\pi}}e^{-ikq''}\\
&=\frac{1}{2\pi}\int dpe^{ip(q'-q'')}\exp(-i\frac{p^2}{2m}(t''-t'))\\
&=\frac{1}{2\pi}\sqrt{-\frac{2m\pi}{t''-t'}}e^{\frac{i}{4}\left(\pi-\frac{-2m(q''-q')^2}{t''-t'}\right)}\\
&=\sqrt{-\frac{im}{2\pi(t''-t')}}e^{-\frac{i}{4}\frac{-2m(q''-q')^2}{t''-t'}}\\
&=\sqrt{\frac{m}{2\pi i(t''-t')}}e^{\frac{-im(q''-q')^2}{2(t''-t')}}
\end{align}
which is the same as in (b).

\subsection{Problem 7.1 - Oscillator Green's function I}
\begin{align}
G(t-t')&=\int_{-\infty}^{+\infty}\frac{dE}{2\pi}\frac{e^{-iE(t-t')}}{-E^2+\omega^2-i\epsilon}\\
&=-\frac{1}{2\pi}\int_{-\infty}^{+\infty}dE\frac{e^{-iE(t-t')}}{E^2-\omega^2+i\epsilon}
\end{align}
with
\begin{align}
E^2-\omega^2+i\epsilon
&=(E+\sqrt{\omega^2-i\epsilon})(E-\sqrt{\omega^2-i\epsilon})\\
&=\left(E+\omega\sqrt{1-\frac{i\epsilon}{\omega^2}}\right)\left(E-\omega\sqrt{1-\frac{i\epsilon}{\omega^2}}\right)\\
&\simeq\left(E+\omega-\frac{i\epsilon}{2\omega}\right)\left(E-\omega+\frac{i\epsilon}{2\omega^2}\right)
\end{align}
we can simplify
\begin{align}
G(\Delta t)
&=-\frac{1}{2\pi}\int_{-\infty}^{+\infty}dEe^{-iE\Delta t}\left(\frac{1}{E+\omega-\frac{i\epsilon}{2\omega}}+\frac{1}{E-\omega+\frac{i\epsilon}{2\omega}}\right)\\
&=-\frac{1}{2\pi}\frac{1}{2\left(\omega-\frac{i\epsilon}{2\omega}\right)}\int_{-\infty}^{+\infty}dEe^{-iE\Delta t}\left(-\frac{1}{E+\omega-\frac{i\epsilon}{2\omega}}+\frac{1}{E-\omega+\frac{i\epsilon}{2\omega}}\right)
\end{align}
Integrating along the closed contour along the lower half plane (seeing that the exponential function makes the arc part vanish - for $\Delta t>0$) and using the residual theorem (only one pole is inside) we get (with $\epsilon\rightarrow0$)
\begin{align}
G(\Delta t)
&=+\frac{1}{2\pi}\frac{1}{2\left(\omega-\frac{i\epsilon}{2\omega}\right)}(2\pi i)e^{-i(\omega-\frac{i\epsilon}{2\omega})\Delta t}\\
&=\frac{i}{2\omega}e^{-i\omega\Delta t}
\end{align}
For $\Delta t<0$ we integrate along the contour of the upper plane - combining both results we get
\begin{align}
G(t)&=\frac{i}{2\omega}e^{-i\omega|t|}
\end{align}


\subsection{Problem 7.2 - Oscillator Green's function II}
We can rewrite the Greens function using the Heaviside theta function 
\begin{align}
|t|
&=(2\theta(t)-1)t\\
\frac{d}{dt}|t|
&=2\theta'(t)t+(2\theta(t)-1)\\
&=2\underbrace{\delta(t)t}_{=0}+2\theta(t)-1\\
&=2\theta(t)-1
\end{align}
and then differentiate and use $\theta'(t)=\delta(t)$
\begin{align}
G(t)=&\frac{i}{2\omega}e^{-i\omega(2\theta(t)-1)t}\\
\partial_t G(t)
&=\frac{i}{2\omega}e^{-i\omega(2\theta(t)-1)t}(-i\omega)(2\theta(t)-1))\\
&=(-i\omega)G(t)\left(2\theta(t)-1\right)\\
\partial_{tt} G(t)
&=(-i\omega)\partial_tG(t)\left(2\theta(t)-1\right)+(-2i\omega)G(t)\delta(t)\\
&=(-i\omega)^2G(t)\left(2\theta(t)-1\right)^2+(-2i\omega)G(t)\delta(t)\\
&=-\omega^2G(t)+e^{-i\omega|t|}\delta(t)
\end{align}
where we used $\left(2\theta(t)-1\right)^2\equiv1$
\begin{align}
\left(\partial_{tt}+\omega^2\right)G(t)
&=\left(-\omega^2+\omega^2\right)G(t)+\delta(t)=\delta(t)
\end{align}

\subsection{Problem 7.3 - Harmonic Oscillator  - Heisenberg and Schroedinger picture}
(a) With $\hbar=1$ and
\begin{align}
H&=\frac{1}{2}P^2+\frac{1}{2}m\omega^2Q^2\\
[Q,P]&=QP-PQ=i\\
[Q,Q]&=[P,P]=0
\end{align}
we obtain for the commutators
\begin{align}
[P^2,Q]
&=P(PQ)-QP^2\\
&=P(QP-i)-QP^2\\
&=(PQ)P-Pi-QP^2\\
&=(QP-i)P-Pi-QP^2\\
&=-2Pi\\
[Q^2,P]
&=Q(QP)-PQ^2\\
&=Q(PQ+i)-PQ^2\\
&=(QP)Q+iQ-PQ^2\\
&=(PQ+i)Q+iQ-PQ^2\\
&=2Qi
\end{align}
Then the Heisenberg equations are
\begin{align}
\dot Q(t)&=i[H,Q(t)]=i\frac{1}{2m}[P^2(t),Q(t)]=\frac{1}{m}P(t)\\
\dot P(t)&=i[H,P(t)]=i\frac{1}{2}m\omega^2[Q^2(t),P(t)]=-m\omega^2Q(t)\\
\rightarrow &\ddot Q(t)=\frac{1}{m}\dot P(t)=-\omega^2 Q(t)
\end{align}
with the solutions (initial conditions $Q(0)=Q, P(0)=P$)
\begin{align}
Q(t)&=A\cos\omega t+B\sin\omega t\qquad\rightarrow A=Q,\quad\omega B=\frac{1}{m}P\\
&=Q\cos\omega t+\frac{1}{\omega m}P\sin\omega t\\
P(t)&=m\dot Q(t)\\
&=-m\omega Q\sin\omega t+P\cos\omega t
\end{align}
(b) Using Diracs trick from QM (rewriting $H$ in terms of $a$ and $a^\dagger$)
\begin{align}
a&=\sqrt{\frac{m\omega}{2}}(Q+\frac{i}{m\omega}P)\\
a^\dagger&=\sqrt{\frac{m\omega}{2}}(Q-\frac{i}{m\omega}P)
\end{align}
we can invert the relation
\begin{align}
Q&=\frac{1}{\sqrt{2m\omega}}(a^\dagger+a)\\
P&=i\sqrt{\frac{m\omega}{2}}(a^\dagger-a)
\end{align}
and
\begin{align}
Q(t)&=Q\cos\omega t+\frac{1}{\omega m}P\sin\omega t\\
&=\frac{1}{\sqrt{2m\omega}}(a^\dagger+a)\cos\omega t+\frac{1}{\omega m}i\sqrt{\frac{m\omega}{2}}(a^\dagger-a)\sin\omega t\\
&=\frac{1}{\sqrt{2m\omega}}\left((a^\dagger+a)\cos\omega t+i(a^\dagger-a)\sin\omega t\right)\\
&=\frac{1}{\sqrt{2m\omega}}\left(a^\dagger(\cos\omega t+i\sin\omega t)+a(\cos\omega t-i\sin\omega t) \right)\\
&=\frac{1}{\sqrt{2m\omega}}\left(a^\dagger e^{i\omega t}+a e^{-i\omega t} \right)\\
P(t)&=i\sqrt{\frac{m\omega}{2}}\left(a^\dagger e^{i\omega t}-a e^{-i\omega t} \right)\\
\end{align}
(c) Now with $t_1<t_2$ and the time ordering operator (larger time to the left)
\begin{align}
\langle0|TQ(t_1)Q(t_2)|0\rangle
&=\frac{1}{2m\omega}\langle0|T\left(a^\dagger e^{i\omega t_1}+a e^{-i\omega t_1} \right)\left(a^\dagger e^{i\omega t_2}+a e^{-i\omega t_2} \right)|0\rangle\\
&=\frac{1}{2m\omega}\langle0|\left(a^\dagger e^{i\omega t_2}+a e^{-i\omega t_2} \right)\left(a^\dagger e^{i\omega t_1}+a e^{-i\omega t_1} \right)|0\rangle\\
&=\frac{1}{2m\omega}\langle0|a e^{-i\omega t_2}a^\dagger e^{i\omega t_1}|0\rangle
\end{align}
all other terms are vanishing because of $a|0\rangle=0$ and $\langle0|a^\dagger=0$. Then
\begin{align}
\langle0|TQ(t_1)Q(t_2)|0\rangle
&=\frac{1}{2m\omega}e^{-i\omega (t_2-t_1)}\underbrace{\langle0|a a^\dagger |0\rangle}_{=1}\\
&=\frac{1}{2m\omega}e^{-i\omega (t_2-t_1)}\\
&\equiv\frac{1}{i}G(t_2-t_1)
\end{align}
And now the next case with $t_1>t_2>t_3>t_4$
\begin{align}
\langle0|TQ(t_1)Q(t_2)Q(t_3)Q(t_4)|0\rangle
&=\frac{1}{(2m\omega)^2}...
\end{align}

\subsection{Problem 7.4 - Harmonic Oscillator with perturbation}
As $f(t)$ is a real function we have $\tilde f(-E)=(\tilde f(E))^*$ then with (7.10)
\begin{align}
\langle0|0\rangle_f
&=\exp\left[\frac{i}{2}\int_{-\infty}^{+\infty}\frac{dE}{2\pi}\frac{\tilde f(E)\tilde f(-E)}{-E^2+\omega^2-i\epsilon}\right]\\
&=\exp\left[\frac{i}{2}\int_{-\infty}^{+\infty}\frac{dE}{2\pi}\frac{\tilde f(E)\tilde f(E)^*}{-E^2+\omega^2-i\epsilon}\right]
\end{align}
But we actually need to calculate $|\langle0|0\rangle_f|^2$ therefore we observe with
\begin{align}
e^{iz}&=e^{i(x+iy)}=e^{-y}e^{ix}=e^{-y}(\cos x+i\sin x)\\
&\quad\rightarrow\quad (e^{iz})^*=e^{-y}(\cos x-i\sin x)=e^{-y-ix}e^{-i(x-iy)}=e^{-iz^*}\\
\langle0|0\rangle_f&=e^{iA}\quad\rightarrow\quad|\langle0|0\rangle_f|^2=e^{iA}(e^{iA})^*=e^{iA}e^{-iA^*}=e^{i(A-A^*)}=e^{-2\Im A}
\end{align}
Now we calculate the imaginary part of the integral
\begin{align}
\Im\frac{1}{4\pi}\int_{-\infty}^{+\infty}dE\frac{\tilde f(E)\tilde f(E)^*}{-E^2+\omega^2-i\epsilon}
&=\frac{1}{4\pi}\int_{-\infty}^{+\infty}dE\Im\frac{\tilde f(E)\tilde f(E)^*}{-E^2+\omega^2-i\epsilon}\\
&=\frac{1}{4\pi}\int_{-\infty}^{+\infty}dE\tilde f(E)\tilde f(E)^*\Im\frac{1}{-E^2+\omega^2-i\epsilon}\\
&=\frac{1}{4\pi}\int_{-\infty}^{+\infty}dE\tilde f(E)\tilde f(E)^*\Im\frac{-E^2+\omega^2+i\epsilon}{(-E^2+\omega^2)^2+\epsilon^2}\\
&=\frac{1}{4\pi}\int_{-\infty}^{+\infty}dE\tilde f(E)\tilde f(E)^*\frac{\epsilon}{(-E^2+\omega^2)^2+\epsilon^2}\\
&\simeq\frac{1}{4\pi}\int_{-\infty}^{+\infty}dE\tilde f(E)\tilde f(E)^*\pi\delta(-E^2+\omega^2)\\
&\simeq\frac{1}{4\pi}\int_{-\infty}^{+\infty}dE\tilde f(E)\tilde f(E)^*\pi\delta((\omega+E)(\omega-E))\\
&\simeq\frac{1}{4\cdot 2\omega}(\tilde f(\omega)\tilde f(\omega)^*+\tilde f(-\omega)\tilde f(-\omega)^*)\\
&\simeq\frac{1}{8\omega}(\tilde f(\omega)\tilde f(\omega)^*+\tilde f(\omega)^*\tilde f(\omega))\\
&\simeq\frac{1}{4\omega}\tilde f(\omega)\tilde f(\omega)^*
\end{align}
then
\begin{align}
|\langle0|0\rangle_f|^2
&=e^{-2\left(\frac{1}{4\omega}\right)\tilde f(\omega)\tilde f(\omega)^*}\\
&=e^{-\frac{1}{2\omega}\tilde f(\omega)\tilde f(\omega)^*}\\
\end{align}

\subsection{Problem 8.1 - Feynman propagator is Greens function Klein-Gordon equation}
With
\begin{align}
\Delta(x-x')=\frac{1}{(2\pi)^4}\int d^4k\frac{e^{ik(x-x')}}{k^2+m^2-i\epsilon}
\end{align}
we have
\begin{align}
(-\partial_x^2+m^2)\Delta(x-x')
&=\frac{1}{(2\pi)^4}\int d^4k(-i^2k^2+m^2)\frac{e^{ik(x-x')}}{k^2+m^2-i\epsilon}\\
&=\frac{1}{(2\pi)^4}\int d^4k\frac{k^2+m^2}{k^2+m^2-i\epsilon}e^{ik(x-x')}\\
&\simeq\frac{1}{(2\pi)^4}\int d^4ke^{ik(x-x')}\\
&=\delta^4(x-x')
\end{align}

\subsection{Problem 8.2 - Feynman propagator II}
With $\widetilde{dk}=d^3k/((2\pi)^32\omega_k)$ and $\omega_k=\sqrt{\vec{k}^2+m^2}$
\begin{align}
\Delta(x-x')
&=\frac{1}{(2\pi)^4}\int d^4k\frac{e^{ik(x-x')}}{k^2+m^2-i\epsilon}\\
&=\frac{1}{(2\pi)^4}\int d^3k\int dk^0e^{-ik^0(t-t')}\frac{e^{i\vec{k}(\vec{x}-\vec{x}')}}{-(k^0)^2+\vec{k}^2+m^2-i\epsilon}\\
&=\frac{1}{(2\pi)^4}\int d^3k\,e^{i\vec{k}(\vec{x}-\vec{x}')}\int dE\frac{e^{-iE(t-t')}}{-E^2+\vec{k}^2+m^2-i\epsilon}\\
&=\frac{1}{(2\pi)^4}\int d^3k\,e^{i\vec{k}(\vec{x}-\vec{x}')}2\pi\frac{i}{2(\vec{k}^2+m^2)}e^{-i(\vec{k}^2+m^2)|t-t'|}
\end{align}
where we used exercise (7.1). Then
\begin{align}
\Delta(x-x')
&=\frac{i}{(2\pi)^3}\int d^3k\,e^{i\vec{k}(\vec{x}-\vec{x}')}\frac{i}{2\omega_k}e^{-i\omega_k|t-t'|}\\
&=i\int\frac{d^3k}{(2\pi)^3 2\omega_k}\,e^{i\vec{k}(\vec{x}-\vec{x}')}e^{-i\omega_k|t-t'|}\\
&=i\int\widetilde{dk}\,e^{i\vec{k}(\vec{x}-\vec{x}')-i\omega_k|t-t'|}\\
&=i\theta(t-t')\int\widetilde{dk}\,e^{i\vec{k}(\vec{x}-\vec{x}')-i\omega_k(t-t')}+i\theta(t'-t)\int\widetilde{dk}\,e^{i\vec{k}(\vec{x}-\vec{x}')+i\omega_k(t-t')}\\
&=i\theta(t-t')\int\widetilde{dk}\,e^{ik(x-x')}+i\theta(t'-t)\int\widetilde{dk}\,e^{-i\vec{k}(\vec{x}-\vec{x}')+i\omega_k(t-t')}\\
&=i\theta(t-t')\int\widetilde{dk}\,e^{ik(x-x')}+i\theta(t'-t)\int\widetilde{dk}\,e^{-ik(x-x')}\\
\end{align}

\section{{\sc Coleman} - Lectures of Sidney Coleman on quantum field theory}
\subsection{Problem 1.1 - Momentum space measure}
Boost in $z$-direction
\begin{align}
p_\mu&=\Lambda^\nu_{\,\mu}p_\nu'\\ 
\Lambda&=\left(\begin{array}{cccc}
\gamma & 0 & 0 & -\gamma\beta\\
0 & 1 & 0 & 0\\
0 & 0 & 1 & 0\\
-\gamma\beta & 0 & 0 & \gamma
\end{array}\right)
\end{align}
Combining everything using $dp_i\wedge\,dp_i=0$
\begin{align}
\rightarrow dp_x&=dp_x'\\
\rightarrow dp_y&=dp_y'\\
\rightarrow dp_z
&=-\gamma\beta\,dp_0'+\gamma\,dp_z'\\
&=-\gamma\beta\left(\frac{\partial p_0'}{\partial p_x'}dp_x'+\frac{\partial p_0'}{\partial p_y'}dp_y'+\frac{\partial p_0'}{\partial p_z'}dp_z'\right)+\gamma\,dp_z'\\
&=-\gamma\beta\frac{1}{2\omega_p'}(2p_x'\,dp_x'+2p_y'\,dp_y'+2p_z'\,dp_z')+\gamma\,dp_z'\\
&=-\gamma\beta\frac{p_x'\,dp_x'+p_y'\,dp_y'}{\omega_p'}+\gamma\left(1-\frac{\beta}{\omega_p'}p_z'\right)\,dp_z'
\end{align}
where we used $p_0'=\omega_p'=\sqrt{m^2-{p_x'}^2-{p_y'}^2-{p_z'}^2}$ and 
\begin{align}
\rightarrow \omega_p&=p_0\\
&=\gamma p_0'-\gamma\beta\,p_z'\\
&=\gamma(\omega_p'-\beta\,p_z')
\end{align}
then
\begin{align}
\frac{d^3p}{(2\pi)^3 2\omega_p}
&=\frac{dp_x\,dp_y\,dp_z}{(2\pi)^3 2\omega_p}\\
&=\frac{dp_x'\,dp_y'\gamma\left(1-\frac{\beta}{\omega_p'}p_z'\right)\,dp_z'}{(2\pi)^3 2\gamma(\omega_p'-\beta\,p_z')}\\
&=\frac{dp_x'\,dp_y'\gamma\left(1-\frac{\beta}{\omega_p'}p_z'\right)\,dp_z'}{(2\pi)^3 2\omega_p'\gamma\left(1-\frac{\beta}{\omega_p'}\,p_z'\right)}\\
&=\frac{dp_x'\,dp_y'\,dp_z'}{(2\pi)^3 2\omega_p'}
\end{align}

%%%%%%%%%%%%%%%%%%%%%%%%%%%%%%%%%%%%%%%%%%%%%%%%%%%%%
%%%%%%%%%%% Kachelriess %%%%%%%%%%%%%%%%%%%%%%%%
%%%%%%%%%%%%%%%%%%%%%%%%%%%%%%%%%%%%%%%%%%%%%%%%%%%%%
\section{{\sc Kachelriess} - Quantum Fields - From the Hubble to the Planck scale}
\subsection{Problem 1.1 - Units}
\begin{enumerate}
\item The fundamental constants are given by
\begin{align}
    k    &=1.381\cdot10^{-23} \text{m}^2\text{s}^{-2}\text{kg}^{ 1}\text{K}^{-1}\\
    G    &=6.674\cdot10^{-11} \text{m}^3\text{s}^{-2}\text{kg}^{-1}\\
    \hbar&=1.054\cdot10^{-34} \text{m}^2\text{s}^{-1}\text{kg}^{ 1}\\
    c    &=2.998\cdot10^{-8}  \text{m}^1\text{s}^{-1}
\end{align}
A newly constructed Planck constant has the general form
\begin{align}
    X_P=c^{\alpha_c}\cdot G^{\alpha_G}\cdot \hbar^{\alpha_\hbar}\cdot k^{\alpha_k}
\end{align}
and the dimension of $X_P$ is given by $\text{m}^{\beta_m}\text{s}^{\beta_s}\text{kg}^{\beta_{kg}}\text{K}^{\beta_K}$ are determined by
\begin{align}
    \text{Meter}\quad    &\beta_m=2\alpha_k+3\alpha_G+2\alpha_h+\alpha_c\\
    \text{Second}\quad   &\beta_s=-2\alpha_k-2\alpha_G-\alpha_c-\alpha_h\\
    \text{Kilogram}\quad &\beta_{kg}=\alpha_k-\alpha_G+\alpha_h\\
    \text{Kelvin}\quad   &\beta_K=-\alpha_k
\end{align}
Solving the linear system gives
\begin{align}
    l_P&=\sqrt{\frac{\hbar G}{c^3}}=1.616\cdot 10^{-35}\text{m}\\
    m_P&=\sqrt{\frac{\hbar c}{G}}=2.176\cdot 10^{-8}\text{kg}\\
    t_P&=\sqrt{\frac{\hbar G}{c^5}}=5.391\cdot 10^{-44}\text{s}\\
    T_P&=\sqrt{\frac{\hbar c^5}{Gk^2}}=1.417\cdot 10^{-32}\text{K}\\
\end{align}
As the constants are made up from QM, SR and GR constants they indicate magnitudes at which a quantum theory of gravity is needed to make a sensible predictions.

\item We use the definition $1\text{barn}=10^{-28}\text{m}^2$
\begin{align}
    1\text{cm}^2  &= 10^{-4}\text{m}^2\\
    1\text{mbarn} &= 10^{-31}\text{m}^2\\
                  &= 10^{-27}\text{cm}^2
\end{align}
We also have $1\text{eV} =1.602\cdot10^{-19}\text{As}\cdot1\text{V}=1.602\cdot10^{-19}\text{J}$
\begin{align}
    E=mc^2\quad
    &\rightarrow\quad 1\text{kg}\cdot c^2 = 8.987\cdot10^{16}\text{J}=5.609\cdot10^{35}\text{eV}\\
    &\rightarrow\quad 1\text{GeV} = 1.782\cdot10^{-27}\text{kg}\\
    E=\hbar\omega\quad
    &\rightarrow\quad \frac{1}{1\text{s}}\cdot \hbar = 1.054\cdot10^{-34}\text{J}=6.582\cdot10^{-16}\text{eV}\\
    &\rightarrow\quad 1\text{GeV}^{-1} = 6.582\cdot10^{-25}\text{s}\\
    E= \frac{\hbar c}{\lambda}\quad
    &\rightarrow\quad \frac{1}{1\text{m}}\cdot \hbar c = 3.161\cdot10^{-26}\text{J}=1.973\cdot10^{-7}\text{eV}\\
    &\rightarrow\quad 1\text{GeV}^{-1} = 1.973\cdot10^{-16}\text{m}\\
    E\sim pc\quad
    &\rightarrow\quad 1\text{kg}\text{m}\text{s}^{-1}\cdot c= 2.998\cdot10^{8}\text{J}=1.871\cdot10^{27}\text{eV}\\
    &\rightarrow\quad 1\text{GeV} = 5.344\cdot10^{-19}\text{kg}\text{m}\text{s}^{-1}
\end{align}
therefore
\begin{align}
    1\text{GeV}^{-2} &= (1.973\cdot10^{-16}\text{m})^2\\
    &=3.893\cdot10^{-32}\text{m}^2\\
    &=0.389\text{mbarn}
\end{align}
\end{enumerate}

\subsection{Problem 3.2 - Maxwell Lagrangian}
\begin{enumerate}
    \item First we observe that
    \begin{align}
        F_{\mu\nu}F^{\mu\nu}
        &=(\partial_\mu A_\nu-\partial_\nu A_\mu)(\partial^\mu A^\nu-\partial^\nu A^\mu)\\
        &=(\partial_\mu A_\nu)(\partial^\mu A^\nu)-(\partial_\mu A_\nu)(\partial^\nu A^\mu)-\underbrace{(\partial_\nu A_\mu)(\partial^\mu A^\nu)}_{=(\partial_\mu A_\nu)(\partial^\nu A^\mu)}+\underbrace{(\partial_\nu A_\mu)(\partial^\nu A^\mu)}_{=(\partial_\mu A_\nu)(\partial^\mu A^\nu)}\\
        &=2\left((\partial_\mu A_\nu)(\partial^\mu A^\nu)-(\partial_\mu A_\nu)(\partial^\nu A^\mu)\right)\\
        &=2(\partial_\mu A_\nu)F^{\mu\nu}.
    \end{align}
    The variation is then given by
    \begin{align}
        \delta\left(F_{\mu\nu}F^{\mu\nu}\right)
        &=2\delta\left((\partial_\mu A_\nu)F^{\mu\nu}\right)\\
        &=2\left[\delta\left(\partial_\mu A_\nu\right)F^{\mu\nu}+(\partial_\mu A_\nu)\delta F^{\mu\nu}\right]\\
        &=2[\delta\left(\partial_\mu A_\nu\right)\underbrace{(\partial^\mu A^\nu-\partial^\nu A^\mu)}_{=F^{\mu\nu}}+(\partial_\mu A_\nu)\underbrace{\left(\delta(\partial^\mu A^\nu-\partial^\nu A^\mu)\right)}_{\delta F^{\mu\nu}}]\\
        &=2\left[\delta\left(\partial_\mu A_\nu\right)\partial^\mu A^\nu-\delta\left(\partial_\mu A_\nu\right)\partial^\nu A^\mu+(\partial_\mu A_\nu)\delta(\partial^\mu A^\nu)-(\partial_\mu A_\nu)\delta(\partial^\nu A^\mu) \right]\\
        &=4\left[\delta\left(\partial_\mu A_\nu\right)\partial^\mu A^\nu-\delta\left(\partial_\mu A_\nu\right)\partial^\nu A^\mu \right]\\
        &=4(\partial^\mu A^\nu-\partial^\nu A^\mu)\;\delta(\partial_\mu A_\nu)\\
        &=4F^{\mu\nu}\;\delta(\partial_\mu A_\nu)\\
        &=4F^{\mu\nu}\;\partial_\mu(\delta A_\nu)
    \end{align}
    We start with the source free Maxwell equations $\partial_\mu F^{\mu\nu}=0$
    \begin{align}
        0
        &=\int_\Omega d^4x\;(\delta A_\nu) \partial_\mu F^{\mu\nu}\\
        &=F^{\mu\nu}(\delta A_\nu)|_{\partial\Omega}-\int_\Omega d^4x\;\underbrace{\partial_\mu(\delta A_\nu)F^{\mu\nu}}_{=\frac{1}{4}\delta(F_{\mu\nu}F^{\mu\nu})}\\
        &=\int_{\Omega}d^4x\;\delta\left(\frac{1}{4}F_{\mu\nu}F^{\mu\nu}\right)
    \end{align}
    and therefore $\mathscr{L}_\text{ph}=\frac{1}{4}F_{\mu\nu}F^{\mu\nu}$.
    \item So we see that the Lagrangian $\mathscr{L}_\text{ph}=\frac{1}{4}F_{\mu\nu}F^{\mu\nu}=2(\partial_\mu A_\nu)F^{\mu\nu}$ yields the inhomogeneous Maxwell equations 
    \begin{align}
        \frac{\partial\mathscr{L}_\text{ph}}{\partial A_\alpha}-\partial_\beta\frac{\partial\mathscr{L}_\text{ph}}{\partial(\partial_\beta A_\alpha)}=0\\
        -\partial_\beta\left[(2\delta_{\alpha\mu}\delta_{\beta\nu}F^{\mu\nu}+2(\partial_\mu A_\nu)(\delta_\alpha^\mu\delta_\beta^\nu-\delta_\alpha^\nu\delta_\beta^\mu)\right]=0\\
        -\partial_\beta\left[(2F^{\alpha\beta}+2(\partial^\alpha A^\beta-\partial^\beta A^\alpha)\right]=0\\
        \partial_\beta(F^{\alpha\beta})=0
    \end{align}
    but not the homogeneous ones. They are fulfilled trivially - by construction of $F^{\mu\nu}$.
    \item The conjugated momentum is given by
    \begin{align}
        \pi_\mu
        &=\frac{\partial\mathscr{L}_\text{ph}}{\partial \dot{A}^\mu}\\
        &= F_{0\mu}
    \end{align}
\end{enumerate}

\subsection{Problem 3.3 - Dimension of \texorpdfstring{$\phi$}{TEXT}}
\begin{enumerate}
    \item With $c=1=\hbar$ we see
    \begin{align}
        E=mc^2        &\rightarrow E\sim M\\
        E=\hbar\omega &\rightarrow T\sim E^{-1}\sim M^{-1}\\
        s=ct          &\rightarrow L\sim T\sim M^{-1}
    \end{align}
    As $\mathscr{L}$ is an action density we have
    \begin{align}
        \mathscr{L}\sim \frac{E\cdot T}{TL^3}\sim M\cdot M^{d-1}=M^d
    \end{align}
    From the explicit form of the scalar Lagrangian we derive
    \begin{align}
        \mathscr{L}\sim \frac{[\phi^2]}{M^{-2}}=[\phi^2]M^{-2}
    \end{align}
    and therefore $[\phi]=M^{(d-2)/2}$
    
    \item Using the previous result we see
    \begin{align}
        \lambda\phi^3: &\qquad M^d \sim [\lambda] M^{3(d-2)/2}\rightarrow d=6\\
        \lambda\phi^4: &\qquad M^d \sim [\lambda] M^{4(d-2)/2}\rightarrow d=4
    \end{align}
    
    \item With
    \begin{align}
        \mathscr{L}
        &=\frac{1}{2}\eta^{\mu\nu}(\partial_\mu\phi)(\partial_\nu\phi)-\frac{1}{2}m^2\phi^2+\lambda\phi^4\\
        &=\frac{1}{2}\eta^{\mu\nu}\left(\partial_\mu\frac{\tilde\phi}{\sqrt{\lambda}}\right)\left(\partial_\nu\frac{\tilde\phi}{\sqrt{\lambda}}\right)-\frac{1}{2}m^2\frac{\tilde\phi^2}{\lambda}+\lambda\frac{\tilde\phi^4}{\lambda^2}\\
        &=\frac{1}{\lambda}\left[\frac{1}{2}\eta^{\mu\nu}(\partial_\mu\tilde\phi)(\partial_\nu\tilde\phi)-\frac{1}{2}m^2\tilde\phi^2+\tilde\phi^4\right]
    \end{align}
\end{enumerate}


\subsection{Problem 3.5 - Yukawa potential}
Integration in spherical coordinates yields (with $x=kr$)
\begin{align}
    \int d^3k\frac{e^{-ik\cdot r}}{k^2+m^2}
    &=2\pi\int  \frac{e^{-ikr\cos{\theta}}}{k^2+m^2} k^2\sin\theta d\theta\, dk\\
    &=-2\pi\int \frac{e^{-ikr\cos{\theta}}}{k^2+m^2} k^2 \; d(\cos\theta)\, dk\\
    &=-2\pi\int \left.\frac{k^2}{ikr}\frac{e^{-ikr\cos{\theta}}}{k^2+m^2} \right|_{-1}^{+1} \; dk\\
    &=-2\pi\int \frac{k}{ir}\frac{e^{-ikr}-e^{+ikr}}{k^2+m^2}  \; dk\\
    &=\frac{4\pi}{r}\int_0^\infty  \frac{k\sin{kr}}{k^2+m^2}  \; dk\\
    &=\frac{4\pi}{r^2}\int_0^\infty  \frac{\frac{x}{r}\sin{x}}{\frac{x^2}{r^2}+m^2}  \; dx\\
    &=\frac{4\pi}{r}\int_0^\infty  \frac{x\sin{x}}{x^2+m^2r^2}  \; dx\\
\end{align}
Now we use a small trick
\begin{align}
    &=\frac{2\pi}{ir}\int_0^\infty  \frac{x(e^{ix}-e^{-ix})}{x^2+m^2r^2}  \; dx\\
    &=\frac{2\pi}{ir}\left[\int_0^\infty\frac{xe^{ix}}{x^2+m^2r^2}  \; dx - \int_0^\infty\frac{xe^{-ix}}{x^2+m^2r^2}  \; dx\right]\\
    &=\frac{2\pi}{ir}\left[\int_0^\infty\frac{xe^{ix}}{x^2+m^2r^2}  \; dx - (-1)^3 \int_{-\infty}^0\frac{ye^{iy}}{y^2+m^2r^2}  \; dy\right]\\
    &=\frac{2\pi}{ir}\int_{-\infty}^\infty\frac{xe^{ix}}{x^2+m^2r^2}  \; dx \\
    &=\frac{2\pi}{ir}\int_{-\infty}^\infty\frac{xe^{ix}}{(x+imr)(x-imr)}  \; dx \\
    &= \frac{2\pi}{ir}\left(2\pi i\cdot \underbrace{\text{Res}_{x=imr}}_{=\frac{imr\exp(i^2mr)}{2imr}}-\int_\text{upper half circle}...\right)\\
    &= \frac{2\pi^2}{r} e^{-mr}
\end{align}
Therefore
\begin{align}
    \frac{1}{(2\pi)^3}\int d^3k\frac{e^{-ik\cdot r}}{k^2+m^2}
    &= \frac{1}{4\pi r} e^{-mr}
\end{align}

\subsection{Problem 3.9 - \texorpdfstring{$\zeta$}{TEXT} function regularization}
\begin{enumerate}
    \item Calculation the Taylor expansion (using L'Hopital's rule for the limits) we obtain
    \begin{align}
        f(t)
        &=\frac{t}{e^t-1}\\
        &=\sum_k\left.\frac{d^kf}{dt^k}\right|_{t=0}t^k\\
        &=1-\frac{1}{2}t+\frac{1}{12}t^2-\frac{1}{12}t^4+...\\
        &\stackrel{!}{=}B_0+B_1t+\frac{B_2}{2}t^2+\frac{B_3}{6}t^2+...\\
        \rightarrow B_n&=\{1,-\frac{1}{2},\frac{1}{6},0,...\}
    \end{align}
    \item Avoiding mathematical rigor we see after playing around for a while
    \begin{align}
        \sum_{n=1}^\infty ne^{-an}
        &=-\frac{d}{da}\sum_{n=1}^\infty e^{-an}\\
        &=-\frac{d}{da}\sum_{n=1}^\infty \left(e^{-a}\right)^n\\
        &=-\frac{d}{da}\frac{1}{1-e^{-a}}\\
        &=-\frac{d}{da}\left(\frac{1}{a}\frac{a}{1-e^{-a}}\right)\\
        &=-\frac{d}{da}\left(\frac{1}{a}f(t)\right)\\
        &=-\frac{d}{da}\left(\frac{1}{a}\sum_{n=0}^\infty\frac{B_n}{n!}a^n\right)\\
        &=-\frac{d}{da}\left(\frac{1}{a}\left[1-\frac{a}{2}+\frac{a^2}{12}-\frac{a^4}{720}+...\right]\right)\\
        &=-\frac{d}{da}\left(\frac{1}{a}-\frac{1}{2}+\frac{a}{12}-\frac{a^3}{720}...\right)\\
        &=\frac{1}{a^2}-\frac{1}{12}+\frac{a}{240}-...\\
        &\stackrel{a\rightarrow0}{\rightarrow}\frac{1}{a^2}-\frac{1}{12}
    \end{align}
    \item Using the definition of the Riemann $\zeta$ function 
    \begin{align}
        \zeta(s)=\sum_{k=1}^\infty \frac{1}{k^s}
    \end{align}
\end{enumerate}


\subsection{Problem 4.1 - \texorpdfstring{$Z[J]$}{Lg} at order \texorpdfstring{$\lambda$}{Lg} in \texorpdfstring{$\phi^4$}{Lg} theory}
Lets start at (4.6a) with $\mathscr{L}_I=-\lambda/4!\phi^4$
\begin{align}
    Z[J]&=\exp\left[\text{i}\int d^4x\mathscr{L}_I\left(\frac{1}{\text{i}}\frac{\delta}{\delta J(x)}\right)\right]\int\mathcal{D}\phi\exp\left[\text{i}\int d^4x(\mathscr{L}_0+J\phi)\right]\\
    &=\exp\left[\text{i}\int d^4x\mathscr{L}_I\left(\frac{1}{\text{i}}\frac{\delta}{\delta J(x)}\right)\right]Z_0[J]\\
    &=\exp\left[-\frac{\text{i}\lambda}{4!}\int d^4x\left(\frac{\delta^4 }{\delta J(x)^4}\right)\right]Z_0[J]\\
    &=Z_0[J]-\frac{\text{i}\lambda}{4!}\int d^4x\left(\frac{\delta^4 Z_0[J]}{\delta J(x)^4}\right)+\dots.
\end{align}
Using (4.7)
\begin{align}
    Z_0[J]&=Z_0[0]\exp\left[-\frac{\text{i}}{2}\int d^4yd^4zJ(y)\Delta_F(y-z)J(z)\right]=Z_0[0]e^{iW_0[J]}\\
    W_0[J]&=-\frac{1}{2}\int d^4yd^4zJ(y)\Delta_F(y-z)J(z)
\end{align}
we derive (4.10) in various steps 
\begin{enumerate}
\item Calculating $\frac{\delta W_0[J]}{\delta J(x)}$
\begin{align}
    \frac{\delta W_0[J]}{\delta J(x)}&=-\frac{1}{2}\lim_{\epsilon\rightarrow0}\int d^4yd^4z\frac{\left(J(y)+\epsilon\delta^{(4)}(y-x)\right)\Delta_F(y-z)\left(J(z)+\epsilon\delta^{(4)}(z-x)\right)-W_0[J]}{\epsilon}\\
    &=-\frac{1}{2}\int d^4yd^4z \left[\delta^{(4)}(y-x)\Delta_F(y-z)J(z)+J(y)\Delta_F(y-z)\delta^{(4)}(z-x)\right]\\
    &=-\frac{1}{2}\int d^4z\Delta_F(x-z)J(z)-\frac{1}{2}\int d^4yJ(y)\Delta_F(y-x)\\
    &=-\int d^4y\Delta_F(y-x)J(y)
\end{align}
where we used $\Delta_F(x)=\Delta_F(-x)$.
\item Calculating $\frac{\delta^2 W_0[J]}{\delta J(x)^2}$
\begin{align}
    \frac{\delta^2 W_0[J]}{\delta J(x)^2}
    &=-\int d^4y\Delta_F(y-x)\frac{\delta J(y)}{\delta J(x)}\\
    &=-\int d^4y\Delta_F(y-x)\delta(y-x)\\
    &=-\Delta_F(0)
\end{align}

\item Calculating $\delta F[J]/\delta J(x)$ for $F[J]=f\left(W_0[J]\right)$
    \begin{align}
        \frac{\delta F[J]}{\delta J(x)}
        &=\lim_{\epsilon\rightarrow0}\frac{1}{\epsilon}f(W_0[\phi(x)+\epsilon\delta(x-y)])-f(W_0[\phi(x)])\\
        &=\lim_{\epsilon\rightarrow0}\frac{1}{\epsilon}f(W_0[\phi(x)]+\epsilon\frac{\delta W_0}{\delta \phi})-f(W_0[\phi(x)])\\
        &=\lim_{\epsilon\rightarrow0}\frac{1}{\epsilon}f(W_0[\phi(x)])+g' \epsilon\frac{\delta W_0}{\delta \phi}-f(W_0[\phi(x)])\\
        &=f'(W_0[J])\frac{\delta W_0}{\delta J}
    \end{align}
\item Calculating first derivative
\begin{align}
    \frac{\delta}{\text{i}\delta J(x)}\exp\left(\text{i}W_0[J]\right)
    &=\frac{\delta W_0[J]}{\delta J(x)}\exp\left(\text{i}W_0[J]\right)
\end{align}
\item Calculating second derivative (using the functional derivative product rule)
\begin{align}
    \left(\frac{\delta}{\text{i}\delta J(x)}\right)^2\exp\left(\text{i}W_0[J]\right)
    &=\left(\left(\frac{\delta W_0[J]}{\delta J(x)}\right)^2+\frac{1}{i}\frac{\delta^2 W_0[J]}{\delta J(x)^2}\right)\exp\left(\text{i}W_0[J]\right)
\end{align}
\item Calculating third derivative
\begin{align}
    \left(\frac{\delta}{\text{i}\delta J(x)}\right)^3\exp\left(\text{i}W_0[J]\right)
    &=\left(\left(\frac{\delta W_0[J]}{\delta J(x)}\right)^3+\frac{3}{i}\frac{\delta^2 W_0[J]}{\delta J(x)^2}\frac{\delta W_0[J]}{\delta J(x)}+\frac{1}{i^2}\frac{\delta^3 W_0[J]}{\delta J(x)^3}\right)\exp\left(\text{i}W_0[J]\right)
\end{align}
\item Calculating fourth derivative
\begin{align*}
    \left(\frac{\delta}{\text{i}\delta J(x)}\right)^4\exp\left(\text{i}W_0[J]\right)
    &=\left(\left(\frac{\delta W_0[J]}{\delta J(x)}\right)^4+\frac{6}{i}\frac{\delta^2 W_0[J]}{\delta J(x)^2}\left(\frac{\delta W_0[J]}{\delta J(x)}\right)^2+\frac{3}{i^2}\left(\frac{\delta^2 W_0[J]}{\delta J(x)^2}\right)^2+\right.\\
    &\qquad\left.+\frac{4}{i^2}\frac{\delta W_0[J]}{\delta J(x)}\frac{\delta^3 W_0[J]}{\delta J(x)^3} +\frac{1}{i^3}\frac{\delta^4 W_0[J]}{\delta J(x)^4}\right)\exp\left(\text{i}W_0[J]\right)\\
    &=\left(\left(\frac{\delta W_0[J]}{\delta J(x)}\right)^4+\frac{6}{i}\frac{\delta^2 W_0[J]}{\delta J(x)^2}\left(\frac{\delta W_0[J]}{\delta J(x)}\right)^2+\frac{3}{i^2}\left(\frac{\delta^2 W_0[J]}{\delta J(x)^2}\right)^2\right)\exp\left(\text{i}W_0[J]\right)
    %=\left(\left(-\int d^4y\Delta_F(y-x)J(y)\right)^4+\frac{6}{i}(-\Delta_F(0))\left(-\int d^4y\Delta_F(y-x)J(y)\right)^2+\frac{3}{i^2}\left(-\Delta_F(0)\right)^2\right)\exp\left(\text{i}W_0[J]\right)
\end{align*}
\item Substituting the functional derivatives
\begin{align*}    
     \left(\frac{\delta}{\text{i}\delta J(x)}\right)^4\exp\left(\text{i}W_0[J]\right)&=\left[\left(\int d^4y\Delta_F(y-x)J(y)\right)^4+6i\Delta_F(0)\left(\int d^4y\Delta_F(y-x)J(y)\right)^2\right.\\
     &\left.+3\left(i\Delta_F(0)\right)^2\right]\exp\left(\text{i}W_0[J]\right)\\
\end{align*}
\end{enumerate}

\subsection{Problem 19.1 - Dynamical stress tensor}
Preliminaries
\begin{itemize} 
\item The Laplace expansion of the determinate by row or column is given by
\begin{align}
    |g|&=\sum_\kappa g_{\kappa\mu}G_{\kappa\mu}\quad\text{(no sum over $\mu$!)}
\end{align}
with the cofactor matrix $G_{\kappa\mu}$ (matrix of determinants of minors of g).
\item The inverse matrix is given by
\begin{align}
    g^{\alpha\beta}=\frac{1}{|g|}G_{\alpha\beta}
\end{align}
\item Therefore we have
\begin{align}
    \frac{\partial|g|}{\delta g_{\alpha\beta}}
    &=\frac{\partial\left(\sum_\kappa g_{\kappa\beta}G_{\kappa\beta}\right)}{\delta g_{\alpha\beta}}\\
    &=\delta_{\kappa\alpha}G_{\kappa\beta}\\
    &=G_{\alpha\beta}\\
    &=|g|g^{\alpha\beta}
\end{align}
\end{itemize}
Now we can calculate
\begin{align}
    \delta\sqrt{|g|}
    &=\frac{\partial\sqrt{|g|}}{\delta g_{\mu\nu}}\delta g_{\mu\nu}
    =\frac{1}{2\sqrt{|g|}}\frac{\partial|g|}{\delta g_{\mu\nu}}\delta g_{\mu\nu}
    =\frac{1}{2}\sqrt{|g|}g^{\mu\nu}\delta g_{\mu\nu}\\
    \frac{\delta\sqrt{|g(x)|}}{\delta g_{\mu\nu}(y)}
    &=\frac{1}{2}\sqrt{|g|}\delta(x-y)
\end{align}
We now use the action and definition (7.49)
\begin{align}
    S_\text{m}&=\int d^4x\sqrt{|g|}\mathscr{L}_\text{m}\\
    T^{\mu\nu}&=\frac{2}{\sqrt{|g|}}\frac{\delta S_\text{m}}{\delta g^{\mu\nu}}\\
    &=\frac{2}{\sqrt{|g|}}\int d^4x\left[\frac{1}{2}\sqrt{|g|}g^{\mu\nu}\mathscr{L}_\text{m}+\sqrt{|g|}\frac{\delta\mathscr{L}_\text{m}}{\delta g_{\mu\nu}}\right]
\end{align}


\subsection{Problem 19.6 - Dirac-Schwarzschild}
\begin{enumerate}
    \item (19.13) - adding the bi-spinor index might be helpful for some readers, see (B.27)
    \item (19.13) vs (B.27) naming of generators $J^{\mu\nu}$ vs $\sigma_{\mu\nu}/2$
\end{enumerate}
The Dirac equation in curved space is obtained (from the covariance principle) by replacing all derivatives $\partial_k$ with covariant tetrad derivatives $\mathscr{D}_k$ 
\begin{align}
    (i\hbar\gamma^k\mathscr{D}_k+mc)\psi=0
\end{align}




Lets start with the Schwarzschild line element
\begin{align}
    ds^2&=\left(1-\frac{2M}{r}\right)dt^2-\left(1-\frac{2M}{r}\right)^{-1}dr^2-r^2(d\vartheta^2+\sin^2\vartheta\,d\phi^2)\\
    &=\eta_{mn}d\xi^md\xi^n
\end{align}
with
\begin{align}
    d\xi^0=\left(1-\frac{2M}{r}\right)^{1/2}dt,\quad d\xi^1=\left(1-\frac{2M}{r}\right)^{-1/2}dr,\quad d\xi^2=rd\vartheta,\quad d\xi^3=r\sin\vartheta d\phi.
\end{align}
and the tetrad fields $e^m_\mu$ can then be derived via $d\xi^m=e^m_\mu(x) dx^\mu$.



\subsection{Problem 23.1 - Conformal transformation}
For a change of coordinates we find in general
\begin{align}
    x^\mu&\mapsto\tilde{x}^\mu\\
    g_{\mu\nu}(x)&\mapsto\tilde{g}_{\mu\nu}(\tilde{x})=\frac{\partial x^\alpha}{\partial \tilde x^\mu}\frac{\partial x^\beta}{\partial \tilde x^\nu}g_{\textcolor{red}{\alpha\beta}}(x)
\end{align}
which for $x\mapsto\tilde{x}=e^{\omega}x$ results in (there might be a sign error in (18.1))
\begin{align}
    g_{\mu\nu}(x)&\mapsto\tilde{g}_{\mu\nu}(\tilde{x})=e^{-2\omega}g_{\textcolor{red}{\alpha\beta}}(x)
\end{align}
while for a conformal transformation we have 
\begin{align}
    g_{\mu\nu}(x)\mapsto &\tilde{g}_{\mu\nu}(x)=\Omega^2g_{\textcolor{red}{\alpha\beta}}(x)\\
    &\tilde{g}_{\mu\nu}(\tilde{x})=\Omega^2g_{\textcolor{red}{\alpha\beta}}(e^{\omega}x)
\end{align}


\subsection{Problem 23.2 - Conformal transformation properties}
\begin{itemize}
\item Christoffel symbol:
\begin{align}
    \tilde{g}_{\mu\nu}(x)&=\Omega^2(x)g_{\mu\nu}(x)=e^{2\omega(x)}g_{\mu\nu}(x)\\
    \tilde{g}_{\mu\nu,\alpha}
    &=2\Omega\Omega_{,\alpha}g_{\mu\nu}+\Omega^2g_{\mu\nu,\alpha}\\
    &=\Omega(2g_{\mu\nu}\Omega_{,\alpha}+\Omega g_{\mu\nu,\alpha})
\end{align}
and
\begin{align}
\delta^\mu_{\;\nu}&=\tilde{g}^{\mu\alpha}\tilde{g}_{\alpha\nu}=\tilde{g}^{\mu\alpha}g_{\alpha\nu}\Omega^2\\
\delta^\mu_{\;\nu}g^{\nu\beta}&=\tilde{g}^{\mu\alpha}g_{\alpha\nu}g^{\nu\beta}\Omega^2\\
g^{\mu\beta}&=\tilde{g}^{\mu\alpha}\delta^\beta_\alpha\Omega^2\\
&\rightarrow\tilde{g}^{\mu\beta}=\Omega^{-2}g^{\mu\beta}
\end{align}
we find by using $\Gamma^\mu_{\;\alpha\beta}=\frac{1}{2}g^{\mu\nu}\left(g_{\alpha\mu,\beta}+g_{\beta\mu,\alpha}-g_{\alpha\beta,\mu}\right)$
\begin{align}
\tilde\Gamma^\mu_{\;\alpha\beta}
&=\frac{1}{2}\tilde{g}^{\mu\nu}\left(\tilde{g}_{\alpha\nu,\beta}+\tilde{g}_{\beta\nu,\alpha}-\tilde{g}_{\alpha\beta,\nu}\right)\\
&=\frac{1}{2}\Omega^{-2}g^{\mu\nu}\left[
 \Omega(2g_{\alpha\nu}\Omega_{,\beta}+\Omega g_{\alpha\nu,\beta})
+\Omega(2g_{\beta\nu}\Omega_{,\alpha}+\Omega g_{\beta\nu,\alpha})
-\Omega(2g_{\alpha\beta}\Omega_{,\nu}+\Omega g_{\alpha\beta,\nu})
\right]\\
&=\Gamma^\mu_{\;\alpha\beta}+\Omega^{-1}g^{\mu\nu}\left[
 g_{\alpha\nu}\Omega_{,\beta}
+g_{\beta\nu}\Omega_{,\alpha}
-g_{\alpha\beta}\Omega_{,\nu}
\right]\\
&=\Gamma^\mu_{\;\alpha\beta}+\Omega^{-1}\left[
 \delta^\mu_\alpha\Omega_{,\beta}
+\delta_\beta^\mu\Omega_{,\alpha}
-g^{\mu\nu}g_{\alpha\beta}\Omega_{,\nu}
\right]
\end{align}

\item Ricci tensor: with
\begin{align}
    \Omega&=e^{2\omega}\\
    \Omega^{-2}\Omega_{,\lambda}
    &=e^{-4\omega}e^{2\omega}2\omega_{,\lambda}\\
    &=2e^{-2\omega}\omega_{,\lambda}\\
    \Omega_{,\lambda\alpha}
    &=\left(2e^{2\omega}\omega_{,\lambda}\right)_{,\alpha}\\
    &=4e^{2\omega}\omega_{,\lambda}\omega_{,\alpha}+2e^{2\omega}\omega_{,\lambda\alpha}\\
    &=2e^{2\omega}\left(2\omega_{,\lambda}\omega_{,\alpha}+\omega_{,\lambda\alpha}\right)
\end{align}
and
\begin{align}
\partial_\lambda\tilde\Gamma^\mu_{\;\alpha\beta}
&=\partial_\lambda\Gamma^\mu_{\;\alpha\beta}
-\Omega^{-2}\Omega_{,\lambda}\left[
 \delta^\mu_\alpha\Omega_{,\beta}
+\delta_\beta^\mu\Omega_{,\alpha}
-g^{\mu\nu}g_{\alpha\beta}\Omega_{,\nu}
\right]
+\Omega^{-1}\left[
 \delta^\mu_\alpha\Omega_{,\beta\lambda}
+\delta_\beta^\mu\Omega_{,\alpha\lambda}
-(g^{\mu\nu}g_{\alpha\beta}\Omega_{,\nu})_{,\lambda}
\right]\\
&=\partial_\lambda\Gamma^\mu_{\;\alpha\beta}
-4\omega_{,\lambda}\left[
 \delta^\mu_\alpha\omega_{,\beta}
+\delta_\beta^\mu\omega_{,\alpha}
-g^{\mu\nu}g_{\alpha\beta}\omega_{,\nu}
\right]
+2\left[
 \delta^\mu_\alpha\left(2\omega_{,\beta}\omega_{,\lambda}+\omega_{,\beta\lambda}\right)
+\delta_\beta^\mu\left(2\omega_{,\alpha}\omega_{,\lambda}+\omega_{,\alpha\lambda}\right)\right]\\
&\qquad-2\left[
 g^{\mu\nu}_{\quad,\lambda}g_{\alpha\beta}\omega_{,\nu}
+g^{\mu\nu}g_{\alpha\beta,\lambda}\omega_{,\nu}
+g^{\mu\nu}g_{\alpha\beta}(2\omega_{,\nu}\omega_{,\lambda}+\omega_{,\nu\lambda})\right]\\
\end{align}
\begin{align}
\partial_\rho\tilde\Gamma^\rho_{\;\mu\nu}
&=\partial_\rho\Gamma^\rho_{\;\mu\nu}
-4\omega_{,\rho}\left[
 \delta^\rho_\mu\omega_{,\nu}
+\delta_\nu^\rho\omega_{,\mu}
-g^{\rho\nu}g_{\mu\nu}\omega_{,\lambda}
\right]
+2\left[
 \delta^\rho_\mu\left(2\omega_{,\nu}\omega_{,\rho}+\omega_{,\nu\rho}\right)
+\delta_\nu^\rho\left(2\omega_{,\mu}\omega_{,\rho}+\omega_{,\mu\rho}\right)\right]\\
&\qquad-2\left[
 g^{\rho\lambda}_{\quad,\rho}g_{\mu\nu}\omega_{,\lambda}
+g^{\rho\lambda}g_{\mu\nu,\rho}\omega_{,\lambda}
+g^{\rho\lambda}g_{\mu\nu}(2\omega_{,\lambda}\omega_{,\rho}+\omega_{,\lambda\rho})\right]\\
%
&=\partial_\rho\Gamma^\rho_{\;\mu\nu}
-4\left[
2\omega_{,\mu}\omega_{,\nu}
-\omega_{,\rho}g^{\rho\nu}g_{\mu\nu}\omega_{,\lambda}
\right]
+4 \left(2\omega_{,\nu}\omega_{,\mu}+\omega_{,\nu\mu}\right)\\
&\qquad-2\left[
 g^{\rho\lambda}_{\quad,\rho}g_{\mu\nu}\omega_{,\lambda}
+g^{\rho\lambda}g_{\mu\nu,\rho}\omega_{,\lambda}
+g^{\rho\lambda}g_{\mu\nu}(2\omega_{,\lambda}\omega_{,\rho}+\omega_{,\lambda\rho})\right]\\
%
&=\partial_\rho\Gamma^\rho_{\;\mu\nu}
+4g^{\rho\nu}g_{\mu\nu}\omega_{,\lambda}\omega_{,\rho}
+4\omega_{,\nu\mu}
-2\left[
 g^{\rho\lambda}_{\quad,\rho}g_{\mu\nu}\omega_{,\lambda}
+g^{\rho\lambda}g_{\mu\nu,\rho}\omega_{,\lambda}
+(2g^{\rho\lambda}g_{\mu\nu}\omega_{,\lambda}\omega_{,\rho}+g^{\rho\lambda}g_{\mu\nu}\omega_{,\lambda\rho})\right]\\
%
&=\partial_\rho\Gamma^\rho_{\;\mu\nu}
+4\omega_{,\lambda}\omega_{,\mu}
+4\omega_{,\nu\mu}
-2\left[
 g^{\rho\lambda}_{\quad,\rho}g_{\mu\nu}\omega_{,\lambda}
+g_{\mu\nu,\rho}\omega^{,\rho}
+2g_{\mu\nu}\omega^{,\rho}\omega_{,\rho}+g_{\mu\nu}\omega^{,\rho}_{\;\;\;\rho}\right]
\end{align}

\begin{align}
\partial_\nu\tilde\Gamma^\rho_{\;\mu\rho}
&=\partial_\nu\Gamma^\rho_{\;\mu\rho}
-4\omega_{,\nu}\left[
 \delta^\rho_\mu\omega_{,\rho}
+\delta_\rho^\rho\omega_{,\mu}
-g^{\rho\kappa}g_{\mu\rho}\omega_{,\kappa}
\right]
+2\left[
 \delta^\rho_\mu\left(2\omega_{,\rho}\omega_{,\nu}+\omega_{,\rho\nu}\right)
+\delta_\rho^\rho\left(2\omega_{,\mu}\omega_{,\nu}+\omega_{,\mu\nu}\right)\right]\\
&\qquad-2\left[
 g^{\rho\kappa}_{\quad,\nu}g_{\mu\rho}\omega_{,\kappa}
+g^{\rho\kappa}g_{\mu\rho,\nu}\omega_{,\kappa}
+g^{\rho\kappa}g_{\mu\rho}(2\omega_{,\kappa}\omega_{,\nu}+\omega_{,\kappa\nu})\right]\\
%
&=\partial_\nu\Gamma^\rho_{\;\mu\rho}
-4\left[
(d+1)\omega_{,\mu}\omega_{,\nu}
-\omega_{,\mu}\omega_{,\nu}
\right]
+2(d+1)\left(2\omega_{,\mu}\omega_{,\nu}+\omega_{,\mu\nu}\right)\\
&\qquad-2\left[
 g^{\rho\kappa}_{\quad,\nu}g_{\mu\rho}\omega_{,\kappa}
+g^{\rho\kappa}g_{\mu\rho,\nu}\omega_{,\kappa}
+\delta^\kappa_\mu(2\omega_{,\kappa}\omega_{,\nu}+\omega_{,\kappa\nu})\right]\\
%
&=\partial_\nu\Gamma^\rho_{\;\mu\rho}
+4\omega_{,\mu}\omega_{,\nu}+2(d+1)\omega_{,\mu\nu}-2\left[
 g^{\rho\kappa}_{\quad,\nu}g_{\mu\rho}\omega_{,\kappa}
+g^{\rho\kappa}g_{\mu\rho,\nu}\omega_{,\kappa}
+(2\omega_{,\mu}\omega_{,\nu}+\omega_{,\mu\nu})\right]\\
%
&=\partial_\nu\Gamma^\rho_{\;\mu\rho}
+2d\cdot\omega_{,\mu\nu}-2\left[
 g^{\rho\kappa}_{\quad,\nu}g_{\mu\rho}\omega_{,\kappa}
+g_{\mu\rho,\nu}\omega^{,\rho}\right]
\end{align}

\begin{align}
\tilde\Gamma^\mu_{\;\alpha\beta}
&=\Gamma^\mu_{\;\alpha\beta}+\Omega^{-1}\left[
 \delta^\mu_\alpha\Omega_{,\beta}
+\delta_\beta^\mu\Omega_{,\alpha}
-g^{\mu\nu}g_{\alpha\beta}\Omega_{,\nu}
\right]\\
\end{align}

\begin{align}
\tilde\Gamma^\rho_{\;\mu\nu}\tilde\Gamma^\sigma_{\;\rho\sigma}
&=\left(\Gamma^\rho_{\;\mu\nu}+\Omega^{-1}\left[
 \delta^\rho_\mu\Omega_{,\nu}
+\delta_\nu^\rho\Omega_{,\mu}
-g^{\rho\lambda}g_{\mu\nu}\Omega_{,\lambda}
\right]\right)\left(\Gamma^\sigma_{\;\rho\sigma}+d\cdot\Omega^{-1}\Omega_{,\rho}\right)\\
&=\Gamma^\rho_{\;\mu\nu}\Gamma^\sigma_{\;\rho\sigma}
+\Gamma^\rho_{\;\mu\nu}d\cdot\Omega^{-1}\Omega_{,\rho}
+\Gamma^\sigma_{\;\rho\sigma}\Omega^{-1}\left[
 \delta^\rho_\mu\Omega_{,\nu}
+\delta_\nu^\rho\Omega_{,\mu}
-g^{\rho\lambda}g_{\mu\nu}\Omega_{,\lambda}
\right]\\
&\qquad+
d\cdot\Omega^{-2}\left[
 \delta^\rho_\mu\Omega_{,\nu}
+\delta_\nu^\rho\Omega_{,\mu}
-g^{\rho\lambda}g_{\mu\nu}\Omega_{,\lambda}
\right]\Omega_{,\rho}
\end{align}



\begin{align}
    \tilde{R}_{\mu\nu}
    &=\tilde{R}^\rho_{\;\mu\rho\nu}\\
    &=\partial_\rho\tilde\Gamma^\rho_{\;\mu\nu}
    -\partial_\nu\tilde\Gamma^\rho_{\;\mu\rho}
    +\tilde\Gamma^\rho_{\;\mu\nu}\tilde\Gamma^\sigma_{\;\rho\sigma}
    -\tilde\Gamma^\sigma_{\;\nu\rho}\tilde\Gamma^\rho_{\;\mu\sigma}
\end{align}

\item Curvature scalar
\begin{align}
    \tilde{R}
    &=\tilde{g}^{\mu\nu}\tilde{R}_{\mu\nu}\\
    &=\tilde{g}^{\mu\nu}\left[R_{\mu\nu}-g_{\mu\nu}\Box\omega-(d-2)\nabla_\mu\nabla_\nu\omega+(d-2)\nabla_\mu\omega\nabla_\nu\omega-(d-2)g_{\mu\nu}\nabla^\lambda\omega\nabla_\lambda\omega\right]\\
    &=\Omega^{-2}\left[R-d\Box\omega-(d-2)\Box\omega+(d-2)\nabla^\mu\omega\nabla_\mu\omega-(d-2)d\nabla^\lambda\omega\nabla_\lambda\omega\right]\\
    &=\Omega^{-2}\left[R-2(d-1)\Box\omega-(d-2)(d-1)\nabla^\lambda\omega\nabla_\lambda\omega\right]\\
\end{align}

\end{itemize}


\subsection{Problem 23.6 - Reflection formula}
\begin{align}
    \Gamma(z)=\int_0^\infty x^{z-1}e^{-x}dx
\end{align}



\subsection{Problem 23.7 - Unruh temperature}

\subsection{Problem 24.14 - Jeans length and the \textcolor{red}{speed of sound}}
We start with the Euler equations
\begin{align}
    \frac{D\rho}{Dt}=-\rho\nabla\cdot\vec{u}\quad&\rightarrow\quad\frac{\partial\rho}{\partial t}+\vec{u}\cdot(\nabla\rho)+\rho(\nabla\cdot\vec{u})=0\\
    \frac{D\vec{u}}{Dt}=-\nabla\left(\frac{P}{\rho}\right)+\vec{g}\quad&\rightarrow\quad\frac{\partial\vec{u}}{\partial t}+\vec{u}\cdot(\nabla\vec{u})+\frac{\nabla P}{\rho}=\vec{g}.
\end{align}
With the perturbation ansatz (small perturbation in a resting fluid)
\begin{align}
    \rho&=\rho_0+\varepsilon \rho_1(x,t)\\
    P&=P_0+\varepsilon P_1(x,t)\\
    \vec{u}&=\varepsilon \vec{u}_1(x,t)
\end{align}
and the Newton equation
\begin{align}
    \triangle\phi=4\pi G\rho\quad&\rightarrow\quad\nabla\cdot\vec{g}_1=-4\pi G\rho_1
\end{align}
we obtain (with the EoS $P=w\rho$) in order $\varepsilon$
\begin{align}
    \frac{\partial\rho_1}{\partial t}+\rho_0(\nabla\cdot\vec{u}_1)&=0\\
    \frac{\partial\vec{u}_1}{\partial t}+\underbrace{\frac{1}{\rho_0}\nabla P_1}_{=\frac{w}{\rho_0}\nabla\rho_1}=\vec{g}_1.
\end{align}
Differentiating both (with respect to space and time) we obtain a wave equation
\begin{align}
    \frac{\partial^2\rho_1}{\partial t^2}-w\triangle\rho_1=4\pi G\rho_0\rho_1
\end{align}
with the speed of sound $c_s^2=w$. Inserting the wave ansatz $\rho_1\sim\exp[i(\vec{k}\cdot\vec{x}-\omega t)]$ yields the dispersion relation
\begin{align}
    \omega^2=c_s^2k^2-4\pi G\rho_0.
\end{align}
For wave numbers $k_J<\sqrt{4\pi G/c_s^2}$ the $\omega$ becomes complex which gives rise to exponentially growing modes. Therefore the Jeans length is given by
\begin{align}
    \lambda_J=\frac{2\pi}{k_J}=c_s\sqrt{\frac{\pi}{G\rho_0}}=\sqrt{\frac{\pi w}{G\rho_0}}.
\end{align}

\subsection{Problem 25.1 - Schwarzschild metric}
The simplified vacuum Einstein equations are given by
\begin{align}
R_{\mu\nu}&-\frac{1}{2}Rg_{\mu\nu}=0\\
&\quad\rightarrow\quad R-\frac{1}{2}R\cdot 4=0\quad\rightarrow\quad R=0\\
R_{\mu\nu}&=0
\end{align}
Lets start with the metric ansatz (25.4) 
\begin{align}
g_{\mu\nu}&=\text{diag}(A(r),-B(r),-r^2,-r^2\sin^2\theta)\\
g^{\mu\nu}&=\text{diag}(1/A(r),-1/B(r),-1/r^2,-1/r^2\sin^2\theta)
\end{align}
The non-vanishing Chistoffel symbols are then
\begin{align}
\Gamma^\mu_{\nu\lambda}&=\frac{1}{2}g^{\mu\kappa}(g_{\kappa\lambda,\nu}+g_{\nu\kappa,\lambda}-g_{\nu\lambda,\kappa})\\
\Gamma^0_{01}&=\frac{A'}{2A},\quad
\Gamma^1_{00}=\frac{A'}{2B}\quad
\Gamma^1_{11}=\frac{B'}{2B}\quad
\Gamma^1_{22}=-\frac{r}{B}\quad
\Gamma^1_{33}=\frac{r\sin^2\theta}{B}\\
\Gamma^2_{12}&=1/r\quad
\Gamma^2_{12}=-\cos\theta\sin\theta\quad
\Gamma^2_{12}=1/r\quad
\Gamma^2_{12}=\cot\theta
\end{align}
The non-vanishing components of the Ricci tensor are
\begin{align}
R_{00}&=\frac{A'}{rB}-\frac{A'^2}{4AB}-\frac{A'B'}{4B^2}+\frac{A''}{2B}\\
R_{11}&=\frac{A'^2}{4A^2}+\frac{B'}{rB}+\frac{A'B'}{4AB}-\frac{A''}{2A}\\
R_{22}&=-\frac{1}{B}+1-\frac{rA'}{2AB}+\frac{rB''}{2B^2}\\
R_{33}&=R_{22}\sin^2\theta
\end{align}
As there are only the two unknown functions $A, B$ we only need two vacuum equations $R_{00}=0$ and $R_{11}=0$. Multiplying the first by $B/A$ and leaving the second  one untouched we obtain the system
\begin{align}
\frac{A'}{rA}-\frac{A'^2}{4A^2}-\frac{A'B'}{4AB}+\frac{A''}{2A}&=0\\
\frac{B'}{rB}+\frac{A'^2}{4A^2}+\frac{A'B'}{4AB}-\frac{A''}{2A}&=0
\end{align}
Adding bot we get $B'/B=-A'/A$ which we can substitude into the first one obtaining
\begin{align}
\frac{A'}{rA}+\frac{A''}{2A}&=0\\
&\rightarrow A'(r)=\frac{c_1}{r^2}\\
&\rightarrow A(r)=c_2-\frac{c_1}{r}
\end{align}
now we can solve for $B(r)$
\begin{align}
\frac{B'}{B}&=-\frac{A'}{A}\\
&\rightarrow B(r)=\frac{c_3 r}{c_1-rc_2}=\frac{-c_3}{c_2-\frac{c_1}{r}}
\end{align}





\subsection{Problem 26.4 - Fixed points of (26.18)}
We start with
\begin{align}
    \text{(F1)}\qquad H^2&=\frac{8\pi G}{3}\left(\frac{1}{2}\dot{\phi}^2+V+\rho\right)\\
    \text{(F2)}\qquad \dot{H}&=-4\pi G\left[\dot{\phi}^2+(1+w_m)\rho\right]\\
    \text{(KG)}\qquad \ddot{\phi}&=-3H\dot{\phi}-V_{,\phi}.
\end{align}
Using $H=\dot{a}/a$, $N=\ln(a)$ and $\lambda=-V_{,\phi}/(\sqrt{8\pi G}V)$ we obtain for the time derivatives of $x$ and $y$
\begin{align}
    \dot{V}&=\frac{dV}{d\phi}\frac{d\phi}{dt}=V_{,\phi}\dot{\phi}\\
    x&=\sqrt{\frac{4}{3}\pi G}\frac{\dot{\phi}}{H}\quad\rightarrow\quad\frac{dx}{dt}=\frac{dx}{dN}\frac{d\ln(a)}{dt}
    =\frac{dx}{dN}H
    =\sqrt{\frac{4}{3}\pi G}\frac{\ddot{\phi}H-\dot{\phi}\dot{H}}{H^2}\\
    y&=\sqrt{\frac{8}{3}\pi G}\frac{\sqrt{V}}{H}\quad\rightarrow\quad\frac{dy}{dt}=\frac{dy}{dN}\frac{d\ln(a)}{dt}
    =\frac{dy}{dN}H
    =\sqrt{\frac{8}{3}\pi G}\frac{\frac{V_{,\phi}\dot{\phi}}{2\sqrt{V}}-\sqrt{V}\dot{H}}{H^2}.
\end{align}
With the substitutions
\begin{align}
    \dot{H}&=-4\pi G\left[\dot{\phi}^2+(1+w_m)\rho\right]\\
    \ddot{\phi}&=-3H\dot{\phi}-V_{,\phi}\\
    V_{,\phi}&=-\sqrt{8\pi G}\lambda V\\
    \rho&=\frac{3H^2}{8\pi G}-\frac{1}{2}\dot{\phi}^2-V\\
    \dot{\phi}&=xH/\sqrt{\frac{4}{3}\pi G}\\
    \sqrt{V}&=yH/\sqrt{\frac{8}{3}\pi G}
\end{align}
we obtain
\begin{align}
    \frac{dx}{dN}
    &=-3x+\frac{\sqrt{6}}{2}\lambda y^2+\frac{3}{2}x[(1-w_m)x^2+(1+w_m)(1-y^2)]\\
    \frac{dy}{dN}
    &=-\frac{\sqrt{6}}{2}\lambda xy+\frac{3}{2}y[(1-w_m)x^2+(1+w_m)(1-y^2)].
\end{align}
To find the fix points of (26.17) we need to solve
\begin{align}
    -3x+\frac{\sqrt{6}}{2}\lambda y^2+\frac{3}{2}x[(1-w_m)x^2+(1+w_m)(1-y^2)]=0\\
    \quad\,\,-\frac{\sqrt{6}}{2}\lambda xy+\frac{3}{2}y[(1-w_m)x^2+(1+w_m)(1-y^2)]=0.
\end{align}
\begin{itemize}
\item An obvious solution is 
\begin{align}
    x_0=0, y_0=0.
\end{align}
\item Two semi-obvious solutions can be found for $y=0$ which solves the second equation and transforms the first to the quadratic equation $x^2-1=0$ which gives
\begin{align}
    x_1&=+1, y_1=0\\
    x_2&=-1, y_2=0.
\end{align}
\item Substituting the square bracket of the second equation into the first and simplifying the second gives
\begin{align}
    -3x+\frac{\sqrt{6}}{2}\lambda (x^2+y^2)=0\\
    -\frac{\sqrt{6}}{2}\lambda x+\frac{3}{2}[1+2x^2-(x^2+y^2)-w_m((x^2+x^2)-1)]=0.
\end{align}
Now we can eliminate $x^2+y^2$ and obtain a single quadratic equation in $x$
\begin{align}
    -\frac{\sqrt{6}}{2}\lambda x+\frac{3}{2}\left[1+2x^2-\frac{\sqrt{6}}{\lambda}x-w_m\left(\frac{\sqrt{6}}{\lambda}x-1\right)\right]=0
\end{align}
which can be simplified to
\begin{align}
    x^2-\frac{3(1+w_m)+\lambda^2}{\sqrt{6}\lambda}x+\frac{1+w_m}{2}=0.
\end{align}
This gives us two more solutions
\begin{align}
    x_3&=\frac{\lambda}{\sqrt{6}}, y_3=\sqrt{1-\frac{\lambda^2}{6}}\qquad\qquad\qquad\quad(\lambda^2<6)\\
    x_4&=\sqrt{\frac{3}{2}}\frac{1+w_m}{\lambda}, y_4=\sqrt{\frac{3}{2}}\frac{\sqrt{1-w_m^2}}{\lambda}\qquad(w_m^2<1).
\end{align}

\item Let's quickly check the stability of the fix points. The characteristic equation for the fix points of a 2d system is given by
\begin{align}
    \alpha^2+a_1(x_i,y_i)\alpha+a_2(x_i,y_i)=0\\
    a_1(x_i,y_i)=-\left(\frac{df_x}{dx}+\frac{df_y}{dy}\right)_{x=x_i,y=y_i}\\
    a_2(x_i,y_i)=\left.\frac{df_x}{dx}\frac{df_y}{dy}-\frac{df_x}{dy}\frac{df_y}{dx}\right|_{x=x_i,y=y_i}
\end{align}
with the stability classification (assuming for EoS parameter $w_m^2<1$)
\begin{center}
\begin{tabular}{ c c c c c}
\hline\hline
 type            & condition            & fix point 0 & fix point 1 & fix point 2   \\ \hline
 saddle node     & $a_2<0$              & $-1<w_m<1$ & $\lambda>\sqrt{6}$ & $\lambda<-\sqrt{6}$\\  
 unstable node   & $0<a_2<a_1^2/4$      & - & $\lambda<\sqrt{6}$ & $\lambda>-\sqrt{6}$\\  
 unstable spiral & $a_1^2/4<a_2, a_1<0$ & - & - & -\\
 center          & $0<a_2, a_1=0$       & - & - & -\\
 stable spiral   & $a_1^2/4<a_2, a_1>0$ & - & - & -\\
 stable node     & $0<a_2<a_1^2/4$      & - & - & -\\ \hline\hline
\end{tabular}
\end{center}

\begin{center}
\begin{tabular}{ c c c }
\hline\hline
 type            & fix point 3 & fix point 4  \\ \hline
 saddle node     & $3(1+w_m)<\lambda^2<6$ & -\\  
 unstable node   & - & -\\
 unstable spiral & - & -\\
 center          & - & -\\
 stable spiral   & - & $\lambda^2>\frac{24(1+w_m)^2}{7+9w_m}$\\
 stable node     & $\lambda^2<3(1+w_m)$ & $\lambda^2<\frac{24(1+w_m)^2}{7+9w_m}$ \\ \hline\hline
\end{tabular}
\end{center}

\end{itemize} 

\subsection{Problem 26.5 - Tracker solution}
Inserting the ansatz
\begin{align}
    \phi(t)=C(\alpha,n)M^{1+\nu}t^\nu
\end{align}
into the ODE
\begin{align}
    \ddot\phi+\frac{3\alpha}{t}\dot\phi-\frac{M^{4+n}}{\phi^{n+1}}=0
\end{align}
gives
\begin{align}
    CM^{1+\nu}\nu(\nu-1)t^{\nu-2}+CM^{1+\nu}\frac{3\alpha}{t}t^{\nu-1}-\frac{M^{4+n}}{C^{n+1}M^{(n+1)(1+\nu)}t^{\nu(n+1)}}&=0\\
    CM^{1+\nu}\left[\nu(\nu-1)+3\alpha\right]t^{\nu-2}-\frac{M^{3-\nu(n+1)}}{C^{n+1}}t^{-\nu(n+1)}&=0
\end{align}
From equating coefficients and powers (in $t$) we obtain
\begin{align}
    \nu&=\frac{2}{2+n}\\
    C(\alpha,n)&=\left(\frac{(2+n)^2}{6\alpha(2+n)-2n}\right)^\frac{1}{2+n}.
\end{align}

\newpage
\section{{\sc Veltman} - Diagrammatica}
\subsection{Problem 1.1 - Matrix exponential}
We compare
\begin{align}
e^\alpha&=1+\alpha+\frac{1}{2!}\alpha^2+\frac{1}{3!}\alpha^3+...+\frac{1}{n!}\alpha^n+...\\
\left[1+\frac{1}{n}\alpha\right]^n&=\sum_k \binom{n}{k}\frac{1}{n^k}\alpha^k=\sum_k \frac{n!}{k!(n-k)!}\frac{1}{n^k}\alpha^k\\
&=\frac{n!}{0!(n-0)!}\frac{1}{n^0}\alpha^0+\frac{n!}{1!(n-1)!}\frac{1}{n}\alpha^1+\frac{n!}{2!(n-2)!}\frac{1}{n^2}\alpha^2+\frac{n!}{3!(n-3)!}\frac{1}{n^3}\alpha^3+...\\
&=1+\alpha+\frac{1}{2!}\underbrace{\frac{n(n-1)}{n^2}}_{\rightarrow1}\alpha^2+\frac{1}{3!}\underbrace{\frac{n(n-1)(n-2)}{n^3}}_{\rightarrow1}\alpha^3+...
\end{align}

\subsection{Problem 1.2 - Lorentz rotation}
Calculating the matrix product in first order we obtain
\begin{align}
R R^T&=\left(
\begin{array}{cccc}
 a^2+b^2+(g+1)^2 & a (h+1)+b c+d (g+1) & a f+b (k+1)+e (g+1) & 0 \\
 a (h+1)+b c+d (g+1) & c^2+d^2+(h+1)^2 & c (k+1)+d e+f (h+1) & 0 \\
 a f+b (k+1)+e (g+1) & c (k+1)+d e+f (h+1) & e^2+f^2+(k+1)^2 & 0 \\
 0 & 0 & 0 & 1 \\
\end{array}
\right)\\
&\simeq
\left(
\begin{array}{cccc}
 1+2g & a+d & b+e & 0 \\
 a+d & 1+2h & cf & 0 \\
 b+e & c+f & 1+2k & 0 \\
 0 & 0 & 0 & 1 \\
\end{array}
\right)
\end{align}
This only becomes the identity for $g=h=k=0$ as well as $a=-d$, $b=-e$ and $c=-f$.

\newpage
\section{{\sc Tong} - Quantum Field Theory}
\subsection{Example Sheet 1 Oct 2007 Problem 1 - Vibrating string}
Using the orthogonality of $\sin mx, \cos mx$
\begin{align}
\frac{\partial y}{\partial t}
&=\sqrt{\frac{2}{a}}\sum_{n=1}\sin\left(\frac{n\pi x}{a}\right)\dot{q}_n\\
\left(\frac{\partial y}{\partial t}\right)^2
&=\frac{2}{a}\left(\sum_n\sin\left(\frac{n\pi x}{a}\right)\dot{q}_n\right)^2\\
&=\frac{2}{a}\sum_{n}\sin^2\left(\frac{n\pi x}{a}\right)\dot{q}_n^2+\frac{2}{a}\sum_{n,m}2\sin\left(\frac{n\pi x}{a}\right)\sin\left(\frac{m\pi x}{a}\right)\dot{q}_n\dot{q}_m\\
\int_0^a \left(\frac{\partial y}{\partial t}\right)^2dx
&=\frac{2}{a}\dot{q}_n^2\sum_{n}\frac{a}{2}\\\
%
\frac{\partial y}{\partial x}
&=\sqrt{\frac{2}{a}}\sum_{n=1}\cos\left(\frac{n\pi x}{a}\right)\frac{n\pi}{a}q_n\\
\left(\frac{\partial y}{\partial x}\right)^2
&=\frac{2}{a}\left(\sum_n\cos\left(\frac{n\pi x}{a}\right)\frac{n\pi}{a}q_n\right)^2\\
&=\frac{2}{a}\sum_n\cos^2\left(\frac{n\pi x}{a}\right)\frac{n^2\pi^2}{a^2}q_n^2+\frac{2}{a}\sum_{n,m}2\cos\left(\frac{n\pi x}{a}\right)\cos\left(\frac{m\pi x}{a}\right)\frac{nm\pi^2}{a^2}q_nq_m\\
\int_0^a \left(\frac{\partial y}{\partial x}\right)^2dx
&=\frac{2}{a}q_n^2\sum_{n}\frac{a}{2}\left(\frac{n\pi}{a}\right)^2
\end{align}
Then we see
\begin{align}
L=\sum_{n}\left[\frac{\sigma}{2}\dot{q}_n^2-\frac{T}{2}\left(\frac{n\pi}{a}\right)^2q_n^2\right]
\end{align}
and therefore
\begin{align}
\frac{\partial L}{\partial q_n}-\frac{d}{dt}\frac{\partial L}{\partial \dot{q}_n}&=0\\
-\frac{T}{2}\left(\frac{n\pi}{a}\right)^2 2q_n-\frac{d}{dt}\frac{\sigma}{2}2\dot{q}_n&=0\\
-T\left(\frac{n\pi}{a}\right)^2 q_n-\sigma\ddot{q}_n&=0\\
\ddot{q}_n+\frac{T}{\sigma}\left(\frac{n\pi}{a}\right)^2q_n&=0
\end{align}

\subsection{Example Sheet 1 Oct 2007 Problem 2 - Lorentz transformation of the Klein-Gordon equation}
\textcolor{blue}{Show directly that if $\phi(x)$ satisfies the Klein-Gordon equation, then $\phi(\Lambda^{-1}x)$  also satisfies this equation for any Lorentz transformation $\Lambda$.}\newline


With $x'=\Lambda x$ or ($x=\Lambda^{-1}x'$) and
\begin{align}
\phi(x)\rightarrow\phi'(x')\equiv\phi(x)=\phi(\Lambda^{-1}x')
\end{align}
we need to calculate the first derivative
\begin{align}
\partial'_\mu\phi'(x')
=\partial'_\mu\phi(x)
&=\frac{\partial x^\alpha}{\partial x'^\mu}\frac{\partial}{\partial x^\alpha}\phi(x)\\
&=(\Lambda^{-1})^\alpha_{\,\beta}\delta^\beta_\mu\frac{\partial}{\partial x^\alpha}\phi(x)\\
&=(\Lambda^{-1})^\alpha_{\,\mu}\frac{\partial}{\partial x^\alpha}\phi(x)
\end{align}
and the second derivative
\begin{align}
\eta^{\mu\nu}\partial'_\nu\partial'_\mu\phi'(x')
&=\underbrace{\eta^{\mu\nu}(\Lambda^{-1})^\alpha_{\,\mu}(\Lambda^{-1})^\beta_{\,\nu}}_{=\eta^{\alpha\beta}}\partial_\beta\partial_\alpha\phi(x)\\
&=\eta^{\alpha\beta}\partial_\beta\partial_\alpha\phi(x)
\end{align}
and therefore
\begin{align}
(\partial'^\mu\partial_\mu+m^2)\phi'(x')
&=\partial'^\mu\partial_\mu\phi'(x')+m^2\phi'(x')\\
&=\partial^\mu\partial_\mu\phi(x)+m^2\phi'(x')\\
&=\partial^\mu\partial_\mu\phi(x)+m^2\phi(x)\\
&=0
\end{align}

\subsection{Example Sheet 1 Oct 2007 Problem 3 - Complex Klein-Gordon field}
With
\begin{align}
\mathcal{L}
&=\eta^{\mu\nu}\partial_\mu\psi^*\partial_\nu\psi-m^2\psi^*\psi-\frac{\lambda}{2}(\psi^*\psi)^2\\
\frac{\partial \mathcal{L}}{\partial\psi}
&=-m^2\psi^*-\lambda(\psi^*\psi)\psi^*\\
\frac{\partial \mathcal{L}}{\partial\psi^*}
&=-m^2\psi-\lambda(\psi^*\psi)\psi\\
\frac{\partial \mathcal{L}}{\partial(\partial_\alpha\psi)}
&=\eta^{\mu\nu}\partial_\mu\psi^*\delta^\alpha_\nu
=\eta^{\mu\alpha}\partial_\mu\psi^*
=\partial^\alpha\psi^*\\
\frac{\partial \mathcal{L}}{\partial(\partial_\alpha\psi^*)}
&=\partial^\alpha\psi
\end{align}
then we calculate the equation of motions
\begin{align}
\partial_\alpha\partial^\alpha\psi^*+m^2\psi^*+\lambda(\psi^*\psi)\psi^*&=0\\
\partial_\alpha\partial^\alpha\psi+m^2\psi+\lambda(\psi^*\psi)\psi&=0
\end{align}
Infinitesimal variation of the Lagrangian
\begin{align}
\delta\mathcal{L}
&=\frac{\partial\mathcal{L}}{\partial\psi_a}\delta\psi_a+\frac{\partial\mathcal{L}}{\partial(\partial_\mu\psi_a)}\overbrace{\delta(\partial_\mu\psi_a)}^{=\partial_\mu(\delta\psi_a)}\\
&=\left[\frac{\partial\mathcal{L}}{\partial\psi_a}-\partial_\mu\frac{\partial\mathcal{L}}{\partial(\partial_\mu\psi_a)}\right]\delta\psi_a+\partial_\mu\underbrace{\left(\frac{\partial\mathcal{L}}{\partial(\partial_\mu\psi_a)}\delta\psi_a\right)}_{=j^\mu}
\end{align}
Lagrangian invariance - substitute infinitesimal trafo $\delta\psi, \delta\psi^*$
\begin{align}
\delta\mathcal{L}
&=\frac{\partial\mathcal{L}}{\partial\psi_a}\delta\psi_a+\frac{\partial\mathcal{L}}{\partial(\partial_\mu\psi_a)}\overbrace{\delta(\partial_\mu\psi_a)}^{=\partial_\mu(\delta\psi_a)}\\
&=i\alpha\left[-m^2\underbrace{(\psi^*\psi-\psi\psi^*)}_{=0}-\lambda(\psi^*\psi)\underbrace{(\psi^*\psi-\psi\psi^*)}_{=0}+\underbrace{(\partial^\mu\psi^*)\partial_\mu\psi-(\partial^\mu\psi)\partial_\mu\psi^*}_{=0}\right]\\
&=0
\end{align}
Noether current 
\begin{align}
\partial_\mu j^\mu&=\partial_\mu\left(\frac{\partial\mathcal{L}}{\partial(\partial_\mu\psi_a)}\delta\psi_a\right)\\
&=\partial_\mu\left(\partial^\mu\psi^*\delta\psi+\partial^\mu\psi\delta\psi^*\right)\\
&=i\alpha\partial_\mu\left[(\partial^\mu\psi^*)\psi-(\partial^\mu\psi)\psi^*\right]\\
&=i\alpha\left[(\partial_\mu\partial^\mu\psi^*)\psi-(\partial_\mu\partial^\mu\psi)\psi^*+(\partial^\mu\psi^*)(\partial_\mu\psi)-(\partial^\mu\psi)(\partial_\mu\psi^*)\right]\\
&=i\alpha\left[(\partial_\mu\partial^\mu\psi^*)\psi-(\partial_\mu\partial^\mu\psi)\psi^*\right]\\
&=i\alpha\left[(m^2\psi^*+(\psi^*\psi)\psi^*)\psi-(m^2\psi+(\psi^*\psi)\psi)\psi^*\right]\\
&=0
\end{align}

\subsection{Example Sheet 1 Oct 2007 Problem 4 - Lagrangian for a triplet of real fields - NOT FINISHED}
\begin{align}
\frac{\partial \mathcal{L}}{\partial\phi_a}
&=-m^2\phi_a\\
\frac{\partial \mathcal{L}}{\partial(\partial_\alpha\phi_a)}
&=\eta^{\mu\nu}\partial_\mu\phi_a\delta^\alpha_\nu
=\eta^{\mu\alpha}\partial_\mu\phi_a
=\partial^\alpha\phi_a\\
\delta\mathcal{L}
&=\frac{\partial\mathcal{L}}{\partial\phi_a}\delta\phi_a+\frac{\partial\mathcal{L}}{\partial(\partial_\mu\phi_a)}\overbrace{\delta(\partial_\mu\phi_a)}^{=\partial_\mu(\delta\phi_a)}\\
&=-m\phi_a\theta\epsilon_{abc}n_b\phi_c+(\partial^\mu\phi_a)\theta\epsilon_{abc}n_b\partial_\mu\phi_c\\
&=\theta[\epsilon_{abc}n_b(\partial^\mu\phi_a)(\partial_\mu\phi_c)-m\epsilon_{abc}n_b\phi_a\phi_c]\\
&=\theta[-n_b\epsilon_{bac}(\partial^\mu\phi_a)(\partial_\mu\phi_c)+m n_b\epsilon_{bac}\phi_a\phi_c]\\
&=\theta[-\vec{n}\cdot(\partial_\mu\phi\times\partial_\mu\phi)+m\vec{n}\cdot(\vec{\phi}\times\vec{\phi})]\\
&=0
\end{align}
Noether current
\begin{align}
j^\mu&=\theta(\partial^\mu\phi_a)\epsilon_{abc}n_b\phi_c\\
j^0&=-\theta n_b\epsilon_{bac}\phi_c\dot{\phi}_a
\end{align}

\subsection{Example Sheet 1 Oct 2007 Problem 5 - Lorentz transformation}
\begin{align}
\eta_{\mu\nu}x^\mu x^\nu
&=\eta_{\mu\nu}x'^\mu x'^\nu\\
&=\eta_{\sigma\tau}(\Lambda^\sigma_{\,\mu}x^\mu)(\Lambda^\tau_{\,\nu}x^\nu)\\
&=\eta_{\sigma\tau}\Lambda^\sigma_{\,\mu}\Lambda^\tau_{\,\nu}x^\mu x^\nu\\
&\rightarrow \eta_{\mu\nu}=\eta_{\sigma\tau}\Lambda^\sigma_{\,\mu}\Lambda^\tau_{\,\nu}
\end{align}
then
\begin{align}
\eta_{\mu\nu}
&=\eta_{\sigma\tau}\Lambda^\sigma_{\,\mu}\Lambda^\tau_{\,\nu}\\
&=\eta_{\sigma\tau}(\delta^\sigma_{\,\mu}+\omega^\sigma_{\,\mu})(\delta^\tau_{\,\nu}+\omega^\tau_{\,\nu})\\
&=\eta_{\sigma\tau}\delta^\sigma_{\,\mu}\delta^\tau_{\,\nu}
+\eta_{\sigma\tau}\delta^\sigma_{\,\mu}\omega^\tau_{\,\nu}
+\eta_{\sigma\tau}\omega^\sigma_{\,\mu}\delta^\tau_{\,\nu}+\mathcal{O}(\omega^2)\\
&\simeq\eta_{\mu\nu}+\eta_{\mu\tau}\omega^\tau_{\,\nu}+\eta_{\sigma\nu}\omega^\sigma_{\,\mu}\\
&\simeq\eta_{\mu\nu}+\omega^{\mu\nu}+\omega^{\nu\mu}\\
&\rightarrow\omega^{\mu\nu}=-\omega^{\nu\mu}
\end{align}
Rotation in the $x-y$ plane ($t$ and $z$ are undisturbed)
\begin{align}
\omega^\mu_{\,\nu}=
\left(
\begin{matrix}
0 & 0 & 0 & 0\\
0 & 0 & \epsilon & 0\\
0 & -\epsilon & 0 & 0\\
0 & 0 & 0 & 0\\
\end{matrix}
\right)
\end{align}
Boost in the $x$ direction ($y$ and $z$ are undisturbed)
\begin{align}
\omega^\mu_{\,\nu}=
\left(
\begin{matrix}
0 & \epsilon & 0 & 0\\
\epsilon & 0 & 0 & 0\\
0 & 0 & 0 & 0\\
0 & 0 & 0 & 0\\
\end{matrix}
\right)
\end{align}
Note that $\omega^\mu_{\,\nu}$ for the boost is symmetric and becomes antisymmetric when $\omega_{\alpha\nu}=\eta_{\alpha\mu}\omega^\mu_{\,\nu}$.

\subsection{Example Sheet 1 Oct 2007 Problem 6 - Lorentz transformation of a scalar field - NOT FINISHED}
For $x'=\Lambda x$ the transformation of the scalar field is given by
\begin{align}
\phi(x)\rightarrow\phi'(x')
&\equiv\phi(x)\\
&=\phi(\Lambda^{-1}x')\\
&\simeq\phi(x')+\partial_\mu\phi(x')[(\Lambda^{-1})^\mu_{\;\alpha}x'^\alpha-x'^\mu]\\
&=\phi(x')+\partial_\mu\phi(x')[(\delta^\mu_{\;\alpha}-\omega^\mu_{\;\alpha})x'^\alpha-x'^\mu]\\
&=\phi(x')-\partial_\mu\phi(x')\omega^\mu_{\;\alpha}x'^\alpha
\end{align}
Checking the expression for the inverse $\Lambda^{-1}$
\begin{align}
\Lambda^{-1}\Lambda&=1\\
(\Lambda^{-1})^\mu_{\;\alpha}\Lambda^\alpha_{\;\nu}&=(\delta^\mu_{\;\alpha}-\omega^\mu_{\;\alpha})(\delta^\alpha_{\;\nu}+\omega^\alpha_{\;\nu})\\
&=\delta^\mu_{\,\nu}-\omega^\mu_{\,\nu}+\omega^\mu_{\,\nu}\\
&=\delta^\mu_{\,\nu}
\end{align}

\subsection{Example Sheet 1 Oct 2007 Problem 7 - Energy momentum tensor  field for Maxwell field - NOT DONE YET}
\begin{itemize}
\item Checking invariance
\begin{align}
\mathcal{L}'&=-F'_{\mu\nu}F'^{\mu\nu}\\
&=-(\partial_\mu[A_\nu+\partial_\nu\xi]-\partial_\nu[A_\mu+\partial_\mu\xi])(\partial^\mu[A^\nu+\partial^\nu\xi]-\partial^\nu[A^\mu+\partial^\mu\xi])\\
&=-(\partial_\mu A_\nu+\partial_\mu\partial_\nu\xi-\partial_\nu A_\mu-\partial_\nu\partial_\mu\xi)
(\partial^\mu A^\nu+\partial^\mu\partial^\nu\xi-\partial^\mu\partial^\nu A^\mu-\partial^\nu\partial^\mu\xi])\\
&=-F_{\mu\nu}F^{\mu\nu}\\
&=\mathcal{L}
\end{align}
so $\mathcal{L}$ is invariant.
\item Noether theorem: the action being invariant under the transform
\begin{align}
A_\mu(x)\rightarrow A'_\mu(x)=A_\mu(x)+\epsilon G_i(A(x))
\end{align}
means that $\mathcal{L}$ can only differ by a total divergence
\begin{align}
\delta\mathcal{L}
&=\mathcal{L}(A',\partial A')-\mathcal{L}(A,\partial A)\\
&\overset{!}{=}\epsilon\partial_\mu X^\mu(A(x))
\end{align}
but
\begin{align}
\delta\mathcal{L}
&=\frac{\partial\mathcal{L}}{\partial A_\mu}\delta A_\mu+\frac{\partial\mathcal{L}}{\partial (\partial_\nu A_\mu)}\delta (\partial_\nu A_\mu)\\
&=\frac{\partial\mathcal{L}}{\partial A_\mu}\delta A_\mu+\frac{\partial\mathcal{L}}{\partial (\partial_\nu A_\mu)} \partial_\nu (\delta A_\mu)\\
&=\frac{\partial\mathcal{L}}{\partial A_\mu}\delta A_\mu+\partial_\nu
\left(\frac{\partial\mathcal{L}}{\partial (\partial_\nu A_\mu)}  (\delta A_\mu)\right)-\left(\frac{\partial\mathcal{L}}{\partial (\partial_\nu A_\mu)}\right)  (\delta A_\mu)\\
&=\partial_\nu\left(\frac{\partial\mathcal{L}}{\partial (\partial_\nu A_\mu)}  (\delta A_\mu)\right)\\
&=\epsilon\partial_\nu\left(\frac{\partial\mathcal{L}}{\partial (\partial_\nu A_\mu)}\partial_\mu X^\mu\right)
\end{align}
\item
\item
\end{itemize}



\subsection{Example Sheet 1 Oct 2007 Problem 8 - Massive vector field}

With $\mathcal{L}=-\frac{1}{4}F_{\mu\nu}F^{\mu\nu}+\frac{1}{2}m^2C_\mu C^\mu$
\begin{align}
\frac{\partial\mathcal{L}}{\partial C_\alpha}&=m^2C^\alpha\\
\frac{\partial\mathcal{L}}{\partial(\partial_\beta C_\alpha)}&=-\frac{2}{4}(\delta^\beta_\mu\delta^\alpha_\nu-\delta^\beta_\nu\delta^\alpha_\mu)F^{\mu\nu}=-\frac{1}{2}(F^{\beta\alpha}-F^{\alpha\beta})=F^{\alpha\beta}
\end{align}
resulting in the equations of motion
\begin{align}
-\partial_\beta F^{\alpha\beta}+m^2C^\alpha=0\\
-\partial_\beta (\partial^\alpha C^\beta-\partial^\beta C^\alpha)+m^2C^\alpha=0\\
-\partial^\alpha \partial_\beta C^\beta+\partial_\beta\partial^\beta C^\alpha+m^2C^\alpha=0
\end{align}
One more differentiation $\partial_\alpha$ and rearranging the differential operators we see
\begin{align}
-\partial_\alpha\partial^\alpha\partial_\beta C^\beta+\partial_\beta\partial^\beta \partial_\alpha C^\alpha+m^2\partial_\alpha C^\alpha=0\\
\rightarrow\partial_\alpha C^\alpha=0\\
\rightarrow\partial_0C^0=\partial_iC^i
\end{align}
Therefore the equations of motions simplify
\begin{align}
\partial_\beta\partial^\beta C^\alpha+m^2C^\alpha=0\\
(\partial_0\partial^0-\partial_i\partial^i) C^\alpha+m^2C^\alpha=0\\
\partial_0\partial^0C^\alpha-\partial_i\partial^i C^\alpha+m^2C^\alpha=0
\end{align}
then for $\alpha=0$
\begin{align}
\partial^0\underbrace{\partial_0C^0}_{=\partial_iC^i}-\partial_i\partial^i C^0+m^2C^0&=0\\
\partial_i\partial^i C^0-m^2C^0&=\partial_i\dot{C}^i\qquad\text{sign missing!?!}.
\end{align}
which means $C^0$ can be calculated from $C^i$ by solving the PDE. Now
\begin{align}
\Pi_\mu
&=\frac{\partial\mathcal{L}}{\partial(\partial_0C^\mu)}=F^{\mu0}=\partial^\mu C^0-\partial^0C^\mu\\
\Pi_0&=0\\
\Pi_i&=\partial^iC^0-\partial^0C^i
\end{align}
then with $F^{00}=0$
\begin{align}
\mathcal{H}
&=\Pi_\mu\partial_0C^\mu-\mathcal{L}\\
&=\Pi_i\partial_0C^i+\frac{1}{4}F_{\mu\nu}F^{\mu\nu}-\frac{1}{2}m^2C_\mu C^\mu\\
&=\Pi_i(\partial_iC^0-\Pi_i)+\frac{1}{4}F_{ij}F^{ij}+\frac{1}{4}F_{0j}F^{0j}+\frac{1}{4}F_{i0}F^{i0}-\frac{1}{2}m^2C_\mu C^\mu\\
&=\Pi_i(\partial_iC^0-\Pi_i)+\frac{1}{4}F_{ij}F^{ij}+\frac{1}{4}\Pi_j\Pi_j+\frac{1}{4}\Pi_i\Pi_i-\frac{1}{2}m^2C_\mu C^\mu\\
&=-\frac{1}{2}\Pi_i\Pi_i+\Pi_i\partial_iC^0+\frac{1}{4}F_{ij}F^{ij}-\frac{1}{2}m^2C_\mu C^\mu
\end{align}

\subsection{Example Sheet 1 Oct 2007 Problem 9 - Scale invariance}
With $x'=\lambda x$ or ($x=\lambda^{-1}x'$) and
\begin{align}
\phi(x)\rightarrow\phi'(x)=\lambda^{-D}\phi(\lambda^{-1}x)
\end{align}
we need to calculate the first derivative
\begin{align}
\partial'_\mu\phi'(x')
=\partial'_\mu\phi'(\lambda x)
&=\frac{\partial x^\alpha}{\partial x'^\mu}\frac{\partial}{\partial x^\alpha}\lambda^{-D}\phi(x)\\
&=\lambda^{-D-1}\partial_\mu\phi(x)
\end{align}
then
\begin{align}
S
&=\int d^nx\,(\partial_\mu\phi(x))(\partial^\mu\phi(x))+...\\
\rightarrow S'
&=\int d^nx'\,(\partial'_\mu\phi'(x'))(\partial'^\mu\phi'(x'))+...\\
&=\int \lambda^{n+1}d^nx\,\lambda^{2(-D-1)}(\partial_\mu\phi(x))(\partial^\mu\phi(x))-\frac{1}{2}m^2\lambda^{-2D}\phi^2-g\lambda^{-pD}\phi^p\\
&\rightarrow \lambda^{n+1-2(D+1)}=1\\
&\rightarrow D=\frac{n-1}{2}
\end{align}
It is a symmetry of the theory if
\begin{align}
n+1-2D=0\quad\rightarrow\quad D=\frac{n+1}{2} \quad\rightarrow\quad m=0
\end{align}
and
\begin{align}
n+1-pD=0\quad\rightarrow\quad p=\frac{n+1}{D}\quad\rightarrow\quad p=2\frac{n+1}{n-1}.
\end{align}
The scale invariant Lagrangian in 3+1 is the given by
\begin{align}
\mathcal{L}=\frac{1}{2}(\partial_\mu\phi)(\partial^\mu\phi)-g\phi^4
\end{align} 
Now calculating the Noether current for $n=3$, $D=1$ and $p=4$
\begin{align}
\delta\phi&=\lambda^{-1}\phi(\lambda^{-1}x)-\phi(x)\\
&=\lambda^{-1}\left(\phi(x)+\partial_\alpha\phi(x)[\lambda^{-1}x^\alpha-x^\mu]+...\right)-\phi(x)\\
&=(\lambda^{-1}-1)\phi(x)+\partial_\alpha x^\alpha\phi(x)(\lambda^{-1}-1)+...\\
&=(\lambda^{-1}-1)(\phi(x)+x^\alpha\partial_\alpha\phi(x))+...\\
&=\frac{1-\lambda}{\lambda}(\phi(x)+x^\alpha\partial_\alpha\phi(x))+...\\
&=\frac{\lambda-1}{\lambda}(-\phi(x)-x^\alpha\partial_\alpha\phi(x))+...
\end{align}
alternatively
\begin{align}
\delta\phi&=\lim_{\lambda\rightarrow1}\frac{d\, \lambda^{-1}\phi(\lambda^{-1}x)}{d\lambda}\\
&=-\phi(x)-x^\alpha\partial_\alpha\phi(x)\\
\delta\mathcal{L}
&=\lim_{\lambda\rightarrow1}\frac{d\mathcal{L}(d\, \lambda^{-1}\phi(\lambda^{-1}x))}{d\lambda}\\
&=\lim_{\lambda\rightarrow1}\frac{d}{d\lambda}\lambda^{-4}\mathcal{L}\\
&=\lim_{\lambda\rightarrow1}-4\lambda^3\mathcal{L}-\partial_\mu\mathcal{L}\frac{\partial(\lambda^{-1}x^\mu)}{\partial\lambda}\\
&=-4\mathcal{L}-x^\mu\partial_\mu\mathcal{L}\\
&=\partial_\mu(x^\mu\mathcal{L})
\end{align}
then
\begin{align}
j^\mu&=\frac{\partial\mathcal{L}}{\partial(\partial_\mu\phi)}\delta\phi-K^\mu\\
&=-\partial_\mu\phi (\phi(x)+x^\alpha\partial_\alpha\phi(x))+x^\mu\mathcal{L}
\end{align}

\newpage
\section{{\sc Liu} - Relativistic Quantum Field Theory I - MIT 2023 Spring}

\subsection{Problem 1.2 - Lorentz invariance of various $\delta$-functions}
\begin{enumerate}[(a)]
\item Using $px=\tilde{p}\tilde{x}$ is a Lorentz scalar and 
\begin{align}
\eta
&=\Lambda\eta\Lambda^T\\
\det\eta
&=\det\Lambda\cdot\det\eta\cdot\det\Lambda^T\quad
\rightarrow\quad 1=(\det\Lambda)^2
\end{align}
we see when rewriting the single $\delta$-functions by their Fourier representation
\begin{align}
\delta^{(4)}(p)
&=\delta(p^0)\delta(p^1)\delta(p^2)\delta(p^3)\\
&=\frac{1}{2\pi}\int (-1)\cdot e^{-ip^0x^0}dx^0\cdot ...\cdot\frac{1}{2\pi}\int 1\cdot e^{ip^3x^3}dx^3\\
&=-\frac{1}{2\pi}\iiiint d^4x\, e^{ipx}\\
&=-\frac{1}{2\pi}\iiiint \underbrace{|\det\Lambda^{-1}|^4}_{=1}d^4\tilde{x}\, e^{i\tilde{p}\tilde{x}}\\
&=\frac{1}{2\pi}\int (-1)\cdot e^{-i\tilde{p}^0\tilde{x}^0}d\tilde{x}^0\cdot ...\cdot\frac{1}{2\pi}\int 1\cdot e^{i\tilde{p}^3\tilde{x}^3}d\tilde{x}^3\\
&=\delta^{(4)}(\tilde{p})
\end{align}

\item 
\item 

\end{enumerate}




\newpage
\section{{\sc Banks} - Quantum Field Theory}
\subsection{Problem 2.2 - Time evolution operator in the Dirac picture}
With the definitions
\begin{align}
i\partial_tU_S&=(H_0+V)U_S\\
U_D(t,t_0)&=e^{iH_0t}U_S(t,t_0)e^{-iH_0t_0}
\end{align}
we can start rewriting
\begin{align}
i\partial_t U_D(t,t_0)
&=i\partial_t \left(e^{iH_0t}U_S(t,t_0)e^{-iH_0t_0}\right)\\
&=i^2H_0\underbrace{e^{iH_0t}U_S(t,t_0)e^{-iH_0t_0}}_{=U_D}+e^{iH_0t}i[\partial_tU_S(t,t_0)]e^{-iH_0t_0}\\
&=-H_0 U_D(t,t_0)+e^{iH_0t}i[\partial_tU_S(t,t_0)]e^{-iH_0t_0}\\
&=-H_0 U_D(t,t_0)+e^{iH_0t}(H_0+V)U_S(t,t_0)e^{-iH_0t_0}\\
&=-H_0 U_D(t,t_0)+H_0\underbrace{e^{iH_0t}U_S(t,t_0)e^{-iH_0t_0}}_{=U_D}+e^{iH_0t}VU_S(t,t_0)e^{-iH_0t_0}\\
&=e^{iH_0t}VU_S(t,t_0)e^{-iH_0t_0}\\
&=e^{iH_0t}V\underbrace{e^{-iH_0t}e^{iH_0t}}_{=1}U_S(t,t_0)e^{-iH_0t_0}\\
&=e^{iH_0t}Ve^{-iH_0t}U_D(t,t_0)
\end{align}


\newpage
\section{{\sc Kugo} - Eichtheorie}
\subsection{Problem 1.1}
With $\Lambda^\alpha_{\;\mu}\approx\delta^\alpha_\mu+\epsilon^\alpha_{\;\;\mu}$ we obtain
\begin{align}
g_{\mu\nu}
&=\Lambda^\alpha_{\;\mu}\Lambda^\beta_{\;\nu}g_{\alpha\beta}\\
&\simeq\left(\delta^\alpha_\mu+\epsilon^\alpha_{\;\;\mu}\right)\left(\delta^\beta_\nu+\epsilon^\beta_{\;\;\nu}\right)g_{\alpha\beta}\\
&\simeq g_{\mu\nu}+\epsilon^\alpha_{\;\;\mu}\delta^\beta_\nu g_{\alpha\beta}+\epsilon^\beta_{\;\;\nu}\delta^\alpha_\mu g_{\alpha\beta}+\mathcal{O}(\epsilon^2)\\
&\simeq g_{\mu\nu}+\epsilon_{\nu\mu}+\epsilon_{\mu\nu}+\mathcal{O}(\epsilon^2)
\end{align}
which means that $\epsilon$ is antisymmetric $\epsilon_{\nu\mu}=-\epsilon_{\mu\nu}$ and we can write
\begin{align}
\epsilon_{\nu\mu}=\frac{1}{2}\left(\epsilon_{\nu\mu}-\epsilon_{\mu\nu}\right).
\end{align}
The infinitesimal Poincare transformation can then be written as
\begin{align}
x'^\mu
&=\Lambda^\mu_{\;\alpha}x^\alpha+a^\mu\\
&\simeq\left(\delta^\mu_\alpha+\epsilon^\mu_{\;\;\alpha}\right)x^\alpha+a^\mu\\
&\simeq x^\mu+\epsilon^\mu_{\;\;\alpha}x^\alpha+a^\mu.
\end{align}
The inverted PT is then given by
\begin{align}
x&=\Lambda^{-1}(x'-a)\\
&=\Lambda^{-1}x'-\Lambda^{-1}a\\
x^\mu&\simeq\left(\delta^\mu_\alpha-\epsilon^\mu_{\;\;\alpha}\right)x'^\alpha-\left(\delta^\mu_\alpha-\epsilon^\mu_{\;\;\alpha}\right)a^\alpha\\
&\simeq x'^\mu-\epsilon^\mu_{\;\;\alpha}x'^\alpha-a^\mu+\mathcal{O}(\epsilon\cdot a)
\end{align}
Because of 
\begin{align}
\phi'(x')=\phi(x)
&\quad\Leftrightarrow\quad\phi'(\Lambda x+a)=\phi(x)\\
&\quad\Leftrightarrow\quad\phi'(x)=\phi(\Lambda^{-1}(x-a))
\end{align}
we can now calculate 
\begin{align}
\delta\phi(x)
&\equiv\phi'(x)-\phi(x)\\
&=\phi(\Lambda^{-1}(x-a))-\phi(x)\\
&\simeq\phi(x^\mu-\epsilon^\mu_{\;\;\alpha}x^\alpha-a^\mu)-\phi(x)\\
&\simeq\phi(x)+\partial_\mu\phi(x)\cdot(-\epsilon^\mu_{\;\;\alpha}x^\alpha-a^\mu)-\phi(x)\\
&\simeq-(a^\mu+\epsilon^\mu_{\;\;\alpha}x^\alpha)\partial_\mu\phi(x)\\
&\simeq-(a^\mu+\epsilon^{\mu\alpha}x_\alpha)\partial_\mu\phi(x)\\
&\simeq-\left(a^\mu+\frac{1}{2}\left(\epsilon^{\mu\alpha}-\epsilon^{\alpha\mu}\right)x_\alpha\right)\partial_\mu\phi(x)\\
&\simeq-\left(a^\mu\partial_\mu+\frac{1}{2}\left(\epsilon^{\mu\alpha}x_\alpha\partial_\mu-\epsilon^{\alpha\mu}x_\alpha\partial_\mu\right)\right)\phi(x)\\
&\simeq-\left(a^\mu\partial_\mu+\frac{1}{2}\left(\epsilon^{\mu\alpha}x_\alpha\partial_\mu-\epsilon^{\mu\alpha}x_\mu\partial_\alpha\right)\right)\phi(x)\\
&\simeq i\left(a^\mu i\partial_\mu+\frac{1}{2}\epsilon^{\mu\alpha}i\left(x_\alpha\partial_\mu-x_\mu\partial_\alpha\right)\right)\phi(x)\\
&\simeq i\left(a^\mu i\partial_\mu-\frac{1}{2}\epsilon^{\mu\alpha}i\left(x_\mu\partial_\alpha-x_\alpha\partial_\mu\right)\right)\phi(x)\\
&\simeq i\left(a^\mu P_\mu-\frac{1}{2}\epsilon^{\mu\alpha}M_{\mu\alpha}\right)\phi(x)
\end{align}
Calculating the commutators
\begin{align}
[P_\mu,P_\nu]=0
\end{align}
\begin{align}
[M_{\mu\nu},P_\rho]
&=i^2(x_\mu\partial_\nu-x_\nu\partial_\mu)\partial_\rho-i^2\partial_\rho(x_\mu\partial_\nu-x_\nu\partial_\mu)\\
&=-(x_\mu\partial_\nu-x_\nu\partial_\mu)\partial_\rho+\partial_\rho(x_\mu\partial_\nu-x_\nu\partial_\mu)\\
&=-x_\mu\partial_\nu\partial_\rho+x_\nu\partial_\mu\partial_\rho+ (\partial_\rho g_{\mu\alpha}x^\alpha)\partial_\nu+x_\mu\partial_\rho\partial_\nu
-(\partial_\rho g_{\nu\alpha}x^\alpha)\partial_\mu-x_\nu\partial_\rho\partial_\mu\\
&=(\partial_\rho g_{\mu\alpha}x^\alpha)\partial_\nu
-(\partial_\rho g_{\nu\alpha}x^\alpha)\partial_\mu\\
&=(g_{\mu\alpha}\partial_\rho x^\alpha)\partial_\nu
-(g_{\nu\alpha}\partial_\rho x^\alpha)\partial_\mu\\
&=(g_{\mu\alpha}\delta^\alpha_\rho)\partial_\nu
-(g_{\nu\alpha}\delta^\alpha_\rho)\partial_\mu\\
&=g_{\mu\rho}\partial_\nu
-g_{\nu\rho}\partial_\mu\\
&=-i(g_{\mu\rho}i\partial_\nu
-g_{\nu\rho}i\partial_\mu)\\
&=-i(g_{\mu\rho}P_\nu
-g_{\nu\rho}P_\mu)
\end{align}
\begin{align}
[M_{\mu\nu},M_{\rho,\sigma}]
&=...\text{painful}
\end{align}

\section{{\sc LeBellac} - Quantum and Statistical Field Theory}
\subsection{Problem 1.1}
Some simple geometry
\begin{align}
l&=2a\cos\theta\\
x&=l\sin\theta\\
&=2a\cos\theta\sin\theta\\
h&=x\tan\theta\\
&=2a\sin^2\theta
\end{align}
Then the potential is given by
\begin{align}
V(\phi)
&=2mga\sin^2\theta+\frac{1}{2}Ca^2(2\cos\theta-1)^2\\
\frac{\partial V}{\partial\theta}
&=4mga\sin\theta\cos\theta-2Ca^2(2\cos\theta-1)\sin\theta\\
&=2a\sin\theta\left(2mg\cos\theta-Ca(2\cos\theta-1)\right)\\
&=2a\sin\theta\left(2(mg-Ca)\cos\theta+Ca\right)\\
&\rightarrow\theta_0=0\\
&\rightarrow\theta_{1,2}=\arccos\frac{Ca}{2(Ca-mg)}
\end{align}
Stability
\begin{align}
\frac{\partial^2 V}{\partial\theta^2}(\theta_{1,2})
&=2a(2mg-Ca)\\
\frac{\partial^2 V}{\partial\theta^2}(\theta_{0})
&=2a(2mg-Ca)
\end{align}




\section{{\sc de Witt} - Dynamical theory of groups and fields}
\subsection{Problem 1 - Functional derivatives of actions}
\begin{align}
    \delta F
    &=\int dx \frac{\delta F[\phi]}{\delta\phi(x)}\cdot\delta\phi(x)\\
    &=\int dx \frac{\delta F[\phi]}{\delta\phi(x)}\cdot\epsilon\delta(x-y)\\
    &=\epsilon\frac{\delta F[\phi]}{\delta\phi(y)}\\
    &=F[\phi+\epsilon\delta(x-y)]-F[\phi]
\end{align}
which means
\begin{align}
    \frac{\delta F[\phi]}{\delta\phi(y)}&=\lim_{\epsilon\rightarrow0}\frac{F[\phi+\epsilon\delta(x-y)]-F[\phi]}{\epsilon}\\
    F[\phi+\epsilon\delta(x-y)]&=F[\phi]+\epsilon\frac{\delta F[\phi]}{\delta\phi(y)}\\
    &=F[\phi]+\epsilon\int dx \frac{\delta F[\phi]}{\delta\phi(x)}\cdot\delta(x-y)
\end{align}
Now
\begin{enumerate}[(a)]
\item Neutral scalar meson
\begin{align}
S
&=\int dx\; L(\varphi,\varphi_{,\mu})\\
&=-\frac{1}{2}\int dx\;(\varphi_{,\mu}\varphi^{,\mu}+m^2\varphi^2)\\
&=-\frac{1}{2}\left(\int dx\;(\varphi_{,\mu}\varphi^{,\mu}+m^2\varphi^2)\right)\\
&=-\frac{1}{2}\left(\int dx\;\varphi_{,\mu}\varphi^{,\mu}+\int dx\;m^2\varphi^2)\right)
\end{align}
Now we calculate the first part (all derivatives are with respect to $x$) neglecting $\mathcal{O}(\epsilon^2)$
\begin{align}
\frac{\delta S_1[\varphi]}{\delta\varphi(y)}
&=\lim_{\epsilon\rightarrow0}\frac{1}{\epsilon}\left(\int dx\;g^{\mu\nu}(\varphi(x)+\epsilon\delta(x-y))_{,\mu}(\varphi(x)+\epsilon\delta(x-y))_{,\nu}-\int dx\;g^{\mu\nu}\varphi(x)_{,\mu}\varphi(x)_{,\nu}\right)\\
&=\lim_{\epsilon\rightarrow0}\frac{1}{\epsilon}\left(\int dx\;g^{\mu\nu}(\varphi(x)_{,\mu}+\epsilon\partial_{\mu}\delta(x-y))(\varphi(x)_{,\nu}+\epsilon\partial_{\nu}\delta(x-y))-\int dx\;g^{\mu\nu}\varphi(x)_{,\mu}\varphi(x)_{,\nu}\right)\\
&=\lim_{\epsilon\rightarrow0}\frac{1}{\epsilon}\left(\int dx\;g^{\mu\nu}(\varphi_{,\mu}\varphi_{,\nu}+\epsilon\varphi_{,\nu}\partial_{\mu}\delta(x-y)+\epsilon\varphi_{,\mu}\partial_{\nu}\delta(x-y))-\int dx\;g^{\mu\nu}\varphi_{,\mu}\varphi_{,\nu}\right)\\
&=\int dx\,g^{\mu\nu}(\varphi_{,\nu}\partial_{\mu}\delta(x-y)+\varphi_{,\mu}\partial_{\nu}\delta(x-y))\\
&=-\int dx\,g^{\mu\nu}(\varphi_{,\nu\mu}\delta(x-y)+\varphi_{,\mu\nu}\delta(x-y))\\
&=-2\varphi_{\;,\mu}^{,\mu}(y)
\end{align}
\begin{align}
\frac{\delta S_2[\varphi]}{\delta\varphi(y)}
&=\lim_{\epsilon\rightarrow0}\frac{1}{\epsilon}m^2\left(\int dx\;(\varphi(x)+\epsilon\delta(x-y))(\varphi(x)+\epsilon\delta(x-y))-\int dx\;g^{\mu\nu}\varphi(x)\varphi(x)\right)\\
&=\lim_{\epsilon\rightarrow0}\frac{1}{\epsilon}m^2\left(\int dx\;(\varphi(x)\varphi(x)+\epsilon\delta(x-y)\varphi(x)+\epsilon\varphi(x)\delta(x-y))-\int dx\;g^{\mu\nu}\varphi(x)\varphi(x)\right)\\
&=m^2\int dx\;(\delta(x-y)\varphi(x)+\varphi(x)\delta(x-y))\\
&=2m^2\varphi(y)
\end{align}
and therefore
\begin{align}
\frac{\delta S[\varphi]}{\delta\varphi(y)}=\varphi_{\;,\mu}^{,\mu}(y)-m^2\varphi(y)
\end{align}

\item Neutral vector meson
\begin{align}
S_1&=-\frac{1}{4}F_{\mu\nu}F^{\mu\nu}=(\varphi_{\nu,\mu}-\varphi_{\mu,\nu})(\varphi^{\nu,\mu}-\varphi^{\mu,\nu})\\
\frac{\delta S_1[\varphi]}{\delta\varphi_\alpha(y)}&=-\frac{1}{4}\lim_{\epsilon\rightarrow0}\frac{1}{\epsilon}\int dx\,
[(\varphi_{\nu}(x)+\epsilon\delta^\alpha_\nu\delta(x-y))_{,\mu}-(\varphi_{\mu}(x)+\epsilon\delta^\alpha_\mu\delta(x-y))_{,\nu}]\\
%
&\cdot[(\varphi^{\nu}(x)+\epsilon\delta^{\alpha\nu}\delta(x-y))^{,\mu}-(\varphi^{\mu}(x)+\epsilon\delta^{\alpha\mu}\delta(x-y))^{,\nu}]-
[\varphi_{\nu,\mu}-\varphi_{\mu,\nu}]
[\varphi^{\nu,\mu}-\varphi^{\mu,\nu}]\\
%
&=-\frac{1}{4}\int dx\left((\delta_\nu^\alpha\partial_\mu\delta(x-y)-\delta_\mu^\alpha\partial_\nu\delta(x-y))[\varphi^{\nu,\mu}-\varphi^{\mu,\nu}]\right.\\
&\qquad\left.+[\varphi_{\nu,\mu}-\varphi_{\mu,\nu}](\delta^{\nu\alpha}\partial^\mu\delta(x-y)-\delta^{\mu\alpha}\partial^\nu\delta(x-y))\right)\\
&=-\frac{1}{4}\int dx\left(
 \partial_\mu\delta(x-y))[\varphi^{\alpha,\mu}-\varphi^{\mu,\alpha}]
-\partial_\nu\delta(x-y))[\varphi^{\nu,\alpha}-\varphi^{\alpha,\nu}]\right.\\
&\qquad\left.
+[\varphi^\alpha_{\;,\mu}-\varphi_{\mu}^{\;,\alpha}]\partial^\mu\delta(x-y)
-[\varphi_{\nu}^{\;,\alpha}-\varphi^{\alpha}_{\;,\nu}]\partial^\nu\delta(x-y)
\right)\\
&=\frac{1}{4}\int dx\,\delta(x-y)\left(
[\varphi^{\alpha,\mu}_{\;,\mu}-\varphi^{\mu,\alpha}_{\;,\mu}]
-[\varphi^{\nu,\alpha}_{,\nu}-\varphi^{\alpha,\nu}_{,\nu}]
+[\varphi^{\alpha\;,\mu}_{,\mu}-\varphi_{\mu}^{,\alpha\mu}]
-[\varphi_{\nu}^{,\alpha\nu}-\varphi^{\alpha\;,\nu}_{,\nu}]
\right)\\
&=\frac{1}{4}\int dx\,\delta(x-y)\left(
4\varphi^{\alpha,\mu}_{\;\;,\mu}-
2\varphi^{\mu,\alpha}_{\;,\mu}-
2\varphi_{\mu}^{,\alpha\mu}
\right)\\
&=\varphi(y)^{\alpha,\mu}_{\;\;,\mu}-\varphi(y)^{\mu,\alpha}_{\;,\mu}
\end{align}
and
\begin{align}
S_2&=-\frac{m^2}{2}\varphi_\mu\varphi^\mu\\
\frac{\delta S_2[\varphi]}{\delta\varphi_\alpha(y)}&=-\frac{m^2}{2}
\lim_{\epsilon\rightarrow0}\frac{1}{\epsilon}\int dx
\left[(\varphi_\mu+\epsilon\delta_\mu^\alpha\delta(x-y))(\varphi^\mu+\epsilon\delta^{\mu\alpha}\delta(x-y))-\varphi_\mu\varphi^\mu\right]\\
&=-\frac{m^2}{2}\int dx\left[\delta_\mu^\alpha\delta(x-y)\varphi^\mu+\varphi_\mu\delta^{\mu\alpha}\delta(x-y)\right]\\
&=-\frac{m^2}{2}\int dx\left[\delta(x-y)\varphi^\alpha+\varphi^\alpha\delta(x-y)\right]\\
&=-m^2\varphi^\alpha(y)
\end{align}
therefore
\begin{align}
\frac{\delta S[\varphi]}{\delta\varphi^\alpha(y)}=\varphi(y)^{\alpha,\mu}_{\;\;\;,\mu}-\varphi(y)^{\mu,\alpha}_{\;\;,\mu}-m^2\varphi^\alpha
\end{align}

\item Neutral tensor meson

\item Two-level mass spectrum

Using results from (a)
\begin{align}
S_2&=\frac{1}{2}\varphi_{,\mu}\varphi^{,\mu}\frac{m_1^2+m_2^2}{m_1^2-m_2^2}\\
\frac{\delta S_2[\varphi]}{\delta\varphi(y)}&=-\frac{m_1^2+m_2^2}{m_1^2-m_2^2}\varphi^{,\mu}_{\;\;,\mu}\\
S_3&=\frac{1}{2}\varphi^2\frac{m_1^2m_2^2}{m_1^2-m_2^2}\\
\frac{\delta S_3[\varphi]}{\delta\varphi(y)}&=\frac{m_1^2m_2^2}{m_1^2-m_2^2}\varphi
\end{align}
and
\begin{align}
S_1&=\varphi^{,\mu\nu}\varphi_{,\mu\nu}\\
\frac{\delta S_1[\varphi]}{\delta\varphi(y)}
&=...\\
&=\int dx \left(\partial^{\mu\nu}\delta(x-y)\varphi_{,\mu\nu}+\varphi^{,\mu\nu}\partial_{\mu\nu}\delta(x-y)\right)\\
&=\int dx \left(\delta(x-y)\varphi_{,\mu\nu}^{\;\;\;,\mu\nu}+\varphi^{,\mu\nu}_{\;\;\;,\mu\nu}\delta(x-y)\right)\\
&=2\varphi^{,\mu\nu}_{\;\;\;,\mu\nu}(y)
\end{align}
Resulting in
\begin{align}
\frac{\delta S[\varphi]}{\delta\varphi(y)}
&=\frac{1}{m_1^2-m_2^2}\varphi^{,\mu\nu}_{\;\;\;,\mu\nu}(y)
-\frac{m_1^2+m_2^2}{m_1^2-m_2^2}\varphi^{,\mu}_{\;\;,\mu}
+\frac{m_1^2m_2^2}{m_1^2-m_2^2}\varphi\\
&=\frac{1}{m_1^2-m_2^2}(\partial^\mu\partial_{\mu}-m_1^2)(\partial^\nu\partial_{\nu}-m_2^2)\varphi
\end{align}
\end{enumerate}

\subsection{Problem 2 - More Lagrangians}
\begin{enumerate}[(a)]
\item Notation is a bit odd - vector field $\varphi^\mu$ and scalar field $\varphi$
\begin{align}
\frac{\partial L}{\partial\varphi^\beta}-\partial_\alpha\frac{\partial L}{\partial\varphi^\beta_{,\alpha}}&=0\\
\varphi_\beta-\frac{1}{2}\varphi_{,\beta}-\partial_\alpha\left(\frac{1}{2}\varphi\delta^{\mu\alpha}\delta_{\mu\beta}\right)&=0\\
\rightarrow\varphi_\beta-\varphi_{,\beta}&=0\\
\frac{\partial L}{\partial\varphi}-\partial_\alpha\frac{\partial L}{\partial\varphi_{,\alpha}}&=0\\
\frac{1}{2}\varphi^\mu_{\;\;,\mu}-m^2\varphi-\partial_\alpha\left(-\frac{1}{2}\varphi^\alpha\right)&=0\\
\rightarrow \varphi^\mu_{\;\;,\mu}-m^2\varphi&=0
\end{align}
now we can separate both equations of motion by
\begin{align}
\varphi^\alpha-\varphi^{,\alpha}&=0\quad\rightarrow\quad \varphi^\alpha_{\;\;,\alpha}-\varphi^{,\alpha}_{\;\;,\alpha}=0\\
\varphi^\mu_{\;\;,\mu\alpha}-m^2\varphi_{,\alpha}&=0
\end{align}
and obtain
\begin{align}
\varphi^{,\alpha}_{\;\;,\alpha}-m^2\varphi&=0\\
\varphi^\mu_{\;\;,\mu\alpha}-m^2\varphi_\alpha&=0\qquad\text{or better}\qquad\varphi_{,\beta}=\varphi_\beta
\end{align}
\item
\item
\end{enumerate}

\subsection{Problem 3 - Implied equations of motion}
\begin{enumerate}[(a)]
\item Nothing to do
\item 
\item
\end{enumerate}


\end{document}