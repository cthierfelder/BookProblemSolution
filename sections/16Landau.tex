\documentclass[../main.tex]{subfiles}

%\graphicspath{{\subfix{../images/}}}

\begin{document}

\section{{\sc Landau Lifschitz} - Vol 2 - Classical Theory of fields 3rd ed}
\subsection{39.1 - Charge in repulsive Coulomb field}
As usual we assume angular momentum conservation (movement in a plane and $M=r\cdot p_\varphi$) 
\begin{align}
\mathcal{E}
&=c\sqrt{p^2+m^2c^2}+\frac{\alpha}{r}\\
&=c\sqrt{p_r^2+p_\varphi^2+m^2c^2}+\frac{\alpha}{r}\\
&=c\sqrt{p_r^2+\frac{M^2}{r^2}+m^2c^2}+\frac{\alpha}{r}\\
&\rightarrow v=c\sqrt{1-\frac{m^2c^4}{\mathcal{E}^2}}\\
&\rightarrow \mathcal{E}=\gamma mc^2\\
&\rightarrow r_\text{min}=\frac{\alpha\gamma+\sqrt{\alpha^2+M^2c^2(\gamma^2-1)}}{mc^2(\gamma^2-1)}
\end{align}
then with $(\gamma mc,\gamma m\vec{v})=(\mathcal{E}/c,\vec{p})$ we get two equations for $\dot\varphi$ and $\dot r$
\begin{align}
p_r^2&=\gamma^2m^2v_r^2=\gamma^2m^2\dot{r}^2=\frac{1}{c^2}\left(\mathcal{E}-\frac{\alpha}{r}\right)^2-\frac{M^2}{r^2}-m^2c^2\\
p_\varphi&=\gamma m v_\varphi=\gamma m r\dot{\varphi}=\frac{M}{r}
\end{align}
dividing them gives us $\varphi(r)$ by integration
\begin{align}
\frac{\dot\varphi}{\dot{r}}
&=\frac{d\varphi}{dr}
=\frac{M}{r^2}\frac{1}{\sqrt{\frac{1}{c^2}\left(\mathcal{E}-\frac{\alpha}{r}\right)^2-\frac{M^2}{r^2}-m^2c^2}}
\end{align}
\begin{align}
\varphi_0
&=\int^\infty_{r_\text{min}} dr\frac{M}{r\sqrt{\frac{1}{c^2}\left(\gamma m c^2r-\alpha\right)^2-M^2-m^2c^2r^2}}\\
&=\left.-\frac{2 c M}{\sqrt{c^2 M^2-\alpha ^2}} \arctan\left(\frac{c\sqrt{m^2c^2r^2 \left(\gamma ^2-1\right)}-c
   \sqrt{m^2c^2r^2 \left(\gamma ^2-1\right) +\frac{\alpha ^2}{c^2}-2 \alpha  \gamma  mr-M^2}}{\sqrt{c^2 M^2-\alpha ^2}}\right)\right|_{r_\text{min}}^\infty\\
&=-\frac{2 c M}{\sqrt{c^2 M^2-\alpha ^2}} \arctan\left(\frac{\alpha  \gamma + \sqrt{\alpha
+\left(\gamma ^2-1\right) M^2c^2}}{\sqrt{ \left(\gamma ^2-1\right)} \sqrt{c^2 M^2-\alpha ^2}}\right)
\end{align}
Then with $2\arctan x =\arctan\frac{2x}{1-x^2}$ and $\arctan(-x)=-\arctan x$
\begin{align}
\chi&=\pi-2\varphi_0\\
&=\pi-(-1)\frac{2\cdot2 c M}{\sqrt{c^2 M^2-\alpha ^2}} \arctan\left(\frac{\alpha  \gamma + \sqrt{\alpha
+\left(\gamma ^2-1\right) M^2c^2}}{\sqrt{ \left(\gamma^2-1\right)} \sqrt{c^2 M^2-\alpha ^2}}\right)\\
&=\pi-(-1)\frac{2 c M}{\sqrt{c^2 M^2-\alpha ^2}} \arctan\left(-\frac{\sqrt{M^2c^2-\alpha^2}\sqrt{-1+\gamma^2}}{\alpha\gamma}\right)\\
&=\pi-\frac{2 c M}{\sqrt{c^2 M^2-\alpha ^2}} \arctan\left(\frac{v\sqrt{M^2c^2-\alpha^2}}{\alpha c}\right)
\end{align}
where we used $\gamma/\sqrt{-1+\gamma^2}=c/v$.

Now we recover the classical relation between impact parameter $b$ and the deflection angle $\chi$. First we express the conserved angular momentum at $r\rightarrow\infty$
\begin{align}
M=p_r^\infty r \sin\alpha=p_r^\infty r\frac{b}{r}=\gamma m vb\simeq mvb
\end{align}
Calculating the non-relativistic approximation
\begin{align}
\chi
&=\pi-\frac{2Mc}{Mc\left(1-\frac{\alpha^2}{2M^2c^2}+...\right)}\arctan\frac{vMc\left(1-\frac{\alpha^2}{2M^2c^2}+...\right)}{\alpha c}\\
&\simeq\pi-2\arctan\frac{vM}{\alpha}
\end{align}
then (assuming a positively charged ($q=ze$) particle scattered at a atomic nucleus ($Q=Ze$) which means ($\alpha=zZe^2/4\pi\epsilon_0$) and $\cot\left(\pi/2-\arctan x\right)=x$
\begin{align}
\frac{\chi}{2}&=\frac{\pi}{2}-\arctan \frac{4\pi\epsilon_0}{zZe^2}mv^2b\\
\cot\frac{\chi}{2}&=\frac{4\pi\epsilon_0mv^2b}{zZe^2}\\
b&=\frac{zZe^2}{4\pi\epsilon_0mv^2}
\end{align}

\end{document}